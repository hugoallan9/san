\renewcommand{\thepage}{\roman{page}}
\clearpage

$\ $
\vspace{14.5cm}

\noindent\begin{tabular}{p{0.9cm}p{6.8cm}}
& 2016.$\,$ Guatemala, Centro América \\
&\Bold Instituto Nacional de Estadística\\[-0.4cm]
&\color{blue!50!black}\url{www.ine.gob.gt}\\[0.9cm]
\end{tabular}\\
\noindent\begin{tabular}{p{0.9cm}p{6.8cm}}
& Está permitida la reproducción parcial o total de los contenidos de este documento con la mención de la fuente. \\[0.5cm]
 
& Este documento fue elaborado empleando  {\Sans R}, Inkscape y {\Logos \XeLaTeX}.\\
\end{tabular} 
\pagestyle{estandar}

\clearpage


%	\includepdf{portadaazul.pdf}
	
	\clearpage
	\newpage $\ $
	\newpage $\ $
$\ $
\vspace{7cm}

\begin{center}
\Bold \LARGE COMPENDIO ESTADÍSTICO DE\\
SEGURIDAD ALIMENTARIA Y NUTRICIONAL 2015
\end{center}
\cleardoublepage
\pagestyle{estandar}
%
%$\ $
%\vspace{0.0cm}
%
%\begin{center}
%{\Bold \LARGE AUTORIDADES}\\[1cm]
%
%
%{\Bold \large \color{color1!89!black} JUNTA  DIRECTIVA} \\[0.4cm]
%
%
%{\Bold Ministerio de Economía}\\
%Titular: Rubén Estuardo Morales Monroy\\
%Suplente: Abel Francisco Cruz Calderón\\[0.4cm]
%
%
%{\Bold Ministerio de Finanzas}\\
%Titular: Julio Héctor Estrada\\
%Suplente: Victor Manuel Martínez Ruiz\\[0.4cm]
%
%
%{\Bold Ministerio de Agricultura, Ganadería y Alimentación}\\
%Titular: Mario Méndez Montenegro\\
%Suplente:  Rosa Elvira Pacheco Mangandi\\[0.4cm]
%
%
%{\Bold Ministerio de Energía y Minas}\\
%Titular: Juan Pelayo Castañón\\
%Suplente: César Roberto Velásquez Barrera\\[0.4cm]
%
%
%{\Bold Secretaría de Planificación y Programación de la Presidencia}\\
%Titular: Miguel Ángel Moir Mérida\\
%Suplente: Edna Abigail Álvarez\\[0.4cm]
%
%
%{\Bold Banco de Guatemala}\\
%Titular: Julio Roberto Suárez Guerra\\
%Suplente: Sergio Francisco Recinos Rivera\\[0.4cm]
%
%
%
%{\Bold Universidad de San Carlos de Guatemala}\\
%Titular: Murphy Olimpo Paiz\\
%Suplente:  Óscar René Paniagua \\[0.4cm]
%
%
%{\Bold Universidades Privadas}\\
%Titular: Miguel Ángel Franco de León\\
%Suplente: Ariel Rivera Irías\\[0.4cm]
%
%
%{\Bold Comité Coordinador de Asociaciones\\ Agrícolas, Comerciales, Industriales y Financieras}\\
%Titular: Juan Raúl Aguilar Kaehler\\
%Suplente:  Oscar Augusto Sequeira García\\[0.8cm]
%
%
%{\Bold \large \color{color1!89!black} GERENCIA}\\[0.2cm]
%Gerente: Néstor Mauricio Guerra Morales.\\
%Subgerente Técnico: Jaime Roberto Mejía Salguero\\
%Subgerente Administrativo Financiero: Orlando Roberto Monzón Girón\\
%
%
%\end{center}

\clearpage

$\ $
\vspace{1cm}

\begin{center}
{\Bold \LARGE AUTORIDADES INSTITUCIONALES}\\[2cm]

{\Bold \large \color{color1!89!black} Dirección de Censos y Encuestas}\\[0.2cm]
Carlos Enrique Mancía Chúa\\[0.8cm]


{\Bold \large \color{color1!89!black} Dirección de Índices y Estadísticas Continuas}\\[0.2cm]
Luis Eduardo Arroyo Gálvez\\[0.8cm]

{\Bold \large \color{color1!89!black} Dirección Administrativa}\\[0.2cm]
Edgar Rolando Elías Pichillá\\[0.8cm]

{\Bold \large \color{color1!89!black} Dirección Financiera}\\[0.2cm]
María Elena Galindo Rodríguez\\[0.8cm]


{\Bold \large \color{color1!89!black} Dirección de Informática}\\[0.2cm]
César Calderón Barillas\\[0.8cm]


\end{center}\cleardoublepage

%%%%%%%%%%%%%%%%%%%%%%%%%%%%%%%%%%%
\clearpage

$\ $
\vspace{1cm}

\begin{center}
	{\Bold \LARGE EQUIPO RESPONSABLE}\\[2cm]
	
	{\Bold \large \color{color1!89!black} Coordinación general:}\\[0.2cm]
Nestor Mauricio Guerra Morales – Gerente INE \\
Edwin Portillo Portillo – Sub-gerente administrativo financiero INE \\
Fredy Arizmendy Gómez Gómez – Sub-gerente técnico INE \\
Raúl Maas – Director Iarna-URL\\[0.8cm]

{\Bold \large \color{color1!89!black} Coordinación técnica y contenido:}\\[0.2cm]
Lorena Ninel Estrada – Iarna-URL\\
Haydee Azucena Barrientos Osorio– INE\\
Juan Enrique Lee Samayoa- INE\\
Luis Eduardo Arroyo Gálvez – INE\\
Gonzalo Adolfo Hernández – Sesan\\
Nora Griselda Cano Escalante– Sesan\\
Héctor Antonio Tuy - Iarna-URL\\
Jaime Luis Carrera - Iarna-URL\\[0.8cm]
%Lorena Ninel Estrada – Iarna-URL\\
%Haydeé Barrientos - INE\\
%Juan Lee - INE\\
%Luis Eduardo Arroyo Gálvez - INE\\
%Gonzalo Hernández - Sesan\\
%Nora Cano - Sesan\\[0.8cm]
{\Bold \large \color{color1!89!black} Recopilación de datos:}\\[0.2cm]

Renato Vargas – Iarna-URL\\
Vivian Guzmán – Iarna-URL \\
Hugo Allan García - Iarna-URL \\
Fabiola Beatriz Ramírez - Iarna-URL \\[0.8cm]

	{\Bold \large \color{color1!89!black} Edición:}\\[0.2cm]
Haydee Azucena Barrientos Osorio - INE\\
Lorena Ninel Estrada - Iarna-URL \\[0.8cm]


		
\end{center}\cleardoublepage
%\swapgeometry

%%%%%%%%%%%%%%%%%%%%%%%%%%%%%%%%%%%%%%%



\clearpage

$\ $
\vspace{1cm}

\begin{center}
	{\Bold \LARGE Oficina Coordinadora Sectorial de Estadísticas de Seguridad Alimentaria y Nutricional (OCSESAN)}\\[2cm]
	
\begin{itemize}
	\item Asociación de Investigación y Estudios Sociales (Asies)
	\item 	Asociación Nacional de Municipalidades (ANAM)
	\item 	Instituto de Investigación y Proyección sobre Ambiente Natural y Sociedad  (Iarna-URL)
	\item 	Instituto Guatemalteco de Seguridad Social (IGSS)
	\item 	Ministerio de Ambiente y Recursos Naturales (MARN)
	\item 	Ministerio de Economía de Guatemala (Mineco)
	\item 	Ministerio de Educación (Mineduc)
	\item 	Ministerio de Finanzas Públicas (Minfin)
	\item 	Ministerio de Salud Pública y Asistencia Social (Mspas)
	\item 	Ministerio de Trabajo y Previsión Social (Mintrab)
	\item 	Procuraduría de los Derechos Humanos de Guatemala (PDH)
	\item 	Secretaría de Seguridad Alimentaria y Nutricional (Sesan)
	\item 	Secretaría General de Planificación y Programación de la Presidencia (Segeplan)
	\item 	Usaid/DevTech - Programa de Monitoreo y Evaluación
	\item 	Viceministerio de Seguridad Alimentaria y Nutricional (Visan)
	\item 	Ministerio de Agricultura, Ganadería y Alimentación (MAGA)
\end{itemize}	

\end{center}\cleardoublepage
%\swapgeometry


%%%%%%%%%%%%%%%%%%%%%%%%%%%


\clearpage

$\ $
\vspace{1cm}

\begin{center}
	{\Bold \LARGE Fuentes primarias de información}\\[2cm]
	
	\begin{itemize}
			\item Banco Mundial (BM)
			\item 	Banco de Guatemala (Banguat)
			\item 	Instituto Guatemalteco de Seguridad Social (IGSS)
			\item 	Instituto Nacional de Estadística (INE)
			\item 	Ministerio de Agricultura, Ganadería y Alimentación (MAGA) – Dirección de Planeamiento (Diplan)
			\item 	Ministerio de Finanzas (Minfin) - Sistema de Contabilidad Integrada (Sicoin)
			\item 	Ministerio de Salud Pública y Asistencia Social (Mspas) - Sistema de Información Gerencial de Salud (Sigsa)
			\item 	Ministerio de Trabajo y Previsión Social (Mintrab)
			
	\end{itemize}	
	
\end{center}\cleardoublepage
%\swapgeometry




\cleardoublepage
\pagestyle{estandar}
\setlength{\arrayrulewidth}{1.0pt}





\cleardoublepage


$\ $\\[2cm]
\indent\titulo{\Bold \huge Presentación}
\addcontentsline{toc}{chapter}{Presentación}
\addtocontents{toc}{\protect\thispagestyle{estandar}}




$\ $\\[0.5cm]
\large
\indent El Instituto Nacional de Estadística 	(INE) se complace en presentar el Compendio Estadístico para Seguridad Alimentaria y Nutricional 2015, con la finalidad de contribuir a mejorar el nivel de comprensión sobre el estado de la Seguridad Alimentaria y Nutricional (SAN) y sus condicionantes en el país.

El INE como ente rector y coordinador del Sistema Estadístico Nacional (Decreto Ley 3-85 Ley Orgánica del Instituto Nacional de Estadística), impulsa diferentes iniciativas tendientes a mejorar la calidad de la estadística en SAN, así como los procesos de recopilación, procesamiento, análisis y difusión de la misma, siendo este compendio uno de estos procesos.

La información contenida en este documento se organizó en siete capítulos con sus respectivos indicadores y sub-indicadores. El primer capítulo es una caracterización poblacional del país. Los siguientes cinco capítulos corresponden específicamente a las dimensiones de seguridad alimentaria y nutricional, y son: a) Disponibilidad de alimentos, b) acceso a los alimentos, c) consumo de los alimentos, d) utilización biológica de los alimentos, y e) situación y atención a la desnutrición/malnutrición. El último capítulo corresponde a  inversión pública en seguridad alimentaria y nutricional. 

Las dimensiones, indicadores y sub-indicadores en que se basó el compendio fueron socializados durante las reuniones ordinarias y extraordinarias de la Ocsesan, por lo que los representantes de las instituciones miembros que asistieron tuvieron la oportunidad de brindar retroalimentación. Los cuadros con la información estadística, de la cual se derivan las gráficas, se encuentran disponibles en la versión electrónica del compendio SAN, en formato xlsx.

Asimismo, mediante un proceso de retroalimentación usuario-productor, el INE espera recibir comentarios acerca de este trabajo, con el fin de mejorar continuamente la información presentada, así como mejorar los procesos que engloban la generación, procesamiento y difusión del Compendio Estadístico SAN 2015.


$\ $

\cleardoublepage
	


\indent\titulo{Agradecimientos}
\addcontentsline{toc}{chapter}{Agradecimientos}
\addtocontents{toc}{\protect\thispagestyle{estandar}}
\\


Detrás de este esfuerzo de integración, procesamiento, análisis y difusión de la información, hay una cantidad de instituciones representadas por profesionales y técnicos, que participaron en la producción del Compendio Estadístico en Seguridad Alimentaria y Nutricional (SAN) 2015, en forma directa o indirecta.

El INE deja constancia de su agradecimiento a las siguientes entidades: La Secretaría de Seguridad Alimentaria y Nutricional (Sesan), el Sistema de Información Gerencial de Salud (Sigsa) del Ministerio de Salud Pública y Asistencia Social (Mspas), El Ministerio de Finanzas Públicas (Minfin), la Secretaría General de Planificación y Programación de la Presidencia (Segeplan), el Ministerio de Educación (Mineduc), el Viceministerio de Seguridad Alimentaria y Nutricional (Visan), y la Dirección de Planeamiento (Diplan) del Ministerio de Agricultura, Ganadería y Alimentación (MAGA), el Ministerio de Ambiente y Recursos Naturales (MARN), la Procuraduría de los Derechos Humanos de Guatemala (PDH), el Instituto Guatemalteco de Seguridad Social (IGSS), el Ministerio de Economía de Guatemala (Mineco), la Asociación de Investigación y Estudios Sociales (Asies), el Ministerio de Trabajo y Previsión Social (Mintrab), y la Asociación Nacional de Municipalidades (Anam).

Asimismo, se agradece el apoyo del Instituto de Investigación y Proyección sobre Ambiente Natural y Sociedad (Iarna) de la Universidad Rafael Landívar (URL), y a DevTech Systems por el apoyo técnico y financiero derivado del sub-contrato No. 1096-SC-13-001-00.\\[16mm]



%%%%%%%%%%%%%%%%%%%%%%%%%%%
\newpage
	$\ $\\[-1cm]
 \textcolor{color2}{\Huge Acrónimos y abreviaturas}\\
	\normalsize	$\ $\\
\begin{center}\fontsize{3.8mm}{1.4em}\selectfont \setlength{\arrayrulewidth}{0.7pt}
	$\ $\\[-1.5cm]
	$\!$\begin{tabular}{p{4.3cm}l}	\hline
	ANAM	&	Asociación Nacional de Municipalidades	\\
\rowcolor{color1!5!white}	Asies	&	Asociación de Investigación y Estudios Sociales	\\
	Banguat	&	Banco de Guatemala	\\
\rowcolor{color1!5!white}	Cacif	&	Comité Coordinador de Asociaciones Agrícolas, Comerciales, Industriales y Financieras	\\
	CMA	&	Cumbre Mundial sobre la Alimentación	\\
\rowcolor{color1!5!white}	Diplan	&	Dirección de Planeamiento	\\
	Encovi	&	Encuesta Nacional de Condiciones de Vida	\\
\rowcolor{color1!5!white}	ENEI	&	Encuesta Nacional de Empleo e Ingresos	\\
	Ensmi	&	Encuesta Nacional de Salud Materno Infantil	\\
\rowcolor{color1!5!white}	FAO	&	Organización de las Naciones Unidas para la Alimentación y la Agricultura, por sus siglas en inglés	\\
	FIDA	&	Fondo Internacional de Desarrollo Agrícola	\\
\rowcolor{color1!5!white}	GHI	&	Índice del Hambre Global, siglas en inglés	\\
	GSARS	&	Estrategia Global para el Mejoramiento de las Estadísticas	\\
	\rowcolor{color1!5!white}Iarna	&	Instituto de Agricultura, Recursos Naturales y Ambiente	\\
		IDH	&	Índice de Desarrollo Humano\\
\rowcolor{color1!5!white}	Ifpri	&	Instituto de Investigación de Política Alimentaria	\\
	IGSS	&	Instituto Guatemalteco de Seguridad Social	\\
		\rowcolor{color1!5!white} INE	&	Instituto Nacional de Estadística\\
	MAGA	&	Ministerio de Agricultura, Ganadería y Alimentación	\\
	\rowcolor{color1!5!white}MARN	&	Ministerio de Ambiente y Recursos Naturales	\\
	MEM	&	Ministerio de Energía y Minas	\\
	\rowcolor{color1!5!white}Mineco	&	Ministerio de Economía de Guatemala	\\
	Mineduc	&	Ministerio de Educación	\\
	\rowcolor{color1!5!white}Minfin	&	Ministerio de Finanzas Públicas	\\
	Mspas	&	Ministerio de Salud Pública y Asistencia Social	\\
	\rowcolor{color1!5!white}Mintrab	&	Ministerio de Trabajo y Previsión Social	\\
	Ocsesan	&	Oficina Coordinadora Sectorial de Estadísticas de Seguridad Alimentaria y Nutricional	\\
	\rowcolor{color1!5!white} ODM	&	Objetivos del Desarrollo del Milenio	\\
	PDH	&	Procuraduría de los Derechos Humanos de Guatemala	\\
	\rowcolor{color1!5!white}PMA	&	Programa Mundial de Alimentos	\\
	PPH0	&	Plan del Pacto Hambre Cero	\\
	\rowcolor{color1!5!white}SAN	&	Seguridad Alimentaria y Nutricional	\\
	Segeplan	&	Secretaría General de Planificación y Programación de la Presidencia	\\
	\rowcolor{color1!5!white}Sesan	&	Secretaría de Seguridad Alimentaria y Nutricional	\\
	Sicoin	&	Sistema de Contabilidad Integrada (Sicoin)	\\
	\rowcolor{color1!5!white}Sigsa	&	Sistema de Información Gerencial de Salud	\\
	URL	&	Universidad Rafael Landívar	\\
	\rowcolor{color1!5!white}Usaid	&	Agencia de los Estados Unidos para el Desarrollo Internacional, siglas en inglés	\\
	Visan	&	Viceministerio de Seguridad Alimentaria y Nutricional	\\
		\hline
		&\\[-0.28cm]
		%			\multicolumn{9}{l}{\footnotesize Fuente: Informe Nacional de Desarrollo Humano (PNUD), con base en las Encuestas Nacionales de Condiciones de Vida (Encovi).}
	\end{tabular}
\end{center}



$\ $\\[1cm]
\newpage
\renewcommand{\contentsname}{Contenidos}
\tableofcontents
\clearpage
\thispagestyle{estandar}
\listofexample


