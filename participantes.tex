
\clearpage

$\ $
\vspace{14.5cm}

\noindent\begin{tabular}{p{0.9cm}p{6.8cm}}
& 2016.$\,$ Guatemala, Centro América \\
&\Bold Instituto Nacional de Estadística\\[-0.4cm]
&\color{blue!50!black}\url{www.ine.gob.gt}\\[0.9cm]
\end{tabular}\\
\noindent\begin{tabular}{p{0.9cm}p{6.8cm}}
& Está permitida la reproducción parcial o total de los contenidos de este documento con la mención de la fuente. \\[0.5cm]
 
& Este documento fue elaborado empleando  {\Sans R}, Inkscape y {\Logos \XeLaTeX}.\\
\end{tabular} 
\pagestyle{empty}

\clearpage


%	\includepdf{portadaazul.pdf}
	
	\clearpage
	\newpage $\ $
	\newpage $\ $
$\ $
\vspace{7cm}

\begin{center}
\Bold \LARGE COMPENDIO ESTADÍSTICO\\
SEGURIDAD ALIMENTARIA 2015
\end{center}
\cleardoublepage


$\ $
\vspace{0.0cm}

\begin{center}
{\Bold \LARGE AUTORIDADES}\\[1cm]


{\Bold \large \color{color1!89!black} JUNTA  DIRECTIVA} \\[0.4cm]


{\Bold Ministerio de Economía}\\
Titular: Rubén Estuardo Morales Monroy\\
Suplente: Abel Francisco Cruz Calderón\\[0.4cm]


{\Bold Ministerio de Finanzas}\\
Titular: Julio Héctor Estrada\\
Suplente: Victor Manuel Martínez Ruiz\\[0.4cm]


{\Bold Ministerio de Agricultura, Ganadería y Alimentación}\\
Titular: Mario Méndez Montenegro\\
Suplente:  Rosa Elvira Pacheco Mangandi\\[0.4cm]


{\Bold Ministerio de Energía y Minas}\\
Titular: Juan Pelayo Castañón\\
Suplente: César Roberto Velásquez Barrera\\[0.4cm]


{\Bold Secretaría de Planificación y Programación de la Presidencia}\\
Titular: Miguel Ángel Moir Mérida\\
Suplente: Edna Abigail Álvarez\\[0.4cm]


{\Bold Banco de Guatemala}\\
Titular: Julio Roberto Suárez Guerra\\
Suplente: Sergio Francisco Recinos Rivera\\[0.4cm]



{\Bold Universidad de San Carlos de Guatemala}\\
Titular: Murphy Olimpo Paiz\\
Suplente:  Óscar René Paniagua \\[0.4cm]


{\Bold Universidades Privadas}\\
Titular: Miguel Ángel Franco de León\\
Suplente: Ariel Rivera Irías\\[0.4cm]


{\Bold Comité Coordinador de Asociaciones\\ Agrícolas, Comerciales, Industriales y Financieras}\\
Titular: Juan Raúl Aguilar Kaehler\\
Suplente:  Oscar Augusto Sequeira García\\[0.8cm]


{\Bold \large \color{color1!89!black} GERENCIA}\\[0.2cm]
Gerente: Néstor Mauricio Guerra Morales.\\
Subgerente Técnico: Jaime Roberto Mejía Salguero\\
Subgerente Administrativo Financiero: Orlando Roberto Monzón Girón\\


\end{center}

\clearpage

$\ $
\vspace{1cm}

\begin{center}
{\Bold \LARGE AUTORIDADES INSTITUCIONALES}\\[2cm]

{\Bold \large \color{color1!89!black} Dirección de Censos y Encuestas}\\[0.2cm]
Carlos Enrique Mancía Chúa\\[0.8cm]


{\Bold \large \color{color1!89!black} Dirección de Índices y Estadísticas Continuas}\\[0.2cm]
Luis Eduardo Arroyo Gálvez\\[0.8cm]

{\Bold \large \color{color1!89!black} Dirección Administrativa}\\[0.2cm]
Edgar Rolando Elías Pichillá\\[0.8cm]

{\Bold \large \color{color1!89!black} Dirección Financiera}\\[0.2cm]
María Elena Galindo Rodríguez\\[0.8cm]


{\Bold \large \color{color1!89!black} Dirección de Informática}\\[0.2cm]
César Calderón Barillas\\[0.8cm]


\end{center}\setcounter{page}{0}\cleardoublepage

%%%%%%%%%%%%%%%%%%%%%%%%%%%%%%%%%%%
\clearpage

$\ $
\vspace{1cm}

\begin{center}
	{\Bold \LARGE EQUIPO RESPONSABLE}\\[2cm]
	
	{\Bold \large \color{color1!89!black} Coordinación general:}\\[0.2cm]

Héctor Tuy – Iarna-URL\\
Lorena Ninel Estrada – Iarna-URL\\
Haydeé Barrientos - INE\\
Juan Lee - INE\\
Luis Eduardo Arroyo Gálvez - INE\\
Gonzalo Hernández - Sesan\\
Nora Cano - Sesan\\[0.8cm]

	{\Bold \large \color{color1!89!black} Edición:}\\[0.2cm]
Lorena Ninel Estrada – Iarna-URL\\[0.8cm]


	{\Bold \large \color{color1!89!black} Formato:}\\[0.2cm]

Lorena Ninel Estrada – Iarna-URL\\
Alejandro Gándara – Iarna -URL\\[0.8cm]

	{\Bold \large \color{color1!89!black} Recopilación de datos:}\\[0.2cm]
Renato Vargas – consultor Iarna-URL\\
Vivian Guzmán – consultora Iarna-URL\\
Haydeé Barrientos - INE\\
Gonzalo Hernández – Sesan\\
Nora Cano - Sesan\\
Lorena Ninel Estrada – Iarna-URL\\
Hugo Allan García - consultor Iarna-URL\\
Fabiola Ramírez - consultora Iarna-URL\\
Orlando Roberto Monzón Girón - INE\\[0.8cm]

	{\Bold \large \color{color1!89!black} Apoyo a Revisión de Contenidos (Iarna-URL)}\\[0.2cm]

Héctor Tuy – Iarna-URL\\
Jaime Luis Carrera – Iarna-URL\\
Cecilia Cleaves – Iarna-URL\\[0.8cm]
	
	
\end{center}\setcounter{page}{0}\cleardoublepage
%\swapgeometry

%%%%%%%%%%%%%%%%%%%%%%%%%%%%%%%%%%%%%%%



\clearpage

$\ $
\vspace{1cm}

\begin{center}
	{\Bold \LARGE Oficina Coordinadora Sectorial de Estadísticas de Seguridad Alimentaria y Nutricional (OCSESAN)}\\[2cm]
	
\begin{itemize}
	\item Asociación de Investigación y Estudios Sociales (Asies)
	\item 	Asociación Nacional de Municipalidades (ANAM)
	\item 	Instituto de Investigación y Proyección sobre Ambiente Natural y Sociedad  (Iarna-URL)
	\item 	Instituto Guatemalteco de Seguridad Social (IGSS).
	\item 	Ministerio de Agricultura, Ganadería y Alimentación (MAGA)
	\item 	Ministerio de Ambiente y Recursos Naturales (MARN)
	\item 	Ministerio de Economía de Guatemala (Mineco)
	\item 	Ministerio de Educación (Mineduc)
	\item 	Ministerio de Finanzas Públicas (Minfin)
	\item 	Ministerio de Salud Pública y Asistencia Social (Mspas)
	\item 	Ministerio de Trabajo y Previsión Social (Mintrab)
	\item 	Procuraduría de los Derechos Humanos de Guatemala (PDH)
	\item 	Secretaría de Seguridad Alimentaria y Nutricional (Sesan)
	\item 	Secretaría General de Planificación y Programación de la Presidencia (Segeplan)
	\item 	Usaid/DevTech - Programa de Monitoreo y Evaluación
	\item 	Viceministerio de Seguridad Alimentaria y Nutricional (Visan)
	
\end{itemize}	

\end{center}\setcounter{page}{0}\cleardoublepage
%\swapgeometry


%%%%%%%%%%%%%%%%%%%%%%%%%%%


\clearpage

$\ $
\vspace{1cm}

\begin{center}
	{\Bold \LARGE Fuentes primarias de información}\\[2cm]
	
	\begin{itemize}
			\item Banco Mundial
			\item 	Banco de Guatemala (Banguat)
			\item 	Instituto Guatemalteco de Seguridad Social (IGSS)
			\item 	Instituto Nacional de Estadística (INE)
			\item 	Ministerio de Agricultura, Ganadería y Alimentación (MAGA) – Dirección de Planeamiento (Diplan)
			\item 	Ministerio de Finanzas (Minfin) - Sistema de Contabilidad Integrada (Sicoin)
			\item 	Ministerio de Salud Pública y Asistencia Social (Mspas) - Sistema de Información Gerencial de Salud (Sigsa)
			\item 	Ministerio de Trabajo y Previsión Social (Mintrab)
			
	\end{itemize}	
	
\end{center}\setcounter{page}{0}\cleardoublepage
%\swapgeometry


%%%%%%%%%%%%%%%%%%%%%%%%%%%

$\ $\\[1cm]

\tableofcontents

\cleardoublepage
	\pagestyle{estandar}
	\setcounter{page}{1}
	\setlength{\arrayrulewidth}{1.0pt}





\cleardoublepage


$\ $\\[2cm]
\thispagestyle{empty}
\indent\titulo{\Bold \huge Presentación}




$\ $\\[0.5cm]
\large
\indent El Instituto Nacional de Estadística 	(INE) se complace en presentar el Compendio Estadístico para Seguridad Alimentaria y Nutricional 2015, con la finalidad de contribuir a mejorar el nivel de comprensión sobre el estado de la Seguridad Alimentaria y Nutricional (SAN) en el país.

El INE como ente rector y coordinador del Sistema Estadístico Nacional (Decreto Ley 3-85 Ley Orgánica del Instituto Nacional de Estadística), impulsa diferentes iniciativas tendientes a mejorar la calidad de la estadística en SAN, así como los procesos de recopilación, procesamiento, análisis y difusión de la misma, siendo este compendio uno de estos procesos.

La información contenida en este documento se organizó en siete dimensiones con sus respectivos indicadores y sub-indicadores. La primera dimensión es una caracterización poblacional del país. Las siguientes seis dimensiones corresponden específicamente a seguridad alimentaria y nutricional, y son: a) Disponibilidad de alimentos, b) acceso a los alimentos, c) consumo de los alimentos, d) utilización biológica de los alimentos, e) situación y atención a la desnutrición/malnutrición, e f) inversión pública en seguridad alimentaria y nutricional. 

Las dimensiones, indicadores y sub-indicadores en que se basó el compendio fueron socializados durante las reuniones ordinarias y extraordinarias de la Ocsesan, por lo que los representantes de las instituciones miembros que asistieron tuvieron la oportunidad de brindar retroalimentación.

Asimismo, mediante un proceso de retroalimentación usuario-productor, el INE espera recibir comentarios acerca de este trabajo, con el fin de mejorar continuamente la información presentada, así como mejorar los procesos que engloban la generación, procesamiento y difusión del Compendio Estadístico SAN 2015.


$\ $

\cleardoublepage



\indent\titulo{Agradecimientos}
\\


Detrás de este esfuerzo de integración, procesamiento, análisis y difusión de la información, hay una cantidad de instituciones representadas por profesionales y técnicos, que participaron en la producción del Compendio Estadístico en Seguridad Alimentaria y Nutricional (SAN) 2015, en forma directa o indirecta.

El INE deja constancia de su agradecimiento a las siguientes entidades: La Secretaría de Seguridad Alimentaria y Nutricional (Sesan), el Sistema de Información Gerencial de Salud (Sigsa) del Ministerio de Salud Pública y Asistencia Social (Mspas), El Ministerio de Finanzas Públicas (Minfin), la Secretaría General de Planificación y Programación de la Presidencia (Segeplan), el Ministerio de Educación (Mineduc), el Viceministerio de Seguridad Alimentaria y Nutricional (Visan), y la Dirección de Planeamiento (Diplan) del Ministerio de Agricultura, Ganadería y Alimentación (MAGA), el Ministerio de Ambiente y Recursos Naturales (MARN), la Procuraduría de los Derechos Humanos de Guatemala (PDH), el Instituto Guatemalteco de Seguridad Social (IGSS), el Ministerio de Economía de Guatemala (Mineco), la Asociación de Investigación y Estudios Sociales (Asies), el Ministerio de Trabajo y Previsión Social (Mintrab), y la Asociación Nacional de Municipalidades (Anam).

Asimismo, se agradece el apoyo del Instituto de Investigación y Proyección sobre Ambiente Natural y Sociedad (Iarna) de la Universidad Rafael Landívar (URL) y a DevTech Systems por el apoyo técnico y financiero derivado del sub-contrato No. 1096-SC-13-001-00.\\[16mm]



