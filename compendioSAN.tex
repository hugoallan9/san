\documentclass[11pt,twoside]{book}
%Versión 3.91

%Paquetes estándar
\usepackage{amsmath}
\usepackage{amsfonts}
\usepackage{amssymb}
\usepackage{graphicx}
\usepackage{pdfpages}
\usepackage{setspace} 
\usepackage{xltxtra}
\usepackage{enumitem}
\usepackage{fixltx2e}

%Agregado por Hugo
\usepackage{color}

%Tikz
\usepackage{tikz}
\usetikzlibrary{calc}
\usetikzlibrary{positioning}


% Estilo de página en blanco en el índice
\newcommand{\blank}{\addtocontents{toc}{\protect\thispagestyle{empty}}}


%If Then Else, definiciones libres de comandos
\usepackage{xifthen}
\usepackage{xargs} %varios argumentos opcionales


%título y fuente
\newcommand{\titulodoc}{Nombre del documento}
\newcommand{\lafuente}{Estadísticas INE}


%Tablas de Excel convertidas a LaTeX, y otros aspectos de cuadros
\usepackage{booktabs}
\usepackage{multirow}
\usepackage{hhline}

\newcounter{Cuadro}[chapter]
\renewcommand{\theCuadro}{\thechapter.\arabic{Cuadro}}


\newcommand{\titulocuadro}[1]{\addtocounter{Cuadro}{1}
{\Bold\color{color1!80!black}{\normalsize Cuadro \theCuadro $\,-$  #1 }}
}







%Columnas definibles en ancho
\usepackage{array}
\newcolumntype{x}[1]{%
	>{\centering\arraybackslash}p{#1}}%

\newcolumntype{g}[1]{%
	>{\raggedleft\arraybackslash}p{#1}}%
	
\newcolumntype{q}[1]{%
	>{\raggedright\arraybackslash}p{#1}}%
	
\usepackage[input-decimal-markers={.}, input-ignore={,}, group-separator={,}]{siunitx}


%Para pruebas
\usepackage{lipsum}


%Para compilar en XeLaTeX con tildes
\usepackage{polyglossia}
\setmainlanguage{spanish}

%Tipo de letra
\usepackage{fontspec}
\setmainfont[
BoldFont = OpenSans-CondBold.ttf ,
ItalicFont = OpenSans-CondLightItalic.ttf ,
BoldItalicFont = OpenSans-CondLightItalic.ttf ]{OpenSans-CondLight.ttf}
\newfontfamily\Bold{Open Sans Condensed Bold}

\newfontfamily\Sans{Open Sans}
\newfontfamily\SansBold{Open Sans Bold}
\newfontfamily\Italic{Open Sans Condensed Light Italic}
\newfontfamily\Logos{Latin Modern Roman}

\newfontfamily\Cinzel{Cinzel}


%%%%%%%%%%% Diseño global del documento
\usepackage[paperwidth=8.5in, paperheight=11in, left=1.25in, right=0.75in, top=0.950in, bottom=0.925in]{geometry}
%	\setlength{\headsep}{0pt}
%	\setlength{\footskip}{46pt}
\setlength{\parindent}{1.5em}		%sangría
\setlength{\parskip}{2ex}		%separación entre párrafos  

%Distancias
\newlength{\cuadri} 
\setlength{\cuadri}{0.125in}


%Tabla de contenidos y vinculaciones
\usepackage{tocloft}
\usepackage[hidelinks, verbose]{hyperref}
\usepackage{url}


%Formato de contenidos
\setlength{\cftbeforetoctitleskip}{0em}
\AtBeginDocument{\addtocontents{toc}{\protect\thispagestyle{empty}}} 

\makeatletter
\renewcommand*\l@subsection{\@dottedtocline{2}{5.2em}{3.2em}}
\makeatother

\renewcommand{\thesection}{\thechapter.\arabic{section}}

\cftsetpnumwidth{2\cuadri}
\cftsetrmarg{8\cuadri}
\renewcommand{\cftsecnumwidth}{2.5\cuadri}
\renewcommand{\cftchapnumwidth}{2\cuadri}
\renewcommand{\cftsecindent}{2\cuadri}


% Elementos geométricos de diseño del cuerpo (cajas de colores, etc.)
\usepackage{colortbl}

\usepackage{multicol}
\setlength{\columnsep}{1.1cm}


%Cambios de márgenes y según paridad de hojas
\usepackage{changepage}
\strictpagecheck


%Colores base del documento
\definecolor{color1}{rgb}{0,0,1}
\definecolor{color2}{rgb}{0.3,0.5,1}


%Para que las páginas en blanco no estén numeradas
\let\origdoublepage\cleardoublepage
\newcommand{\clearemptydoublepage}{
	\clearpage
	{\pagestyle{empty}\origdoublepage}}
\let\cleardoublepage\clearemptydoublepage


%%%%%%%% Llamadas y notas al pie
\makeatletter
\newcommand{\markerspace}{\@ifnextchar.%
{$\!$}{\@ifnextchar,%
    {$\!$}{\@ifnextchar;%
        {$\!$}{\@ifnextchar:%
            {$\!$}{$\ $}
        }
    }
}}
\makeatother

\newcounter{numllamada}
\newcounter{numtextollamada}
\setcounter{numllamada}{0}
\setcounter{numtextollamada}{0}


\newcommand{\llamada}[1][\thenumllamada]{\stepcounter{numllamada}\begingroup\setcounter{mpfootnote}{#1}\renewcommand\thefootnote\thempfootnote$\hspace{0.2ex}$\footnotemark\endgroup\markerspace}

\newcommand{\notita}[2][\thenumtextollamada]{\stepcounter{numtextollamada}\stepcounter{numllamada}$\hspace{0.2ex}$\footnote[#1]{#2}}


\newcommand{\textollamada}[2][\thenumtextollamada]{%
\ifthenelse{\equal{#1}{*}}{\begingroup\renewcommand\thefootnote\thempfootnote\footnotetext[0]{#2\\[-1.7ex]}\endgroup}{\stepcounter{numtextollamada}\begingroup\renewcommand\thefootnote\thempfootnote\footnotetext[#1]{#2}\endgroup}
}

%redefinición de footnote
\newlength{\footnoterulewidth} \setlength{\footnoterulewidth}{2.5cm} \newlength{\footnoteruleheight} \setlength{\footnoteruleheight}{.4pt} 
\makeatletter
 \renewcommand{\footnoterule}{ \kern -3pt   \color{color2}\hrule width \footnoterulewidth height \footnoteruleheight   \kern \dimexpr 3pt - \footnoteruleheight \relax } 
\makeatother

\makeatletter
\renewcommand\@makefntext[1]{%
  \noindent\makebox[1em][r]{\scriptsize\@makefnmark}\scriptsize#1}
\makeatother



%%%%%%%%%%% tablas
\usepackage{pdflscape}
\usepackage{rotating}
\usepackage{bigstrut}
\usepackage{longtable}
\LTcapwidth=1.234\textwidth
\setlength{\arrayrulewidth}{0.8pt}
\setlength\doublerulesep{1.3pt}
\arrayrulecolor{color2}

\newlength{\Oldarrayrulewidth}
\newcommand{\Cline}[2]{%
	\noalign{\global\setlength{\Oldarrayrulewidth}{\arrayrulewidth}}%
	\noalign{\global\setlength{\arrayrulewidth}{#1}}\cline{#2}%
	\noalign{\global\setlength{\arrayrulewidth}{\Oldarrayrulewidth}}}


%%%%%%%%%%% Paquete Tcolorbox
\usepackage[skins, breakable, hooks]{tcolorbox}


% Definición del comando cajita

\newtcolorbox{cajita-arriba}{width=52\cuadri, height=35\cuadri, enlarge left by=0pt, enlarge top by =1\cuadri, enlarge bottom by=2\cuadri, nobeforeafter, colframe=white, colback=white, left=-3pt, right=0pt, bottom= 0pt, top=-3pt, arc=0pt, boxrule=0pt}

\newtcolorbox{cajita-abajo}{width=52\cuadri, height=35\cuadri, enlarge top by=0pt, enlarge left by=0pt, enlarge bottom by=-2\cuadri, nobeforeafter, colframe=white, colback=white, left=-3pt, right=0pt, bottom= 0pt, top=-3pt, arc=0pt, boxrule=0pt}

\newtcolorbox{cajota-unica}{width=52\cuadri, height=72\cuadri,  enlarge left by=0pt, enlarge top by =1\cuadri, enlarge bottom by=-2\cuadri, nobeforeafter, colframe=white, colback=white, left=-3pt, right=0pt, bottom= 0pt, top=-3pt, arc=0pt, boxrule=0pt}

\newtcolorbox{descripcion-cajita}{width=18\cuadri, height=30.8\cuadri, nobeforeafter, colframe=white, colback=white, left=-3pt, right=-3pt, bottom= -3pt, top=-8pt, arc=0pt, boxrule=0pt, enlarge bottom by=-29.78\cuadri}

\newtcolorbox{grafica-cajita}{width=32\cuadri, height=30.8\cuadri, nobeforeafter, colframe=white, colback=white, left=-3pt, right=-3pt, bottom= -3pt, top=-2pt, arc=0pt, boxrule=0pt, enlarge bottom by=-29.78\cuadri}

% Cajas para fondos de capítulo y encabezado

\newtcolorbox{fondo-capitulo}{width=8.5in, height=11in, skin=enhancedmiddle,  nobeforeafter,  watermark graphics=fondo-capitulo.pdf, watermark opacity=1.0, watermark overzoom=1.0, enlarge left by=-1.525in, enlarge top by=-0.95in,  enlarge bottom by=-20\cuadri, boxrule=0pt, colframe=white, left=-3pt, bottom=-1pt, top=8\cuadri, right=-3pt, arc=0pt}

\newtcolorbox{parte-toc}{width=52\cuadri, enlarge left by=0pt, enlarge top by =0\cuadri, enlarge bottom by=0\cuadri, nobeforeafter, colframe=color1!90!black, colback=white, left=3pt, right=0pt, bottom= 0pt, top=0pt, arc=0pt, boxrule=0pt, leftrule=4pt}

\newtcolorbox{fondo-parte}{width=8.5in, height=11in, skin=enhancedmiddle,  nobeforeafter,  watermark graphics=parte.pdf, watermark opacity=1.0, watermark overzoom=1.0, enlarge left by=-1.455in, enlarge top by=-0.95in,  enlarge bottom by=-20\cuadri, boxrule=0pt, colframe=white, left=-3pt, bottom=-1pt, top=8\cuadri, right=-3pt, arc=0pt}

\newtcolorbox{fondo-capitulo-no-descripcion}{width=8.5in, height=11in, skin=enhancedmiddle,  nobeforeafter,  watermark graphics=fondo-capitulo-no-descripcion.pdf, watermark opacity=1.0, watermark overzoom=1.0, enlarge left by=-1.525in, enlarge top by=-0.95in,  enlarge bottom by=-20\cuadri, boxrule=0pt, colframe=white, left=-3pt, bottom=-1pt, top=8\cuadri, right=-3pt, arc=0pt}

\newtcolorbox{numcapitulo}{height=1.2in, width=1.12in, enlarge top by= -1.3in, enlarge bottom by=-0.47in,boxrule=3pt,arc=3pt, colframe=color1, colback=color1, right=2pt, left=3pt, top=14pt, bottom=2pt}

\newtcolorbox{encabezadoimpar}{width=8.5in, height=0.86in, skin=enhancedmiddle,  nobeforeafter,  watermark graphics=topodd3.pdf, watermark opacity=1.0, watermark overzoom=1.0, enlarge left by=-1.25in, enlarge top by=-0.45in,  enlarge bottom by=3\cuadri, boxrule=0pt, colframe=white, left=0.71in, bottom=-1pt, top=2.5\cuadri, right=0.71in, arc=0pt}

\newtcolorbox{encabezadopar}{width=8.5in, height=0.86in, skin=enhancedmiddle,  nobeforeafter,  watermark graphics=topeven3.pdf, watermark opacity=1.0, watermark overzoom=1.0, enlarge left by=-0.75in, enlarge top by=-0.45in,  enlarge bottom by=3\cuadri, boxrule=0pt, colframe=white, left=0.71in, bottom=-1pt, top=2.5\cuadri, right=0.71in, arc=0pt}

\newtcbox{numpag}{colback=color1, colframe=color1, arc=0pt, top=3pt, bottom=2pt, left=3pt, right=3pt, nobeforeafter, enlarge top by=-0.125in}

%%%%%%%%%%% Encabezado y pie de página
\usepackage{fancyhdr}

\fancypagestyle{estandar}{%
\fancyhf{}
\renewcommand{\headrulewidth}{0pt}
\fancyhead[CO]{%
\begin{encabezadoimpar}
\end{encabezadoimpar}
}
\fancyfoot[RO]{%
\color{color2}{\capituloencabezado \ \ } \color{black}\raisebox{0.5mm}{$\mid$}\color{color1}\textbf{ \ \ \ \thepage}
}
\fancyhead[CE]{%
\begin{encabezadopar}
\end{encabezadopar}
}
\fancyfoot[LE]{%
\color{color1}\textbf{\thepage \ \ \ }\color{black}\raisebox{0.5mm}{$\mid$}\color{color2}{ \ \  \titulodoc}
}
}


\fancypagestyle{soloarriba}{%
	\fancyhf{}
	\renewcommand{\headrulewidth}{0pt}
	\fancyhead[CO]{%
		\begin{encabezadoimpar}
		\end{encabezadoimpar}
	}
	\fancyfoot[RO]{}
	\fancyhead[CE]{%
		\begin{encabezadopar}
		\end{encabezadopar}
	}
	\fancyfoot[LE]{}
}


\pagestyle{estandar}



%%%%%%%%%%% Macros de cajitas

%  El comando se escribe así: \cajita[Sección en índice]{Sección en cuerpo}{Descripción}{Título gráfica}{Desagregación}{Gráfica con \includegraphics o tikz}{Fuente}

\newcounter{updown}
\setcounter{updown}{0}


\newcommand{\cajitaalternante}[1]{%
\ifthenelse{\equal{\theupdown}{0}}{\setcounter{numllamada}{0}\setcounter{numtextollamada}{0}\noindent\begin{cajita-arriba}\phantomsection\stepcounter{section} #1\end{cajita-arriba}\setcounter{updown}{1}}{\noindent\begin{cajita-abajo}\phantomsection\stepcounter{section} #1\end{cajita-abajo}\setcounter{updown}{0}}%
}


\newcommand{\titulizador}[1]{
\begin{tabular}{@{}p{3.5\cuadri}|p{3pt}@{}p{46.0\cuadri}}
& &\\[-1\cuadri]
\textbf{\color{color2}\Large \thesection}$\ $ & &\textbf{\Large #1}\\[-1\cuadri]
 & &\\[-0.8pt] \cline{2-3}
\end{tabular}

}


\newcommand{\titulizadormanual}[2]{
\begin{tabular}{@{}p{3.5\cuadri}|p{3pt}@{}p{46.0\cuadri}}
& &\\[-1\cuadri]
\textbf{\color{color2}\Large #2}$\ $ & &\textbf{\Large #1}\\[-1\cuadri]
 & &\\[-0.8pt] \cline{2-3}
\end{tabular}

}




\newcommand{\cajitaderecha}[7][]{%
\ifthenelse{\isempty{#1}}{
\cajitaalternante{
\addcontentsline{toc}{section}{\numberline{\thesection} #2}
\addtocontents{toc}{\protect\thispagestyle{empty}}
\titulizador{#2}

\begin{tabular}[b]{@{}p{17.5\cuadri}@{}p{1.5\cuadri}@{} x{34\cuadri}@{}}
& &\\[0.5\cuadri]
\begin{descripcion-cajita}
\parskip 6pt\parindent 1em%
#3
\end{descripcion-cajita}
  & &
\begin{grafica-cajita}
\begin{center}
\ifthenelse{\isempty{#4}}{}{{\textbf{#4}}\\[-1pt]}%
\ifthenelse{\isempty{#5}}{$\ $\\[-0.1\cuadri]}{
	{\footnotesize\texttwelveudash$\,\,$#5$\,\,$\texttwelveudash}\\[0.6\cuadri]}
#6
\begin{flushleft}
$\ $\\[-2\cuadri]
\ \ \ \footnotesize Fuente: #7
\end{flushleft}
\end{center}
\end{grafica-cajita}
\end{tabular}
}}
{
\cajitaalternante{
\addcontentsline{toc}{section}{\numberline{\thesection} #1}
\addtocontents{toc}{\protect\thispagestyle{empty}}
\titulizador{#2}

\begin{tabular}[b]{@{}p{17.5\cuadri}@{}p{1.5\cuadri}@{} x{34\cuadri}@{}}
& &\\[0.5\cuadri]
\begin{descripcion-cajita}
\parskip 6pt\parindent 1em%
#3
\end{descripcion-cajita}
  & &
\begin{grafica-cajita}
\begin{center}
\ifthenelse{\isempty{#4}}{}{{\textbf{#4}}\\[-1pt]}%
\ifthenelse{\isempty{#5}}{$\ $\\[-0.1\cuadri]}{
	{\footnotesize\texttwelveudash$\,\,$#5$\,\,$\texttwelveudash}\\[0.6\cuadri]}
#6
\begin{flushleft}
$\ $\\[-2\cuadri]
\ \ \ \footnotesize Fuente: #7
\end{flushleft}
\end{center}
\end{grafica-cajita}
\end{tabular}
}}
}


\newcommand{\cajitaizquierda}[7][]{%
\ifthenelse{\isempty{#1}}{
\cajitaalternante{
\addcontentsline{toc}{section}{\numberline{\thesection} #2}
\addtocontents{toc}{\protect\thispagestyle{empty}}
\titulizador{#2}

\begin{tabular}[b]{@{}p{34\cuadri}@{}p{0\cuadri}@{} x{17.5\cuadri}@{}}
& &\\[0.5\cuadri]
\begin{grafica-cajita}
\begin{center}
\ifthenelse{\isempty{#4}}{}{{\textbf{#4}}\\[-1pt]}%
\ifthenelse{\isempty{#5}}{$\ $\\[-0.1\cuadri]}{
	{\footnotesize\texttwelveudash$\,\,$#5$\,\,$\texttwelveudash}\\[0.6\cuadri]}
#6
\begin{flushleft}
$\ $\\[-2\cuadri]
\ \ \ \footnotesize Fuente: #7
\end{flushleft}
\end{center}
\end{grafica-cajita}
  & &
\begin{descripcion-cajita}
\parskip 6pt\parindent 1em%
#3
\end{descripcion-cajita}
\end{tabular}
}}
{
\cajitaalternante{
\addcontentsline{toc}{section}{\numberline{\thesection} #1}
\addtocontents{toc}{\protect\thispagestyle{empty}}
\titulizador{#2}

\begin{tabular}[b]{@{}p{34\cuadri}@{}p{0\cuadri}@{} x{17.5\cuadri}@{}}
& &\\[0.5\cuadri]
\begin{grafica-cajita}
\begin{center}
\ifthenelse{\isempty{#4}}{}{{\textbf{#4}}\\[-1pt]}%
\ifthenelse{\isempty{#5}}{$\ $\\[-0.1\cuadri]}{
	{\footnotesize\texttwelveudash$\,\,$#5$\,\,$\texttwelveudash}\\[0.6\cuadri]}
#6
\begin{flushleft}
$\ $\\[-2\cuadri]
\ \ \ \footnotesize Fuente: #7
\end{flushleft}
\end{center}
\end{grafica-cajita}
  & &
\begin{descripcion-cajita}
\parskip 6pt\parindent 1em%
#3
\end{descripcion-cajita}
\end{tabular}
}}
}


\newcommand{\cajita}[7][]{%
\ifthenelse{\equal{\theupdown}{0}}{
\cajitaizquierda[#1]{#2}{#3}{#4}{#5}{#6}{#7}


}{%
\cajitaderecha[#1]{#2}{#3}{#4}{#5}{#6}{#7}


}
}



%%%%%%%%%%%%%%%  Cajita manual



\newcommand{\cajitaderechamanual}[8][]{%
\ifthenelse{\isempty{#1}}{
\cajitaalternante{
\addcontentsline{toc}{section}{\numberline{\thesection} #2}
\addtocontents{toc}{\protect\thispagestyle{empty}}
\titulizadormanual{#2}{#8}

\begin{tabular}[b]{@{}p{17.5\cuadri}@{}p{1.5\cuadri}@{} x{34\cuadri}@{}}
& &\\[0.5\cuadri]
\begin{descripcion-cajita}
\parskip 6pt\parindent 1em%
#3
\end{descripcion-cajita}
  & &
\begin{grafica-cajita}
\begin{center}
\ifthenelse{\isempty{#4}}{}{{\textbf{#4}}\\[-1pt]}%
\ifthenelse{\isempty{#5}}{$\ $\\[-0.1\cuadri]}{
	{\footnotesize\texttwelveudash$\,\,$#5$\,\,$\texttwelveudash}\\[0.6\cuadri]}
#6
\begin{flushleft}
$\ $\\[-2\cuadri]
\ \ \ \footnotesize Fuente: #7
\end{flushleft}
\end{center}
\end{grafica-cajita}
\end{tabular}
}}
{
\cajitaalternante{
\addcontentsline{toc}{section}{\numberline{\thesection} #1}
\addtocontents{toc}{\protect\thispagestyle{empty}}
\titulizadormanual{#2}{#8}

\begin{tabular}[b]{@{}p{17.5\cuadri}@{}p{1.5\cuadri}@{} x{34\cuadri}@{}}
& &\\[0.5\cuadri]
\begin{descripcion-cajita}
\parskip 6pt\parindent 1em%
#3
\end{descripcion-cajita}
  & &
\begin{grafica-cajita}
\begin{center}
\ifthenelse{\isempty{#4}}{}{{\textbf{#4}}\\[-1pt]}%
\ifthenelse{\isempty{#5}}{$\ $\\[-0.1\cuadri]}{
	{\footnotesize\texttwelveudash$\,\,$#5$\,\,$\texttwelveudash}\\[0.6\cuadri]}
#6
\begin{flushleft}
$\ $\\[-2\cuadri]
\ \ \ \footnotesize Fuente: #7
\end{flushleft}
\end{center}
\end{grafica-cajita}
\end{tabular}
}}
}


\newcommand{\cajitaizquierdamanual}[8][]{%
\ifthenelse{\isempty{#1}}{
\cajitaalternante{
\addcontentsline{toc}{section}{\numberline{\thesection} #2}
\addtocontents{toc}{\protect\thispagestyle{empty}}
\titulizadormanual{#2}{#8}

\begin{tabular}[b]{@{}p{34\cuadri}@{}p{0\cuadri}@{} x{17.5\cuadri}@{}}
& &\\[0.5\cuadri]
\begin{grafica-cajita}
\begin{center}
\ifthenelse{\isempty{#4}}{}{{\textbf{#4}}\\[-1pt]}%
\ifthenelse{\isempty{#5}}{$\ $\\[-0.1\cuadri]}{
	{\footnotesize\texttwelveudash$\,\,$#5$\,\,$\texttwelveudash}\\[0.6\cuadri]}
#6
\begin{flushleft}
$\ $\\[-2\cuadri]
\ \ \ \footnotesize Fuente: #7
\end{flushleft}
\end{center}
\end{grafica-cajita}
  & &
\begin{descripcion-cajita}
\parskip 6pt\parindent 1em%
#3
\end{descripcion-cajita}
\end{tabular}
}}
{
\cajitaalternante{
\addcontentsline{toc}{section}{\numberline{\thesection} #1}
\addtocontents{toc}{\protect\thispagestyle{empty}}
\titulizadormanual{#2}{#8}

\begin{tabular}[b]{@{}p{34\cuadri}@{}p{0\cuadri}@{} x{17.5\cuadri}@{}}
& &\\[0.5\cuadri]
\begin{grafica-cajita}
\begin{center}
\ifthenelse{\isempty{#4}}{}{{\textbf{#4}}\\[-1pt]}%
\ifthenelse{\isempty{#5}}{$\ $\\[-0.1\cuadri]}{
	{\footnotesize\texttwelveudash$\,\,$#5$\,\,$\texttwelveudash}\\[0.6\cuadri]}
#6
\begin{flushleft}
$\ $\\[-2\cuadri]
\ \ \ \footnotesize Fuente: #7
\end{flushleft}
\end{center}
\end{grafica-cajita}
  & &
\begin{descripcion-cajita}
\parskip 6pt\parindent 1em%
#3
\end{descripcion-cajita}
\end{tabular}
}}
}


\newcommand{\cajitamanual}[8][]{%
\ifthenelse{\equal{\theupdown}{0}}{
\cajitaizquierdamanual[#1]{#2}{#3}{#4}{#5}{#6}{#7}{#8}


}{%
\cajitaderechamanual[#1]{#2}{#3}{#4}{#5}{#6}{#7}{#8}


}
}








%%%%%%%%%%%%%%%  Cajita de tabla

\newcommand{\cajitaizquierdatabla}[7][]{%
	\ifthenelse{\isempty{#1}}{
		\cajitaalternante{
			\addcontentsline{toc}{section}{\numberline{\thesection} #2}
			\addtocontents{toc}{\protect\thispagestyle{empty}}
			\titulizador{#2}
			
			\begin{tabular}[b]{@{}p{34\cuadri}@{}p{0\cuadri}@{} x{17.5\cuadri}@{}}
				& &\\[0.5\cuadri]
				\begin{grafica-cajita}
					\begin{center}
						\ifthenelse{\isempty{#4}}{}{{\textbf{#4}}\\[-1pt]}%
						\ifthenelse{\isempty{#5}}{$\ $\\[-0.1\cuadri]}{
							{\footnotesize\texttwelveudash$\,\,$#5$\,\,$\texttwelveudash}\\[0.6\cuadri]}
						\ifthenelse{\isempty{#7}}{\renewcommand{\lafuente}{Estadísticas INE}}{\renewcommand{\lafuente}{#7}}%
						\begin{tabular}{l}
							#6\\[4mm]
							$\ $\\[-3.5mm]
							{\footnotesize $\ \ \ $Fuente: \lafuente}
						\end{tabular}
					\end{center}
				\end{grafica-cajita}
				& &
				\begin{descripcion-cajita}
					\parskip 6pt\parindent 1em%
					#3
				\end{descripcion-cajita}
			\end{tabular}
		}}
		{
			\cajitaalternante{
				\addcontentsline{toc}{section}{\numberline{\thesection} #1}
				\addtocontents{toc}{\protect\thispagestyle{empty}}
				\titulizador{#2}
				
				\begin{tabular}[b]{@{}p{34\cuadri}@{}p{0\cuadri}@{} x{17.5\cuadri}@{}}
					& &\\[0.5\cuadri]
					\begin{grafica-cajita}
						\begin{center}
							\ifthenelse{\isempty{#4}}{}{{\textbf{#4}}\\[-1pt]}%
							\ifthenelse{\isempty{#5}}{$\ $\\[-0.1\cuadri]}{
								{\footnotesize\texttwelveudash$\,\,$#5$\,\,$\texttwelveudash}\\[0.6\cuadri]}
							\ifthenelse{\isempty{#7}}{\renewcommand{\lafuente}{Estadísticas INE}}{\renewcommand{\lafuente}{#7}}%
							\begin{tabular}{l}
								#6\\[4mm]
								$\ $\\[-3.5mm]
								{\footnotesize $\ \ \ $Fuente: \lafuente}
							\end{tabular}
						\end{center}
					\end{grafica-cajita}
					& &
					\begin{descripcion-cajita}
						\parskip 6pt\parindent 1em%
						#3
					\end{descripcion-cajita}
				\end{tabular}
			}}
		}


\newcommand{\cajitaderechatabla}[7][]{%
	\ifthenelse{\isempty{#1}}{
		\cajitaalternante{
			\addcontentsline{toc}{section}{\numberline{\thesection} #2}
			\addtocontents{toc}{\protect\thispagestyle{empty}}
			\titulizador{#2}
			
			\begin{tabular}[b]{@{}p{17.5\cuadri}@{}p{1.5\cuadri}@{} x{34\cuadri}@{}}
				& &\\[0.5\cuadri]
				\begin{descripcion-cajita}
					\parskip 6pt\parindent 1em%
					#3
				\end{descripcion-cajita}
				& &
				\begin{grafica-cajita}
					\begin{center}
						\ifthenelse{\isempty{#4}}{}{{\textbf{#4}}\\[-1pt]}%
						\ifthenelse{\isempty{#5}}{$\ $\\[-0.1\cuadri]}{
							{\footnotesize\texttwelveudash$\,\,$#5$\,\,$\texttwelveudash}\\[0.6\cuadri]}
						\ifthenelse{\isempty{#7}}{\renewcommand{\lafuente}{Estadísticas INE}}{\renewcommand{\lafuente}{#7}}%
						\begin{tabular}{l}
							#6\\[4mm]
							$\ $\\[-3.5mm]
							{\footnotesize $\ \ \ $Fuente: \lafuente}
						\end{tabular}
					\end{center}
				\end{grafica-cajita}
			\end{tabular}
		}}
		{
			\cajitaalternante{
				\addcontentsline{toc}{section}{\numberline{\thesection} #1}
				\addtocontents{toc}{\protect\thispagestyle{empty}}
				\titulizador{#2}
				
				\begin{tabular}[b]{@{}p{17.5\cuadri}@{}p{1.5\cuadri}@{} x{34\cuadri}@{}}
					& &\\[0.5\cuadri]
					\begin{descripcion-cajita}
						\parskip 6pt\parindent 1em%
						#3
					\end{descripcion-cajita}
					& &
					\begin{grafica-cajita}
						\begin{center}
							\ifthenelse{\isempty{#4}}{}{{\textbf{#4}}\\[-1pt]}%
							\ifthenelse{\isempty{#5}}{$\ $\\[-0.1\cuadri]}{
								{\footnotesize\texttwelveudash$\,\,$#5$\,\,$\texttwelveudash}\\[0.6\cuadri]}
							\ifthenelse{\isempty{#7}}{\renewcommand{\lafuente}{Estadísticas INE}}{\renewcommand{\lafuente}{#7}}%
							\begin{tabular}{l}
								#6\\[4mm]
								$\ $\\[-3.5mm]
								{\footnotesize $\ \ \ $Fuente: \lafuente}
							\end{tabular}
						\end{center}
					\end{grafica-cajita}
				\end{tabular}
			}}
		}





\newcommand{\cajitatabla}[7][]{%
	\ifthenelse{\equal{\theupdown}{0}}{
		\cajitaizquierdatabla[#1]{#2}{#3}{#4}{#5}{#6}{#7}
		
		
	}{%
	\cajitaderechatabla[#1]{#2}{#3}{#4}{#5}{#6}{#7}
	
	
}
}



%%%%%%%%%%%%%%%% Macro de cajota

\newcommand{\cajota}[7][]{%
	\setcounter{numllamada}{0}\setcounter{numtextollamada}{0}
\noindent\begin{cajota-unica}\phantomsection\stepcounter{section}
\ifthenelse{\equal{\theupdown}{0}}{}{\setcounter{updown}{0}}
\ifthenelse{\isempty{#1}}{%
\addcontentsline{toc}{section}{\numberline{\thesection} #2}
\addtocontents{toc}{\protect\thispagestyle{empty}}
\titulizador{#2}}{%
\addcontentsline{toc}{section}{\numberline{\thesection} #1}
\addtocontents{toc}{\protect\thispagestyle{empty}}
\titulizador{#2}
}
\parskip 6pt\parindent 2em%
$\ $\\

#3$\ $\\[-1\cuadri]

\begin{center}
\ifthenelse{\isempty{#4}}{}{{\textbf{#4}}\\[-1pt]}%
\ifthenelse{\isempty{#5}}{$\ $\\[0.2\cuadri]}{
	{\footnotesize\texttwelveudash$\,\,$#5$\,\,$\texttwelveudash}\\[2.0\cuadri]}
#6
\begin{flushright}
$\ $\\[-1.5\cuadri]
\footnotesize Fuente: #7 $\ \ \ $ 
\end{flushright}
\end{center}
$\ $\\[-3\cuadri]

\end{cajota-unica}


}





%%%%%%%%% Cajota de Tabla

\newcommand{\cajotatabla}[7][]{%
	\setcounter{numllamada}{0}\setcounter{numtextollamada}{0}
	\noindent\begin{cajota-unica}\phantomsection\stepcounter{section}
		\ifthenelse{\equal{\theupdown}{0}}{}{\setcounter{updown}{0}}
		\ifthenelse{\isempty{#1}}{%
			\addcontentsline{toc}{section}{\numberline{\thesection} #2}
			\addtocontents{toc}{\protect\thispagestyle{empty}}
			\titulizador{#2}}{%
			\addcontentsline{toc}{section}{\numberline{\thesection} #1}
			\addtocontents{toc}{\protect\thispagestyle{empty}}
			\titulizador{#2}
		}
		\parskip 6pt\parindent 2em%
		$\ $\\
		
		#3$\ $\\[-1\cuadri]
		
		\begin{center}
			\ifthenelse{\isempty{#4}}{}{{\textbf{#4}}\\[-1pt]}%
			\ifthenelse{\isempty{#5}}{$\ $\\[0.2\cuadri]}{
			{\footnotesize\texttwelveudash$\,\,$#5$\,\,$\texttwelveudash}\\[2.0\cuadri]}
		\ifthenelse{\isempty{#7}}{\renewcommand{\lafuente}{Estadísticas INE}}{\renewcommand{\lafuente}{#7}}%
			\begin{tabular}{l}
				#6\\[4mm]
				$\ $\\[-3.5mm]
				{\footnotesize $\ \ \ $Fuente: \lafuente}
			\end{tabular}
		\end{center}
		$\ $\\[-3\cuadri]
		
	\end{cajota-unica}
	
	
}




%%%%%%%%%% Macro de capítulo

%previos

\newcommand{\capituloencabezado}{}

\newcommand{\capitulocondescripcion}[2]{%
\begin{fondo-capitulo}
$\ $\\[7.00\cuadri]
\begin{tabular}{p{1.28in}p{1.2in}p{4.75in}}
 & \begin{numcapitulo}\fontsize{0.875in}{1em}\selectfont\color{white}\centering\textbf{\thechapter}\end{numcapitulo} & \fontsize{0.4in}{3em}\selectfont \Bold \begin{tabular}[t]{p{4.75in}} #1 
 \end{tabular} \\ 
\end{tabular}\\[0.75in]

\begin{tabular}{p{2in}p{5.4in}}
 &\parskip 2.5ex \parindent 2em \Large #2
 
\\ 
\end{tabular} 
\end{fondo-capitulo}
}

\newcommand{\capitulosindescripcion}[1]{%
\begin{fondo-capitulo-no-descripcion}
$\ $\\[22.65\cuadri]
\begin{tabular}{p{1.28in}p{1.2in}p{4.75in}}
 & \begin{numcapitulo}\fontsize{0.875in}{1em}\selectfont\color{white}\centering\textbf{\thechapter}\end{numcapitulo} & \fontsize{0.4in}{3em}\selectfont \Bold \begin{tabular}[t]{p{4.75in}} #1 
 \end{tabular} \\ 
\end{tabular}
\end{fondo-capitulo-no-descripcion}
}



\newcommand{\rayitatoc}{\addtocontents{toc}{\protect\addvspace{0.2\baselineskip}{\color{color2}\hrule height 0.9pt} \addvspace{0.6\baselineskip} \color{black}} }

% El macro definitivo

\newcommand{\partes}{}

\newcommand{\INEchaptercarta}[3][]{%
	\setcounter{numllamada}{0}\setcounter{numtextollamada}{0}\setcounter{footnote}{0}
	\cleardoublepage\stepcounter{chapter}\addtocontents{toc}{\protect\addvspace{0.6\baselineskip}\color{color1}}%
	\phantomsection
	\ifthenelse{\isempty{#1}}{%
		\addcontentsline{toc}{chapter}{\numberline{\thechapter}#2}%
		\renewcommand{\capituloencabezado}{#2}%
	}{
	\addcontentsline{toc}{chapter}{\numberline{\thechapter}#1}%
	\renewcommand{\capituloencabezado}{#1}%
}
\thispagestyle{empty}%

\ifthenelse{\equal{\partes}{}}{\rayitatoc}{} %
\addtocontents{toc}{\color{black}\addvspace{0.2\baselineskip}}
\ifthenelse{\equal{\unexpanded{#3}}{}}{%
	\capitulosindescripcion{#2}
}{%
\capitulocondescripcion{#2}{#3}
}

\cleardoublepage
\setcounter{section}{0}
\setcounter{updown}{0}
}





%%%%%%%%%%  PARTES



\newcommand{\partesindescripcion}[1]{%
\begin{fondo-parte}
$\ $\\[17.65\cuadri]
\begin{tabular}{p{1.4in}p{5.75in}}
 & \begin{tabular}[t]{x{5.75in}}
\Cinzel \fontsize{0.4in}{3.5em}\selectfont \textbf{PARTE \Roman{parte}}\\[0.4in] \hline
\\[-0.08in]
\Cinzel\fontsize{0.4in}{3.5em}\selectfont {\color{color1!70!black} #1}  \\[0.2in]\hline
 \end{tabular} \\ 
\end{tabular}
\end{fondo-parte}
}

\newcommand{\partecondescripcion}[2]{%
\begin{fondo-parte}
$\ $\\[7.00\cuadri]
\begin{tabular}{p{1.4in}p{5.75in}}
 &  \begin{tabular}[t]{x{5.75in}} 
\Cinzel\fontsize{0.4in}{3.5em}\selectfont \textbf{PARTE \Roman{parte}}\\[0.4in] \hline
\\[-0.08in]
\Cinzel\fontsize{0.4in}{3.5em}\selectfont {\color{color1!70!black} #1} 
  \\[0.2in]\hline
 \end{tabular} \\ 
\end{tabular}\\[0.85in]

\begin{tabular}{p{1.4in}p{5.75in}}
 &\parskip 2.5ex \parindent 2em \Logos\Large
 \begin{multicols}{2}
 #2
\end{multicols}
\\ 
\end{tabular} 
\end{fondo-parte}
}


% El macro de parte definitivo


\newcounter{parte}
\setcounter{parte}{0}

\newcommand{\INEpartecarta}[3][]{%
	\setcounter{numllamada}{0}\setcounter{numtextollamada}{0}\setcounter{footnote}{0}
	\cleardoublepage\stepcounter{parte}\addtocontents{toc}{\protect\addvspace{3.1\baselineskip}\color{color2!60!black}}%
	\phantomsection
	\ifthenelse{\isempty{#1}}{
	\addtocontents{toc}{\protect \textbf{\Cinzel\large PARTE {\Roman{parte}} }\\[1.0mm] {\Cinzel\large #2 }\\[-2.5mm]
			\protect \par} \rayitatoc	
		}{
\addtocontents{toc}{\protect \textbf{\Cinzel\large PARTE {\Roman{parte}} }\\[1.0mm] {\Cinzel\large #1 }\\[-2.5mm]
	\protect \par} \rayitatoc
}
\thispagestyle{empty}%
\addtocontents{toc}{\protect\addvspace{-0.1\baselineskip}\color{black}}%
\ifthenelse{\equal{\unexpanded{#3}}{}}{%
	\partesindescripcion{#2}
}{%
\partecondescripcion{#2}{#3}
}

\cleardoublepage
\setcounter{section}{0}
\setcounter{updown}{0}
}


%%%%%%%%



\let\oldappendix\appendix

\renewcommand{\appendix}{
\cleardoublepage
\oldappendix

$\ $
\vspace{6.6cm}

\thispagestyle{empty}
\begin{center}
	\fontsize{16mm}{1em}\selectfont\Bold \color{color2!80!black} APÉNDICES
\end{center}
\addtocontents{toc}{\protect\addvspace{0.6\baselineskip}}
\addcontentsline{toc}{chapter}{APÉNDICES}
\cleardoublepage
\renewcommand{\rayitatoc}{}
}







%Comandos agregados por Hugo
\newcommand*\up{\textcolor{green}{%
		\ensuremath{\blacktriangle}}}
\newcommand*\down{\textcolor{red}{%
		\ensuremath{\blacktriangledown}}}





% % % % % % % % % % % % % % % % % % % % % % % % % % % %
%  Parte en construcción

% De la ENEI vieja.
\newtcolorbox{fondo}{width=6.5in, height=9in, nobeforeafter, boxrule=0pt, colframe=white, left=-3pt, bottom=-1in, top=-13pt, right=0pt, arc=0pt, enlarge bottom by= -3in, colback= white}
\newcommand{\hoja}[1]{\noindent\begin{fondo} #1 \end{fondo}\clearpage}


\newcommand{\titulo}[1]{
$\ $\\[0.3in]
\noindent{\color{color1!80!black}\LARGE \textbf{#1}}\\[-0.1in]
{\color{color1}\hrule}
$\ $\\[-0.1in]
}




\newcommand{\findoc}{
	\clearpage
	
	\checkoddpage
	\ifthenelse{\boolean{oddpage}}{%
	}{%
	\thispagestyle{empty}
	$\ $
	\clearpage
}


\thispagestyle{empty}
$\ $\\[20.35cm]
\begin{center}
	\color{gray!90!black}\large Instituto Nacional de Estadística\\[0.5mm]
	\textbf{\titulodoc}
\end{center}
}



\usepackage{hyperref}
\usepackage{setspace}
\usepackage{lscape}
\usepackage{dcolumn}
\usepackage{rotating}
\renewcommand{\titulodoc}{Compendio SAN}



\begin{document}

\clearpage

$\ $
\vspace{14.5cm}

\noindent\begin{tabular}{p{0.9cm}p{6.8cm}}
& 2016.$\,$ Guatemala, Centro América \\
&\Bold Instituto Nacional de Estadística\\[-0.4cm]
&\color{blue!50!black}\url{www.ine.gob.gt}\\[0.9cm]
\end{tabular}\\
\noindent\begin{tabular}{p{0.9cm}p{6.8cm}}
& Está permitida la reproducción parcial o total de los contenidos de este documento con la mención de la fuente. \\[0.5cm]
 
& Este documento fue elaborado empleando  {\Sans R}, Inkscape y {\Logos \XeLaTeX}.\\
\end{tabular} 
\pagestyle{empty}

\clearpage


%	\includepdf{portadaazul.pdf}
	
	\clearpage
	\newpage $\ $
	\newpage $\ $
$\ $
\vspace{7cm}

\begin{center}
\Bold \LARGE COMPENDIO ESTADÍSTICO\\
SEGURIDAD ALIMENTARIA 2015
\end{center}
\cleardoublepage


$\ $
\vspace{0.0cm}

\begin{center}
{\Bold \LARGE AUTORIDADES}\\[1cm]


{\Bold \large \color{color1!89!black} JUNTA  DIRECTIVA} \\[0.4cm]


{\Bold Ministerio de Economía}\\
Titular: Rubén Estuardo Morales Monroy\\
Suplente: Abel Francisco Cruz Calderón\\[0.4cm]


{\Bold Ministerio de Finanzas}\\
Titular: Julio Héctor Estrada\\
Suplente: Victor Manuel Martínez Ruiz\\[0.4cm]


{\Bold Ministerio de Agricultura, Ganadería y Alimentación}\\
Titular: Mario Méndez Montenegro\\
Suplente:  Rosa Elvira Pacheco Mangandi\\[0.4cm]


{\Bold Ministerio de Energía y Minas}\\
Titular: Juan Pelayo Castañón\\
Suplente: César Roberto Velásquez Barrera\\[0.4cm]


{\Bold Secretaría de Planificación y Programación de la Presidencia}\\
Titular: Miguel Ángel Moir Mérida\\
Suplente: Edna Abigail Álvarez\\[0.4cm]


{\Bold Banco de Guatemala}\\
Titular: Julio Roberto Suárez Guerra\\
Suplente: Sergio Francisco Recinos Rivera\\[0.4cm]



{\Bold Universidad de San Carlos de Guatemala}\\
Titular: Murphy Olimpo Paiz\\
Suplente:  Óscar René Paniagua \\[0.4cm]


{\Bold Universidades Privadas}\\
Titular: Miguel Ángel Franco de León\\
Suplente: Ariel Rivera Irías\\[0.4cm]


{\Bold Comité Coordinador de Asociaciones\\ Agrícolas, Comerciales, Industriales y Financieras}\\
Titular: Juan Raúl Aguilar Kaehler\\
Suplente:  Oscar Augusto Sequeira García\\[0.8cm]


{\Bold \large \color{color1!89!black} GERENCIA}\\[0.2cm]
Gerente: Néstor Mauricio Guerra Morales.\\
Subgerente Técnico: Jaime Roberto Mejía Salguero\\
Subgerente Administrativo Financiero: Orlando Roberto Monzón Girón\\


\end{center}

\clearpage

$\ $
\vspace{1cm}

\begin{center}
{\Bold \LARGE AUTORIDADES INSTITUCIONALES}\\[2cm]

{\Bold \large \color{color1!89!black} Dirección de Censos y Encuestas}\\[0.2cm]
Carlos Enrique Mancía Chúa\\[0.8cm]


{\Bold \large \color{color1!89!black} Dirección de Índices y Estadísticas Continuas}\\[0.2cm]
Luis Eduardo Arroyo Gálvez\\[0.8cm]

{\Bold \large \color{color1!89!black} Dirección Administrativa}\\[0.2cm]
Edgar Rolando Elías Pichillá\\[0.8cm]

{\Bold \large \color{color1!89!black} Dirección Financiera}\\[0.2cm]
María Elena Galindo Rodríguez\\[0.8cm]


{\Bold \large \color{color1!89!black} Dirección de Informática}\\[0.2cm]
César Calderón Barillas\\[0.8cm]


\end{center}\setcounter{page}{0}\cleardoublepage

%%%%%%%%%%%%%%%%%%%%%%%%%%%%%%%%%%%
\clearpage

$\ $
\vspace{1cm}

\begin{center}
	{\Bold \LARGE EQUIPO RESPONSABLE}\\[2cm]
	
	{\Bold \large \color{color1!89!black} Coordinación general:}\\[0.2cm]

Héctor Tuy – Iarna-URL\\
Lorena Ninel Estrada – Iarna-URL\\
Haydeé Barrientos - INE\\
Juan Lee - INE\\
Luis Eduardo Arroyo Gálvez - INE\\
Gonzalo Hernández - Sesan\\
Nora Cano - Sesan\\[0.8cm]

	{\Bold \large \color{color1!89!black} Edición:}\\[0.2cm]
Lorena Ninel Estrada – Iarna-URL\\[0.8cm]


	{\Bold \large \color{color1!89!black} Formato:}\\[0.2cm]

Lorena Ninel Estrada – Iarna-URL\\
Alejandro Gándara – Iarna -URL\\[0.8cm]

	{\Bold \large \color{color1!89!black} Recopilación de datos:}\\[0.2cm]
Renato Vargas – consultor Iarna-URL\\
Vivian Guzmán – consultora Iarna-URL\\
Haydeé Barrientos - INE\\
Gonzalo Hernández – Sesan\\
Nora Cano - Sesan\\
Lorena Ninel Estrada – Iarna-URL\\
Hugo Allan García - consultor Iarna-URL\\
Fabiola Ramírez - consultora Iarna-URL\\
Orlando Roberto Monzón Girón - INE\\[0.8cm]

	{\Bold \large \color{color1!89!black} Apoyo a Revisión de Contenidos (Iarna-URL)}\\[0.2cm]

Héctor Tuy – Iarna-URL\\
Jaime Luis Carrera – Iarna-URL\\
Cecilia Cleaves – Iarna-URL\\[0.8cm]
	
	
\end{center}\setcounter{page}{0}\cleardoublepage
%\swapgeometry

%%%%%%%%%%%%%%%%%%%%%%%%%%%%%%%%%%%%%%%



\clearpage

$\ $
\vspace{1cm}

\begin{center}
	{\Bold \LARGE Oficina Coordinadora Sectorial de Estadísticas de Seguridad Alimentaria y Nutricional (OCSESAN)}\\[2cm]
	
\begin{itemize}
	\item Asociación de Investigación y Estudios Sociales (Asies)
	\item 	Asociación Nacional de Municipalidades (ANAM)
	\item 	Instituto de Investigación y Proyección sobre Ambiente Natural y Sociedad  (Iarna-URL)
	\item 	Instituto Guatemalteco de Seguridad Social (IGSS).
	\item 	Ministerio de Agricultura, Ganadería y Alimentación (MAGA)
	\item 	Ministerio de Ambiente y Recursos Naturales (MARN)
	\item 	Ministerio de Economía de Guatemala (Mineco)
	\item 	Ministerio de Educación (Mineduc)
	\item 	Ministerio de Finanzas Públicas (Minfin)
	\item 	Ministerio de Salud Pública y Asistencia Social (Mspas)
	\item 	Ministerio de Trabajo y Previsión Social (Mintrab)
	\item 	Procuraduría de los Derechos Humanos de Guatemala (PDH)
	\item 	Secretaría de Seguridad Alimentaria y Nutricional (Sesan)
	\item 	Secretaría General de Planificación y Programación de la Presidencia (Segeplan)
	\item 	Usaid/DevTech - Programa de Monitoreo y Evaluación
	\item 	Viceministerio de Seguridad Alimentaria y Nutricional (Visan)
	
\end{itemize}	

\end{center}\setcounter{page}{0}\cleardoublepage
%\swapgeometry


%%%%%%%%%%%%%%%%%%%%%%%%%%%


\clearpage

$\ $
\vspace{1cm}

\begin{center}
	{\Bold \LARGE Fuentes primarias de información}\\[2cm]
	
	\begin{itemize}
			\item Banco Mundial
			\item 	Banco de Guatemala (Banguat)
			\item 	Instituto Guatemalteco de Seguridad Social (IGSS)
			\item 	Instituto Nacional de Estadística (INE)
			\item 	Ministerio de Agricultura, Ganadería y Alimentación (MAGA) – Dirección de Planeamiento (Diplan)
			\item 	Ministerio de Finanzas (Minfin) - Sistema de Contabilidad Integrada (Sicoin)
			\item 	Ministerio de Salud Pública y Asistencia Social (Mspas) - Sistema de Información Gerencial de Salud (Sigsa)
			\item 	Ministerio de Trabajo y Previsión Social (Mintrab)
			
	\end{itemize}	
	
\end{center}\setcounter{page}{0}\cleardoublepage
%\swapgeometry


%%%%%%%%%%%%%%%%%%%%%%%%%%%

$\ $\\[1cm]

\tableofcontents

\cleardoublepage
	\pagestyle{estandar}
	\setcounter{page}{1}
	\setlength{\arrayrulewidth}{1.0pt}





\cleardoublepage


$\ $\\[2cm]
\thispagestyle{empty}
\indent\titulo{\Bold \huge Presentación}




$\ $\\[0.5cm]
\large
\indent El Instituto Nacional de Estadística 	(INE) se complace en presentar el Compendio Estadístico para Seguridad Alimentaria y Nutricional 2015, con la finalidad de contribuir a mejorar el nivel de comprensión sobre el estado de la Seguridad Alimentaria y Nutricional (SAN) en el país.

El INE como ente rector y coordinador del Sistema Estadístico Nacional (Decreto Ley 3-85 Ley Orgánica del Instituto Nacional de Estadística), impulsa diferentes iniciativas tendientes a mejorar la calidad de la estadística en SAN, así como los procesos de recopilación, procesamiento, análisis y difusión de la misma, siendo este compendio uno de estos procesos.

La información contenida en este documento se organizó en siete dimensiones con sus respectivos indicadores y sub-indicadores. La primera dimensión es una caracterización poblacional del país. Las siguientes seis dimensiones corresponden específicamente a seguridad alimentaria y nutricional, y son: a) Disponibilidad de alimentos, b) acceso a los alimentos, c) consumo de los alimentos, d) utilización biológica de los alimentos, e) situación y atención a la desnutrición/malnutrición, e f) inversión pública en seguridad alimentaria y nutricional. 

Las dimensiones, indicadores y sub-indicadores en que se basó el compendio fueron socializados durante las reuniones ordinarias y extraordinarias de la Ocsesan, por lo que los representantes de las instituciones miembros que asistieron tuvieron la oportunidad de brindar retroalimentación.

Asimismo, mediante un proceso de retroalimentación usuario-productor, el INE espera recibir comentarios acerca de este trabajo, con el fin de mejorar continuamente la información presentada, así como mejorar los procesos que engloban la generación, procesamiento y difusión del Compendio Estadístico SAN 2015.


$\ $

\cleardoublepage



\indent\titulo{Agradecimientos}
\\


Detrás de este esfuerzo de integración, procesamiento, análisis y difusión de la información, hay una cantidad de instituciones representadas por profesionales y técnicos, que participaron en la producción del Compendio Estadístico en Seguridad Alimentaria y Nutricional (SAN) 2015, en forma directa o indirecta.

El INE deja constancia de su agradecimiento a las siguientes entidades: La Secretaría de Seguridad Alimentaria y Nutricional (Sesan), el Sistema de Información Gerencial de Salud (Sigsa) del Ministerio de Salud Pública y Asistencia Social (Mspas), El Ministerio de Finanzas Públicas (Minfin), la Secretaría General de Planificación y Programación de la Presidencia (Segeplan), el Ministerio de Educación (Mineduc), el Viceministerio de Seguridad Alimentaria y Nutricional (Visan), y la Dirección de Planeamiento (Diplan) del Ministerio de Agricultura, Ganadería y Alimentación (MAGA), el Ministerio de Ambiente y Recursos Naturales (MARN), la Procuraduría de los Derechos Humanos de Guatemala (PDH), el Instituto Guatemalteco de Seguridad Social (IGSS), el Ministerio de Economía de Guatemala (Mineco), la Asociación de Investigación y Estudios Sociales (Asies), el Ministerio de Trabajo y Previsión Social (Mintrab), y la Asociación Nacional de Municipalidades (Anam).

Asimismo, se agradece el apoyo del Instituto de Investigación y Proyección sobre Ambiente Natural y Sociedad (Iarna) de la Universidad Rafael Landívar (URL) y a DevTech Systems por el apoyo técnico y financiero derivado del sub-contrato No. 1096-SC-13-001-00.\\[16mm]




\newpage
%	\tableofcontents
%		\pagestyle{estandar}
%		\setcounter{page}{0}
$\ $

\cleardoublepage



\indent\titulo{Introducción}
\\


El Compendio Estadístico de Seguridad Alimentaria y Nutricional (SAN) 2015 se integra bajo el enfoque de dimensiones/pilares de la seguridad alimentaria y nutricional, con la finalidad de brindar información de utilidad para evaluar el estado de SAN para Guatemala, y apoyar de esta forma las políticas públicas sobre el tema, así como fortalecer la toma de decisiones informada y la participación ciudadana. 

Es por ello que el Compendio se ha integrado en las dimensiones/pilares de población, disponibilidad de alimentos, acceso a los alimentos, consumo de los alimentos, utilización biológica de los alimentos,  situación y atención a la desnutrición/malnutrición e inversión pública en SAN.

Para la recopilación de la información se llevó a cabo revisión bibliográfica, consultas con expertos, y socialización de la información con los representantes de la Oficina Coordinadora Sectorial de Estadísticas de Seguridad Alimentaria y Nutricional (Ocsesan).

Como resultado, se presenta un Compendio Estadístico de SAN 2015 que reúne información sobre los indicadores y sub-indicadores identificados. No obstante, la información presentada en este Compendio dependió de la disponibilidad y calidad de los datos, por lo que no todos los indicadores se encuentran representados en este informe. El detalle de la información faltante se da al inicio de cada dimensión, y se espera que la misma pueda integrarse a futuras versiones del compendio.\\[16mm]


\newpage

\titulo{Capítulos del compendio: Indicadores y sub-indicadores}


%\noindent\textbf{Recopilación:}

El Compendio Estadístico 2015 está organizado en siete capítulos: a) Población, b) disponibilidad de alimentos, c) acceso a los alimentos, d) consumo de los alimentos, e) utilización biológica de los alimentos, f) situación y atención a la desnutrición/malnutrición, e g) inversión pública en SAN. El primero reúne información sobre la caracterización de la población objeto de estudio; los siguientes cinco  capítulos están basados en los pilares de la seguridad alimentaria y nutricional; el séptimo capítulo busca explicar la situación de SAN en Guatemala (Figura \ref{figura3}). 

Como su nombre lo indica, el primer capítulo del Compendio sobre \textbf{población}, reúne información sobre la caracterización poblacional de Guatemala, incluyendo densidad poblacional, vivienda, pobreza y trabajo. 

El segundo capítulo del Compendio corresponde a garantizar la disponibilidad de una alimentación adecuada para todos: \textbf{dimensión de disponibilidad de alimentos. }Según la FAO (2015), es importante abordar el tema de la disponibilidad de alimentos adecuados, especialmente para los estratos más pobres de la sociedad. Asimismo, la producción agrícola debe ser ambiental, económica y socialmente sostenible, y debe garantizar la disponibilidad de alimentos suficientes para los que sufren las consecuencias de la inseguridad alimentaria y nutricional. Además, es importante asegurar la resiliencia de las comunidades y de los recursos naturales, incluyendo el mantenimiento de la diversidad genética.

Por otro lado, y según Coneval (2010), la disponibilidad de alimentos es el resultado de la producción interna, tanto de productos primarios como industrializados, del nivel de las reservas, importaciones y exportaciones, ayuda alimentaria y capacidad de almacenamiento y movilización. La disponibilidad debe ser estable, de forma que existan alimentos suficientes durante todo el año. Asimismo, debe ser adecuada a las condiciones sociales y culturales, y contar con productos sin sustancias dañinas para la salud.

El tercer capítulo se refiere a garantizar el acceso a una alimentación adecuada: \textbf{dimensión de acceso a los alimentos.} Las políticas para alcanzar esta meta van desde asegurar la cobertura de transferencias sociales a largo plazo, con base en el desarrollo de cadenas de valor, hasta enfrentar los desequilibrios comerciales y la volatilidad de los precios de los alimentos en mercados internacionales; esto último es un factor clave que afecta el acceso a alimentos nutritivos. Asimismo, se hace necesario enfrentar la integración de la producción de alimentos y el sistema de distribución, lo cual aumenta la propagación de alimentos de mala calidad (FAO, 2015).

En este sentido, los alimentos deben estar disponibles física y económicamente a toda la población. El acceso de alimentos saludables, así como su precio, dependen de la oferta y la demanda. La conducta del consumidor y sus preferencias pueden explicar las diferencias en los tipos de alimentos y la estructura de la oferta, así como las variaciones entre regiones con respecto a la disponibilidad de alimentos y la clase de establecimientos que los ofrecen. Además, el acceso económico de los hogares depende del ingreso y precio de los alimentos (Coneval, 2010).

El cuarto capítulo sobre la \textbf{dimensión de consumo de los alimentos}, se refiere a lo que consumen los miembros de cada hogar, ya sea que provenga de la autoproducción o del intercambio, ayudas o adquisiciones en los mercados, así como preparación y distribución intrafamiliar. El consumo es el resultado del poder de compra de los hogares y de quién realiza las compras y prepara los alimentos; además de hábitos y cultura, los cuales se pueden ver influidos por la publicidad y los medios de comunicación (Coneval, 2010). 

Asimismo, las elecciones de consumo de tipo de alimentos y de su estado (alimentos frescos, congelados, enlatados y preparados), dependen del tiempo que se tenga para obtener los ingredientes y preparar los alimentos. La limitación de tiempo es mayor en hogares con niños y cuando las mujeres trabajan fuera de la casa. Además, los individuos necesitan información sobre la elección de comida que hacen (Coneval, 2010).

El quinto capítulo, \textbf{dimensión de utilización biológica de los alimentos}, se refiere a que la desnutrición tiende a estar más concentrada en las poblaciones más vulnerables. Las políticas y programas de nutrición deben ser lo suficientemente amplios para enfrentar el retraso en el crecimiento y la deficiencia en micronutrientes (FAO, 2015). Asimismo, Coneval (2010) indica que el aprovechamiento biológico de los alimentos depende de las condiciones de salud del individuo, en particular del predominio de enfermedades infecciosas, así como de aspectos de saneamiento del medio; por ejemplo, el acceso a agua potable. Otros factores a tomar en cuenta son las condiciones del lugar, la forma de preparar los alimentos, y el consumo y almacenaje de los mismos que pueden contribuir a su contaminación (Coneval, 2010). 

\begin{figure}
	\centering
	\includegraphics[width=0.7\textwidth]{Mapacap}
	\caption{Dimensiones/pilares que componen el Compendio Estadístico de Seguridad Alimentaria y Nutricional 2015. \textbf{Elaborado por:} Iarna-URL, 2015.}
	\label{figura3}
\end{figure}

El capítulo seis se refiere a la  \textbf{dimensión de la situación y atención a la desnutrición/malnutrición}. La desnutrición aguda, también conocida como emaciación o bajo peso para la talla, es causada por una ingesta insuficiente de alimentos y por la presencia de infecciones graves durante períodos prolongados. Por otro lado, la desnutrición crónica, también conocida como demedro o baja talla para la edad, es el retardo en el crecimiento lineal de los niños, como resultado de los efectos negativos acumulados de períodos de alimentación inadecuada, o por efectos fatales de infecciones agudas repetidas. Esta condición se presenta en los primeros tres años de vida, y sus consecuencias son irreversibles, ya que afectan negativamente el desarrollo motor, las funciones cognoscitivas y el desempeño escolar de los niños (Coneval, 2010). 

Por otro lado, también debe tomarse en cuenta la obesidad y el sobrepeso, los cuales son factores de riesgo para el desarrollo de enfermedades como diabetes, osteoartritis, enfermedades cardiovasculares, e incluso el desarrollo de ciertos tipos de cáncer, discapacidades o muerte prematura. De acuerdo con la Organización Mundial de la Salud (OMS), la obesidad y el sobrepeso se definen como la acumulación anormal o excesiva de grasa, la cual puede ser perjudicial para la salud, y que tiene como causa principal el desequilibrio entre la ingesta calórica y el consumo de energía (Coneval, 2010). Por último, se desarrolló una dimensión sobre la inversión pública en SAN, para describir los esfuerzos hechos en dicha materia mediante la implementación del Plan del Pacto Hambre Cero (PPH0).

El séptimo capítulo sobre la \textbf{inversión pública en SAN} en Guatemala, busca recopilar los logros en materia de SAN para el país en los últimos años.

Cada capítulo está dividido en una serie de indicadores y sub-indicadores elegidos para describir la situación de cada tema. Los indicadores fueron presentados en las reuniones de la Oficina Coordinadora Sectorial de Estadísticas de Seguridad Alimentaria y Nutricional (Ocsesan), y se le dio la oportunidad a los participantes de brindar su opinión sobre el listado de los mismos. Al final, no se tomaron en cuenta todos los indicadores disponibles, sino aquellos que se consideraron adecuados para que la información en el Compendio fuese útil y manejable.





\begin{landscape}
	
	\begin{figure}
		\centering
		\includegraphics[width=1.4\textwidth]{cuadro1}
%		\caption{Dimensiones/pilares que componen el Compendio Estadístico de Seguridad Alimentaria y Nutricional 2015. \textbf{Elaborado por:} Iarna-URL, 2015.}
		\label{cuadro1}
	\end{figure}
\end{landscape}

\begin{landscape}
	
	\begin{figure}
		\centering
		\includegraphics[width=1.4\textwidth]{cuadro2}
				\caption{Elaborado por: Iarna-URL, 2015.}
		\label{cuadro2}
	\end{figure}
\end{landscape}

	\INEchaptercarta{Población}{La dimensión de Población reúne información sobre la caracterización poblacional de Guatemala e incluye la densidad poblacional, vivienda, pobreza y trabajo. 
		
		Esta dimensión se compone de cuatro indicadores y 8 sub-indicadores.
		
		El indicador \textbf{1.1 Población}, cuenta con cinco sub-indicadores:%\\[-.3cm] 
		
		\begin{itemize}
			\item[a)] proyecciones al 2015, %\\[-.3cm]
			\item[b)] densidad, 
			\item[c)] grupos quinquenales de edad, 
			\item[d)] población y densidad a nivel departamental, 
			\item[e)] Índice de Desarrollo Humano (IDH).
		\end{itemize} 
		
		El indicador\textbf{ 1.2 Vivienda}, se compone del sub-indicador de necesidades básicas insatisfechas.
		
		El indicador \textbf{1.3 Pobreza}, se compone del sub-indicador de Mapa de Pobreza.
		
		Finalmente, el indicador \textbf{1.4 Trabajo}, se compone del sub-indicador de Población Económicamente Activa (PEA). 
	}
		%#########################1########################

 \cajita{%
Evolución de la población }%
{%
 }%
{%
 Número de habitantes} %
{%
 República de Guatemala, serie histórica, en miles de millones} %
{%
 \begin{tikzpicture}[x=1pt,y=1pt]  % Created by tikzDevice version 0.9 on 2016-03-03 01:12:06
% !TEX encoding = UTF-8 Unicode
\definecolor{fillColor}{RGB}{255,255,255}
\path[use as bounding box,fill=fillColor,fill opacity=0.00] (0,0) rectangle (289.08,198.74);
\begin{scope}
\path[clip] (  0.00,  0.00) rectangle (289.08,198.74);

\path[] (  0.00,  0.00) rectangle (289.08,198.74);
\end{scope}
\begin{scope}
\path[clip] (  0.00,  0.00) rectangle (289.08,198.74);

\path[] ( 12.62, 15.61) rectangle (280.54,191.48);

\path[] ( 12.62, 21.50) --
	(280.54, 21.50);

\path[] ( 12.62, 63.55) --
	(280.54, 63.55);

\path[] ( 12.62,105.60) --
	(280.54,105.60);

\path[] ( 12.62,147.65) --
	(280.54,147.65);

\path[] ( 12.62,189.70) --
	(280.54,189.70);

\path[] ( 12.62, 42.52) --
	(280.54, 42.52);

\path[] ( 12.62, 84.58) --
	(280.54, 84.58);

\path[] ( 12.62,126.63) --
	(280.54,126.63);

\path[] ( 12.62,168.68) --
	(280.54,168.68);

\path[] ( 43.53, 15.61) --
	( 43.53,191.48);

\path[] ( 95.06, 15.61) --
	( 95.06,191.48);

\path[] (146.58, 15.61) --
	(146.58,191.48);

\path[] (198.11, 15.61) --
	(198.11,191.48);

\path[] (249.63, 15.61) --
	(249.63,191.48);
\definecolor{drawColor}{RGB}{0,0,255}

\path[draw=drawColor,line width= 1.7pt,line join=round] ( 43.53, 60.50) --
	( 95.06, 90.75) --
	(146.58,121.44) --
	(198.11,152.42) --
	(249.63,183.49);
\definecolor{drawColor}{RGB}{0,0,0}

\node[text=drawColor,anchor=base,inner sep=0pt, outer sep=0pt, scale=  1.02] at ( 43.53, 48.59) {14,713.8};

\node[text=drawColor,anchor=base east,inner sep=0pt, outer sep=0pt, scale=  1.02] at ( 88.80, 90.75) {15,073.4};

\node[text=drawColor,anchor=base east,inner sep=0pt, outer sep=0pt, scale=  1.02] at (140.33,121.44) {15,438.4};

\node[text=drawColor,anchor=base east,inner sep=0pt, outer sep=0pt, scale=  1.02] at (191.85,152.42) {15,806.7};

\node[text=drawColor,anchor=base,inner sep=0pt, outer sep=0pt, scale=  1.02] at (249.63,187.46) {16,176.1};

\path[draw=drawColor,line width= 0.1pt,line join=round] ( 12.62, 23.61) -- (280.54, 23.61);

\path[] ( 12.62, 15.61) rectangle (280.54,191.48);
\end{scope}
\begin{scope}
\path[clip] (  0.00,  0.00) rectangle (289.08,198.74);

\path[] ( 12.62, 15.61) --
	( 12.62,191.48);
\end{scope}
\begin{scope}
\path[clip] (  0.00,  0.00) rectangle (289.08,198.74);
\definecolor{drawColor}{RGB}{255,255,255}

\node[text=drawColor,text opacity=0.00,anchor=base east,inner sep=0pt, outer sep=0pt, scale=  1.00] at (  7.67, 38.62) {14500};

\node[text=drawColor,text opacity=0.00,anchor=base east,inner sep=0pt, outer sep=0pt, scale=  1.00] at (  7.67, 80.67) {15000};

\node[text=drawColor,text opacity=0.00,anchor=base east,inner sep=0pt, outer sep=0pt, scale=  1.00] at (  7.67,122.72) {15500};

\node[text=drawColor,text opacity=0.00,anchor=base east,inner sep=0pt, outer sep=0pt, scale=  1.00] at (  7.67,164.77) {16000};
\end{scope}
\begin{scope}
\path[clip] (  0.00,  0.00) rectangle (289.08,198.74);

\path[] (  9.87, 42.52) --
	( 12.62, 42.52);

\path[] (  9.87, 84.58) --
	( 12.62, 84.58);

\path[] (  9.87,126.63) --
	( 12.62,126.63);

\path[] (  9.87,168.68) --
	( 12.62,168.68);
\end{scope}
\begin{scope}
\path[clip] (  0.00,  0.00) rectangle (289.08,198.74);

\path[] ( 12.62, 15.61) --
	(280.54, 15.61);
\end{scope}
\begin{scope}
\path[clip] (  0.00,  0.00) rectangle (289.08,198.74);

\path[] ( 43.53, 12.86) --
	( 43.53, 15.61);

\path[] ( 95.06, 12.86) --
	( 95.06, 15.61);

\path[] (146.58, 12.86) --
	(146.58, 15.61);

\path[] (198.11, 12.86) --
	(198.11, 15.61);

\path[] (249.63, 12.86) --
	(249.63, 15.61);
\end{scope}
\begin{scope}
\path[clip] (  0.00,  0.00) rectangle (289.08,198.74);
\definecolor{drawColor}{RGB}{0,0,0}

\node[text=drawColor,anchor=base,inner sep=0pt, outer sep=0pt, scale=  1.00] at ( 43.53,  2.85) {2011};

\node[text=drawColor,anchor=base,inner sep=0pt, outer sep=0pt, scale=  1.00] at ( 95.06,  2.85) {2012};

\node[text=drawColor,anchor=base,inner sep=0pt, outer sep=0pt, scale=  1.00] at (146.58,  2.85) {2013};

\node[text=drawColor,anchor=base,inner sep=0pt, outer sep=0pt, scale=  1.00] at (198.11,  2.85) {2014};

\node[text=drawColor,anchor=base,inner sep=0pt, outer sep=0pt, scale=  1.00] at (249.63,  2.85) {2015};
\end{scope}
  \end{tikzpicture}}%
{%
 Instituto Nacional de Estadística} %



%#################################2#######################


 \cajita{%
 	Densidad de población }%
 {%
 }%
 {%
 	Número de habitantes por kilómetro cuadrado} %
 {%
 	República de Guatemala, serie histórica, en habitantes por kilómetro cuadrado} %
 {%
 	\begin{tikzpicture}[x=1pt,y=1pt]  % Created by tikzDevice version 0.9 on 2016-03-03 01:12:07
% !TEX encoding = UTF-8 Unicode
\definecolor{fillColor}{RGB}{255,255,255}
\path[use as bounding box,fill=fillColor,fill opacity=0.00] (0,0) rectangle (289.08,198.74);
\begin{scope}
\path[clip] (  0.00,  0.00) rectangle (289.08,198.74);

\path[] (  0.00,  0.00) rectangle (289.08,198.74);
\end{scope}
\begin{scope}
\path[clip] (  0.00,  0.00) rectangle (289.08,198.74);

\path[] (  3.86, 15.61) rectangle (280.54,191.48);

\path[] (  3.86, 36.45) --
	(280.54, 36.45);

\path[] (  3.86, 82.24) --
	(280.54, 82.24);

\path[] (  3.86,128.03) --
	(280.54,128.03);

\path[] (  3.86,173.82) --
	(280.54,173.82);

\path[] (  3.86, 59.35) --
	(280.54, 59.35);

\path[] (  3.86,105.13) --
	(280.54,105.13);

\path[] (  3.86,150.92) --
	(280.54,150.92);

\path[] ( 35.79, 15.61) --
	( 35.79,191.48);

\path[] ( 88.99, 15.61) --
	( 88.99,191.48);

\path[] (142.20, 15.61) --
	(142.20,191.48);

\path[] (195.41, 15.61) --
	(195.41,191.48);

\path[] (248.62, 15.61) --
	(248.62,191.48);
\definecolor{drawColor}{RGB}{0,0,255}

\path[draw=drawColor,line width= 1.7pt,line join=round] ( 35.79, 60.50) --
	( 88.99, 90.75) --
	(142.20,121.44) --
	(195.41,152.42) --
	(248.62,183.49);
\definecolor{drawColor}{RGB}{0,0,0}

\node[text=drawColor,anchor=base,inner sep=0pt, outer sep=0pt, scale=  1.02] at ( 35.79, 48.59) {135.1};

\node[text=drawColor,anchor=base east,inner sep=0pt, outer sep=0pt, scale=  1.02] at ( 84.98, 90.75) {138.4};

\node[text=drawColor,anchor=base east,inner sep=0pt, outer sep=0pt, scale=  1.02] at (138.18,121.44) {141.8};

\node[text=drawColor,anchor=base east,inner sep=0pt, outer sep=0pt, scale=  1.02] at (191.39,152.42) {145.2};

\node[text=drawColor,anchor=base,inner sep=0pt, outer sep=0pt, scale=  1.02] at (248.62,187.46) {148.6};

\path[draw=drawColor,line width= 0.1pt,line join=round] (  3.86, 23.61) -- (280.54, 23.61);

\path[] (  3.86, 15.61) rectangle (280.54,191.48);
\end{scope}
\begin{scope}
\path[clip] (  0.00,  0.00) rectangle (289.08,198.74);

\path[] (  3.86, 15.61) --
	(  3.86,191.48);
\end{scope}
\begin{scope}
\path[clip] (  0.00,  0.00) rectangle (289.08,198.74);

\path[] (  1.11, 59.35) --
	(  3.86, 59.35);

\path[] (  1.11,105.13) --
	(  3.86,105.13);

\path[] (  1.11,150.92) --
	(  3.86,150.92);
\end{scope}
\begin{scope}
\path[clip] (  0.00,  0.00) rectangle (289.08,198.74);

\path[] (  3.86, 15.61) --
	(280.54, 15.61);
\end{scope}
\begin{scope}
\path[clip] (  0.00,  0.00) rectangle (289.08,198.74);

\path[] ( 35.79, 12.86) --
	( 35.79, 15.61);

\path[] ( 88.99, 12.86) --
	( 88.99, 15.61);

\path[] (142.20, 12.86) --
	(142.20, 15.61);

\path[] (195.41, 12.86) --
	(195.41, 15.61);

\path[] (248.62, 12.86) --
	(248.62, 15.61);
\end{scope}
\begin{scope}
\path[clip] (  0.00,  0.00) rectangle (289.08,198.74);
\definecolor{drawColor}{RGB}{0,0,0}

\node[text=drawColor,anchor=base,inner sep=0pt, outer sep=0pt, scale=  1.00] at ( 35.79,  2.85) {2011};

\node[text=drawColor,anchor=base,inner sep=0pt, outer sep=0pt, scale=  1.00] at ( 88.99,  2.85) {2012};

\node[text=drawColor,anchor=base,inner sep=0pt, outer sep=0pt, scale=  1.00] at (142.20,  2.85) {2013};

\node[text=drawColor,anchor=base,inner sep=0pt, outer sep=0pt, scale=  1.00] at (195.41,  2.85) {2014};

\node[text=drawColor,anchor=base,inner sep=0pt, outer sep=0pt, scale=  1.00] at (248.62,  2.85) {2015};
\end{scope}
  \end{tikzpicture}}%
 {%
 	Instituto Nacional de Estadística} %
 
 
 
 
 
%%%%%%%%%%%%%%%%%%%%%%%%%%%%%%%%%%3%%%%%%%%%%%%%%%%%%%%%%%%

 \cajita{%
 	Población por grupos de edad, año 2008 }%
 {%
 }%
 {%
 	Distribución de la población del 2008 por grupos de edades} %
 {%
 	República de Guatemala, 2008, en porcentaje} %
 {%
 	\begin{tikzpicture}[x=1pt,y=1pt]  % Created by tikzDevice version 0.9 on 2016-03-03 01:20:43
% !TEX encoding = UTF-8 Unicode
\definecolor{fillColor}{RGB}{255,255,255}
\path[use as bounding box,fill=fillColor,fill opacity=0.00] (0,0) rectangle (289.08,198.74);
\begin{scope}
\path[clip] (  0.00,  0.00) rectangle (289.08,198.74);

\path[] (  0.00,  0.00) rectangle (289.08,198.74);
\end{scope}
\begin{scope}
\path[clip] (  0.00,  0.00) rectangle (289.08,198.74);

\path[] ( 33.72,  0.00) rectangle (267.09,198.74);

\path[] ( 33.72,  8.40) --
	(267.09,  8.40);

\path[] ( 33.72, 22.39) --
	(267.09, 22.39);

\path[] ( 33.72, 36.39) --
	(267.09, 36.39);

\path[] ( 33.72, 50.39) --
	(267.09, 50.39);

\path[] ( 33.72, 64.38) --
	(267.09, 64.38);

\path[] ( 33.72, 78.38) --
	(267.09, 78.38);

\path[] ( 33.72, 92.37) --
	(267.09, 92.37);

\path[] ( 33.72,106.37) --
	(267.09,106.37);

\path[] ( 33.72,120.37) --
	(267.09,120.37);

\path[] ( 33.72,134.36) --
	(267.09,134.36);

\path[] ( 33.72,148.36) --
	(267.09,148.36);

\path[] ( 33.72,162.35) --
	(267.09,162.35);

\path[] ( 33.72,176.35) --
	(267.09,176.35);

\path[] ( 33.72,190.34) --
	(267.09,190.34);
\definecolor{drawColor}{RGB}{0,0,255}
\definecolor{fillColor}{RGB}{0,0,255}

\path[draw=drawColor,line width= 0.6pt,line join=round,fill=fillColor] ( 33.72,  4.20) rectangle (102.23, 12.60);

\path[draw=drawColor,line width= 0.6pt,line join=round,fill=fillColor] ( 33.72, 18.19) rectangle ( 65.88, 26.59);

\path[draw=drawColor,line width= 0.6pt,line join=round,fill=fillColor] ( 33.72, 32.19) rectangle ( 71.66, 40.59);

\path[draw=drawColor,line width= 0.6pt,line join=round,fill=fillColor] ( 33.72, 46.19) rectangle ( 76.82, 54.58);

\path[draw=drawColor,line width= 0.6pt,line join=round,fill=fillColor] ( 33.72, 60.18) rectangle ( 86.10, 68.58);

\path[draw=drawColor,line width= 0.6pt,line join=round,fill=fillColor] ( 33.72, 74.18) rectangle ( 98.00, 82.58);

\path[draw=drawColor,line width= 0.6pt,line join=round,fill=fillColor] ( 33.72, 88.17) rectangle (114.32, 96.57);

\path[draw=drawColor,line width= 0.6pt,line join=round,fill=fillColor] ( 33.72,102.17) rectangle (134.96,110.57);

\path[draw=drawColor,line width= 0.6pt,line join=round,fill=fillColor] ( 33.72,116.17) rectangle (157.49,124.56);

\path[draw=drawColor,line width= 0.6pt,line join=round,fill=fillColor] ( 33.72,130.16) rectangle (179.13,138.56);

\path[draw=drawColor,line width= 0.6pt,line join=round,fill=fillColor] ( 33.72,144.16) rectangle (207.45,152.56);

\path[draw=drawColor,line width= 0.6pt,line join=round,fill=fillColor] ( 33.72,158.15) rectangle (229.56,166.55);

\path[draw=drawColor,line width= 0.6pt,line join=round,fill=fillColor] ( 33.72,172.15) rectangle (250.84,180.55);

\path[draw=drawColor,line width= 0.6pt,line join=round,fill=fillColor] ( 33.72,186.15) rectangle (267.09,194.54);
\definecolor{drawColor}{RGB}{0,0,0}

\path[draw=drawColor,line width= 0.1pt,line join=round] ( 33.72,  0.00) -- ( 33.72,198.74);

\node[text=drawColor,anchor=base west,inner sep=0pt, outer sep=0pt, scale=  1.02] at (104.47,  4.43) {4.4};

\node[text=drawColor,anchor=base west,inner sep=0pt, outer sep=0pt, scale=  1.02] at ( 68.12, 18.42) {2.0};

\node[text=drawColor,anchor=base west,inner sep=0pt, outer sep=0pt, scale=  1.02] at ( 73.90, 32.42) {2.4};

\node[text=drawColor,anchor=base west,inner sep=0pt, outer sep=0pt, scale=  1.02] at ( 79.05, 46.41) {2.7};

\node[text=drawColor,anchor=base west,inner sep=0pt, outer sep=0pt, scale=  1.02] at ( 88.33, 60.41) {3.3};

\node[text=drawColor,anchor=base west,inner sep=0pt, outer sep=0pt, scale=  1.02] at (100.23, 74.41) {4.1};

\node[text=drawColor,anchor=base west,inner sep=0pt, outer sep=0pt, scale=  1.02] at (116.56, 88.40) {5.1};

\node[text=drawColor,anchor=base west,inner sep=0pt, outer sep=0pt, scale=  1.02] at (137.19,102.40) {6.5};

\node[text=drawColor,anchor=base west,inner sep=0pt, outer sep=0pt, scale=  1.02] at (159.73,116.39) {7.9};

\node[text=drawColor,anchor=base west,inner sep=0pt, outer sep=0pt, scale=  1.02] at (181.37,130.39) {9.3};

\node[text=drawColor,anchor=base west,inner sep=0pt, outer sep=0pt, scale=  1.02] at (210.58,144.39) {11.1};

\node[text=drawColor,anchor=base west,inner sep=0pt, outer sep=0pt, scale=  1.02] at (232.69,158.38) {12.5};

\node[text=drawColor,anchor=base west,inner sep=0pt, outer sep=0pt, scale=  1.02] at (253.97,172.38) {13.8};

\node[text=drawColor,anchor=base west,inner sep=0pt, outer sep=0pt, scale=  1.02] at (270.21,186.37) {14.9};

\path[] ( 33.72,  0.00) rectangle (267.09,198.74);
\end{scope}
\begin{scope}
\path[clip] (  0.00,  0.00) rectangle (289.08,198.74);

\path[] ( 33.72,  0.00) --
	( 33.72,198.74);
\end{scope}
\begin{scope}
\path[clip] (  0.00,  0.00) rectangle (289.08,198.74);
\definecolor{drawColor}{RGB}{0,0,0}

\node[text=drawColor,anchor=base east,inner sep=0pt, outer sep=0pt, scale=  1.00] at ( 30.97,  4.49) {65 y más};

\node[text=drawColor,anchor=base east,inner sep=0pt, outer sep=0pt, scale=  1.00] at ( 30.97, 18.49) {60-64};

\node[text=drawColor,anchor=base east,inner sep=0pt, outer sep=0pt, scale=  1.00] at ( 30.97, 32.48) {55-59};

\node[text=drawColor,anchor=base east,inner sep=0pt, outer sep=0pt, scale=  1.00] at ( 30.97, 46.48) {50-54};

\node[text=drawColor,anchor=base east,inner sep=0pt, outer sep=0pt, scale=  1.00] at ( 30.97, 60.47) {45-49};

\node[text=drawColor,anchor=base east,inner sep=0pt, outer sep=0pt, scale=  1.00] at ( 30.97, 74.47) {40-44};

\node[text=drawColor,anchor=base east,inner sep=0pt, outer sep=0pt, scale=  1.00] at ( 30.97, 88.46) {35-39};

\node[text=drawColor,anchor=base east,inner sep=0pt, outer sep=0pt, scale=  1.00] at ( 30.97,102.46) {30-34};

\node[text=drawColor,anchor=base east,inner sep=0pt, outer sep=0pt, scale=  1.00] at ( 30.97,116.46) {25-29};

\node[text=drawColor,anchor=base east,inner sep=0pt, outer sep=0pt, scale=  1.00] at ( 30.97,130.45) {20-24};

\node[text=drawColor,anchor=base east,inner sep=0pt, outer sep=0pt, scale=  1.00] at ( 30.97,144.45) {15-19};

\node[text=drawColor,anchor=base east,inner sep=0pt, outer sep=0pt, scale=  1.00] at ( 30.97,158.44) {10-14};

\node[text=drawColor,anchor=base east,inner sep=0pt, outer sep=0pt, scale=  1.00] at ( 30.97,172.44) {5- 9};

\node[text=drawColor,anchor=base east,inner sep=0pt, outer sep=0pt, scale=  1.00] at ( 30.97,186.44) {0- 4};
\end{scope}
\begin{scope}
\path[clip] (  0.00,  0.00) rectangle (289.08,198.74);

\path[] ( 30.97,  8.40) --
	( 33.72,  8.40);

\path[] ( 30.97, 22.39) --
	( 33.72, 22.39);

\path[] ( 30.97, 36.39) --
	( 33.72, 36.39);

\path[] ( 30.97, 50.39) --
	( 33.72, 50.39);

\path[] ( 30.97, 64.38) --
	( 33.72, 64.38);

\path[] ( 30.97, 78.38) --
	( 33.72, 78.38);

\path[] ( 30.97, 92.37) --
	( 33.72, 92.37);

\path[] ( 30.97,106.37) --
	( 33.72,106.37);

\path[] ( 30.97,120.37) --
	( 33.72,120.37);

\path[] ( 30.97,134.36) --
	( 33.72,134.36);

\path[] ( 30.97,148.36) --
	( 33.72,148.36);

\path[] ( 30.97,162.35) --
	( 33.72,162.35);

\path[] ( 30.97,176.35) --
	( 33.72,176.35);

\path[] ( 30.97,190.34) --
	( 33.72,190.34);
\end{scope}
  \end{tikzpicture}}%
 {%
 	Instituto Nacional de Estadística} %
 
 %%%%%%%%%%%%%%%%%%%%%%%%%%%%%%%%%%4%%%%%%%%%%%%%%%%%%%%%%%%
 
 \cajita{%
 	Población por grupos de edad, año 2015 }%
 {%
 }%
 {%
 	Distribución de la población del 2015 por grupos de edades} %
 {%
 	República de Guatemala, 2015, en porcentaje} %
 {%
 	\begin{tikzpicture}[x=1pt,y=1pt]  % Created by tikzDevice version 0.9 on 2016-03-03 01:22:03
% !TEX encoding = UTF-8 Unicode
\definecolor{fillColor}{RGB}{255,255,255}
\path[use as bounding box,fill=fillColor,fill opacity=0.00] (0,0) rectangle (289.08,198.74);
\begin{scope}
\path[clip] (  0.00,  0.00) rectangle (289.08,198.74);

\path[] (  0.00,  0.00) rectangle (289.08,198.74);
\end{scope}
\begin{scope}
\path[clip] (  0.00,  0.00) rectangle (289.08,198.74);

\path[] ( 33.72,  0.00) rectangle (267.09,198.74);

\path[] ( 33.72,  8.40) --
	(267.09,  8.40);

\path[] ( 33.72, 22.39) --
	(267.09, 22.39);

\path[] ( 33.72, 36.39) --
	(267.09, 36.39);

\path[] ( 33.72, 50.39) --
	(267.09, 50.39);

\path[] ( 33.72, 64.38) --
	(267.09, 64.38);

\path[] ( 33.72, 78.38) --
	(267.09, 78.38);

\path[] ( 33.72, 92.37) --
	(267.09, 92.37);

\path[] ( 33.72,106.37) --
	(267.09,106.37);

\path[] ( 33.72,120.37) --
	(267.09,120.37);

\path[] ( 33.72,134.36) --
	(267.09,134.36);

\path[] ( 33.72,148.36) --
	(267.09,148.36);

\path[] ( 33.72,162.35) --
	(267.09,162.35);

\path[] ( 33.72,176.35) --
	(267.09,176.35);

\path[] ( 33.72,190.34) --
	(267.09,190.34);
\definecolor{drawColor}{RGB}{0,0,255}
\definecolor{fillColor}{RGB}{0,0,255}

\path[draw=drawColor,line width= 0.6pt,line join=round,fill=fillColor] ( 33.72,  4.20) rectangle (110.00, 12.60);

\path[draw=drawColor,line width= 0.6pt,line join=round,fill=fillColor] ( 33.72, 18.19) rectangle ( 67.84, 26.59);

\path[draw=drawColor,line width= 0.6pt,line join=round,fill=fillColor] ( 33.72, 32.19) rectangle ( 72.63, 40.59);

\path[draw=drawColor,line width= 0.6pt,line join=round,fill=fillColor] ( 33.72, 46.19) rectangle ( 81.11, 54.58);

\path[draw=drawColor,line width= 0.6pt,line join=round,fill=fillColor] ( 33.72, 60.18) rectangle ( 91.84, 68.58);

\path[draw=drawColor,line width= 0.6pt,line join=round,fill=fillColor] ( 33.72, 74.18) rectangle (106.66, 82.58);

\path[draw=drawColor,line width= 0.6pt,line join=round,fill=fillColor] ( 33.72, 88.17) rectangle (125.49, 96.57);

\path[draw=drawColor,line width= 0.6pt,line join=round,fill=fillColor] ( 33.72,102.17) rectangle (147.08,110.57);

\path[draw=drawColor,line width= 0.6pt,line join=round,fill=fillColor] ( 33.72,116.17) rectangle (166.43,124.56);

\path[draw=drawColor,line width= 0.6pt,line join=round,fill=fillColor] ( 33.72,130.16) rectangle (193.95,138.56);

\path[draw=drawColor,line width= 0.6pt,line join=round,fill=fillColor] ( 33.72,144.16) rectangle (216.94,152.56);

\path[draw=drawColor,line width= 0.6pt,line join=round,fill=fillColor] ( 33.72,158.15) rectangle (238.83,166.55);

\path[draw=drawColor,line width= 0.6pt,line join=round,fill=fillColor] ( 33.72,172.15) rectangle (254.69,180.55);

\path[draw=drawColor,line width= 0.6pt,line join=round,fill=fillColor] ( 33.72,186.15) rectangle (267.09,194.54);
\definecolor{drawColor}{RGB}{0,0,0}

\path[draw=drawColor,line width= 0.1pt,line join=round] ( 33.72,  0.00) -- ( 33.72,198.74);

\node[text=drawColor,anchor=base west,inner sep=0pt, outer sep=0pt, scale=  1.02] at (112.24,  4.43) {4.6};

\node[text=drawColor,anchor=base west,inner sep=0pt, outer sep=0pt, scale=  1.02] at ( 70.08, 18.42) {2.0};

\node[text=drawColor,anchor=base west,inner sep=0pt, outer sep=0pt, scale=  1.02] at ( 74.87, 32.42) {2.3};

\node[text=drawColor,anchor=base west,inner sep=0pt, outer sep=0pt, scale=  1.02] at ( 83.35, 46.41) {2.8};

\node[text=drawColor,anchor=base west,inner sep=0pt, outer sep=0pt, scale=  1.02] at ( 94.08, 60.41) {3.5};

\node[text=drawColor,anchor=base west,inner sep=0pt, outer sep=0pt, scale=  1.02] at (108.90, 74.41) {4.4};

\node[text=drawColor,anchor=base west,inner sep=0pt, outer sep=0pt, scale=  1.02] at (127.72, 88.40) {5.5};

\node[text=drawColor,anchor=base west,inner sep=0pt, outer sep=0pt, scale=  1.02] at (149.32,102.40) {6.8};

\node[text=drawColor,anchor=base west,inner sep=0pt, outer sep=0pt, scale=  1.02] at (168.67,116.39) {8.0};

\node[text=drawColor,anchor=base west,inner sep=0pt, outer sep=0pt, scale=  1.02] at (196.19,130.39) {9.6};

\node[text=drawColor,anchor=base west,inner sep=0pt, outer sep=0pt, scale=  1.02] at (220.07,144.39) {11.0};

\node[text=drawColor,anchor=base west,inner sep=0pt, outer sep=0pt, scale=  1.02] at (241.95,158.38) {12.3};

\node[text=drawColor,anchor=base west,inner sep=0pt, outer sep=0pt, scale=  1.02] at (257.81,172.38) {13.2};

\node[text=drawColor,anchor=base west,inner sep=0pt, outer sep=0pt, scale=  1.02] at (270.21,186.37) {14.0};

\path[] ( 33.72,  0.00) rectangle (267.09,198.74);
\end{scope}
\begin{scope}
\path[clip] (  0.00,  0.00) rectangle (289.08,198.74);

\path[] ( 33.72,  0.00) --
	( 33.72,198.74);
\end{scope}
\begin{scope}
\path[clip] (  0.00,  0.00) rectangle (289.08,198.74);
\definecolor{drawColor}{RGB}{0,0,0}

\node[text=drawColor,anchor=base east,inner sep=0pt, outer sep=0pt, scale=  1.00] at ( 30.97,  4.49) {65 y más};

\node[text=drawColor,anchor=base east,inner sep=0pt, outer sep=0pt, scale=  1.00] at ( 30.97, 18.49) {60-64};

\node[text=drawColor,anchor=base east,inner sep=0pt, outer sep=0pt, scale=  1.00] at ( 30.97, 32.48) {55-59};

\node[text=drawColor,anchor=base east,inner sep=0pt, outer sep=0pt, scale=  1.00] at ( 30.97, 46.48) {50-54};

\node[text=drawColor,anchor=base east,inner sep=0pt, outer sep=0pt, scale=  1.00] at ( 30.97, 60.47) {45-49};

\node[text=drawColor,anchor=base east,inner sep=0pt, outer sep=0pt, scale=  1.00] at ( 30.97, 74.47) {40-44};

\node[text=drawColor,anchor=base east,inner sep=0pt, outer sep=0pt, scale=  1.00] at ( 30.97, 88.46) {35-39};

\node[text=drawColor,anchor=base east,inner sep=0pt, outer sep=0pt, scale=  1.00] at ( 30.97,102.46) {30-34};

\node[text=drawColor,anchor=base east,inner sep=0pt, outer sep=0pt, scale=  1.00] at ( 30.97,116.46) {25-29};

\node[text=drawColor,anchor=base east,inner sep=0pt, outer sep=0pt, scale=  1.00] at ( 30.97,130.45) {20-24};

\node[text=drawColor,anchor=base east,inner sep=0pt, outer sep=0pt, scale=  1.00] at ( 30.97,144.45) {15-19};

\node[text=drawColor,anchor=base east,inner sep=0pt, outer sep=0pt, scale=  1.00] at ( 30.97,158.44) {10-14};

\node[text=drawColor,anchor=base east,inner sep=0pt, outer sep=0pt, scale=  1.00] at ( 30.97,172.44) {5- 9};

\node[text=drawColor,anchor=base east,inner sep=0pt, outer sep=0pt, scale=  1.00] at ( 30.97,186.44) {0- 4};
\end{scope}
\begin{scope}
\path[clip] (  0.00,  0.00) rectangle (289.08,198.74);

\path[] ( 30.97,  8.40) --
	( 33.72,  8.40);

\path[] ( 30.97, 22.39) --
	( 33.72, 22.39);

\path[] ( 30.97, 36.39) --
	( 33.72, 36.39);

\path[] ( 30.97, 50.39) --
	( 33.72, 50.39);

\path[] ( 30.97, 64.38) --
	( 33.72, 64.38);

\path[] ( 30.97, 78.38) --
	( 33.72, 78.38);

\path[] ( 30.97, 92.37) --
	( 33.72, 92.37);

\path[] ( 30.97,106.37) --
	( 33.72,106.37);

\path[] ( 30.97,120.37) --
	( 33.72,120.37);

\path[] ( 30.97,134.36) --
	( 33.72,134.36);

\path[] ( 30.97,148.36) --
	( 33.72,148.36);

\path[] ( 30.97,162.35) --
	( 33.72,162.35);

\path[] ( 30.97,176.35) --
	( 33.72,176.35);

\path[] ( 30.97,190.34) --
	( 33.72,190.34);
\end{scope}
  \end{tikzpicture}}%
 {%
 	Instituto Nacional de Estadística} %
 
 
 
  %%%%%%%%%%%%%%%%%%%%%%%%%%%%%%%%%%5%%%%%%%%%%%%%%%%%%%%%%%%
  
  \cajota{%
  	Población departamental del 2008 }%
  {%
  }%
  {%
  	Número de habitantes en el 2008 por departamento
  	} %
  {%
  	República de Guatemala, por departamentos 2008, en número de habitantes} %
  {%
  	 \includegraphics[width=52\cuadri]{graficas/1_02.pdf}}%
  {%
  	Instituto Nacional de Estadística} %
  
  
   %%%%%%%%%%%%%%%%%%%%%%%%%%%%%%%%%%6%%%%%%%%%%%%%%%%%%%%%%%%
   
   \cajota{%
   	Densidad poblacional departamental del 2008 }%
   {%
   }%
   {%
   	Densidad de población en el 2008 por departamento
   } %
   {%
   		República de Guatemala, por departamentos 2008, en número de habitantes por kilómetro cuadrado} %
   {%
   	\includegraphics[width=52\cuadri]{graficas/1_05.pdf}}%
   {%
   	Instituto Nacional de Estadística} %
   
   
     %%%%%%%%%%%%%%%%%%%%%%%%%%%%%%%%%%7%%%%%%%%%%%%%%%%%%%%%%%%
     
     \cajota{%
     	Población departamental del 2015 }%
     {%
     }%
     {%
     	Número de habitantes en el 2015 por departamento
     } %
     {%
     	República de Guatemala, por departamentos 2015, en número de habitantes} %
     {%
     	\includegraphics[width=52\cuadri]{graficas/1_02.pdf}}%
     {%
     	Instituto Nacional de Estadística} %
     
     
     %%%%%%%%%%%%%%%%%%%%%%%%%%%%%%%%%%8%%%%%%%%%%%%%%%%%%%%%%%%
     
     \cajota{%
     	Densidad poblacional departamental del 2015 }%
     {%
     }%
     {%
     	Densidad de población en el 2015 por departamento
     } %
     {%
     	República de Guatemala, por departamentos 2015, en número de habitantes por kilómetro cuadrado} %
     {%
     	\includegraphics[width=52\cuadri]{graficas/1_05.pdf}}%
     {%
     	Instituto Nacional de Estadística} %

 %%%%%%%%%%%%%%%%%%%%%%%%%%%%%%%%%%9%%%%%%%%%%%%%%%%%%%%%%%%
 
 \cajita{%
 	Índice de desarrollo humano }%
 {%
 }%
 {%
 	Índice de desarrollo humano y sus componentes} %
 {%
 	República de Guatemala, años 2006 y 2015, adimensional} %
 {%
 	\begin{tikzpicture}[x=1pt,y=1pt]  % Created by tikzDevice version 0.9 on 2016-03-03 01:36:33
% !TEX encoding = UTF-8 Unicode
\definecolor{fillColor}{RGB}{255,255,255}
\path[use as bounding box,fill=fillColor,fill opacity=0.00] (0,0) rectangle (289.08,198.74);
\begin{scope}
\path[clip] (  0.00,  0.00) rectangle (289.08,198.74);

\path[] (  0.00,  0.00) rectangle (289.08,198.74);
\end{scope}
\begin{scope}
\path[clip] (  0.00,  0.00) rectangle (289.08,198.74);

\path[] (  0.00, 18.46) rectangle (289.08,171.32);

\path[] ( 41.30, 18.46) --
	( 41.30,171.32);

\path[] (110.13, 18.46) --
	(110.13,171.32);

\path[] (178.95, 18.46) --
	(178.95,171.32);

\path[] (247.78, 18.46) --
	(247.78,171.32);
\definecolor{drawColor}{RGB}{0,0,255}
\definecolor{fillColor}{RGB}{0,0,255}

\path[draw=drawColor,line width= 0.6pt,line join=round,fill=fillColor] ( 12.05, 18.46) rectangle ( 39.58,124.11);
\definecolor{drawColor}{RGB}{157,187,255}
\definecolor{fillColor}{RGB}{157,187,255}

\path[draw=drawColor,line width= 0.6pt,line join=round,fill=fillColor] ( 43.02, 18.46) rectangle ( 70.55,126.18);
\definecolor{drawColor}{RGB}{0,0,255}
\definecolor{fillColor}{RGB}{0,0,255}

\path[draw=drawColor,line width= 0.6pt,line join=round,fill=fillColor] ( 80.87, 18.46) rectangle (108.40,171.32);
\definecolor{drawColor}{RGB}{157,187,255}
\definecolor{fillColor}{RGB}{157,187,255}

\path[draw=drawColor,line width= 0.6pt,line join=round,fill=fillColor] (111.85, 18.46) rectangle (139.38,168.38);
\definecolor{drawColor}{RGB}{0,0,255}
\definecolor{fillColor}{RGB}{0,0,255}

\path[draw=drawColor,line width= 0.6pt,line join=round,fill=fillColor] (149.70, 18.46) rectangle (177.23, 96.35);
\definecolor{drawColor}{RGB}{157,187,255}
\definecolor{fillColor}{RGB}{157,187,255}

\path[draw=drawColor,line width= 0.6pt,line join=round,fill=fillColor] (180.68, 18.46) rectangle (208.21,102.44);
\definecolor{drawColor}{RGB}{0,0,255}
\definecolor{fillColor}{RGB}{0,0,255}

\path[draw=drawColor,line width= 0.6pt,line join=round,fill=fillColor] (218.53, 18.46) rectangle (246.06,117.51);
\definecolor{drawColor}{RGB}{157,187,255}
\definecolor{fillColor}{RGB}{157,187,255}

\path[draw=drawColor,line width= 0.6pt,line join=round,fill=fillColor] (249.50, 18.46) rectangle (277.03,117.72);
\definecolor{drawColor}{RGB}{0,0,0}

\path[draw=drawColor,line width= 0.6pt,line join=round] (  0.00, 18.46) -- (289.08, 18.46);

\node[text=drawColor,rotate= 90.00,anchor=base west,inner sep=0pt, outer sep=0pt, scale=  0.83] at ( 29.05,125.94) {0.6};

\node[text=drawColor,rotate= 90.00,anchor=base west,inner sep=0pt, outer sep=0pt, scale=  0.83] at ( 60.02,128.00) {0.6};

\node[text=drawColor,rotate= 90.00,anchor=base west,inner sep=0pt, outer sep=0pt, scale=  0.83] at ( 97.87,173.15) {0.8};

\node[text=drawColor,rotate= 90.00,anchor=base west,inner sep=0pt, outer sep=0pt, scale=  0.83] at (128.85,170.21) {0.8};

\node[text=drawColor,rotate= 90.00,anchor=base west,inner sep=0pt, outer sep=0pt, scale=  0.83] at (166.70, 98.18) {0.4};

\node[text=drawColor,rotate= 90.00,anchor=base west,inner sep=0pt, outer sep=0pt, scale=  0.83] at (197.68,104.27) {0.5};

\node[text=drawColor,rotate= 90.00,anchor=base west,inner sep=0pt, outer sep=0pt, scale=  0.83] at (235.53,119.33) {0.5};

\node[text=drawColor,rotate= 90.00,anchor=base west,inner sep=0pt, outer sep=0pt, scale=  0.83] at (266.50,119.54) {0.5};

\path[] (  0.00, 18.46) rectangle (289.08,171.32);
\end{scope}
\begin{scope}
\path[clip] (  0.00,  0.00) rectangle (289.08,198.74);

\path[] (  0.00, 18.46) --
	(289.08, 18.46);
\end{scope}
\begin{scope}
\path[clip] (  0.00,  0.00) rectangle (289.08,198.74);

\path[] ( 41.30, 15.71) --
	( 41.30, 18.46);

\path[] (110.13, 15.71) --
	(110.13, 18.46);

\path[] (178.95, 15.71) --
	(178.95, 18.46);

\path[] (247.78, 15.71) --
	(247.78, 18.46);
\end{scope}
\begin{scope}
\path[clip] (  0.00,  0.00) rectangle (289.08,198.74);
\definecolor{drawColor}{RGB}{0,0,0}

\node[text=drawColor,anchor=base,inner sep=0pt, outer sep=0pt, scale=  1.00] at ( 41.30,  5.69) {IDH};

\node[text=drawColor,anchor=base,inner sep=0pt, outer sep=0pt, scale=  1.00] at (110.13,  5.69) {IDH Salud};

\node[text=drawColor,anchor=base,inner sep=0pt, outer sep=0pt, scale=  1.00] at (178.95,  5.69) {IDH Educación};

\node[text=drawColor,anchor=base,inner sep=0pt, outer sep=0pt, scale=  1.00] at (247.78,  5.69) {IDH Ingresos};
\end{scope}
\begin{scope}
\path[clip] (  0.00,  0.00) rectangle (289.08,198.74);
\coordinate (apoyo) at (61.19,191.48);
\coordinate (longitudFicticia) at (7.11,7.26);
\coordinate (longitud) at (7.11,7.11);
\coordinate (desX) at (138.78,0);
\coordinate (desY) at (0,0.07);
\definecolor[named]{ct1}{HTML}{
0000FF
}
\definecolor[named]{ct2}{HTML}{
9DBBFF
}
\definecolor[named]{ctb1}{HTML}{
0000FF
}
\definecolor[named]{ctb2}{HTML}{
9DBBFF
}
\path [fill=none] (apoyo) rectangle ($(apoyo)+(longitudFicticia)$)
node [xshift=0.3cm,inner sep=0pt, outer sep=0pt,midway,right,scale = 0.9]{2006};
\draw [color = ctb1,fill=ct1] ( $(apoyo)  + (desY) $) rectangle ($(apoyo)+ (desY) +(longitud)$);
\path [fill=none] ($(apoyo)+(desX)$) rectangle ($(apoyo)+(desX)+(longitudFicticia)$)
node [xshift=0.3cm,inner sep=0pt, outer sep=0pt,midway,right,scale = 0.9]{2011};
\draw [color = ctb2 ,fill=ct2] ( $(apoyo)  + (desY) + (desX) $) rectangle ($(apoyo)+ (desY)+ (desX) +(longitud)$);
\end{scope}
  \end{tikzpicture}}%
 {%
 	Informe Nacional de Desarrollo Humano (PNUD), con base en las Encuestas Nacionales de Condiciones de Vida (Encovi).} % 
   
   
   
   %%%%%%%%%%%%%%%%%%%%%%%%%%%%%%%%%%10%%%%%%%%%%%%%%%%%%%%%%%%
   
   \cajota{%
   	Índice de desarrollo humano para los departamentos}%
   	{%
   	}%
   	{%
   		Índice de desarrollo humano por departamento
   	}%
   	{%
   		República de Guatemala, por departamentos, adimensional} %
   	{%
   		\includegraphics[width=52\cuadri]{graficas/1_11.pdf}}%
   	{%
   		Instituto Nacional de Estadística} %
 
 
 
 
      %%%%%%%%%%%%%%%%%%%%%%%%%%%%%%%%%%11%%%%%%%%%%%%%%%%%%%%%%%%
      
      \cajota{%
      	Hogares que no poseen vivienda}%
      {%
      }%
      {%
      	Hogares que no satisfacen la necesidad de vivienda por departamento
      } %
      {%
      	República de Guatemala, por departamentos, en porcentaje} %
      {%
      	\includegraphics[width=52\cuadri]{graficas/1_12.pdf}}%
      {%
      	Instituto Nacional de Estadística} %
      
      
        %%%%%%%%%%%%%%%%%%%%%%%%%%%%%%%%%%12%%%%%%%%%%%%%%%%%%%%%%%%
        
        \cajota{%
        	Hogares que viven en hacinamiento}%
        {%
        }%
        {%
        	Hogares en hacinamiento por departamento
        } %
        {%
        	República de Guatemala, por departamentos, en porcentaje} %
        {%
        	\includegraphics[width=52\cuadri]{graficas/1_13.pdf}}%
        {%
        	Instituto Nacional de Estadística} %
        
        
          
          %%%%%%%%%%%%%%%%%%%%%%%%%%%%%%%%%%13%%%%%%%%%%%%%%%%%%%%%%%%
          
          \cajota{%
          	Hogares sin acceso a agua}%
          {%
          }%
          {%
          	Hogares que no poseen acceso a agua entubada
          } %
          {%
          	República de Guatemala, por departamentos, en porcentaje} %
          {%
          	\includegraphics[width=52\cuadri]{graficas/1_14.pdf}}%
          {%
          	Instituto Nacional de Estadística} %
          
          
                    %%%%%%%%%%%%%%%%%%%%%%%%%%%%%%%%%%14%%%%%%%%%%%%%%%%%%%%%%%%
                    
                    \cajota{%
                    	Hogares sin acceso a servicios de saneamiento}%
                    {%
                    }%
                    {%
                    	Hogares sin acceso a servicios de saneamiento por departamentos
                    } %
                    {%
                    	República de Guatemala, por departamentos, en porcentaje} %
                    {%
                    	\includegraphics[width=52\cuadri]{graficas/1_15.pdf}}%
                    {%
                    	Instituto Nacional de Estadística} %

                    %%%%%%%%%%%%%%%%%%%%%%%%%%%%%%%%%%15%%%%%%%%%%%%%%%%%%%%%%%%
                    
                    \cajota{%
                    	Pobreza extrema rural}%
                    {%
                    }%
                    {%
                    	Pobreza extrema rural por departamentos
                    } %
                    {%
                    	República de Guatemala, por departamentos año 2014, en porcentaje} %
                    {%
                    	\includegraphics[width=52\cuadri]{graficas/1_16.pdf}}%
                    {%
                    	ENCOVI 2014} %
                    
                    %%%%%%%%%%%%%%%%%%%%%%%%%%%%%%%%%%16%%%%%%%%%%%%%%%%%%%%%%%%
                    
                    \cajota{%
                    	Pobreza total rural}%
                    {%
                    }%
                    {%
                    	Pobreza total rural por departamentos
                    } %
                    {%
                    	República de Guatemala, por departamentos año 2014, en porcentaje} %
                    {%
                    	\includegraphics[width=52\cuadri]{graficas/1_17.pdf}}%
                    {%
                    	ENCOVI 2014} %
                    %%%%%%%%%%%%%%%%%%%%%%%%%%%%%%%%%%17%%%%%%%%%%%%%%%%%%%%%%%%
                    
                    \cajota{%
                    	Pobreza extrema }%
                    {%
                    }%
                    {%
                    	Pobreza extrema por departamentos
                    } %
                    {%
                    	República de Guatemala, por departamentos año 2014, en porcentaje} %
                    {%
                    	\includegraphics[width=52\cuadri]{graficas/1_18.pdf}}%
                    {%
                    	ENCOVI 2014} %
                    
                    %%%%%%%%%%%%%%%%%%%%%%%%%%%%%%%%%%18%%%%%%%%%%%%%%%%%%%%%%%%
                    
                    \cajota{%
                    	Pobreza total }%
                    {%
                    }%
                    {%
                    	Pobreza total  por departamentos
                    } %
                    {%
                    	República de Guatemala, por departamentos año 2014, en porcentaje} %
                    {%
                    	\includegraphics[width=52\cuadri]{graficas/1_19.pdf}}%
                    {%
                    	ENCOVI 2014} %
         
 %%%%%%%%%%%%%%%%%%%%%%%%%%%%%%%%%%19%%%%%%%%%%%%%%%%%%%%%%%%
 
 \cajita{%
 	PET }%
 {%
 }%
 {%
 	Población en edad de trabajar} %
 {%
 	República de Guatemala, serie histórica por ENEI, en número de personas} %
 {%
 	\begin{tikzpicture}[x=1pt,y=1pt]  % Created by tikzDevice version 0.9 on 2016-03-03 00:54:40
% !TEX encoding = UTF-8 Unicode
\definecolor{fillColor}{RGB}{255,255,255}
\path[use as bounding box,fill=fillColor,fill opacity=0.00] (0,0) rectangle (289.08,198.74);
\begin{scope}
\path[clip] (  0.00,  0.00) rectangle (289.08,198.74);

\path[] (  0.00,  0.00) rectangle (289.08,198.74);
\end{scope}
\begin{scope}
\path[clip] (  0.00,  0.00) rectangle (289.08,198.74);

\path[] ( 25.76, 15.61) rectangle (280.54,191.48);

\path[] ( 25.76, 31.48) --
	(280.54, 31.48);

\path[] ( 25.76, 74.96) --
	(280.54, 74.96);

\path[] ( 25.76,118.43) --
	(280.54,118.43);

\path[] ( 25.76,161.90) --
	(280.54,161.90);

\path[] ( 25.76, 53.22) --
	(280.54, 53.22);

\path[] ( 25.76, 96.69) --
	(280.54, 96.69);

\path[] ( 25.76,140.16) --
	(280.54,140.16);

\path[] ( 25.76,183.64) --
	(280.54,183.64);

\path[] ( 62.15, 15.61) --
	( 62.15,191.48);

\path[] (122.82, 15.61) --
	(122.82,191.48);

\path[] (183.48, 15.61) --
	(183.48,191.48);

\path[] (244.15, 15.61) --
	(244.15,191.48);
\definecolor{drawColor}{RGB}{0,0,255}

\path[draw=drawColor,line width= 1.7pt,line join=round] ( 62.15, 60.50) --
	(122.82, 99.42) --
	(183.48,131.03) --
	(244.15,183.49);
\definecolor{drawColor}{RGB}{0,0,0}

\node[text=drawColor,anchor=base,inner sep=0pt, outer sep=0pt, scale=  1.02] at ( 62.15, 48.59) {9,083,763};

\node[text=drawColor,anchor=base east,inner sep=0pt, outer sep=0pt, scale=  1.02] at (115.67, 99.42) {9,531,370};

\node[text=drawColor,anchor=base east,inner sep=0pt, outer sep=0pt, scale=  1.02] at (176.34,131.03) {9,894,951};

\node[text=drawColor,anchor=base,inner sep=0pt, outer sep=0pt, scale=  1.02] at (244.15,187.46) {10,498,289};

\path[draw=drawColor,line width= 0.1pt,line join=round] ( 25.76, 23.61) -- (280.54, 23.61);

\path[] ( 25.76, 15.61) rectangle (280.54,191.48);
\end{scope}
\begin{scope}
\path[clip] (  0.00,  0.00) rectangle (289.08,198.74);

\path[] ( 25.76, 15.61) --
	( 25.76,191.48);
\end{scope}
\begin{scope}
\path[clip] (  0.00,  0.00) rectangle (289.08,198.74);
\definecolor{drawColor}{RGB}{255,255,255}

\node[text=drawColor,text opacity=0.00,anchor=base east,inner sep=0pt, outer sep=0pt, scale=  1.00] at ( 20.81, 49.31) {9000000};

\node[text=drawColor,text opacity=0.00,anchor=base east,inner sep=0pt, outer sep=0pt, scale=  1.00] at ( 20.81, 92.78) {9500000};

\node[text=drawColor,text opacity=0.00,anchor=base east,inner sep=0pt, outer sep=0pt, scale=  1.00] at ( 20.81,136.26) {10000000};

\node[text=drawColor,text opacity=0.00,anchor=base east,inner sep=0pt, outer sep=0pt, scale=  1.00] at ( 20.81,179.73) {10500000};
\end{scope}
\begin{scope}
\path[clip] (  0.00,  0.00) rectangle (289.08,198.74);

\path[] ( 23.01, 53.22) --
	( 25.76, 53.22);

\path[] ( 23.01, 96.69) --
	( 25.76, 96.69);

\path[] ( 23.01,140.16) --
	( 25.76,140.16);

\path[] ( 23.01,183.64) --
	( 25.76,183.64);
\end{scope}
\begin{scope}
\path[clip] (  0.00,  0.00) rectangle (289.08,198.74);

\path[] ( 25.76, 15.61) --
	(280.54, 15.61);
\end{scope}
\begin{scope}
\path[clip] (  0.00,  0.00) rectangle (289.08,198.74);

\path[] ( 62.15, 12.86) --
	( 62.15, 15.61);

\path[] (122.82, 12.86) --
	(122.82, 15.61);

\path[] (183.48, 12.86) --
	(183.48, 15.61);

\path[] (244.15, 12.86) --
	(244.15, 15.61);
\end{scope}
\begin{scope}
\path[clip] (  0.00,  0.00) rectangle (289.08,198.74);
\definecolor{drawColor}{RGB}{0,0,0}

\node[text=drawColor,anchor=base,inner sep=0pt, outer sep=0pt, scale=  1.00] at ( 62.15,  2.85) {2011};

\node[text=drawColor,anchor=base,inner sep=0pt, outer sep=0pt, scale=  1.00] at (122.82,  2.85) {2012};

\node[text=drawColor,anchor=base,inner sep=0pt, outer sep=0pt, scale=  1.00] at (183.48,  2.85) {2013};

\node[text=drawColor,anchor=base,inner sep=0pt, outer sep=0pt, scale=  1.00] at (244.15,  2.85) {2014};
\end{scope}
  \end{tikzpicture}}%
 {%
 	ENEI 2014, 2013, 2012 y 2011} % 
 
 
  %%%%%%%%%%%%%%%%%%%%%%%%%%%%%%%%%%20%%%%%%%%%%%%%%%%%%%%%%%%
 
 \cajita{%
 	PEA }%
 {%
 }%
 {%
 	Población económicamente activa} %
 {%
 	República de Guatemala, serie histórica por ENEI, en número de personas} %
 {%
 	\begin{tikzpicture}[x=1pt,y=1pt]  % Created by tikzDevice version 0.9 on 2016-03-03 02:00:13
% !TEX encoding = UTF-8 Unicode
\definecolor{fillColor}{RGB}{255,255,255}
\path[use as bounding box,fill=fillColor,fill opacity=0.00] (0,0) rectangle (289.08,198.74);
\begin{scope}
\path[clip] (  0.00,  0.00) rectangle (289.08,198.74);

\path[] (  0.00,  0.00) rectangle (289.08,198.74);
\end{scope}
\begin{scope}
\path[clip] (  0.00,  0.00) rectangle (289.08,198.74);

\path[] ( 21.38, 15.61) rectangle (280.54,191.48);

\path[] ( 21.38, 26.79) --
	(280.54, 26.79);

\path[] ( 21.38, 68.42) --
	(280.54, 68.42);

\path[] ( 21.38,110.05) --
	(280.54,110.05);

\path[] ( 21.38,151.68) --
	(280.54,151.68);

\path[] ( 21.38, 47.60) --
	(280.54, 47.60);

\path[] ( 21.38, 89.23) --
	(280.54, 89.23);

\path[] ( 21.38,130.87) --
	(280.54,130.87);

\path[] ( 21.38,172.50) --
	(280.54,172.50);

\path[] ( 58.40, 15.61) --
	( 58.40,191.48);

\path[] (120.11, 15.61) --
	(120.11,191.48);

\path[] (181.81, 15.61) --
	(181.81,191.48);

\path[] (243.52, 15.61) --
	(243.52,191.48);
\definecolor{drawColor}{RGB}{0,0,255}

\path[draw=drawColor,line width= 1.7pt,line join=round] ( 58.40, 60.50) --
	(120.11,170.01) --
	(181.81,129.27) --
	(243.52,183.49);
\definecolor{drawColor}{RGB}{0,0,0}

\node[text=drawColor,anchor=base,inner sep=0pt, outer sep=0pt, scale=  1.02] at ( 58.40, 48.59) {5,577,471};

\node[text=drawColor,anchor=base,inner sep=0pt, outer sep=0pt, scale=  1.02] at (120.11,173.98) {6,235,064};

\node[text=drawColor,anchor=base,inner sep=0pt, outer sep=0pt, scale=  1.02] at (181.81,117.36) {5,990,436};

\node[text=drawColor,anchor=base,inner sep=0pt, outer sep=0pt, scale=  1.02] at (243.52,187.46) {6,316,005};

\path[draw=drawColor,line width= 0.1pt,line join=round] ( 21.38, 23.61) -- (280.54, 23.61);

\path[] ( 21.38, 15.61) rectangle (280.54,191.48);
\end{scope}
\begin{scope}
\path[clip] (  0.00,  0.00) rectangle (289.08,198.74);

\path[] ( 21.38, 15.61) --
	( 21.38,191.48);
\end{scope}
\begin{scope}
\path[clip] (  0.00,  0.00) rectangle (289.08,198.74);
\definecolor{drawColor}{RGB}{255,255,255}

\node[text=drawColor,text opacity=0.00,anchor=base east,inner sep=0pt, outer sep=0pt, scale=  1.00] at ( 16.43, 43.69) {5500000};

\node[text=drawColor,text opacity=0.00,anchor=base east,inner sep=0pt, outer sep=0pt, scale=  1.00] at ( 16.43, 85.32) {5750000};

\node[text=drawColor,text opacity=0.00,anchor=base east,inner sep=0pt, outer sep=0pt, scale=  1.00] at ( 16.43,126.96) {6000000};

\node[text=drawColor,text opacity=0.00,anchor=base east,inner sep=0pt, outer sep=0pt, scale=  1.00] at ( 16.43,168.59) {6250000};
\end{scope}
\begin{scope}
\path[clip] (  0.00,  0.00) rectangle (289.08,198.74);

\path[] ( 18.63, 47.60) --
	( 21.38, 47.60);

\path[] ( 18.63, 89.23) --
	( 21.38, 89.23);

\path[] ( 18.63,130.87) --
	( 21.38,130.87);

\path[] ( 18.63,172.50) --
	( 21.38,172.50);
\end{scope}
\begin{scope}
\path[clip] (  0.00,  0.00) rectangle (289.08,198.74);

\path[] ( 21.38, 15.61) --
	(280.54, 15.61);
\end{scope}
\begin{scope}
\path[clip] (  0.00,  0.00) rectangle (289.08,198.74);

\path[] ( 58.40, 12.86) --
	( 58.40, 15.61);

\path[] (120.11, 12.86) --
	(120.11, 15.61);

\path[] (181.81, 12.86) --
	(181.81, 15.61);

\path[] (243.52, 12.86) --
	(243.52, 15.61);
\end{scope}
\begin{scope}
\path[clip] (  0.00,  0.00) rectangle (289.08,198.74);
\definecolor{drawColor}{RGB}{0,0,0}

\node[text=drawColor,anchor=base,inner sep=0pt, outer sep=0pt, scale=  1.00] at ( 58.40,  2.85) {2011};

\node[text=drawColor,anchor=base,inner sep=0pt, outer sep=0pt, scale=  1.00] at (120.11,  2.85) {2012};

\node[text=drawColor,anchor=base,inner sep=0pt, outer sep=0pt, scale=  1.00] at (181.81,  2.85) {2013};

\node[text=drawColor,anchor=base,inner sep=0pt, outer sep=0pt, scale=  1.00] at (243.52,  2.85) {2014};
\end{scope}
  \end{tikzpicture}}%
 {%
 	ENEI 2014, 2013, 2012 y 2011} % 
 
 
  %%%%%%%%%%%%%%%%%%%%%%%%%%%%%%%%%%21%%%%%%%%%%%%%%%%%%%%%%%%
  
  \cajita{%
  	Población ocupada }%
  {%
  }%
  {%
  	Población ocupada a nivel nacional} %
  {%
  	República de Guatemala, serie histórica por ENEI, en número de personas} %
  {%
  	\begin{tikzpicture}[x=1pt,y=1pt]  % Created by tikzDevice version 0.9 on 2016-03-03 02:00:19
% !TEX encoding = UTF-8 Unicode
\definecolor{fillColor}{RGB}{255,255,255}
\path[use as bounding box,fill=fillColor,fill opacity=0.00] (0,0) rectangle (289.08,198.74);
\begin{scope}
\path[clip] (  0.00,  0.00) rectangle (289.08,198.74);

\path[] (  0.00,  0.00) rectangle (289.08,198.74);
\end{scope}
\begin{scope}
\path[clip] (  0.00,  0.00) rectangle (289.08,198.74);

\path[] ( 21.38, 15.61) rectangle (280.54,191.48);

\path[] ( 21.38, 25.65) --
	(280.54, 25.65);

\path[] ( 21.38, 64.84) --
	(280.54, 64.84);

\path[] ( 21.38,104.02) --
	(280.54,104.02);

\path[] ( 21.38,143.21) --
	(280.54,143.21);

\path[] ( 21.38,182.39) --
	(280.54,182.39);

\path[] ( 21.38, 45.25) --
	(280.54, 45.25);

\path[] ( 21.38, 84.43) --
	(280.54, 84.43);

\path[] ( 21.38,123.62) --
	(280.54,123.62);

\path[] ( 21.38,162.80) --
	(280.54,162.80);

\path[] ( 58.40, 15.61) --
	( 58.40,191.48);

\path[] (120.11, 15.61) --
	(120.11,191.48);

\path[] (181.81, 15.61) --
	(181.81,191.48);

\path[] (243.52, 15.61) --
	(243.52,191.48);
\definecolor{drawColor}{RGB}{0,0,255}

\path[draw=drawColor,line width= 1.7pt,line join=round] ( 58.40, 60.50) --
	(120.11,171.55) --
	(181.81,133.21) --
	(243.52,183.49);
\definecolor{drawColor}{RGB}{0,0,0}

\node[text=drawColor,anchor=base,inner sep=0pt, outer sep=0pt, scale=  1.02] at ( 58.40, 48.59) {5,347,334};

\node[text=drawColor,anchor=base,inner sep=0pt, outer sep=0pt, scale=  1.02] at (120.11,175.52) {6,055,826};

\node[text=drawColor,anchor=base,inner sep=0pt, outer sep=0pt, scale=  1.02] at (181.81,121.29) {5,811,193};

\node[text=drawColor,anchor=base,inner sep=0pt, outer sep=0pt, scale=  1.02] at (243.52,187.46) {6,131,995};

\path[draw=drawColor,line width= 0.1pt,line join=round] ( 21.38, 23.61) -- (280.54, 23.61);

\path[] ( 21.38, 15.61) rectangle (280.54,191.48);
\end{scope}
\begin{scope}
\path[clip] (  0.00,  0.00) rectangle (289.08,198.74);

\path[] ( 21.38, 15.61) --
	( 21.38,191.48);
\end{scope}
\begin{scope}
\path[clip] (  0.00,  0.00) rectangle (289.08,198.74);
\definecolor{drawColor}{RGB}{255,255,255}

\node[text=drawColor,text opacity=0.00,anchor=base east,inner sep=0pt, outer sep=0pt, scale=  1.00] at ( 16.43, 41.34) {5250000};

\node[text=drawColor,text opacity=0.00,anchor=base east,inner sep=0pt, outer sep=0pt, scale=  1.00] at ( 16.43, 80.52) {5500000};

\node[text=drawColor,text opacity=0.00,anchor=base east,inner sep=0pt, outer sep=0pt, scale=  1.00] at ( 16.43,119.71) {5750000};

\node[text=drawColor,text opacity=0.00,anchor=base east,inner sep=0pt, outer sep=0pt, scale=  1.00] at ( 16.43,158.89) {6000000};
\end{scope}
\begin{scope}
\path[clip] (  0.00,  0.00) rectangle (289.08,198.74);

\path[] ( 18.63, 45.25) --
	( 21.38, 45.25);

\path[] ( 18.63, 84.43) --
	( 21.38, 84.43);

\path[] ( 18.63,123.62) --
	( 21.38,123.62);

\path[] ( 18.63,162.80) --
	( 21.38,162.80);
\end{scope}
\begin{scope}
\path[clip] (  0.00,  0.00) rectangle (289.08,198.74);

\path[] ( 21.38, 15.61) --
	(280.54, 15.61);
\end{scope}
\begin{scope}
\path[clip] (  0.00,  0.00) rectangle (289.08,198.74);

\path[] ( 58.40, 12.86) --
	( 58.40, 15.61);

\path[] (120.11, 12.86) --
	(120.11, 15.61);

\path[] (181.81, 12.86) --
	(181.81, 15.61);

\path[] (243.52, 12.86) --
	(243.52, 15.61);
\end{scope}
\begin{scope}
\path[clip] (  0.00,  0.00) rectangle (289.08,198.74);
\definecolor{drawColor}{RGB}{0,0,0}

\node[text=drawColor,anchor=base,inner sep=0pt, outer sep=0pt, scale=  1.00] at ( 58.40,  2.85) {2011};

\node[text=drawColor,anchor=base,inner sep=0pt, outer sep=0pt, scale=  1.00] at (120.11,  2.85) {2012};

\node[text=drawColor,anchor=base,inner sep=0pt, outer sep=0pt, scale=  1.00] at (181.81,  2.85) {2013};

\node[text=drawColor,anchor=base,inner sep=0pt, outer sep=0pt, scale=  1.00] at (243.52,  2.85) {2014};
\end{scope}
  \end{tikzpicture}}%
  {%
  	ENEI 2014, 2013, 2012 y 2011} %  
  
  
  
  %%%%%%%%%%%%%%%%%%%%%%%%%%%%%%%%%%22%%%%%%%%%%%%%%%%%%%%%%%%
  
  \cajita{%
  	Tasa de participación según área de residencia }%
  {%
  }%
  {%
  	Tasa de participación de la población ocupada según el área de residencia} %
  {%
  	República de Guatemala, serie histórica por ENEI, en porcentaje} %
  {%
  	\begin{tikzpicture}[x=1pt,y=1pt]  \input{graficas/1_23.tex}  \end{tikzpicture}}%
  {%
  	ENEI 2014, 2013, 2012 y 2011} %  
  
  
 %%%%%%%%%%%%%%%%%%%%%%%%%%%%%%%%%%23%%%%%%%%%%%%%%%%%%%%%%%%
 
 \cajita{%
 	Tasa de participación según sexo}%
 {%
 }%
 {%
 	Tasa de participación de la población ocupada según sexo} %
 {%
 	República de Guatemala, serie histórica por ENEI, en número de personas} %
 {%
 	\begin{tikzpicture}[x=1pt,y=1pt]  \input{graficas/1_24.tex}  \end{tikzpicture}}%
 {%
 	ENEI 2014, 2013, 2012 y 2011} %    
 

		\INEchaptercarta{Disponibilidad de alimentos}{La dimensión de Disponibilidad de Alimentos se refiere a garantizar la disponibilidad de una alimentación adecuada para todos (FAO, 2015). Esta dimensión se compone de cuatro indicadores y diecisiete sub-indicadores. A pesar de ser una de las dimensiones clave para la seguridad alimentaria y nutricional, fue la que representó más dificultad para los recopiladores de datos, ya que la información es escasa.
			
			El indicador \textbf{2.1 Producción de Alimentos}, cuenta con siete sub-indicadores: a) Producción interna de alimentos (maíz, frijol, arroz, trigo y ajonjolí), b) superficie sembrada de alimentos básicos, c) rendimiento de cultivo de alimentos por hectárea, d) huertos familiares, e) huertos escolares, y f) hoja de balance de alimentos. De estos sub-indicadores, no se encontró información para los huertos familiares y los huertos escolares.
			
			Para los indicadores \textbf{2.2 Pérdida post-cosecha} y \textbf{2.3 Reservas de granos básicos} no se encontró información disponible. Finalmente, para el indicador 2.4 Comercio de alimentos, el cual se divide en 4 sub-indicadores, se encontró información para los sub-indicadores de importación y exportación de alimentos, pero no se encontró información para los sub-indicadores de nivel de dependencia externa de alimentos y ayuda alimentaria.}
		
		%#########################1########################

 \cajita{%
Maíz }%
{%
<<<<<<< HEAD
El área ocupada para la siembra de maíz, durante los períodos de siembra de mayo de un año a abril del siguiente ha tenido una tendencia creciente, llegando a ocupar 1.24 millones de  manzanas en el período agrícola 2014/2015.

Así también, la producción de maíz en el territorio nacional fue de 40.7 millones de quintales en el período agrícola 2014/2015.\textollamada[1]{Datos preliminares}
\textollamada[2]{Datos estimados}}%
=======
 }%
>>>>>>> origin/master
{%
 Área cosechada y producción de maíz} %
{%
 República de Guatemala, serie histórica, en manzanas y  quintales } %
{%
 \begin{tikzpicture}[x=1pt,y=1pt]  % Created by tikzDevice version 0.9 on 2016-03-03 04:14:58
% !TEX encoding = UTF-8 Unicode
\definecolor{fillColor}{RGB}{255,255,255}
\path[use as bounding box,fill=fillColor,fill opacity=0.00] (0,0) rectangle (289.08,198.74);
\begin{scope}
\path[clip] (  0.00,  0.00) rectangle (289.08,198.74);

\path[] (  0.00,  0.00) rectangle (289.08,198.74);
\end{scope}
\begin{scope}
\path[clip] (  0.00,  0.00) rectangle (289.08,198.74);

\path[] (  0.00, 18.46) rectangle (289.08,142.27);

\path[] ( 33.36, 18.46) --
	( 33.36,142.27);

\path[] ( 88.95, 18.46) --
	( 88.95,142.27);

\path[] (144.54, 18.46) --
	(144.54,142.27);

\path[] (200.13, 18.46) --
	(200.13,142.27);

\path[] (255.72, 18.46) --
	(255.72,142.27);
\definecolor{drawColor}{RGB}{0,0,255}
\definecolor{fillColor}{RGB}{0,0,255}

\path[draw=drawColor,line width= 0.6pt,line join=round,fill=fillColor] (  9.73, 18.46) rectangle ( 31.97, 22.03);
\definecolor{drawColor}{RGB}{157,187,255}
\definecolor{fillColor}{RGB}{157,187,255}

\path[draw=drawColor,line width= 0.6pt,line join=round,fill=fillColor] ( 34.75, 18.46) rectangle ( 56.98,128.26);
\definecolor{drawColor}{RGB}{0,0,255}
\definecolor{fillColor}{RGB}{0,0,255}

\path[draw=drawColor,line width= 0.6pt,line join=round,fill=fillColor] ( 65.32, 18.46) rectangle ( 87.56, 22.11);
\definecolor{drawColor}{RGB}{157,187,255}
\definecolor{fillColor}{RGB}{157,187,255}

\path[draw=drawColor,line width= 0.6pt,line join=round,fill=fillColor] ( 90.34, 18.46) rectangle (112.57,130.74);
\definecolor{drawColor}{RGB}{0,0,255}
\definecolor{fillColor}{RGB}{0,0,255}

\path[draw=drawColor,line width= 0.6pt,line join=round,fill=fillColor] (120.91, 18.46) rectangle (143.15, 22.14);
\definecolor{drawColor}{RGB}{157,187,255}
\definecolor{fillColor}{RGB}{157,187,255}

\path[draw=drawColor,line width= 0.6pt,line join=round,fill=fillColor] (145.93, 18.46) rectangle (168.17,133.98);
\definecolor{drawColor}{RGB}{0,0,255}
\definecolor{fillColor}{RGB}{0,0,255}

\path[draw=drawColor,line width= 0.6pt,line join=round,fill=fillColor] (176.51, 18.46) rectangle (198.74, 22.21);
\definecolor{drawColor}{RGB}{157,187,255}
\definecolor{fillColor}{RGB}{157,187,255}

\path[draw=drawColor,line width= 0.6pt,line join=round,fill=fillColor] (201.52, 18.46) rectangle (223.76,138.78);
\definecolor{drawColor}{RGB}{0,0,255}
\definecolor{fillColor}{RGB}{0,0,255}

\path[draw=drawColor,line width= 0.6pt,line join=round,fill=fillColor] (232.10, 18.46) rectangle (254.33, 22.25);
\definecolor{drawColor}{RGB}{157,187,255}
\definecolor{fillColor}{RGB}{157,187,255}

\path[draw=drawColor,line width= 0.6pt,line join=round,fill=fillColor] (257.11, 18.46) rectangle (279.35,142.27);
\definecolor{drawColor}{RGB}{0,0,0}

\path[draw=drawColor,line width= 0.6pt,line join=round] (  0.00, 18.46) -- (289.08, 18.46);

\node[text=drawColor,rotate= 90.00,anchor=base west,inner sep=0pt, outer sep=0pt, scale=  0.83] at ( 24.08, 27.85) {1,175,255};

\node[text=drawColor,rotate= 90.00,anchor=base west,inner sep=0pt, outer sep=0pt, scale=  0.83] at ( 49.10,134.81) {36,117,212};

\node[text=drawColor,rotate= 90.00,anchor=base west,inner sep=0pt, outer sep=0pt, scale=  0.83] at ( 79.67, 27.93) {1,199,900};

\node[text=drawColor,rotate= 90.00,anchor=base west,inner sep=0pt, outer sep=0pt, scale=  0.83] at (104.69,137.29) {36,932,600};

\node[text=drawColor,rotate= 90.00,anchor=base west,inner sep=0pt, outer sep=0pt, scale=  0.83] at (135.27, 27.96) {1,211,900};

\node[text=drawColor,rotate= 90.00,anchor=base west,inner sep=0pt, outer sep=0pt, scale=  0.83] at (160.28,140.52) {37,995,900};

\node[text=drawColor,rotate= 90.00,anchor=base west,inner sep=0pt, outer sep=0pt, scale=  0.83] at (190.86, 28.03) {1,233,300};

\node[text=drawColor,rotate= 90.00,anchor=base west,inner sep=0pt, outer sep=0pt, scale=  0.83] at (215.88,145.33) {39,576,500};

\node[text=drawColor,rotate= 90.00,anchor=base west,inner sep=0pt, outer sep=0pt, scale=  0.83] at (246.45, 28.07) {1,247,100};

\node[text=drawColor,rotate= 90.00,anchor=base west,inner sep=0pt, outer sep=0pt, scale=  0.83] at (271.47,148.82) {40,724,100};

\path[] (  0.00, 18.46) rectangle (289.08,142.27);
\end{scope}
\begin{scope}
\path[clip] (  0.00,  0.00) rectangle (289.08,198.74);

\path[] (  0.00, 18.46) --
	(289.08, 18.46);
\end{scope}
\begin{scope}
\path[clip] (  0.00,  0.00) rectangle (289.08,198.74);

\path[] ( 33.36, 15.71) --
	( 33.36, 18.46);

\path[] ( 88.95, 15.71) --
	( 88.95, 18.46);

\path[] (144.54, 15.71) --
	(144.54, 18.46);

\path[] (200.13, 15.71) --
	(200.13, 18.46);

\path[] (255.72, 15.71) --
	(255.72, 18.46);
\end{scope}
\begin{scope}
\path[clip] (  0.00,  0.00) rectangle (289.08,198.74);
\definecolor{drawColor}{RGB}{0,0,0}

\node[text=drawColor,anchor=base,inner sep=0pt, outer sep=0pt, scale=  1.00] at ( 33.36,  5.69) {2010/2011};

\node[text=drawColor,anchor=base,inner sep=0pt, outer sep=0pt, scale=  1.00] at ( 88.95,  5.69) {2011/2012};

\node[text=drawColor,anchor=base,inner sep=0pt, outer sep=0pt, scale=  1.00] at (144.54,  5.69) {2012/2013};

\node[text=drawColor,anchor=base,inner sep=0pt, outer sep=0pt, scale=  1.00] at (200.13,  5.69) {2013/2014 \llamada};

\node[text=drawColor,anchor=base,inner sep=0pt, outer sep=0pt, scale=  1.00] at (255.72,  5.69) {2014/2015  \llamada};
\end{scope}
\begin{scope}
\path[clip] (  0.00,  0.00) rectangle (289.08,198.74);
\coordinate (apoyo) at (61.96,191.07);
\coordinate (longitudFicticia) at (7.11,7.67);
\coordinate (longitud) at (7.11,7.11);
\coordinate (desX) at (128.08,0);
\coordinate (desY) at (0,0.28);
\definecolor[named]{ct1}{HTML}{
0000FF
}
\definecolor[named]{ct2}{HTML}{
9DBBFF
}
\definecolor[named]{ctb1}{HTML}{
0000FF
}
\definecolor[named]{ctb2}{HTML}{
9DBBFF
}
\path [fill=none] (apoyo) rectangle ($(apoyo)+(longitudFicticia)$)
node [xshift=0.3cm,inner sep=0pt, outer sep=0pt,midway,right,scale = 0.9]{Area};
\draw [color = ctb1,fill=ct1] ( $(apoyo)  + (desY) $) rectangle ($(apoyo)+ (desY) +(longitud)$);
\path [fill=none] ($(apoyo)+(desX)$) rectangle ($(apoyo)+(desX)+(longitudFicticia)$)
node [xshift=0.3cm,inner sep=0pt, outer sep=0pt,midway,right,scale = 0.9]{Producción};
\draw [color = ctb2 ,fill=ct2] ( $(apoyo)  + (desY) + (desX) $) rectangle ($(apoyo)+ (desY)+ (desX) +(longitud)$);
\end{scope}
  \end{tikzpicture}}%
{%
Diplan-MAGA con datos de Banguat (MAGA, 2013).} %


%#########################2########################

\cajita{%
	Rendimiento del maíz }%
{%
<<<<<<< HEAD
El rendimiento de un producto se define como la producción obtenida por cada unidad de tierra sembrada. Así, en los últimos 5 períodos de siembra aumentó tanto el área como la producción obtenida de maíz.

 El indicador de rendimiento también ha aumentado, lo que muestra que se ha hecho ligeramente más eficiente la siembra de maíz. \textollamada[1]{Datos preliminares}
\textollamada[2]{Datos estimados}}%
{%
	Rendimiento de la siembra de maíz } %
=======
}%
{%
	Rendimiento del maíz } %
>>>>>>> origin/master
{%
	República de Guatemala, serie histórica, en quintales sobre manzanas } %
{%
	\begin{tikzpicture}[x=1pt,y=1pt]  \input{graficas/2_02.tex}  \end{tikzpicture}}%
{%
	Diplan-MAGA con datos de Banguat (MAGA, 2013).} %


%#########################3########################

\cajita{%
	Frijol }%
{%
<<<<<<< HEAD
El área ocupada para la siembra de frijol, durante los períodos de siembra de mayo de un año a abril del siguiente ha tenido una tendencia creciente, llegando a ocupar 358.3 miles de  manzanas en el período agrícola 2014/2015.

Así también, la producción de frijol en el territorio nacional fue de 5.2 millones de quintales en el período agrícola 2014/2015.\textollamada[1]{Datos preliminares}
\textollamada[2]{Datos estimados}}%
=======
}%
>>>>>>> origin/master
{%
	Área cosechada y producción de frijol} %
{%
	República de Guatemala, serie histórica, en manzanas y  quintales } %
{%
	\begin{tikzpicture}[x=1pt,y=1pt]  % Created by tikzDevice version 0.9 on 2016-03-03 04:15:01
% !TEX encoding = UTF-8 Unicode
\definecolor{fillColor}{RGB}{255,255,255}
\path[use as bounding box,fill=fillColor,fill opacity=0.00] (0,0) rectangle (289.08,198.74);
\begin{scope}
\path[clip] (  0.00,  0.00) rectangle (289.08,198.74);

\path[] (  0.00,  0.00) rectangle (289.08,198.74);
\end{scope}
\begin{scope}
\path[clip] (  0.00,  0.00) rectangle (289.08,198.74);

\path[] (  0.00, 18.46) rectangle (289.08,146.67);

\path[] ( 33.36, 18.46) --
	( 33.36,146.67);

\path[] ( 88.95, 18.46) --
	( 88.95,146.67);

\path[] (144.54, 18.46) --
	(144.54,146.67);

\path[] (200.13, 18.46) --
	(200.13,146.67);

\path[] (255.72, 18.46) --
	(255.72,146.67);
\definecolor{drawColor}{RGB}{0,0,255}
\definecolor{fillColor}{RGB}{0,0,255}

\path[draw=drawColor,line width= 0.6pt,line join=round,fill=fillColor] (  9.73, 18.46) rectangle ( 31.97, 26.79);
\definecolor{drawColor}{RGB}{157,187,255}
\definecolor{fillColor}{RGB}{157,187,255}

\path[draw=drawColor,line width= 0.6pt,line join=round,fill=fillColor] ( 34.75, 18.46) rectangle ( 56.98,132.55);
\definecolor{drawColor}{RGB}{0,0,255}
\definecolor{fillColor}{RGB}{0,0,255}

\path[draw=drawColor,line width= 0.6pt,line join=round,fill=fillColor] ( 65.32, 18.46) rectangle ( 87.56, 26.85);
\definecolor{drawColor}{RGB}{157,187,255}
\definecolor{fillColor}{RGB}{157,187,255}

\path[draw=drawColor,line width= 0.6pt,line join=round,fill=fillColor] ( 90.34, 18.46) rectangle (112.57,134.86);
\definecolor{drawColor}{RGB}{0,0,255}
\definecolor{fillColor}{RGB}{0,0,255}

\path[draw=drawColor,line width= 0.6pt,line join=round,fill=fillColor] (120.91, 18.46) rectangle (143.15, 27.00);
\definecolor{drawColor}{RGB}{157,187,255}
\definecolor{fillColor}{RGB}{157,187,255}

\path[draw=drawColor,line width= 0.6pt,line join=round,fill=fillColor] (145.93, 18.46) rectangle (168.17,138.35);
\definecolor{drawColor}{RGB}{0,0,255}
\definecolor{fillColor}{RGB}{0,0,255}

\path[draw=drawColor,line width= 0.6pt,line join=round,fill=fillColor] (176.51, 18.46) rectangle (198.74, 27.18);
\definecolor{drawColor}{RGB}{157,187,255}
\definecolor{fillColor}{RGB}{157,187,255}

\path[draw=drawColor,line width= 0.6pt,line join=round,fill=fillColor] (201.52, 18.46) rectangle (223.76,142.82);
\definecolor{drawColor}{RGB}{0,0,255}
\definecolor{fillColor}{RGB}{0,0,255}

\path[draw=drawColor,line width= 0.6pt,line join=round,fill=fillColor] (232.10, 18.46) rectangle (254.33, 27.32);
\definecolor{drawColor}{RGB}{157,187,255}
\definecolor{fillColor}{RGB}{157,187,255}

\path[draw=drawColor,line width= 0.6pt,line join=round,fill=fillColor] (257.11, 18.46) rectangle (279.35,146.67);
\definecolor{drawColor}{RGB}{0,0,0}

\path[draw=drawColor,line width= 0.6pt,line join=round] (  0.00, 18.46) -- (289.08, 18.46);

\node[text=drawColor,rotate= 90.00,anchor=base west,inner sep=0pt, outer sep=0pt, scale=  0.83] at ( 24.08, 31.51) {336,756};

\node[text=drawColor,rotate= 90.00,anchor=base west,inner sep=0pt, outer sep=0pt, scale=  0.83] at ( 49.10,138.37) {4,610,828};

\node[text=drawColor,rotate= 90.00,anchor=base west,inner sep=0pt, outer sep=0pt, scale=  0.83] at ( 79.67, 31.57) {339,200};

\node[text=drawColor,rotate= 90.00,anchor=base west,inner sep=0pt, outer sep=0pt, scale=  0.83] at (104.69,140.68) {4,704,200};

\node[text=drawColor,rotate= 90.00,anchor=base west,inner sep=0pt, outer sep=0pt, scale=  0.83] at (135.27, 31.73) {345,400};

\node[text=drawColor,rotate= 90.00,anchor=base west,inner sep=0pt, outer sep=0pt, scale=  0.83] at (160.28,144.17) {4,845,500};

\node[text=drawColor,rotate= 90.00,anchor=base west,inner sep=0pt, outer sep=0pt, scale=  0.83] at (190.86, 31.90) {352,500};

\node[text=drawColor,rotate= 90.00,anchor=base west,inner sep=0pt, outer sep=0pt, scale=  0.83] at (215.88,148.64) {5,026,200};

\node[text=drawColor,rotate= 90.00,anchor=base west,inner sep=0pt, outer sep=0pt, scale=  0.83] at (246.45, 32.05) {358,300};

\node[text=drawColor,rotate= 90.00,anchor=base west,inner sep=0pt, outer sep=0pt, scale=  0.83] at (271.47,152.49) {5,181,500};

\path[] (  0.00, 18.46) rectangle (289.08,146.67);
\end{scope}
\begin{scope}
\path[clip] (  0.00,  0.00) rectangle (289.08,198.74);

\path[] (  0.00, 18.46) --
	(289.08, 18.46);
\end{scope}
\begin{scope}
\path[clip] (  0.00,  0.00) rectangle (289.08,198.74);

\path[] ( 33.36, 15.71) --
	( 33.36, 18.46);

\path[] ( 88.95, 15.71) --
	( 88.95, 18.46);

\path[] (144.54, 15.71) --
	(144.54, 18.46);

\path[] (200.13, 15.71) --
	(200.13, 18.46);

\path[] (255.72, 15.71) --
	(255.72, 18.46);
\end{scope}
\begin{scope}
\path[clip] (  0.00,  0.00) rectangle (289.08,198.74);
\definecolor{drawColor}{RGB}{0,0,0}

\node[text=drawColor,anchor=base,inner sep=0pt, outer sep=0pt, scale=  1.00] at ( 33.36,  5.69) {2010/2011};

\node[text=drawColor,anchor=base,inner sep=0pt, outer sep=0pt, scale=  1.00] at ( 88.95,  5.69) {2011/2012};

\node[text=drawColor,anchor=base,inner sep=0pt, outer sep=0pt, scale=  1.00] at (144.54,  5.69) {2012/2013};

\node[text=drawColor,anchor=base,inner sep=0pt, outer sep=0pt, scale=  1.00] at (200.13,  5.69) {2013/2014 p};

\node[text=drawColor,anchor=base,inner sep=0pt, outer sep=0pt, scale=  1.00] at (255.72,  5.69) {2014/2015  e};
\end{scope}
\begin{scope}
\path[clip] (  0.00,  0.00) rectangle (289.08,198.74);
\coordinate (apoyo) at (61.96,191.07);
\coordinate (longitudFicticia) at (7.11,7.67);
\coordinate (longitud) at (7.11,7.11);
\coordinate (desX) at (128.08,0);
\coordinate (desY) at (0,0.28);
\definecolor[named]{ct1}{HTML}{
0000FF
}
\definecolor[named]{ct2}{HTML}{
9DBBFF
}
\definecolor[named]{ctb1}{HTML}{
0000FF
}
\definecolor[named]{ctb2}{HTML}{
9DBBFF
}
\path [fill=none] (apoyo) rectangle ($(apoyo)+(longitudFicticia)$)
node [xshift=0.3cm,inner sep=0pt, outer sep=0pt,midway,right,scale = 0.9]{Area};
\draw [color = ctb1,fill=ct1] ( $(apoyo)  + (desY) $) rectangle ($(apoyo)+ (desY) +(longitud)$);
\path [fill=none] ($(apoyo)+(desX)$) rectangle ($(apoyo)+(desX)+(longitudFicticia)$)
node [xshift=0.3cm,inner sep=0pt, outer sep=0pt,midway,right,scale = 0.9]{Producción};
\draw [color = ctb2 ,fill=ct2] ( $(apoyo)  + (desY) + (desX) $) rectangle ($(apoyo)+ (desY)+ (desX) +(longitud)$);
\end{scope}
  \end{tikzpicture}}%
{%
	MAGA.} %

%#########################4	########################

\cajita{%
<<<<<<< HEAD
	Rendimiento frijol }%
{%
El rendimiento de un producto se define como la producción obtenida por cada unidad de tierra sembrada. Así, en los últimos 5 períodos de siembra aumentó tanto el área como la producción obtenida de frijol.

El indicador de rendimiento también ha aumentado, lo que muestra que se ha hecho ligeramente más eficiente la siembra de frijol. \textollamada[1]{Datos preliminares}
\textollamada[2]{Datos estimados}}%
{%
	Rendimiento de la siembra de frijol} %
=======
	Frijol }%
{%
}%
{%
	Rendimiento del frijol} %
>>>>>>> origin/master
{%
	República de Guatemala, serie histórica, en quintales sobre manzanas } %
{%
	\begin{tikzpicture}[x=1pt,y=1pt]  % Created by tikzDevice version 0.9 on 2016-03-03 04:15:04
% !TEX encoding = UTF-8 Unicode
\definecolor{fillColor}{RGB}{255,255,255}
\path[use as bounding box,fill=fillColor,fill opacity=0.00] (0,0) rectangle (289.08,198.74);
\begin{scope}
\path[clip] (  0.00,  0.00) rectangle (289.08,198.74);

\path[] (  0.00,  0.00) rectangle (289.08,198.74);
\end{scope}
\begin{scope}
\path[clip] (  0.00,  0.00) rectangle (289.08,198.74);

\path[] (  0.00, 12.77) rectangle (289.08,181.67);

\path[] ( 33.36, 12.77) --
	( 33.36,181.67);

\path[] ( 88.95, 12.77) --
	( 88.95,181.67);

\path[] (144.54, 12.77) --
	(144.54,181.67);

\path[] (200.13, 12.77) --
	(200.13,181.67);

\path[] (255.72, 12.77) --
	(255.72,181.67);
\definecolor{drawColor}{RGB}{0,0,255}
\definecolor{fillColor}{RGB}{0,0,255}

\path[draw=drawColor,line width= 0.6pt,line join=round,fill=fillColor] ( 18.07, 20.44) rectangle ( 48.64,165.82);

\path[draw=drawColor,line width= 0.6pt,line join=round,fill=fillColor] ( 73.66, 20.44) rectangle (104.24,167.70);

\path[draw=drawColor,line width= 0.6pt,line join=round,fill=fillColor] (129.25, 20.44) rectangle (159.83,169.40);

\path[draw=drawColor,line width= 0.6pt,line join=round,fill=fillColor] (184.84, 20.44) rectangle (215.42,171.84);

\path[draw=drawColor,line width= 0.6pt,line join=round,fill=fillColor] (240.44, 20.44) rectangle (271.01,173.99);
\definecolor{drawColor}{RGB}{0,0,0}

\path[draw=drawColor,line width= 0.1pt,line join=round] (  0.00, 20.44) -- (289.08, 20.44);

\node[text=drawColor,anchor=base,inner sep=0pt, outer sep=0pt, scale=  1.02] at ( 33.36,169.79) {13.7};

\node[text=drawColor,anchor=base,inner sep=0pt, outer sep=0pt, scale=  1.02] at ( 88.95,171.67) {13.9};

\node[text=drawColor,anchor=base,inner sep=0pt, outer sep=0pt, scale=  1.02] at (144.54,173.37) {14.0};

\node[text=drawColor,anchor=base,inner sep=0pt, outer sep=0pt, scale=  1.02] at (200.13,175.81) {14.3};

\node[text=drawColor,anchor=base,inner sep=0pt, outer sep=0pt, scale=  1.02] at (255.72,177.96) {14.5};

\path[] (  0.00, 12.77) rectangle (289.08,181.67);
\end{scope}
\begin{scope}
\path[clip] (  0.00,  0.00) rectangle (289.08,198.74);

\path[] (  0.00, 12.77) --
	(289.08, 12.77);
\end{scope}
\begin{scope}
\path[clip] (  0.00,  0.00) rectangle (289.08,198.74);

\path[] ( 33.36, 10.02) --
	( 33.36, 12.77);

\path[] ( 88.95, 10.02) --
	( 88.95, 12.77);

\path[] (144.54, 10.02) --
	(144.54, 12.77);

\path[] (200.13, 10.02) --
	(200.13, 12.77);

\path[] (255.72, 10.02) --
	(255.72, 12.77);
\end{scope}
\begin{scope}
\path[clip] (  0.00,  0.00) rectangle (289.08,198.74);
\definecolor{drawColor}{RGB}{0,0,0}

\node[text=drawColor,anchor=base,inner sep=0pt, outer sep=0pt, scale=  1.00] at ( 33.36, -0.00) {2010/2011};

\node[text=drawColor,anchor=base,inner sep=0pt, outer sep=0pt, scale=  1.00] at ( 88.95, -0.00) {2011/2012};

\node[text=drawColor,anchor=base,inner sep=0pt, outer sep=0pt, scale=  1.00] at (144.54, -0.00) {2012/2013};

\node[text=drawColor,anchor=base,inner sep=0pt, outer sep=0pt, scale=  1.00] at (200.13, -0.00) {2013/2014 p};

\node[text=drawColor,anchor=base,inner sep=0pt, outer sep=0pt, scale=  1.00] at (255.72, -0.00) {2014/2015  e};
\end{scope}
  \end{tikzpicture}}%
{%
	MAGA.} %

%#########################5########################

\cajita{%
	Arroz }%
{%
<<<<<<< HEAD
El área ocupada para la siembra de arroz, durante los períodos de siembra de mayo de un año a abril del siguiente ha tenido una tendencia creciente, llegando a ocupar 16 mil  manzanas en el período agrícola 2014/2015.

Así también, la producción de arroz en el territorio nacional fue de 732.9 mil quintales en el período agrícola 2014/2015.\textollamada[1]{Datos preliminares}
\textollamada[2]{Datos estimados}}%
=======
}%
>>>>>>> origin/master
{%
	Área cosechada y producción de arroz} %
{%
	República de Guatemala, serie histórica, en manzanas y  quintales } %
{%
	\begin{tikzpicture}[x=1pt,y=1pt]  % Created by tikzDevice version 0.9 on 2016-03-03 04:15:04
% !TEX encoding = UTF-8 Unicode
\definecolor{fillColor}{RGB}{255,255,255}
\path[use as bounding box,fill=fillColor,fill opacity=0.00] (0,0) rectangle (289.08,198.74);
\begin{scope}
\path[clip] (  0.00,  0.00) rectangle (289.08,198.74);

\path[] (  0.00,  0.00) rectangle (289.08,198.74);
\end{scope}
\begin{scope}
\path[clip] (  0.00,  0.00) rectangle (289.08,198.74);

\path[] (  0.00, 18.46) rectangle (289.08,153.33);

\path[] ( 33.36, 18.46) --
	( 33.36,153.33);

\path[] ( 88.95, 18.46) --
	( 88.95,153.33);

\path[] (144.54, 18.46) --
	(144.54,153.33);

\path[] (200.13, 18.46) --
	(200.13,153.33);

\path[] (255.72, 18.46) --
	(255.72,153.33);
\definecolor{drawColor}{RGB}{0,0,255}
\definecolor{fillColor}{RGB}{0,0,255}

\path[draw=drawColor,line width= 0.6pt,line join=round,fill=fillColor] (  9.73, 18.46) rectangle ( 31.97, 21.22);
\definecolor{drawColor}{RGB}{157,187,255}
\definecolor{fillColor}{RGB}{157,187,255}

\path[draw=drawColor,line width= 0.6pt,line join=round,fill=fillColor] ( 34.75, 18.46) rectangle ( 56.98,138.65);
\definecolor{drawColor}{RGB}{0,0,255}
\definecolor{fillColor}{RGB}{0,0,255}

\path[draw=drawColor,line width= 0.6pt,line join=round,fill=fillColor] ( 65.32, 18.46) rectangle ( 87.56, 21.25);
\definecolor{drawColor}{RGB}{157,187,255}
\definecolor{fillColor}{RGB}{157,187,255}

\path[draw=drawColor,line width= 0.6pt,line join=round,fill=fillColor] ( 90.34, 18.46) rectangle (112.57,141.81);
\definecolor{drawColor}{RGB}{0,0,255}
\definecolor{fillColor}{RGB}{0,0,255}

\path[draw=drawColor,line width= 0.6pt,line join=round,fill=fillColor] (120.91, 18.46) rectangle (143.15, 21.29);
\definecolor{drawColor}{RGB}{157,187,255}
\definecolor{fillColor}{RGB}{157,187,255}

\path[draw=drawColor,line width= 0.6pt,line join=round,fill=fillColor] (145.93, 18.46) rectangle (168.17,144.77);
\definecolor{drawColor}{RGB}{0,0,255}
\definecolor{fillColor}{RGB}{0,0,255}

\path[draw=drawColor,line width= 0.6pt,line join=round,fill=fillColor] (176.51, 18.46) rectangle (198.74, 21.35);
\definecolor{drawColor}{RGB}{157,187,255}
\definecolor{fillColor}{RGB}{157,187,255}

\path[draw=drawColor,line width= 0.6pt,line join=round,fill=fillColor] (201.52, 18.46) rectangle (223.76,149.28);
\definecolor{drawColor}{RGB}{0,0,255}
\definecolor{fillColor}{RGB}{0,0,255}

\path[draw=drawColor,line width= 0.6pt,line join=round,fill=fillColor] (232.10, 18.46) rectangle (254.33, 21.40);
\definecolor{drawColor}{RGB}{157,187,255}
\definecolor{fillColor}{RGB}{157,187,255}

\path[draw=drawColor,line width= 0.6pt,line join=round,fill=fillColor] (257.11, 18.46) rectangle (279.35,153.33);
\definecolor{drawColor}{RGB}{0,0,0}

\path[draw=drawColor,line width= 0.6pt,line join=round] (  0.00, 18.46) -- (289.08, 18.46);

\node[text=drawColor,rotate= 90.00,anchor=base west,inner sep=0pt, outer sep=0pt, scale=  0.83] at ( 24.08, 25.22) {15,012};

\node[text=drawColor,rotate= 90.00,anchor=base west,inner sep=0pt, outer sep=0pt, scale=  0.83] at ( 49.10,143.37) {653,140};

\node[text=drawColor,rotate= 90.00,anchor=base west,inner sep=0pt, outer sep=0pt, scale=  0.83] at ( 79.67, 25.25) {15,200};

\node[text=drawColor,rotate= 90.00,anchor=base west,inner sep=0pt, outer sep=0pt, scale=  0.83] at (104.69,146.53) {670,300};

\node[text=drawColor,rotate= 90.00,anchor=base west,inner sep=0pt, outer sep=0pt, scale=  0.83] at (135.27, 25.29) {15,400};

\node[text=drawColor,rotate= 90.00,anchor=base west,inner sep=0pt, outer sep=0pt, scale=  0.83] at (160.28,149.49) {686,400};

\node[text=drawColor,rotate= 90.00,anchor=base west,inner sep=0pt, outer sep=0pt, scale=  0.83] at (190.86, 25.34) {15,700};

\node[text=drawColor,rotate= 90.00,anchor=base west,inner sep=0pt, outer sep=0pt, scale=  0.83] at (215.88,154.00) {710,900};

\node[text=drawColor,rotate= 90.00,anchor=base west,inner sep=0pt, outer sep=0pt, scale=  0.83] at (246.45, 25.40) {16,000};

\node[text=drawColor,rotate= 90.00,anchor=base west,inner sep=0pt, outer sep=0pt, scale=  0.83] at (271.47,158.05) {732,900};

\path[] (  0.00, 18.46) rectangle (289.08,153.33);
\end{scope}
\begin{scope}
\path[clip] (  0.00,  0.00) rectangle (289.08,198.74);

\path[] (  0.00, 18.46) --
	(289.08, 18.46);
\end{scope}
\begin{scope}
\path[clip] (  0.00,  0.00) rectangle (289.08,198.74);

\path[] ( 33.36, 15.71) --
	( 33.36, 18.46);

\path[] ( 88.95, 15.71) --
	( 88.95, 18.46);

\path[] (144.54, 15.71) --
	(144.54, 18.46);

\path[] (200.13, 15.71) --
	(200.13, 18.46);

\path[] (255.72, 15.71) --
	(255.72, 18.46);
\end{scope}
\begin{scope}
\path[clip] (  0.00,  0.00) rectangle (289.08,198.74);
\definecolor{drawColor}{RGB}{0,0,0}

\node[text=drawColor,anchor=base,inner sep=0pt, outer sep=0pt, scale=  1.00] at ( 33.36,  5.69) {2010/2011};

\node[text=drawColor,anchor=base,inner sep=0pt, outer sep=0pt, scale=  1.00] at ( 88.95,  5.69) {2011/2012};

\node[text=drawColor,anchor=base,inner sep=0pt, outer sep=0pt, scale=  1.00] at (144.54,  5.69) {2012/2013};

\node[text=drawColor,anchor=base,inner sep=0pt, outer sep=0pt, scale=  1.00] at (200.13,  5.69) {2013/2014 p};

\node[text=drawColor,anchor=base,inner sep=0pt, outer sep=0pt, scale=  1.00] at (255.72,  5.69) {2014/2015  e};
\end{scope}
\begin{scope}
\path[clip] (  0.00,  0.00) rectangle (289.08,198.74);
\coordinate (apoyo) at (61.96,191.07);
\coordinate (longitudFicticia) at (7.11,7.67);
\coordinate (longitud) at (7.11,7.11);
\coordinate (desX) at (128.08,0);
\coordinate (desY) at (0,0.28);
\definecolor[named]{ct1}{HTML}{
0000FF
}
\definecolor[named]{ct2}{HTML}{
9DBBFF
}
\definecolor[named]{ctb1}{HTML}{
0000FF
}
\definecolor[named]{ctb2}{HTML}{
9DBBFF
}
\path [fill=none] (apoyo) rectangle ($(apoyo)+(longitudFicticia)$)
node [xshift=0.3cm,inner sep=0pt, outer sep=0pt,midway,right,scale = 0.9]{Area};
\draw [color = ctb1,fill=ct1] ( $(apoyo)  + (desY) $) rectangle ($(apoyo)+ (desY) +(longitud)$);
\path [fill=none] ($(apoyo)+(desX)$) rectangle ($(apoyo)+(desX)+(longitudFicticia)$)
node [xshift=0.3cm,inner sep=0pt, outer sep=0pt,midway,right,scale = 0.9]{Producción};
\draw [color = ctb2 ,fill=ct2] ( $(apoyo)  + (desY) + (desX) $) rectangle ($(apoyo)+ (desY)+ (desX) +(longitud)$);
\end{scope}
  \end{tikzpicture}}%
{%
	Diplan-MAGA con datos de Banguat (MAGA, 2013).} %


%#########################6########################

\cajita{%
	Rendimiento del arroz }%
{%
<<<<<<< HEAD
El rendimiento de un producto se define como la producción obtenida por cada unidad de tierra sembrada. Así, en los últimos 5 períodos de siembra aumentó tanto el área como la producción obtenida de arroz.

El indicador de rendimiento también ha aumentado, lo que muestra que se ha hecho ligeramente más eficiente la siembra de arroz. \textollamada[1]{Datos preliminares}
\textollamada[2]{Datos estimados}}%
{%
	Rendimiento de la siembra de arroz} %
=======
}%
{%
	Rendimiento del arroz} %
>>>>>>> origin/master
{%
	República de Guatemala, serie histórica, en quintales sobre manzanas } %
{%
	\begin{tikzpicture}[x=1pt,y=1pt]  \input{graficas/2_06.tex}  \end{tikzpicture}}%
{%
	Diplan-MAGA con datos de Banguat (MAGA, 2013).} %

%#########################7########################

\cajita{%
	Trigo }%
{%
<<<<<<< HEAD
El área ocupada para la siembra de trigo, durante los períodos de siembra de mayo de un año a abril del siguiente ha tenido una tendencia creciente, llegando a ocupar 1,100  manzanas en el período agrícola 2014/2015.

Así también, la producción de arroz en el territorio nacional fue de 35.8 mil quintales en el período agrícola 2014/2015.\textollamada[1]{Datos preliminares}
\textollamada[2]{Datos estimados}}%
=======
}%
>>>>>>> origin/master
{%
	Área cosechada y producción de trigo} %
{%
	República de Guatemala, serie histórica, en manzanas y  quintales } %
{%
	\begin{tikzpicture}[x=1pt,y=1pt]  % Created by tikzDevice version 0.9 on 2016-03-03 04:15:07
% !TEX encoding = UTF-8 Unicode
\definecolor{fillColor}{RGB}{255,255,255}
\path[use as bounding box,fill=fillColor,fill opacity=0.00] (0,0) rectangle (289.08,198.74);
\begin{scope}
\path[clip] (  0.00,  0.00) rectangle (289.08,198.74);

\path[] (  0.00,  0.00) rectangle (289.08,198.74);
\end{scope}
\begin{scope}
\path[clip] (  0.00,  0.00) rectangle (289.08,198.74);

\path[] (  0.00, 18.46) rectangle (289.08,157.72);

\path[] ( 33.36, 18.46) --
	( 33.36,157.72);

\path[] ( 88.95, 18.46) --
	( 88.95,157.72);

\path[] (144.54, 18.46) --
	(144.54,157.72);

\path[] (200.13, 18.46) --
	(200.13,157.72);

\path[] (255.72, 18.46) --
	(255.72,157.72);
\definecolor{drawColor}{RGB}{0,0,255}
\definecolor{fillColor}{RGB}{0,0,255}

\path[draw=drawColor,line width= 0.6pt,line join=round,fill=fillColor] (  9.73, 18.46) rectangle ( 31.97, 22.22);
\definecolor{drawColor}{RGB}{157,187,255}
\definecolor{fillColor}{RGB}{157,187,255}

\path[draw=drawColor,line width= 0.6pt,line join=round,fill=fillColor] ( 34.75, 18.46) rectangle ( 56.98,141.70);
\definecolor{drawColor}{RGB}{0,0,255}
\definecolor{fillColor}{RGB}{0,0,255}

\path[draw=drawColor,line width= 0.6pt,line join=round,fill=fillColor] ( 65.32, 18.46) rectangle ( 87.56, 22.35);
\definecolor{drawColor}{RGB}{157,187,255}
\definecolor{fillColor}{RGB}{157,187,255}

\path[draw=drawColor,line width= 0.6pt,line join=round,fill=fillColor] ( 90.34, 18.46) rectangle (112.57,141.38);
\definecolor{drawColor}{RGB}{0,0,255}
\definecolor{fillColor}{RGB}{0,0,255}

\path[draw=drawColor,line width= 0.6pt,line join=round,fill=fillColor] (120.91, 18.46) rectangle (143.15, 22.35);
\definecolor{drawColor}{RGB}{157,187,255}
\definecolor{fillColor}{RGB}{157,187,255}

\path[draw=drawColor,line width= 0.6pt,line join=round,fill=fillColor] (145.93, 18.46) rectangle (168.17,148.39);
\definecolor{drawColor}{RGB}{0,0,255}
\definecolor{fillColor}{RGB}{0,0,255}

\path[draw=drawColor,line width= 0.6pt,line join=round,fill=fillColor] (176.51, 18.46) rectangle (198.74, 22.74);
\definecolor{drawColor}{RGB}{157,187,255}
\definecolor{fillColor}{RGB}{157,187,255}

\path[draw=drawColor,line width= 0.6pt,line join=round,fill=fillColor] (201.52, 18.46) rectangle (223.76,151.89);
\definecolor{drawColor}{RGB}{0,0,255}
\definecolor{fillColor}{RGB}{0,0,255}

\path[draw=drawColor,line width= 0.6pt,line join=round,fill=fillColor] (232.10, 18.46) rectangle (254.33, 22.74);
\definecolor{drawColor}{RGB}{157,187,255}
\definecolor{fillColor}{RGB}{157,187,255}

\path[draw=drawColor,line width= 0.6pt,line join=round,fill=fillColor] (257.11, 18.46) rectangle (279.35,157.72);
\definecolor{drawColor}{RGB}{0,0,0}

\path[draw=drawColor,line width= 0.6pt,line join=round] (  0.00, 18.46) -- (289.08, 18.46);

\node[text=drawColor,rotate= 90.00,anchor=base west,inner sep=0pt, outer sep=0pt, scale=  0.83] at ( 24.08, 24.40) {968};

\node[text=drawColor,rotate= 90.00,anchor=base west,inner sep=0pt, outer sep=0pt, scale=  0.83] at ( 49.10,145.70) {31,681};

\node[text=drawColor,rotate= 90.00,anchor=base west,inner sep=0pt, outer sep=0pt, scale=  0.83] at ( 79.67, 25.62) {1,000};

\node[text=drawColor,rotate= 90.00,anchor=base west,inner sep=0pt, outer sep=0pt, scale=  0.83] at (104.69,145.38) {31,600};

\node[text=drawColor,rotate= 90.00,anchor=base west,inner sep=0pt, outer sep=0pt, scale=  0.83] at (135.27, 25.62) {1,000};

\node[text=drawColor,rotate= 90.00,anchor=base west,inner sep=0pt, outer sep=0pt, scale=  0.83] at (160.28,152.38) {33,400};

\node[text=drawColor,rotate= 90.00,anchor=base west,inner sep=0pt, outer sep=0pt, scale=  0.83] at (190.86, 26.01) {1,100};

\node[text=drawColor,rotate= 90.00,anchor=base west,inner sep=0pt, outer sep=0pt, scale=  0.83] at (215.88,155.88) {34,300};

\node[text=drawColor,rotate= 90.00,anchor=base west,inner sep=0pt, outer sep=0pt, scale=  0.83] at (246.45, 26.01) {1,100};

\node[text=drawColor,rotate= 90.00,anchor=base west,inner sep=0pt, outer sep=0pt, scale=  0.83] at (271.47,161.72) {35,800};

\path[] (  0.00, 18.46) rectangle (289.08,157.72);
\end{scope}
\begin{scope}
\path[clip] (  0.00,  0.00) rectangle (289.08,198.74);

\path[] (  0.00, 18.46) --
	(289.08, 18.46);
\end{scope}
\begin{scope}
\path[clip] (  0.00,  0.00) rectangle (289.08,198.74);

\path[] ( 33.36, 15.71) --
	( 33.36, 18.46);

\path[] ( 88.95, 15.71) --
	( 88.95, 18.46);

\path[] (144.54, 15.71) --
	(144.54, 18.46);

\path[] (200.13, 15.71) --
	(200.13, 18.46);

\path[] (255.72, 15.71) --
	(255.72, 18.46);
\end{scope}
\begin{scope}
\path[clip] (  0.00,  0.00) rectangle (289.08,198.74);
\definecolor{drawColor}{RGB}{0,0,0}

\node[text=drawColor,anchor=base,inner sep=0pt, outer sep=0pt, scale=  1.00] at ( 33.36,  5.69) {2010/2011};

\node[text=drawColor,anchor=base,inner sep=0pt, outer sep=0pt, scale=  1.00] at ( 88.95,  5.69) {2011/2012};

\node[text=drawColor,anchor=base,inner sep=0pt, outer sep=0pt, scale=  1.00] at (144.54,  5.69) {2012/2013};

\node[text=drawColor,anchor=base,inner sep=0pt, outer sep=0pt, scale=  1.00] at (200.13,  5.69) {2013/2014 \llamada};

\node[text=drawColor,anchor=base,inner sep=0pt, outer sep=0pt, scale=  1.00] at (255.72,  5.69) {2014/2015  \llamada};
\end{scope}
\begin{scope}
\path[clip] (  0.00,  0.00) rectangle (289.08,198.74);
\coordinate (apoyo) at (61.96,191.07);
\coordinate (longitudFicticia) at (7.11,7.67);
\coordinate (longitud) at (7.11,7.11);
\coordinate (desX) at (128.08,0);
\coordinate (desY) at (0,0.28);
\definecolor[named]{ct1}{HTML}{
0000FF
}
\definecolor[named]{ct2}{HTML}{
9DBBFF
}
\definecolor[named]{ctb1}{HTML}{
0000FF
}
\definecolor[named]{ctb2}{HTML}{
9DBBFF
}
\path [fill=none] (apoyo) rectangle ($(apoyo)+(longitudFicticia)$)
node [xshift=0.3cm,inner sep=0pt, outer sep=0pt,midway,right,scale = 0.9]{Area};
\draw [color = ctb1,fill=ct1] ( $(apoyo)  + (desY) $) rectangle ($(apoyo)+ (desY) +(longitud)$);
\path [fill=none] ($(apoyo)+(desX)$) rectangle ($(apoyo)+(desX)+(longitudFicticia)$)
node [xshift=0.3cm,inner sep=0pt, outer sep=0pt,midway,right,scale = 0.9]{Producción};
\draw [color = ctb2 ,fill=ct2] ( $(apoyo)  + (desY) + (desX) $) rectangle ($(apoyo)+ (desY)+ (desX) +(longitud)$);
\end{scope}
  \end{tikzpicture}}%
{%
	MAGA} %


%#########################8########################

\cajita{%
	Rendimiento del trigo}%
{%
<<<<<<< HEAD
El rendimiento de un producto se define como la producción obtenida por cada unidad de tierra sembrada. Así, en los últimos 5 períodos de siembra aumentó tanto el área como la producción obtenida de trigo.

El indicador de rendimiento, sin embargo, presenta una tendencia oscilante en el período 2010/2015. Cabe recalcar que los datos de la siembra 2013/2014 son no oficializados y para la siembra 2014/2015 se trabajó con proyecciones. \textollamada[1]{Datos preliminares}
\textollamada[2]{Datos estimados}}%
{%
	Rendimiento de la siembra de trigo} %
=======
}%
{%
	Rendimiento del trigo} %
>>>>>>> origin/master
{%
	República de Guatemala, serie histórica, en quintales sobre manzanas } %
{%
	\begin{tikzpicture}[x=1pt,y=1pt]  \input{graficas/2_08.tex}  \end{tikzpicture}}%
{%
	MAGA } %

%#########################9########################

\cajita{%
	Ajonjolí }%
{%
<<<<<<< HEAD
El área ocupada para la siembra de ajonjolí, durante los períodos de siembra de mayo de un año a abril del siguiente ha tenido una tendencia oscilante entre 50,000 y 56,000 manzanas.  Se estima que en el período agrícola 2014/2015 esta haya llegado a ser de 54,900 manzanas.

La producción de ajonjolí en el territorio nacional fue de 1.1 millones de quintales en el período agrícola 2014/2015.\\textollamada[1]{Datos preliminares}
\textollamada[2]{Datos estimados}}%
=======
}%
>>>>>>> origin/master
{%
	Área cosechada y producción de ajonjolí} %
{%
	República de Guatemala, serie histórica, en manzanas y  quintales } %
{%
	\begin{tikzpicture}[x=1pt,y=1pt]  \input{graficas/2_09.tex}  \end{tikzpicture}}%
{%
	Diplan-MAGA con datos de Banguat (MAGA, 2013).} %


%#########################10########################

\cajita{%
	Rendimiento del ajonjolí }%
{%
<<<<<<< HEAD
El rendimiento de ajonjolí obtenido en los últimos 5 años ha oscilando entre los 20 y 23.2 quintales por manzana de siembra.\textollamada[1]{Datos preliminares}
\textollamada[2]{Datos estimados}}%
=======
}%
>>>>>>> origin/master
{%
	Rendimiento del ajonjolí} %
{%
	República de Guatemala, serie histórica, en quintales sobre manzanas } %
{%
	\begin{tikzpicture}[x=1pt,y=1pt]  \input{graficas/2_10.tex}  \end{tikzpicture}}%
{%
	Diplan-MAGA con datos de Banguat (MAGA, 2013).} %


%#########################11########################

\cajita{%
	Balanza comercial de maíz blanco }%
{%
<<<<<<< HEAD
En relación al comercio internacional de productos, el maíz blanco tuvo una balanza negativa en el período 2010-2014, esto significa que la cantidad de producto importado fue mayor que la exportada.

En el 2014 se exportaron 2.0 mil toneladas métricas de este producto mientras ques se importaron 23.0 mil toneladas métricas.}%
=======
}%
>>>>>>> origin/master
{%
	Exportaciones, importaciones y balanza comercial de maíz blanco} %
{%
	República de Guatemala, serie histórica, en toneladas métricas } %
{%
	\begin{tikzpicture}[x=1pt,y=1pt]  % Created by tikzDevice version 0.9 on 2016-03-03 04:15:13
% !TEX encoding = UTF-8 Unicode
\definecolor{fillColor}{RGB}{255,255,255}
\path[use as bounding box,fill=fillColor,fill opacity=0.00] (0,0) rectangle (289.08,198.74);
\begin{scope}
\path[clip] (  0.00,  0.00) rectangle (289.08,198.74);

\path[] (  0.00,  0.00) rectangle (289.08,198.74);
\end{scope}
\begin{scope}
\path[clip] (  0.00,  0.00) rectangle (289.08,198.74);

\path[] (  0.00, 18.46) rectangle (289.08,148.78);

\path[] ( 33.36, 18.46) --
	( 33.36,148.78);

\path[] ( 88.95, 18.46) --
	( 88.95,148.78);

\path[] (144.54, 18.46) --
	(144.54,148.78);

\path[] (200.13, 18.46) --
	(200.13,148.78);

\path[] (255.72, 18.46) --
	(255.72,148.78);
\definecolor{drawColor}{RGB}{0,0,255}
\definecolor{fillColor}{RGB}{0,0,255}

\path[draw=drawColor,line width= 0.6pt,line join=round,fill=fillColor] (  9.27, 76.94) rectangle ( 24.09, 80.62);
\definecolor{drawColor}{RGB}{157,187,255}
\definecolor{fillColor}{RGB}{157,187,255}

\path[draw=drawColor,line width= 0.6pt,line join=round,fill=fillColor] ( 25.94, 76.94) rectangle ( 40.77,119.73);
\definecolor{drawColor}{RGB}{200,200,200}
\definecolor{fillColor}{RGB}{200,200,200}

\path[draw=drawColor,line width= 0.6pt,line join=round,fill=fillColor] ( 42.62, 37.84) rectangle ( 57.45, 76.94);
\definecolor{drawColor}{RGB}{0,0,255}
\definecolor{fillColor}{RGB}{0,0,255}

\path[draw=drawColor,line width= 0.6pt,line join=round,fill=fillColor] ( 64.86, 76.94) rectangle ( 79.68,101.43);
\definecolor{drawColor}{RGB}{157,187,255}
\definecolor{fillColor}{RGB}{157,187,255}

\path[draw=drawColor,line width= 0.6pt,line join=round,fill=fillColor] ( 81.54, 76.94) rectangle ( 96.36,148.78);
\definecolor{drawColor}{RGB}{200,200,200}
\definecolor{fillColor}{RGB}{200,200,200}

\path[draw=drawColor,line width= 0.6pt,line join=round,fill=fillColor] ( 98.21, 29.59) rectangle (113.04, 76.94);
\definecolor{drawColor}{RGB}{0,0,255}
\definecolor{fillColor}{RGB}{0,0,255}

\path[draw=drawColor,line width= 0.6pt,line join=round,fill=fillColor] (120.45, 76.94) rectangle (135.27, 81.38);
\definecolor{drawColor}{RGB}{157,187,255}
\definecolor{fillColor}{RGB}{157,187,255}

\path[draw=drawColor,line width= 0.6pt,line join=round,fill=fillColor] (137.13, 76.94) rectangle (151.95,139.87);
\definecolor{drawColor}{RGB}{200,200,200}
\definecolor{fillColor}{RGB}{200,200,200}

\path[draw=drawColor,line width= 0.6pt,line join=round,fill=fillColor] (153.81, 18.46) rectangle (168.63, 76.94);
\definecolor{drawColor}{RGB}{0,0,255}
\definecolor{fillColor}{RGB}{0,0,255}

\path[draw=drawColor,line width= 0.6pt,line join=round,fill=fillColor] (176.04, 76.94) rectangle (190.87, 91.15);
\definecolor{drawColor}{RGB}{157,187,255}
\definecolor{fillColor}{RGB}{157,187,255}

\path[draw=drawColor,line width= 0.6pt,line join=round,fill=fillColor] (192.72, 76.94) rectangle (207.54,108.80);
\definecolor{drawColor}{RGB}{200,200,200}
\definecolor{fillColor}{RGB}{200,200,200}

\path[draw=drawColor,line width= 0.6pt,line join=round,fill=fillColor] (209.40, 59.29) rectangle (224.22, 76.94);
\definecolor{drawColor}{RGB}{0,0,255}
\definecolor{fillColor}{RGB}{0,0,255}

\path[draw=drawColor,line width= 0.6pt,line join=round,fill=fillColor] (231.63, 76.94) rectangle (246.46, 80.39);
\definecolor{drawColor}{RGB}{157,187,255}
\definecolor{fillColor}{RGB}{157,187,255}

\path[draw=drawColor,line width= 0.6pt,line join=round,fill=fillColor] (248.31, 76.94) rectangle (263.14,116.85);
\definecolor{drawColor}{RGB}{200,200,200}
\definecolor{fillColor}{RGB}{200,200,200}

\path[draw=drawColor,line width= 0.6pt,line join=round,fill=fillColor] (264.99, 40.48) rectangle (279.81, 76.94);
\definecolor{drawColor}{RGB}{0,0,0}

\path[draw=drawColor,line width= 0.6pt,line join=round] (  0.00, 76.94) -- (289.08, 76.94);

\node[text=drawColor,rotate= 90.00,anchor=base west,inner sep=0pt, outer sep=0pt, scale=  0.83] at ( 19.91, 84.99) {2,127.5};

\node[text=drawColor,rotate= 90.00,anchor=base west,inner sep=0pt, outer sep=0pt, scale=  0.83] at ( 36.59,124.83) {24,745.3};

\node[text=drawColor,rotate= 90.00,anchor=base west,inner sep=0pt, outer sep=0pt, scale=  0.83] at ( 53.27, 43.39) {-22,617.8};

\node[text=drawColor,rotate= 90.00,anchor=base west,inner sep=0pt, outer sep=0pt, scale=  0.83] at ( 75.50,106.53) {14,164.0};

\node[text=drawColor,rotate= 90.00,anchor=base west,inner sep=0pt, outer sep=0pt, scale=  0.83] at ( 92.18,153.88) {41,547.8};

\node[text=drawColor,rotate= 90.00,anchor=base west,inner sep=0pt, outer sep=0pt, scale=  0.83] at (108.86, 35.15) {-27,383.8};

\node[text=drawColor,rotate= 90.00,anchor=base west,inner sep=0pt, outer sep=0pt, scale=  0.83] at (131.10, 85.76) {2,568.6};

\node[text=drawColor,rotate= 90.00,anchor=base west,inner sep=0pt, outer sep=0pt, scale=  0.83] at (147.77,144.97) {36,393.6};

\node[text=drawColor,rotate= 90.00,anchor=base west,inner sep=0pt, outer sep=0pt, scale=  0.83] at (164.45, 24.01) {-33,825.0};

\node[text=drawColor,rotate= 90.00,anchor=base west,inner sep=0pt, outer sep=0pt, scale=  0.83] at (186.69, 95.52) {8,214.9};

\node[text=drawColor,rotate= 90.00,anchor=base west,inner sep=0pt, outer sep=0pt, scale=  0.83] at (203.37,113.89) {18,422.1};

<<<<<<< HEAD
\node[text=drawColor,rotate= 90.00,anchor=base west,inner sep=0pt, outer sep=0pt, scale=  0.83] at (220.04, 24.85) {-10,207.1};
=======
\node[text=drawColor,rotate= 90.00,anchor=base west,inner sep=0pt, outer sep=0pt, scale=  0.83] at (220.04, 64.85) {-10,207.1};
>>>>>>> origin/master

\node[text=drawColor,rotate= 90.00,anchor=base west,inner sep=0pt, outer sep=0pt, scale=  0.83] at (242.28, 84.76) {1,995.0};

\node[text=drawColor,rotate= 90.00,anchor=base west,inner sep=0pt, outer sep=0pt, scale=  0.83] at (258.96,121.95) {23,081.3};

\node[text=drawColor,rotate= 90.00,anchor=base west,inner sep=0pt, outer sep=0pt, scale=  0.83] at (275.64, 46.04) {-21,086.3};

\path[] (  0.00, 18.46) rectangle (289.08,148.78);
\end{scope}
\begin{scope}
\path[clip] (  0.00,  0.00) rectangle (289.08,198.74);

\path[] (  0.00, 18.46) --
	(289.08, 18.46);
\end{scope}
\begin{scope}
\path[clip] (  0.00,  0.00) rectangle (289.08,198.74);

\path[] ( 33.36, 15.71) --
	( 33.36, 18.46);

\path[] ( 88.95, 15.71) --
	( 88.95, 18.46);

\path[] (144.54, 15.71) --
	(144.54, 18.46);

\path[] (200.13, 15.71) --
	(200.13, 18.46);

\path[] (255.72, 15.71) --
	(255.72, 18.46);
\end{scope}
\begin{scope}
\path[clip] (  0.00,  0.00) rectangle (289.08,198.74);
\definecolor{drawColor}{RGB}{0,0,0}

\node[text=drawColor,anchor=base,inner sep=0pt, outer sep=0pt, scale=  1.00] at ( 33.36,  5.69) {2010};

\node[text=drawColor,anchor=base,inner sep=0pt, outer sep=0pt, scale=  1.00] at ( 88.95,  5.69) {2011};

\node[text=drawColor,anchor=base,inner sep=0pt, outer sep=0pt, scale=  1.00] at (144.54,  5.69) {2012};

\node[text=drawColor,anchor=base,inner sep=0pt, outer sep=0pt, scale=  1.00] at (200.13,  5.69) {2013};

\node[text=drawColor,anchor=base,inner sep=0pt, outer sep=0pt, scale=  1.00] at (255.72,  5.69) {2014*};
\end{scope}
\begin{scope}
\path[clip] (  0.00,  0.00) rectangle (289.08,198.74);
\coordinate (apoyo1) at (25.51,188.79);
\coordinate (apoyo2) at (117.41,188.79);
\coordinate (apoyo3) at (217.03,188.79);
\coordinate (longitudFicticia) at (7.11,9.95);
\coordinate (longitud) at (7.11,7.11);
\coordinate (desY) at (0,1.42);
\definecolor[named]{ct1}{HTML}{
0000FF
}
\definecolor[named]{ct2}{HTML}{
9DBBFF
}
\definecolor[named]{ct3}{HTML}{
C8C8C8
}
\definecolor[named]{ctb1}{HTML}{
0000FF
}
\definecolor[named]{ctb2}{HTML}{
9DBBFF
}
\definecolor[named]{ctb3}{HTML}{
C8C8C8
}
\path [fill=none] (apoyo1) rectangle ($(apoyo1)+(longitudFicticia)$)
node [xshift=0.3cm,inner sep=0pt, outer sep=0pt,text width=1.33333333333333in,midway,right,scale = 0.9]{Exportación};
\draw [color= ctb1, fill=ct1] ( $(apoyo1)  + (desY) $) rectangle ($(apoyo1)+ (desY) +(longitud)$);
\path [fill=none] (apoyo2) rectangle ($(apoyo2)+(longitudFicticia)$)
node [xshift=0.3cm,inner sep=0pt, outer sep=0pt,text width=1.33333333333333in,midway,right,scale = 0.9]{Importación};
\draw [color = ctb2, fill=ct2] ( $(apoyo2)  + (desY) $) rectangle ($(apoyo2)+ (desY) +(longitud)$);
\path [fill=none] (apoyo3) rectangle ($(apoyo3)+(longitudFicticia)$)
node [xshift=0.3cm,inner sep=0pt, outer sep=0pt,text width=1.33333333333333in,midway,right,scale = 0.9]{Balanza};
\path [color = ctb3, fill=ct3] ( $(apoyo3)  + (desY) $) rectangle ($(apoyo3)+ (desY) +(longitud)$);
\end{scope}
  \end{tikzpicture}}%
{%
	Diplan-MAGA con datos de Banguat (MAGA, 2013).} %


%#########################12########################

\cajita{%
	Balanza comercial de maíz amarillo }%
{%
<<<<<<< HEAD
La serie histórica de las importaciones y exportaciones de maíz amarillo, muestra que el 99\% del maíz que transita por las aduanas nacionales tiene fines de importación.

En el 2014 se importaron 519,406 toneladas métricas de maíz amarillo, mientras que se exportaron 25.2 toneladas métricas. }%
=======
}%
>>>>>>> origin/master
{%
	Exportaciones, importaciones y balanza comercial de maíz amarillo} %
{%
	República de Guatemala, serie histórica, en toneladas métricas } %
{%
	\begin{tikzpicture}[x=1pt,y=1pt]  \input{graficas/2_12.tex}  \end{tikzpicture}}%
{%
	Diplan-MAGA con datos de Banguat (MAGA, 2013).} %



%#########################13########################

\cajita{%
	Balanza comercial del frijol }%
{%
<<<<<<< HEAD
En relación al comercio internacional de productos, el frijol tuvo una balanza negativa en el período 2010-2014, esto significa que la cantidad de producto importada fue mayor que la exportada.

En el 2014 se exportaron 2.2 mil toneladas métricas de este producto mientras qu se importaron 3.9 mil toneladas métricas.}%
=======
}%
>>>>>>> origin/master
{%
	Exportaciones, importaciones y balanza comercial del frijol} %
{%
	República de Guatemala, serie histórica, en toneladas métricas } %
{%
	\begin{tikzpicture}[x=1pt,y=1pt]  % Created by tikzDevice version 0.9 on 2016-03-03 04:15:19
% !TEX encoding = UTF-8 Unicode
\definecolor{fillColor}{RGB}{255,255,255}
\path[use as bounding box,fill=fillColor,fill opacity=0.00] (0,0) rectangle (289.08,198.74);
\begin{scope}
\path[clip] (  0.00,  0.00) rectangle (289.08,198.74);

\path[] (  0.00,  0.00) rectangle (289.08,198.74);
\end{scope}
\begin{scope}
\path[clip] (  0.00,  0.00) rectangle (289.08,198.74);

\path[] (  0.00, 18.46) rectangle (289.08,148.78);

\path[] ( 33.36, 18.46) --
	( 33.36,148.78);

\path[] ( 88.95, 18.46) --
	( 88.95,148.78);

\path[] (144.54, 18.46) --
	(144.54,148.78);

\path[] (200.13, 18.46) --
	(200.13,148.78);

\path[] (255.72, 18.46) --
	(255.72,148.78);
\definecolor{drawColor}{RGB}{0,0,255}
\definecolor{fillColor}{RGB}{0,0,255}

\path[draw=drawColor,line width= 0.6pt,line join=round,fill=fillColor] (  9.27, 81.03) rectangle ( 24.09, 85.14);
\definecolor{drawColor}{RGB}{157,187,255}
\definecolor{fillColor}{RGB}{157,187,255}

\path[draw=drawColor,line width= 0.6pt,line join=round,fill=fillColor] ( 25.94, 81.03) rectangle ( 40.77,120.35);
\definecolor{drawColor}{RGB}{200,200,200}
\definecolor{fillColor}{RGB}{200,200,200}

\path[draw=drawColor,line width= 0.6pt,line join=round,fill=fillColor] ( 42.62, 45.83) rectangle ( 57.45, 81.03);
\definecolor{drawColor}{RGB}{0,0,255}
\definecolor{fillColor}{RGB}{0,0,255}

\path[draw=drawColor,line width= 0.6pt,line join=round,fill=fillColor] ( 64.86, 81.03) rectangle ( 79.68, 86.21);
\definecolor{drawColor}{RGB}{157,187,255}
\definecolor{fillColor}{RGB}{157,187,255}

\path[draw=drawColor,line width= 0.6pt,line join=round,fill=fillColor] ( 81.54, 81.03) rectangle ( 96.36,148.78);
\definecolor{drawColor}{RGB}{200,200,200}
\definecolor{fillColor}{RGB}{200,200,200}

\path[draw=drawColor,line width= 0.6pt,line join=round,fill=fillColor] ( 98.21, 18.46) rectangle (113.04, 81.03);
\definecolor{drawColor}{RGB}{0,0,255}
\definecolor{fillColor}{RGB}{0,0,255}

\path[draw=drawColor,line width= 0.6pt,line join=round,fill=fillColor] (120.45, 81.03) rectangle (135.27, 81.75);
\definecolor{drawColor}{RGB}{157,187,255}
\definecolor{fillColor}{RGB}{157,187,255}

\path[draw=drawColor,line width= 0.6pt,line join=round,fill=fillColor] (137.13, 81.03) rectangle (151.95,117.26);
\definecolor{drawColor}{RGB}{200,200,200}
\definecolor{fillColor}{RGB}{200,200,200}

\path[draw=drawColor,line width= 0.6pt,line join=round,fill=fillColor] (153.81, 45.52) rectangle (168.63, 81.03);
\definecolor{drawColor}{RGB}{0,0,255}
\definecolor{fillColor}{RGB}{0,0,255}

\path[draw=drawColor,line width= 0.6pt,line join=round,fill=fillColor] (176.04, 81.03) rectangle (190.87, 86.01);
\definecolor{drawColor}{RGB}{157,187,255}
\definecolor{fillColor}{RGB}{157,187,255}

\path[draw=drawColor,line width= 0.6pt,line join=round,fill=fillColor] (192.72, 81.03) rectangle (207.54,102.20);
\definecolor{drawColor}{RGB}{200,200,200}
\definecolor{fillColor}{RGB}{200,200,200}

\path[draw=drawColor,line width= 0.6pt,line join=round,fill=fillColor] (209.40, 64.84) rectangle (224.22, 81.03);
\definecolor{drawColor}{RGB}{0,0,255}
\definecolor{fillColor}{RGB}{0,0,255}

\path[draw=drawColor,line width= 0.6pt,line join=round,fill=fillColor] (231.63, 81.03) rectangle (246.46, 88.37);
\definecolor{drawColor}{RGB}{157,187,255}
\definecolor{fillColor}{RGB}{157,187,255}

\path[draw=drawColor,line width= 0.6pt,line join=round,fill=fillColor] (248.31, 81.03) rectangle (263.14, 93.70);
\definecolor{drawColor}{RGB}{200,200,200}
\definecolor{fillColor}{RGB}{200,200,200}

\path[draw=drawColor,line width= 0.6pt,line join=round,fill=fillColor] (264.99, 75.70) rectangle (279.81, 81.03);
\definecolor{drawColor}{RGB}{0,0,0}

\path[draw=drawColor,line width= 0.6pt,line join=round] (  0.00, 81.03) -- (289.08, 81.03);

\node[text=drawColor,rotate= 90.00,anchor=base west,inner sep=0pt, outer sep=0pt, scale=  0.83] at ( 19.91, 89.52) {1,246.3};

\node[text=drawColor,rotate= 90.00,anchor=base west,inner sep=0pt, outer sep=0pt, scale=  0.83] at ( 36.59,125.44) {11,913.9};

\node[text=drawColor,rotate= 90.00,anchor=base west,inner sep=0pt, outer sep=0pt, scale=  0.83] at ( 53.27, 51.38) {-10,667.6};

\node[text=drawColor,rotate= 90.00,anchor=base west,inner sep=0pt, outer sep=0pt, scale=  0.83] at ( 75.50, 90.58) {1,568.0};

\node[text=drawColor,rotate= 90.00,anchor=base west,inner sep=0pt, outer sep=0pt, scale=  0.83] at ( 92.18,153.88) {20,531.0};

\node[text=drawColor,rotate= 90.00,anchor=base west,inner sep=0pt, outer sep=0pt, scale=  0.83] at (108.86, 24.01) {-18,963.0};

\node[text=drawColor,rotate= 90.00,anchor=base west,inner sep=0pt, outer sep=0pt, scale=  0.83] at (131.10, 85.02) {216.9};

\node[text=drawColor,rotate= 90.00,anchor=base west,inner sep=0pt, outer sep=0pt, scale=  0.83] at (147.77,122.36) {10,980.1};

\node[text=drawColor,rotate= 90.00,anchor=base west,inner sep=0pt, outer sep=0pt, scale=  0.83] at (164.45, 51.07) {-10,763.2};

\node[text=drawColor,rotate= 90.00,anchor=base west,inner sep=0pt, outer sep=0pt, scale=  0.83] at (186.69, 90.38) {1,508.0};

\node[text=drawColor,rotate= 90.00,anchor=base west,inner sep=0pt, outer sep=0pt, scale=  0.83] at (203.37,106.57) {6,414.1};

\node[text=drawColor,rotate= 90.00,anchor=base west,inner sep=0pt, outer sep=0pt, scale=  0.83] at (220.04, 69.67) {-4,906.1};

\node[text=drawColor,rotate= 90.00,anchor=base west,inner sep=0pt, outer sep=0pt, scale=  0.83] at (242.28, 92.74) {2,222.7};

\node[text=drawColor,rotate= 90.00,anchor=base west,inner sep=0pt, outer sep=0pt, scale=  0.83] at (258.96, 98.07) {3,839.2};

\node[text=drawColor,rotate= 90.00,anchor=base west,inner sep=0pt, outer sep=0pt, scale=  0.83] at (275.64, 80.53) {-1,616.5};

\path[] (  0.00, 18.46) rectangle (289.08,148.78);
\end{scope}
\begin{scope}
\path[clip] (  0.00,  0.00) rectangle (289.08,198.74);

\path[] (  0.00, 18.46) --
	(289.08, 18.46);
\end{scope}
\begin{scope}
\path[clip] (  0.00,  0.00) rectangle (289.08,198.74);

\path[] ( 33.36, 15.71) --
	( 33.36, 18.46);

\path[] ( 88.95, 15.71) --
	( 88.95, 18.46);

\path[] (144.54, 15.71) --
	(144.54, 18.46);

\path[] (200.13, 15.71) --
	(200.13, 18.46);

\path[] (255.72, 15.71) --
	(255.72, 18.46);
\end{scope}
\begin{scope}
\path[clip] (  0.00,  0.00) rectangle (289.08,198.74);
\definecolor{drawColor}{RGB}{0,0,0}

\node[text=drawColor,anchor=base,inner sep=0pt, outer sep=0pt, scale=  1.00] at ( 33.36,  5.69) {2010};

\node[text=drawColor,anchor=base,inner sep=0pt, outer sep=0pt, scale=  1.00] at ( 88.95,  5.69) {2011};

\node[text=drawColor,anchor=base,inner sep=0pt, outer sep=0pt, scale=  1.00] at (144.54,  5.69) {2012};

\node[text=drawColor,anchor=base,inner sep=0pt, outer sep=0pt, scale=  1.00] at (200.13,  5.69) {2013};

\node[text=drawColor,anchor=base,inner sep=0pt, outer sep=0pt, scale=  1.00] at (255.72,  5.69) {2014*};
\end{scope}
\begin{scope}
\path[clip] (  0.00,  0.00) rectangle (289.08,198.74);
\coordinate (apoyo1) at (25.51,188.79);
\coordinate (apoyo2) at (117.41,188.79);
\coordinate (apoyo3) at (217.03,188.79);
\coordinate (longitudFicticia) at (7.11,9.95);
\coordinate (longitud) at (7.11,7.11);
\coordinate (desY) at (0,1.42);
\definecolor[named]{ct1}{HTML}{
0000FF
}
\definecolor[named]{ct2}{HTML}{
9DBBFF
}
\definecolor[named]{ct3}{HTML}{
C8C8C8
}
\definecolor[named]{ctb1}{HTML}{
0000FF
}
\definecolor[named]{ctb2}{HTML}{
9DBBFF
}
\definecolor[named]{ctb3}{HTML}{
C8C8C8
}
\path [fill=none] (apoyo1) rectangle ($(apoyo1)+(longitudFicticia)$)
node [xshift=0.3cm,inner sep=0pt, outer sep=0pt,text width=1.33333333333333in,midway,right,scale = 0.9]{Exportación};
\draw [color= ctb1, fill=ct1] ( $(apoyo1)  + (desY) $) rectangle ($(apoyo1)+ (desY) +(longitud)$);
\path [fill=none] (apoyo2) rectangle ($(apoyo2)+(longitudFicticia)$)
node [xshift=0.3cm,inner sep=0pt, outer sep=0pt,text width=1.33333333333333in,midway,right,scale = 0.9]{Importación};
\draw [color = ctb2, fill=ct2] ( $(apoyo2)  + (desY) $) rectangle ($(apoyo2)+ (desY) +(longitud)$);
\path [fill=none] (apoyo3) rectangle ($(apoyo3)+(longitudFicticia)$)
node [xshift=0.3cm,inner sep=0pt, outer sep=0pt,text width=1.33333333333333in,midway,right,scale = 0.9]{Balanza};
\path [color = ctb3, fill=ct3] ( $(apoyo3)  + (desY) $) rectangle ($(apoyo3)+ (desY) +(longitud)$);
\end{scope}
  \end{tikzpicture}}%
{%
	MAGA} %


%#########################14########################

\cajita{%
	Balanza comercial del arroz }%
{%
<<<<<<< HEAD
	La serie histórica de las importaciones y exportaciones de arroz, muestra que el 99\% del arroz que transita por las aduanas nacionales tiene fines de importación.
	
	En el 2014 se importaron 64,289 toneladas métricas de arroz, mientras que se exportaron 55.8 toneladas métricas. 
=======
>>>>>>> origin/master
}%
{%
	Exportaciones, importaciones y balanza comercial del arroz} %
{%
	República de Guatemala, serie histórica, en toneladas métricas } %
{%
	\begin{tikzpicture}[x=1pt,y=1pt]  % Created by tikzDevice version 0.9 on 2016-03-03 04:15:22
% !TEX encoding = UTF-8 Unicode
\definecolor{fillColor}{RGB}{255,255,255}
\path[use as bounding box,fill=fillColor,fill opacity=0.00] (0,0) rectangle (289.08,198.74);
\begin{scope}
\path[clip] (  0.00,  0.00) rectangle (289.08,198.74);

\path[] (  0.00,  0.00) rectangle (289.08,198.74);
\end{scope}
\begin{scope}
\path[clip] (  0.00,  0.00) rectangle (289.08,198.74);

\path[] (  0.00, 18.46) rectangle (289.08,144.39);

\path[] ( 33.36, 18.46) --
	( 33.36,144.39);

\path[] ( 88.95, 18.46) --
	( 88.95,144.39);

\path[] (144.54, 18.46) --
	(144.54,144.39);

\path[] (200.13, 18.46) --
	(200.13,144.39);

\path[] (255.72, 18.46) --
	(255.72,144.39);
\definecolor{drawColor}{RGB}{0,0,255}
\definecolor{fillColor}{RGB}{0,0,255}

\path[draw=drawColor,line width= 0.6pt,line join=round,fill=fillColor] (  9.27, 80.71) rectangle ( 24.09, 82.25);
\definecolor{drawColor}{RGB}{157,187,255}
\definecolor{fillColor}{RGB}{157,187,255}

\path[draw=drawColor,line width= 0.6pt,line join=round,fill=fillColor] ( 25.94, 80.71) rectangle ( 40.77,125.31);
\definecolor{drawColor}{RGB}{200,200,200}
\definecolor{fillColor}{RGB}{200,200,200}

\path[draw=drawColor,line width= 0.6pt,line join=round,fill=fillColor] ( 42.62, 37.65) rectangle ( 57.45, 80.71);
\definecolor{drawColor}{RGB}{0,0,255}
\definecolor{fillColor}{RGB}{0,0,255}

\path[draw=drawColor,line width= 0.6pt,line join=round,fill=fillColor] ( 64.86, 80.71) rectangle ( 79.68, 81.64);
\definecolor{drawColor}{RGB}{157,187,255}
\definecolor{fillColor}{RGB}{157,187,255}

\path[draw=drawColor,line width= 0.6pt,line join=round,fill=fillColor] ( 81.54, 80.71) rectangle ( 96.36,129.34);
\definecolor{drawColor}{RGB}{200,200,200}
\definecolor{fillColor}{RGB}{200,200,200}

\path[draw=drawColor,line width= 0.6pt,line join=round,fill=fillColor] ( 98.21, 33.00) rectangle (113.04, 80.71);
\definecolor{drawColor}{RGB}{0,0,255}
\definecolor{fillColor}{RGB}{0,0,255}

\path[draw=drawColor,line width= 0.6pt,line join=round,fill=fillColor] (120.45, 80.71) rectangle (135.27, 82.13);
\definecolor{drawColor}{RGB}{157,187,255}
\definecolor{fillColor}{RGB}{157,187,255}

\path[draw=drawColor,line width= 0.6pt,line join=round,fill=fillColor] (137.13, 80.71) rectangle (151.95,144.39);
\definecolor{drawColor}{RGB}{200,200,200}
\definecolor{fillColor}{RGB}{200,200,200}

\path[draw=drawColor,line width= 0.6pt,line join=round,fill=fillColor] (153.81, 18.46) rectangle (168.63, 80.71);
\definecolor{drawColor}{RGB}{0,0,255}
\definecolor{fillColor}{RGB}{0,0,255}

\path[draw=drawColor,line width= 0.6pt,line join=round,fill=fillColor] (176.04, 80.71) rectangle (190.87, 81.02);
\definecolor{drawColor}{RGB}{157,187,255}
\definecolor{fillColor}{RGB}{157,187,255}

\path[draw=drawColor,line width= 0.6pt,line join=round,fill=fillColor] (192.72, 80.71) rectangle (207.54,142.14);
\definecolor{drawColor}{RGB}{200,200,200}
\definecolor{fillColor}{RGB}{200,200,200}

\path[draw=drawColor,line width= 0.6pt,line join=round,fill=fillColor] (209.40, 19.59) rectangle (224.22, 80.71);
\definecolor{drawColor}{RGB}{0,0,255}
\definecolor{fillColor}{RGB}{0,0,255}

\path[draw=drawColor,line width= 0.6pt,line join=round,fill=fillColor] (231.63, 80.71) rectangle (246.46, 80.75);
\definecolor{drawColor}{RGB}{157,187,255}
\definecolor{fillColor}{RGB}{157,187,255}

\path[draw=drawColor,line width= 0.6pt,line join=round,fill=fillColor] (248.31, 80.71) rectangle (263.14,121.07);
\definecolor{drawColor}{RGB}{200,200,200}
\definecolor{fillColor}{RGB}{200,200,200}

\path[draw=drawColor,line width= 0.6pt,line join=round,fill=fillColor] (264.99, 40.39) rectangle (279.81, 80.71);
\definecolor{drawColor}{RGB}{0,0,0}

\path[draw=drawColor,line width= 0.6pt,line join=round] (  0.00, 80.71) -- (289.08, 80.71);

\node[text=drawColor,rotate= 90.00,anchor=base west,inner sep=0pt, outer sep=0pt, scale=  0.83] at ( 19.91, 86.62) {2,445.7};

\node[text=drawColor,rotate= 90.00,anchor=base west,inner sep=0pt, outer sep=0pt, scale=  0.83] at ( 36.59,130.41) {71,041.9};

\node[text=drawColor,rotate= 90.00,anchor=base west,inner sep=0pt, outer sep=0pt, scale=  0.83] at ( 53.27, 43.20) {-68,596.3};

\node[text=drawColor,rotate= 90.00,anchor=base west,inner sep=0pt, outer sep=0pt, scale=  0.83] at ( 75.50, 86.01) {1,472.0};

\node[text=drawColor,rotate= 90.00,anchor=base west,inner sep=0pt, outer sep=0pt, scale=  0.83] at ( 92.18,134.44) {77,464.0};

\node[text=drawColor,rotate= 90.00,anchor=base west,inner sep=0pt, outer sep=0pt, scale=  0.83] at (108.86, 38.56) {-75,992.0};

\node[text=drawColor,rotate= 90.00,anchor=base west,inner sep=0pt, outer sep=0pt, scale=  0.83] at (131.10, 86.50) {2,261.7};

\node[text=drawColor,rotate= 90.00,anchor=base west,inner sep=0pt, outer sep=0pt, scale=  0.83] at (147.77,150.21) {101,424.3};

\node[text=drawColor,rotate= 90.00,anchor=base west,inner sep=0pt, outer sep=0pt, scale=  0.83] at (164.45, 24.01) {-99,162.6};

\node[text=drawColor,rotate= 90.00,anchor=base west,inner sep=0pt, outer sep=0pt, scale=  0.83] at (186.69, 84.29) {490.6};

\node[text=drawColor,rotate= 90.00,anchor=base west,inner sep=0pt, outer sep=0pt, scale=  0.83] at (203.37,147.24) {97,845.5};

\node[text=drawColor,rotate= 90.00,anchor=base west,inner sep=0pt, outer sep=0pt, scale=  0.83] at (220.04, 25.15) {-97,354.9};

\node[text=drawColor,rotate= 90.00,anchor=base west,inner sep=0pt, outer sep=0pt, scale=  0.83] at (242.28, 83.29) {55.8};

\node[text=drawColor,rotate= 90.00,anchor=base west,inner sep=0pt, outer sep=0pt, scale=  0.83] at (258.96,126.17) {64,289.8};

\node[text=drawColor,rotate= 90.00,anchor=base west,inner sep=0pt, outer sep=0pt, scale=  0.83] at (275.64, 45.94) {-64,234.0};

\path[] (  0.00, 18.46) rectangle (289.08,144.39);
\end{scope}
\begin{scope}
\path[clip] (  0.00,  0.00) rectangle (289.08,198.74);

\path[] (  0.00, 18.46) --
	(289.08, 18.46);
\end{scope}
\begin{scope}
\path[clip] (  0.00,  0.00) rectangle (289.08,198.74);

\path[] ( 33.36, 15.71) --
	( 33.36, 18.46);

\path[] ( 88.95, 15.71) --
	( 88.95, 18.46);

\path[] (144.54, 15.71) --
	(144.54, 18.46);

\path[] (200.13, 15.71) --
	(200.13, 18.46);

\path[] (255.72, 15.71) --
	(255.72, 18.46);
\end{scope}
\begin{scope}
\path[clip] (  0.00,  0.00) rectangle (289.08,198.74);
\definecolor{drawColor}{RGB}{0,0,0}

\node[text=drawColor,anchor=base,inner sep=0pt, outer sep=0pt, scale=  1.00] at ( 33.36,  5.69) {2010};

\node[text=drawColor,anchor=base,inner sep=0pt, outer sep=0pt, scale=  1.00] at ( 88.95,  5.69) {2011};

\node[text=drawColor,anchor=base,inner sep=0pt, outer sep=0pt, scale=  1.00] at (144.54,  5.69) {2012};

\node[text=drawColor,anchor=base,inner sep=0pt, outer sep=0pt, scale=  1.00] at (200.13,  5.69) {2013};

\node[text=drawColor,anchor=base,inner sep=0pt, outer sep=0pt, scale=  1.00] at (255.72,  5.69) {2014*};
\end{scope}
\begin{scope}
\path[clip] (  0.00,  0.00) rectangle (289.08,198.74);
\coordinate (apoyo1) at (25.51,188.79);
\coordinate (apoyo2) at (117.41,188.79);
\coordinate (apoyo3) at (217.03,188.79);
\coordinate (longitudFicticia) at (7.11,9.95);
\coordinate (longitud) at (7.11,7.11);
\coordinate (desY) at (0,1.42);
\definecolor[named]{ct1}{HTML}{
0000FF
}
\definecolor[named]{ct2}{HTML}{
9DBBFF
}
\definecolor[named]{ct3}{HTML}{
C8C8C8
}
\definecolor[named]{ctb1}{HTML}{
0000FF
}
\definecolor[named]{ctb2}{HTML}{
9DBBFF
}
\definecolor[named]{ctb3}{HTML}{
C8C8C8
}
\path [fill=none] (apoyo1) rectangle ($(apoyo1)+(longitudFicticia)$)
node [xshift=0.3cm,inner sep=0pt, outer sep=0pt,text width=1.33333333333333in,midway,right,scale = 0.9]{Exportación};
\draw [color= ctb1, fill=ct1] ( $(apoyo1)  + (desY) $) rectangle ($(apoyo1)+ (desY) +(longitud)$);
\path [fill=none] (apoyo2) rectangle ($(apoyo2)+(longitudFicticia)$)
node [xshift=0.3cm,inner sep=0pt, outer sep=0pt,text width=1.33333333333333in,midway,right,scale = 0.9]{Importación};
\draw [color = ctb2, fill=ct2] ( $(apoyo2)  + (desY) $) rectangle ($(apoyo2)+ (desY) +(longitud)$);
\path [fill=none] (apoyo3) rectangle ($(apoyo3)+(longitudFicticia)$)
node [xshift=0.3cm,inner sep=0pt, outer sep=0pt,text width=1.33333333333333in,midway,right,scale = 0.9]{Balanza};
\path [color = ctb3, fill=ct3] ( $(apoyo3)  + (desY) $) rectangle ($(apoyo3)+ (desY) +(longitud)$);
\end{scope}
  \end{tikzpicture}}%
{%
	MAGA} %

%#########################15########################

\cajita{%
	Balanza comercial del trigo }%
{%
<<<<<<< HEAD
	En relación al comercio internacional de productos, el trigo tuvo una balanza negativa en el período 2010-2014, esto significa que la cantidad de producto importada fue mayor que la exportada.
	
	En el 2014 se exportaron 706 toneladas métricas de este producto mientras que se importaron 364.7 miles de toneladas métricas.
=======
>>>>>>> origin/master
}%
{%
	Exportaciones, importaciones y balanza comercial del trigo} %
{%
	República de Guatemala, serie histórica, en toneladas métricas } %
{%
	\begin{tikzpicture}[x=1pt,y=1pt]  % Created by tikzDevice version 0.9 on 2016-03-03 04:15:25
% !TEX encoding = UTF-8 Unicode
\definecolor{fillColor}{RGB}{255,255,255}
\path[use as bounding box,fill=fillColor,fill opacity=0.00] (0,0) rectangle (289.08,198.74);
\begin{scope}
\path[clip] (  0.00,  0.00) rectangle (289.08,198.74);

\path[] (  0.00,  0.00) rectangle (289.08,198.74);
\end{scope}
\begin{scope}
\path[clip] (  0.00,  0.00) rectangle (289.08,198.74);

\path[] (  0.00, 18.46) rectangle (289.08,144.39);

\path[] ( 33.36, 18.46) --
	( 33.36,144.39);

\path[] ( 88.95, 18.46) --
	( 88.95,144.39);

\path[] (144.54, 18.46) --
	(144.54,144.39);

\path[] (200.13, 18.46) --
	(200.13,144.39);

\path[] (255.72, 18.46) --
	(255.72,144.39);
\definecolor{drawColor}{RGB}{0,0,255}
\definecolor{fillColor}{RGB}{0,0,255}

\path[draw=drawColor,line width= 0.6pt,line join=round,fill=fillColor] (  9.27, 81.41) rectangle ( 24.09, 81.41);
\definecolor{drawColor}{RGB}{157,187,255}
\definecolor{fillColor}{RGB}{157,187,255}

\path[draw=drawColor,line width= 0.6pt,line join=round,fill=fillColor] ( 25.94, 81.41) rectangle ( 40.77,141.39);
\definecolor{drawColor}{RGB}{200,200,200}
\definecolor{fillColor}{RGB}{200,200,200}

\path[draw=drawColor,line width= 0.6pt,line join=round,fill=fillColor] ( 42.62, 21.42) rectangle ( 57.45, 81.41);
\definecolor{drawColor}{RGB}{0,0,255}
\definecolor{fillColor}{RGB}{0,0,255}

\path[draw=drawColor,line width= 0.6pt,line join=round,fill=fillColor] ( 64.86, 81.41) rectangle ( 79.68, 81.43);
\definecolor{drawColor}{RGB}{157,187,255}
\definecolor{fillColor}{RGB}{157,187,255}

\path[draw=drawColor,line width= 0.6pt,line join=round,fill=fillColor] ( 81.54, 81.41) rectangle ( 96.36,144.39);
\definecolor{drawColor}{RGB}{200,200,200}
\definecolor{fillColor}{RGB}{200,200,200}

\path[draw=drawColor,line width= 0.6pt,line join=round,fill=fillColor] ( 98.21, 18.46) rectangle (113.04, 81.41);
\definecolor{drawColor}{RGB}{0,0,255}
\definecolor{fillColor}{RGB}{0,0,255}

\path[draw=drawColor,line width= 0.6pt,line join=round,fill=fillColor] (120.45, 81.41) rectangle (135.27, 81.44);
\definecolor{drawColor}{RGB}{157,187,255}
\definecolor{fillColor}{RGB}{157,187,255}

\path[draw=drawColor,line width= 0.6pt,line join=round,fill=fillColor] (137.13, 81.41) rectangle (151.95,144.09);
\definecolor{drawColor}{RGB}{200,200,200}
\definecolor{fillColor}{RGB}{200,200,200}

\path[draw=drawColor,line width= 0.6pt,line join=round,fill=fillColor] (153.81, 18.76) rectangle (168.63, 81.41);
\definecolor{drawColor}{RGB}{0,0,255}
\definecolor{fillColor}{RGB}{0,0,255}

\path[draw=drawColor,line width= 0.6pt,line join=round,fill=fillColor] (176.04, 81.41) rectangle (190.87, 81.54);
\definecolor{drawColor}{RGB}{157,187,255}
\definecolor{fillColor}{RGB}{157,187,255}

\path[draw=drawColor,line width= 0.6pt,line join=round,fill=fillColor] (192.72, 81.41) rectangle (207.54,137.79);
\definecolor{drawColor}{RGB}{200,200,200}
\definecolor{fillColor}{RGB}{200,200,200}

\path[draw=drawColor,line width= 0.6pt,line join=round,fill=fillColor] (209.40, 25.16) rectangle (224.22, 81.41);
\definecolor{drawColor}{RGB}{0,0,255}
\definecolor{fillColor}{RGB}{0,0,255}

\path[draw=drawColor,line width= 0.6pt,line join=round,fill=fillColor] (231.63, 81.41) rectangle (246.46, 81.50);
\definecolor{drawColor}{RGB}{157,187,255}
\definecolor{fillColor}{RGB}{157,187,255}

\path[draw=drawColor,line width= 0.6pt,line join=round,fill=fillColor] (248.31, 81.41) rectangle (263.14,125.84);
\definecolor{drawColor}{RGB}{200,200,200}
\definecolor{fillColor}{RGB}{200,200,200}

\path[draw=drawColor,line width= 0.6pt,line join=round,fill=fillColor] (264.99, 37.06) rectangle (279.81, 81.41);
\definecolor{drawColor}{RGB}{0,0,0}

\path[draw=drawColor,line width= 0.6pt,line join=round] (  0.00, 81.41) -- (289.08, 81.41);

\node[text=drawColor,rotate= 90.00,anchor=base west,inner sep=0pt, outer sep=0pt, scale=  0.83] at ( 19.91, 83.23) {3.0};

\node[text=drawColor,rotate= 90.00,anchor=base west,inner sep=0pt, outer sep=0pt, scale=  0.83] at ( 36.59,147.22) {492,354.0};

\node[text=drawColor,rotate= 90.00,anchor=base west,inner sep=0pt, outer sep=0pt, scale=  0.83] at ( 53.27, 27.70) {-492,351.0};

\node[text=drawColor,rotate= 90.00,anchor=base west,inner sep=0pt, outer sep=0pt, scale=  0.83] at ( 75.50, 84.71) {201.0};

\node[text=drawColor,rotate= 90.00,anchor=base west,inner sep=0pt, outer sep=0pt, scale=  0.83] at ( 92.18,150.21) {516,907.0};

\node[text=drawColor,rotate= 90.00,anchor=base west,inner sep=0pt, outer sep=0pt, scale=  0.83] at (108.86, 24.74) {-516,706.0};

\node[text=drawColor,rotate= 90.00,anchor=base west,inner sep=0pt, outer sep=0pt, scale=  0.83] at (131.10, 84.71) {252.9};

\node[text=drawColor,rotate= 90.00,anchor=base west,inner sep=0pt, outer sep=0pt, scale=  0.83] at (147.77,149.91) {514,445.6};

\node[text=drawColor,rotate= 90.00,anchor=base west,inner sep=0pt, outer sep=0pt, scale=  0.83] at (164.45, 25.04) {-514,192.7};

\node[text=drawColor,rotate= 90.00,anchor=base west,inner sep=0pt, outer sep=0pt, scale=  0.83] at (186.69, 85.91) {1,092.6};

\node[text=drawColor,rotate= 90.00,anchor=base west,inner sep=0pt, outer sep=0pt, scale=  0.83] at (203.37,143.61) {462,758.5};

\node[text=drawColor,rotate= 90.00,anchor=base west,inner sep=0pt, outer sep=0pt, scale=  0.83] at (220.04, 31.44) {-461,665.9};

\node[text=drawColor,rotate= 90.00,anchor=base west,inner sep=0pt, outer sep=0pt, scale=  0.83] at (242.28, 84.77) {706.0};

\node[text=drawColor,rotate= 90.00,anchor=base west,inner sep=0pt, outer sep=0pt, scale=  0.83] at (258.96,131.66) {364,685.9};

\node[text=drawColor,rotate= 90.00,anchor=base west,inner sep=0pt, outer sep=0pt, scale=  0.83] at (275.64, 43.34) {-363,980.0};

\path[] (  0.00, 18.46) rectangle (289.08,144.39);
\end{scope}
\begin{scope}
\path[clip] (  0.00,  0.00) rectangle (289.08,198.74);

\path[] (  0.00, 18.46) --
	(289.08, 18.46);
\end{scope}
\begin{scope}
\path[clip] (  0.00,  0.00) rectangle (289.08,198.74);

\path[] ( 33.36, 15.71) --
	( 33.36, 18.46);

\path[] ( 88.95, 15.71) --
	( 88.95, 18.46);

\path[] (144.54, 15.71) --
	(144.54, 18.46);

\path[] (200.13, 15.71) --
	(200.13, 18.46);

\path[] (255.72, 15.71) --
	(255.72, 18.46);
\end{scope}
\begin{scope}
\path[clip] (  0.00,  0.00) rectangle (289.08,198.74);
\definecolor{drawColor}{RGB}{0,0,0}

\node[text=drawColor,anchor=base,inner sep=0pt, outer sep=0pt, scale=  1.00] at ( 33.36,  5.69) {2010};

\node[text=drawColor,anchor=base,inner sep=0pt, outer sep=0pt, scale=  1.00] at ( 88.95,  5.69) {2011};

\node[text=drawColor,anchor=base,inner sep=0pt, outer sep=0pt, scale=  1.00] at (144.54,  5.69) {2012};

\node[text=drawColor,anchor=base,inner sep=0pt, outer sep=0pt, scale=  1.00] at (200.13,  5.69) {2013};

\node[text=drawColor,anchor=base,inner sep=0pt, outer sep=0pt, scale=  1.00] at (255.72,  5.69) {2014*};
\end{scope}
\begin{scope}
\path[clip] (  0.00,  0.00) rectangle (289.08,198.74);
\coordinate (apoyo1) at (25.51,188.79);
\coordinate (apoyo2) at (117.41,188.79);
\coordinate (apoyo3) at (217.03,188.79);
\coordinate (longitudFicticia) at (7.11,9.95);
\coordinate (longitud) at (7.11,7.11);
\coordinate (desY) at (0,1.42);
\definecolor[named]{ct1}{HTML}{
0000FF
}
\definecolor[named]{ct2}{HTML}{
9DBBFF
}
\definecolor[named]{ct3}{HTML}{
C8C8C8
}
\definecolor[named]{ctb1}{HTML}{
0000FF
}
\definecolor[named]{ctb2}{HTML}{
9DBBFF
}
\definecolor[named]{ctb3}{HTML}{
C8C8C8
}
\path [fill=none] (apoyo1) rectangle ($(apoyo1)+(longitudFicticia)$)
node [xshift=0.3cm,inner sep=0pt, outer sep=0pt,text width=1.33333333333333in,midway,right,scale = 0.9]{Exportación};
\draw [color= ctb1, fill=ct1] ( $(apoyo1)  + (desY) $) rectangle ($(apoyo1)+ (desY) +(longitud)$);
\path [fill=none] (apoyo2) rectangle ($(apoyo2)+(longitudFicticia)$)
node [xshift=0.3cm,inner sep=0pt, outer sep=0pt,text width=1.33333333333333in,midway,right,scale = 0.9]{Importación};
\draw [color = ctb2, fill=ct2] ( $(apoyo2)  + (desY) $) rectangle ($(apoyo2)+ (desY) +(longitud)$);
\path [fill=none] (apoyo3) rectangle ($(apoyo3)+(longitudFicticia)$)
node [xshift=0.3cm,inner sep=0pt, outer sep=0pt,text width=1.33333333333333in,midway,right,scale = 0.9]{Balanza};
\path [color = ctb3, fill=ct3] ( $(apoyo3)  + (desY) $) rectangle ($(apoyo3)+ (desY) +(longitud)$);
\end{scope}
  \end{tikzpicture}}%
{%
	Diplan-MAGA con datos de Banguat (MAGA, 2013).} %

%#########################16########################

\cajita{%
	Balanza comercial del ajonjolí }%
{%
<<<<<<< HEAD
	La serie histórica de las importaciones y exportaciones de ajonjolí, muestra que la balanza comercial es positiva, excepto en el 2011, donde se importaron 835 toneladas métricas más de las que se exportaron.
	
	En el 2014 se importaron 14.6 mil toneladas métricas de arroz, mientras que se exportaron 17.8 mil toneladas métricas. 
=======
>>>>>>> origin/master
}%
{%
	Exportaciones, importaciones y balanza comercial del ajonjolí} %
{%
	República de Guatemala, serie histórica, en toneladas métricas } %
{%
	\begin{tikzpicture}[x=1pt,y=1pt]  % Created by tikzDevice version 0.9 on 2016-03-03 04:15:29
% !TEX encoding = UTF-8 Unicode
\definecolor{fillColor}{RGB}{255,255,255}
\path[use as bounding box,fill=fillColor,fill opacity=0.00] (0,0) rectangle (289.08,198.74);
\begin{scope}
\path[clip] (  0.00,  0.00) rectangle (289.08,198.74);

\path[] (  0.00,  0.00) rectangle (289.08,198.74);
\end{scope}
\begin{scope}
\path[clip] (  0.00,  0.00) rectangle (289.08,198.74);

\path[] (  0.00, 18.46) rectangle (289.08,148.78);

\path[] ( 33.36, 18.46) --
	( 33.36,148.78);

\path[] ( 88.95, 18.46) --
	( 88.95,148.78);

\path[] (144.54, 18.46) --
	(144.54,148.78);

\path[] (200.13, 18.46) --
	(200.13,148.78);

\path[] (255.72, 18.46) --
	(255.72,148.78);
\definecolor{drawColor}{RGB}{0,0,255}
\definecolor{fillColor}{RGB}{0,0,255}

\path[draw=drawColor,line width= 0.6pt,line join=round,fill=fillColor] (  9.27, 21.57) rectangle ( 24.09,107.97);
\definecolor{drawColor}{RGB}{157,187,255}
\definecolor{fillColor}{RGB}{157,187,255}

\path[draw=drawColor,line width= 0.6pt,line join=round,fill=fillColor] ( 25.94, 21.57) rectangle ( 40.77, 54.29);
\definecolor{drawColor}{RGB}{200,200,200}
\definecolor{fillColor}{RGB}{200,200,200}

\path[draw=drawColor,line width= 0.6pt,line join=round,fill=fillColor] ( 42.62, 21.57) rectangle ( 57.45, 75.25);
\definecolor{drawColor}{RGB}{0,0,255}
\definecolor{fillColor}{RGB}{0,0,255}

\path[draw=drawColor,line width= 0.6pt,line join=round,fill=fillColor] ( 64.86, 21.57) rectangle ( 79.68, 88.68);
\definecolor{drawColor}{RGB}{157,187,255}
\definecolor{fillColor}{RGB}{157,187,255}

\path[draw=drawColor,line width= 0.6pt,line join=round,fill=fillColor] ( 81.54, 21.57) rectangle ( 96.36, 91.80);
\definecolor{drawColor}{RGB}{200,200,200}
\definecolor{fillColor}{RGB}{200,200,200}

\path[draw=drawColor,line width= 0.6pt,line join=round,fill=fillColor] ( 98.21, 18.46) rectangle (113.04, 21.57);
\definecolor{drawColor}{RGB}{0,0,255}
\definecolor{fillColor}{RGB}{0,0,255}

\path[draw=drawColor,line width= 0.6pt,line join=round,fill=fillColor] (120.45, 21.57) rectangle (135.27,114.19);
\definecolor{drawColor}{RGB}{157,187,255}
\definecolor{fillColor}{RGB}{157,187,255}

\path[draw=drawColor,line width= 0.6pt,line join=round,fill=fillColor] (137.13, 21.57) rectangle (151.95, 56.22);
\definecolor{drawColor}{RGB}{200,200,200}
\definecolor{fillColor}{RGB}{200,200,200}

\path[draw=drawColor,line width= 0.6pt,line join=round,fill=fillColor] (153.81, 21.57) rectangle (168.63, 79.55);
\definecolor{drawColor}{RGB}{0,0,255}
\definecolor{fillColor}{RGB}{0,0,255}

\path[draw=drawColor,line width= 0.6pt,line join=round,fill=fillColor] (176.04, 21.57) rectangle (190.87,148.78);
\definecolor{drawColor}{RGB}{157,187,255}
\definecolor{fillColor}{RGB}{157,187,255}

\path[draw=drawColor,line width= 0.6pt,line join=round,fill=fillColor] (192.72, 21.57) rectangle (207.54, 62.86);
\definecolor{drawColor}{RGB}{200,200,200}
\definecolor{fillColor}{RGB}{200,200,200}

\path[draw=drawColor,line width= 0.6pt,line join=round,fill=fillColor] (209.40, 21.57) rectangle (224.22,107.49);
\definecolor{drawColor}{RGB}{0,0,255}
\definecolor{fillColor}{RGB}{0,0,255}

\path[draw=drawColor,line width= 0.6pt,line join=round,fill=fillColor] (231.63, 21.57) rectangle (246.46, 88.01);
\definecolor{drawColor}{RGB}{157,187,255}
\definecolor{fillColor}{RGB}{157,187,255}

\path[draw=drawColor,line width= 0.6pt,line join=round,fill=fillColor] (248.31, 21.57) rectangle (263.14, 76.24);
\definecolor{drawColor}{RGB}{200,200,200}
\definecolor{fillColor}{RGB}{200,200,200}

\path[draw=drawColor,line width= 0.6pt,line join=round,fill=fillColor] (264.99, 21.57) rectangle (279.81, 33.35);
\definecolor{drawColor}{RGB}{0,0,0}

\path[draw=drawColor,line width= 0.6pt,line join=round] (  0.00, 21.57) -- (289.08, 21.57);

\node[text=drawColor,rotate= 90.00,anchor=base west,inner sep=0pt, outer sep=0pt, scale=  0.83] at ( 19.91,113.06) {23,143.9};

\node[text=drawColor,rotate= 90.00,anchor=base west,inner sep=0pt, outer sep=0pt, scale=  0.83] at ( 36.59, 58.66) {8,764.1};

\node[text=drawColor,rotate= 90.00,anchor=base west,inner sep=0pt, outer sep=0pt, scale=  0.83] at ( 53.27, 80.35) {14,379.8};

\node[text=drawColor,rotate= 90.00,anchor=base west,inner sep=0pt, outer sep=0pt, scale=  0.83] at ( 75.50, 93.78) {17,977.0};

\node[text=drawColor,rotate= 90.00,anchor=base west,inner sep=0pt, outer sep=0pt, scale=  0.83] at ( 92.18, 96.89) {18,812.0};

\node[text=drawColor,rotate= 90.00,anchor=base west,inner sep=0pt, outer sep=0pt, scale=  0.83] at (108.86, 22.19) {-835.0};

\node[text=drawColor,rotate= 90.00,anchor=base west,inner sep=0pt, outer sep=0pt, scale=  0.83] at (131.10,119.29) {24,812.0};

\node[text=drawColor,rotate= 90.00,anchor=base west,inner sep=0pt, outer sep=0pt, scale=  0.83] at (147.77, 60.59) {9,282.0};

\node[text=drawColor,rotate= 90.00,anchor=base west,inner sep=0pt, outer sep=0pt, scale=  0.83] at (164.45, 84.64) {15,530.1};

\node[text=drawColor,rotate= 90.00,anchor=base west,inner sep=0pt, outer sep=0pt, scale=  0.83] at (186.69,153.88) {34,078.0};

\node[text=drawColor,rotate= 90.00,anchor=base west,inner sep=0pt, outer sep=0pt, scale=  0.83] at (203.37, 67.96) {11,061.2};

\node[text=drawColor,rotate= 90.00,anchor=base west,inner sep=0pt, outer sep=0pt, scale=  0.83] at (220.04,112.59) {23,016.9};

\node[text=drawColor,rotate= 90.00,anchor=base west,inner sep=0pt, outer sep=0pt, scale=  0.83] at (242.28, 93.11) {17,799.0};

\node[text=drawColor,rotate= 90.00,anchor=base west,inner sep=0pt, outer sep=0pt, scale=  0.83] at (258.96, 81.34) {14,645.6};

\node[text=drawColor,rotate= 90.00,anchor=base west,inner sep=0pt, outer sep=0pt, scale=  0.83] at (275.64, 37.72) {3,153.4};

\path[] (  0.00, 18.46) rectangle (289.08,148.78);
\end{scope}
\begin{scope}
\path[clip] (  0.00,  0.00) rectangle (289.08,198.74);

\path[] (  0.00, 18.46) --
	(289.08, 18.46);
\end{scope}
\begin{scope}
\path[clip] (  0.00,  0.00) rectangle (289.08,198.74);

\path[] ( 33.36, 15.71) --
	( 33.36, 18.46);

\path[] ( 88.95, 15.71) --
	( 88.95, 18.46);

\path[] (144.54, 15.71) --
	(144.54, 18.46);

\path[] (200.13, 15.71) --
	(200.13, 18.46);

\path[] (255.72, 15.71) --
	(255.72, 18.46);
\end{scope}
\begin{scope}
\path[clip] (  0.00,  0.00) rectangle (289.08,198.74);
\definecolor{drawColor}{RGB}{0,0,0}

\node[text=drawColor,anchor=base,inner sep=0pt, outer sep=0pt, scale=  1.00] at ( 33.36,  5.69) {2010};

\node[text=drawColor,anchor=base,inner sep=0pt, outer sep=0pt, scale=  1.00] at ( 88.95,  5.69) {2011};

\node[text=drawColor,anchor=base,inner sep=0pt, outer sep=0pt, scale=  1.00] at (144.54,  5.69) {2012};

\node[text=drawColor,anchor=base,inner sep=0pt, outer sep=0pt, scale=  1.00] at (200.13,  5.69) {2013};

\node[text=drawColor,anchor=base,inner sep=0pt, outer sep=0pt, scale=  1.00] at (255.72,  5.69) {2014*};
\end{scope}
\begin{scope}
\path[clip] (  0.00,  0.00) rectangle (289.08,198.74);
\coordinate (apoyo1) at (25.51,188.79);
\coordinate (apoyo2) at (117.41,188.79);
\coordinate (apoyo3) at (217.03,188.79);
\coordinate (longitudFicticia) at (7.11,9.95);
\coordinate (longitud) at (7.11,7.11);
\coordinate (desY) at (0,1.42);
\definecolor[named]{ct1}{HTML}{
0000FF
}
\definecolor[named]{ct2}{HTML}{
9DBBFF
}
\definecolor[named]{ct3}{HTML}{
C8C8C8
}
\definecolor[named]{ctb1}{HTML}{
0000FF
}
\definecolor[named]{ctb2}{HTML}{
9DBBFF
}
\definecolor[named]{ctb3}{HTML}{
C8C8C8
}
\path [fill=none] (apoyo1) rectangle ($(apoyo1)+(longitudFicticia)$)
node [xshift=0.3cm,inner sep=0pt, outer sep=0pt,text width=1.33333333333333in,midway,right,scale = 0.9]{Exportación};
\draw [color= ctb1, fill=ct1] ( $(apoyo1)  + (desY) $) rectangle ($(apoyo1)+ (desY) +(longitud)$);
\path [fill=none] (apoyo2) rectangle ($(apoyo2)+(longitudFicticia)$)
node [xshift=0.3cm,inner sep=0pt, outer sep=0pt,text width=1.33333333333333in,midway,right,scale = 0.9]{Importación};
\draw [color = ctb2, fill=ct2] ( $(apoyo2)  + (desY) $) rectangle ($(apoyo2)+ (desY) +(longitud)$);
\path [fill=none] (apoyo3) rectangle ($(apoyo3)+(longitudFicticia)$)
node [xshift=0.3cm,inner sep=0pt, outer sep=0pt,text width=1.33333333333333in,midway,right,scale = 0.9]{Balanza};
\path [color = ctb3, fill=ct3] ( $(apoyo3)  + (desY) $) rectangle ($(apoyo3)+ (desY) +(longitud)$);
\end{scope}
  \end{tikzpicture}}%
{%
	Diplan-MAGA con datos de Banguat (MAGA, 2013).} %

%#########################17########################

\cajita{%
<<<<<<< HEAD
	Disponibilidad de cereales }%
{%
La disponibilidad de alimentos, en el 2013, fue de 173.1 kilogramos de tortilla, 11.8 de pan y galleta, 10.4 de harina de trito y 9.9 kilogramos de harina de maíz.}%
{%
	Disponibilidad per cápita de cereales} %
{%
	República de Guatemala, año 2013, en kilogramos } %
=======
	Disponibilidad per cápita de cereales }%
{%
}%
{%
	Disponibilidad per cápita de cereales } %
{%
	República de Guatemala, 2013 , en kilogramos } %
>>>>>>> origin/master
{%
	\begin{tikzpicture}[x=1pt,y=1pt]  % Created by tikzDevice version 0.9 on 2016-03-03 04:15:33
% !TEX encoding = UTF-8 Unicode
\definecolor{fillColor}{RGB}{255,255,255}
\path[use as bounding box,fill=fillColor,fill opacity=0.00] (0,0) rectangle (289.08,198.74);
\begin{scope}
\path[clip] (  0.00,  0.00) rectangle (289.08,198.74);

\path[] (  0.00,  0.00) rectangle (289.08,198.74);
\end{scope}
\begin{scope}
\path[clip] (  0.00,  0.00) rectangle (289.08,198.74);

\path[] ( 68.75,  0.00) rectangle (262.69,198.74);

\path[] ( 68.75, 16.56) --
	(262.69, 16.56);

\path[] ( 68.75, 44.16) --
	(262.69, 44.16);

\path[] ( 68.75, 71.77) --
	(262.69, 71.77);

\path[] ( 68.75, 99.37) --
	(262.69, 99.37);

\path[] ( 68.75,126.97) --
	(262.69,126.97);

\path[] ( 68.75,154.58) --
	(262.69,154.58);

\path[] ( 68.75,182.18) --
	(262.69,182.18);
\definecolor{drawColor}{RGB}{0,0,255}
\definecolor{fillColor}{RGB}{0,0,255}

\path[draw=drawColor,line width= 0.6pt,line join=round,fill=fillColor] ( 68.75,  8.28) rectangle ( 70.21, 24.84);

\path[draw=drawColor,line width= 0.6pt,line join=round,fill=fillColor] ( 68.75, 35.88) rectangle ( 74.13, 52.45);

\path[draw=drawColor,line width= 0.6pt,line join=round,fill=fillColor] ( 68.75, 63.49) rectangle ( 70.21, 80.05);

\path[draw=drawColor,line width= 0.6pt,line join=round,fill=fillColor] ( 68.75, 91.09) rectangle ( 81.98,107.65);

\path[draw=drawColor,line width= 0.6pt,line join=round,fill=fillColor] ( 68.75,118.69) rectangle ( 80.41,135.26);

\path[draw=drawColor,line width= 0.6pt,line join=round,fill=fillColor] ( 68.75,146.30) rectangle (262.69,162.86);

\path[draw=drawColor,line width= 0.6pt,line join=round,fill=fillColor] ( 68.75,173.90) rectangle ( 79.85,190.46);
\definecolor{drawColor}{RGB}{0,0,0}

\path[draw=drawColor,line width= 0.1pt,line join=round] ( 68.75,  0.00) -- ( 68.75,198.74);

\node[text=drawColor,anchor=base west,inner sep=0pt, outer sep=0pt, scale=  1.02] at ( 72.45, 12.59) {1.3};

\node[text=drawColor,anchor=base west,inner sep=0pt, outer sep=0pt, scale=  1.02] at ( 76.37, 40.19) {4.8};

\node[text=drawColor,anchor=base west,inner sep=0pt, outer sep=0pt, scale=  1.02] at ( 72.45, 67.80) {1.3};

\node[text=drawColor,anchor=base west,inner sep=0pt, outer sep=0pt, scale=  1.02] at ( 85.10, 95.40) {11.8};

\node[text=drawColor,anchor=base west,inner sep=0pt, outer sep=0pt, scale=  1.02] at ( 83.53,123.00) {10.4};

\node[text=drawColor,anchor=base west,inner sep=0pt, outer sep=0pt, scale=  1.02] at (266.71,150.61) {173.1};

\node[text=drawColor,anchor=base west,inner sep=0pt, outer sep=0pt, scale=  1.02] at ( 82.08,178.21) {9.9};

\path[] ( 68.75,  0.00) rectangle (262.69,198.74);
\end{scope}
\begin{scope}
\path[clip] (  0.00,  0.00) rectangle (289.08,198.74);

\path[] ( 68.75,  0.00) --
	( 68.75,198.74);
\end{scope}
\begin{scope}
\path[clip] (  0.00,  0.00) rectangle (289.08,198.74);
\definecolor{drawColor}{RGB}{0,0,0}

\node[text=drawColor,anchor=base east,inner sep=0pt, outer sep=0pt, scale=  1.00] at ( 66.00, 12.65) {Tortilla (maicillo)};

\node[text=drawColor,anchor=base east,inner sep=0pt, outer sep=0pt, scale=  1.00] at ( 66.00, 40.26) {Arroz oro};

\node[text=drawColor,anchor=base east,inner sep=0pt, outer sep=0pt, scale=  1.00] at ( 66.00, 67.86) {Pastas alimenticias};

\node[text=drawColor,anchor=base east,inner sep=0pt, outer sep=0pt, scale=  1.00] at ( 66.00, 95.46) {Pan y galleta};

\node[text=drawColor,anchor=base east,inner sep=0pt, outer sep=0pt, scale=  1.00] at ( 66.00,123.07) {Harina de trigo};

\node[text=drawColor,anchor=base east,inner sep=0pt, outer sep=0pt, scale=  1.00] at ( 66.00,150.67) {Tortilla};

\node[text=drawColor,anchor=base east,inner sep=0pt, outer sep=0pt, scale=  1.00] at ( 66.00,178.27) {Harina de maíz};
\end{scope}
\begin{scope}
\path[clip] (  0.00,  0.00) rectangle (289.08,198.74);

\path[] ( 66.00, 16.56) --
	( 68.75, 16.56);

\path[] ( 66.00, 44.16) --
	( 68.75, 44.16);

\path[] ( 66.00, 71.77) --
	( 68.75, 71.77);

\path[] ( 66.00, 99.37) --
	( 68.75, 99.37);

\path[] ( 66.00,126.97) --
	( 68.75,126.97);

\path[] ( 66.00,154.58) --
	( 68.75,154.58);

\path[] ( 66.00,182.18) --
	( 68.75,182.18);
\end{scope}
  \end{tikzpicture}}%
{%
	INE y MAGA} %

%#########################18########################

\cajita{%
<<<<<<< HEAD
	Disponibilidad leguminosas, azúcares y/o tubérculos }%
{%
La disponibilidad de frijol fue de 11.8 kilogramos per cápita, en el año 2013.

En cuanto a los azúcares, se dispone de 35.7 kilogramos per cápita, de azúcar blanca y refinada.

En cuanto a los tubérculos, se disponía en el 2013 de 24.6 kilogramos per cápita.\textollamada[*]{Las leguminosas son fuente de fibra y vitaminas del complejo B. Contienen minerales como hierro y calcio.
	Las leguminosas son ricas en proteínas, pero se debe mezclar con productos derivados de cereales o granos para obtener una proteína completa.}}%
{%
	Disponibilidad per cápita de leguminosas, azúcares y/o tubérculos  } %
{%
	República de Guatemala, año 2013, en kilogramos } %
=======
	Disponibilidad per cápita de Leguminosas, azúcares yo tubérculos }%
{%
}%
{%
	Disponibilidad per cápita de Leguminosas, azúcares yo tubérculos  } %
{%
	República de Guatemala, 2013 , en kilogramos } %
>>>>>>> origin/master
{%
	\begin{tikzpicture}[x=1pt,y=1pt]  % Created by tikzDevice version 0.9 on 2016-03-03 04:15:34
% !TEX encoding = UTF-8 Unicode
\definecolor{fillColor}{RGB}{255,255,255}
\path[use as bounding box,fill=fillColor,fill opacity=0.00] (0,0) rectangle (289.08,198.74);
\begin{scope}
\path[clip] (  0.00,  0.00) rectangle (289.08,198.74);

\path[] (  0.00,  0.00) rectangle (289.08,198.74);
\end{scope}
\begin{scope}
\path[clip] (  0.00,  0.00) rectangle (289.08,198.74);

\path[] ( 87.72,  0.00) rectangle (267.09,198.74);

\path[] ( 87.72, 22.93) --
	(267.09, 22.93);

\path[] ( 87.72, 61.15) --
	(267.09, 61.15);

\path[] ( 87.72, 99.37) --
	(267.09, 99.37);

\path[] ( 87.72,137.59) --
	(267.09,137.59);

\path[] ( 87.72,175.81) --
	(267.09,175.81);
\definecolor{drawColor}{RGB}{0,0,255}
\definecolor{fillColor}{RGB}{0,0,255}

\path[draw=drawColor,line width= 0.6pt,line join=round,fill=fillColor] ( 87.72, 11.47) rectangle ( 89.23, 34.40);

\path[draw=drawColor,line width= 0.6pt,line join=round,fill=fillColor] ( 87.72, 49.69) rectangle (211.32, 72.62);

\path[draw=drawColor,line width= 0.6pt,line join=round,fill=fillColor] ( 87.72, 87.91) rectangle ( 97.77,110.84);

\path[draw=drawColor,line width= 0.6pt,line join=round,fill=fillColor] ( 87.72,126.13) rectangle (267.09,149.06);

\path[draw=drawColor,line width= 0.6pt,line join=round,fill=fillColor] ( 87.72,164.34) rectangle (147.01,187.28);
\definecolor{drawColor}{RGB}{0,0,0}

\path[draw=drawColor,line width= 0.1pt,line join=round] ( 87.72,  0.00) -- ( 87.72,198.74);

\node[text=drawColor,anchor=base west,inner sep=0pt, outer sep=0pt, scale=  1.02] at ( 91.46, 18.96) {0.3};

\node[text=drawColor,anchor=base west,inner sep=0pt, outer sep=0pt, scale=  1.02] at (214.44, 57.18) {24.6};

\node[text=drawColor,anchor=base west,inner sep=0pt, outer sep=0pt, scale=  1.02] at (100.01, 95.40) {2.0};

\node[text=drawColor,anchor=base west,inner sep=0pt, outer sep=0pt, scale=  1.02] at (270.21,133.62) {35.7};

\node[text=drawColor,anchor=base west,inner sep=0pt, outer sep=0pt, scale=  1.02] at (150.13,171.84) {11.8};

\path[] ( 87.72,  0.00) rectangle (267.09,198.74);
\end{scope}
\begin{scope}
\path[clip] (  0.00,  0.00) rectangle (289.08,198.74);

\path[] ( 87.72,  0.00) --
	( 87.72,198.74);
\end{scope}
\begin{scope}
\path[clip] (  0.00,  0.00) rectangle (289.08,198.74);
\definecolor{drawColor}{RGB}{0,0,0}

\node[text=drawColor,anchor=base east,inner sep=0pt, outer sep=0pt, scale=  1.00] at ( 84.97, 19.02) {Yuca};

\node[text=drawColor,anchor=base east,inner sep=0pt, outer sep=0pt, scale=  1.00] at ( 84.97, 57.24) {Papa};

\node[text=drawColor,anchor=base east,inner sep=0pt, outer sep=0pt, scale=  1.00] at ( 84.97, 95.46) {Materiales azucarados};

\node[text=drawColor,anchor=base east,inner sep=0pt, outer sep=0pt, scale=  1.00] at ( 84.97,133.68) {Azúcar blanca y refinada};

\node[text=drawColor,anchor=base east,inner sep=0pt, outer sep=0pt, scale=  1.00] at ( 84.97,171.90) {Frijoles};
\end{scope}
\begin{scope}
\path[clip] (  0.00,  0.00) rectangle (289.08,198.74);

\path[] ( 84.97, 22.93) --
	( 87.72, 22.93);

\path[] ( 84.97, 61.15) --
	( 87.72, 61.15);

\path[] ( 84.97, 99.37) --
	( 87.72, 99.37);

\path[] ( 84.97,137.59) --
	( 87.72,137.59);

\path[] ( 84.97,175.81) --
	( 87.72,175.81);
\end{scope}
  \end{tikzpicture}}%
{%
	INE y MAGA} %

%#########################19########################

\cajita{%
<<<<<<< HEAD
	Disponibilidad de hortalizas}%
{%
En el 2013 se contaba mayormente con disponibilidad de tomate (13.7 kilogramos per cápita), seguido de la cebolla (7.8), el güicoy (3.5) y chile pimiento (2.6).}%
{%
	Disponibilidad per cápita de hortalizas } %
{%
	República de Guatemala, año 2013, en kilogramos } %
=======
	Disponibilidad per cápita de Hortalizas}%
{%
}%
{%
	Disponibilidad per cápita de Hortalizas } %
{%
	República de Guatemala, 2013 , en kilogramos } %
>>>>>>> origin/master
{%
	\begin{tikzpicture}[x=1pt,y=1pt]  % Created by tikzDevice version 0.9 on 2016-03-03 04:15:34
% !TEX encoding = UTF-8 Unicode
\definecolor{fillColor}{RGB}{255,255,255}
\path[use as bounding box,fill=fillColor,fill opacity=0.00] (0,0) rectangle (289.08,198.74);
\begin{scope}
\path[clip] (  0.00,  0.00) rectangle (289.08,198.74);

\path[] (  0.00,  0.00) rectangle (289.08,198.74);
\end{scope}
\begin{scope}
\path[clip] (  0.00,  0.00) rectangle (289.08,198.74);

\path[] ( 56.96,  0.00) rectangle (267.09,198.74);

\path[] ( 56.96, 19.23) --
	(267.09, 19.23);

\path[] ( 56.96, 51.29) --
	(267.09, 51.29);

\path[] ( 56.96, 83.34) --
	(267.09, 83.34);

\path[] ( 56.96,115.40) --
	(267.09,115.40);

\path[] ( 56.96,147.45) --
	(267.09,147.45);

\path[] ( 56.96,179.51) --
	(267.09,179.51);
\definecolor{drawColor}{RGB}{0,0,255}
\definecolor{fillColor}{RGB}{0,0,255}

\path[draw=drawColor,line width= 0.6pt,line join=round,fill=fillColor] ( 56.96,  9.62) rectangle ( 96.84, 28.85);

\path[draw=drawColor,line width= 0.6pt,line join=round,fill=fillColor] ( 56.96, 41.67) rectangle (110.64, 60.90);

\path[draw=drawColor,line width= 0.6pt,line join=round,fill=fillColor] ( 56.96, 73.73) rectangle ( 96.84, 92.96);

\path[draw=drawColor,line width= 0.6pt,line join=round,fill=fillColor] ( 56.96,105.78) rectangle ( 90.70,125.02);

\path[draw=drawColor,line width= 0.6pt,line join=round,fill=fillColor] ( 56.96,137.84) rectangle (267.09,157.07);

\path[draw=drawColor,line width= 0.6pt,line join=round,fill=fillColor] ( 56.96,169.89) rectangle (176.59,189.13);
\definecolor{drawColor}{RGB}{0,0,0}

\path[draw=drawColor,line width= 0.1pt,line join=round] ( 56.96,  0.00) -- ( 56.96,198.74);

\node[text=drawColor,anchor=base west,inner sep=0pt, outer sep=0pt, scale=  1.02] at ( 99.08, 15.26) {2.6};

\node[text=drawColor,anchor=base west,inner sep=0pt, outer sep=0pt, scale=  1.02] at (112.88, 47.32) {3.5};

\node[text=drawColor,anchor=base west,inner sep=0pt, outer sep=0pt, scale=  1.02] at ( 99.08, 79.37) {2.6};

\node[text=drawColor,anchor=base west,inner sep=0pt, outer sep=0pt, scale=  1.02] at ( 92.94,111.43) {2.2};

\node[text=drawColor,anchor=base west,inner sep=0pt, outer sep=0pt, scale=  1.02] at (270.21,143.48) {13.7};

\node[text=drawColor,anchor=base west,inner sep=0pt, outer sep=0pt, scale=  1.02] at (178.83,175.54) {7.8};

\path[] ( 56.96,  0.00) rectangle (267.09,198.74);
\end{scope}
\begin{scope}
\path[clip] (  0.00,  0.00) rectangle (289.08,198.74);

\path[] ( 56.96,  0.00) --
	( 56.96,198.74);
\end{scope}
\begin{scope}
\path[clip] (  0.00,  0.00) rectangle (289.08,198.74);
\definecolor{drawColor}{RGB}{0,0,0}

\node[text=drawColor,anchor=base east,inner sep=0pt, outer sep=0pt, scale=  1.00] at ( 54.21, 15.32) {Otras hortalizas};

\node[text=drawColor,anchor=base east,inner sep=0pt, outer sep=0pt, scale=  1.00] at ( 54.21, 47.38) {Güicoy};

\node[text=drawColor,anchor=base east,inner sep=0pt, outer sep=0pt, scale=  1.00] at ( 54.21, 79.44) {Chile pimiento};

\node[text=drawColor,anchor=base east,inner sep=0pt, outer sep=0pt, scale=  1.00] at ( 54.21,111.49) {Zanahoria};

\node[text=drawColor,anchor=base east,inner sep=0pt, outer sep=0pt, scale=  1.00] at ( 54.21,143.55) {Tomate};

\node[text=drawColor,anchor=base east,inner sep=0pt, outer sep=0pt, scale=  1.00] at ( 54.21,175.60) {Cebolla};
\end{scope}
\begin{scope}
\path[clip] (  0.00,  0.00) rectangle (289.08,198.74);

\path[] ( 54.21, 19.23) --
	( 56.96, 19.23);

\path[] ( 54.21, 51.29) --
	( 56.96, 51.29);

\path[] ( 54.21, 83.34) --
	( 56.96, 83.34);

\path[] ( 54.21,115.40) --
	( 56.96,115.40);

\path[] ( 54.21,147.45) --
	( 56.96,147.45);

\path[] ( 54.21,179.51) --
	( 56.96,179.51);
\end{scope}
  \end{tikzpicture}}%
{%
	INE y MAGA} %

%#########################20########################

\cajita{%
<<<<<<< HEAD
	Disponibilidad de frutas}%
{%
En cuanto a las frutas, se disponía en el 2013 de 51.2 kilogramos por persona, de 18.1 kilogramos de cítricos, 11.2 kilogramos de piña y 5.6 kilogramos de melón.}%
{%
	Disponibilidad per cápita de frutas } %
{%
	República de Guatemala, año 2013, en kilogramos } %
=======
	Disponibilidad per cápita de Frutas}%
{%
}%
{%
	Disponibilidad per cápita de Frutas } %
{%
	República de Guatemala, 2013 , en kilogramos } %
>>>>>>> origin/master
{%
	\begin{tikzpicture}[x=1pt,y=1pt]  % Created by tikzDevice version 0.9 on 2016-03-03 04:15:36
% !TEX encoding = UTF-8 Unicode
\definecolor{fillColor}{RGB}{255,255,255}
\path[use as bounding box,fill=fillColor,fill opacity=0.00] (0,0) rectangle (289.08,198.74);
\begin{scope}
\path[clip] (  0.00,  0.00) rectangle (289.08,198.74);

\path[] (  0.00,  0.00) rectangle (289.08,198.74);
\end{scope}
\begin{scope}
\path[clip] (  0.00,  0.00) rectangle (289.08,198.74);

\path[] ( 43.64,  0.00) rectangle (267.09,198.74);

\path[] ( 43.64, 16.56) --
	(267.09, 16.56);

\path[] ( 43.64, 44.16) --
	(267.09, 44.16);

\path[] ( 43.64, 71.77) --
	(267.09, 71.77);

\path[] ( 43.64, 99.37) --
	(267.09, 99.37);

\path[] ( 43.64,126.97) --
	(267.09,126.97);

\path[] ( 43.64,154.58) --
	(267.09,154.58);

\path[] ( 43.64,182.18) --
	(267.09,182.18);
\definecolor{drawColor}{RGB}{0,0,255}
\definecolor{fillColor}{RGB}{0,0,255}

\path[draw=drawColor,line width= 0.6pt,line join=round,fill=fillColor] ( 43.64,  8.28) rectangle ( 70.69, 24.84);

\path[draw=drawColor,line width= 0.6pt,line join=round,fill=fillColor] ( 43.64, 35.88) rectangle ( 92.52, 52.45);

\path[draw=drawColor,line width= 0.6pt,line join=round,fill=fillColor] ( 43.64, 63.49) rectangle ( 68.08, 80.05);

\path[draw=drawColor,line width= 0.6pt,line join=round,fill=fillColor] ( 43.64, 91.09) rectangle ( 65.02,107.65);

\path[draw=drawColor,line width= 0.6pt,line join=round,fill=fillColor] ( 43.64,118.69) rectangle (122.63,135.26);

\path[draw=drawColor,line width= 0.6pt,line join=round,fill=fillColor] ( 43.64,146.30) rectangle (267.09,162.86);

\path[draw=drawColor,line width= 0.6pt,line join=round,fill=fillColor] ( 43.64,173.90) rectangle ( 51.93,190.46);
\definecolor{drawColor}{RGB}{0,0,0}

\path[draw=drawColor,line width= 0.1pt,line join=round] ( 43.64,  0.00) -- ( 43.64,198.74);

\node[text=drawColor,anchor=base west,inner sep=0pt, outer sep=0pt, scale=  1.02] at ( 72.93, 12.59) {6.2};

\node[text=drawColor,anchor=base west,inner sep=0pt, outer sep=0pt, scale=  1.02] at ( 95.64, 40.19) {11.2};

\node[text=drawColor,anchor=base west,inner sep=0pt, outer sep=0pt, scale=  1.02] at ( 70.31, 67.80) {5.6};

\node[text=drawColor,anchor=base west,inner sep=0pt, outer sep=0pt, scale=  1.02] at ( 67.26, 95.40) {4.9};

\node[text=drawColor,anchor=base west,inner sep=0pt, outer sep=0pt, scale=  1.02] at (125.76,123.00) {18.1};

\node[text=drawColor,anchor=base west,inner sep=0pt, outer sep=0pt, scale=  1.02] at (270.21,150.61) {51.2};

\node[text=drawColor,anchor=base west,inner sep=0pt, outer sep=0pt, scale=  1.02] at ( 54.17,178.21) {1.9};

\path[] ( 43.64,  0.00) rectangle (267.09,198.74);
\end{scope}
\begin{scope}
\path[clip] (  0.00,  0.00) rectangle (289.08,198.74);

\path[] ( 43.64,  0.00) --
	( 43.64,198.74);
\end{scope}
\begin{scope}
\path[clip] (  0.00,  0.00) rectangle (289.08,198.74);
\definecolor{drawColor}{RGB}{0,0,0}

\node[text=drawColor,anchor=base east,inner sep=0pt, outer sep=0pt, scale=  1.00] at ( 40.89, 12.65) {Otras frutas};

\node[text=drawColor,anchor=base east,inner sep=0pt, outer sep=0pt, scale=  1.00] at ( 40.89, 40.26) {Piña};

\node[text=drawColor,anchor=base east,inner sep=0pt, outer sep=0pt, scale=  1.00] at ( 40.89, 67.86) {Melón};

\node[text=drawColor,anchor=base east,inner sep=0pt, outer sep=0pt, scale=  1.00] at ( 40.89, 95.46) {Aguacate};

\node[text=drawColor,anchor=base east,inner sep=0pt, outer sep=0pt, scale=  1.00] at ( 40.89,123.07) {Cítricos};

\node[text=drawColor,anchor=base east,inner sep=0pt, outer sep=0pt, scale=  1.00] at ( 40.89,150.67) {Banano};

\node[text=drawColor,anchor=base east,inner sep=0pt, outer sep=0pt, scale=  1.00] at ( 40.89,178.27) {Plátano};
\end{scope}
\begin{scope}
\path[clip] (  0.00,  0.00) rectangle (289.08,198.74);

\path[] ( 40.89, 16.56) --
	( 43.64, 16.56);

\path[] ( 40.89, 44.16) --
	( 43.64, 44.16);

\path[] ( 40.89, 71.77) --
	( 43.64, 71.77);

\path[] ( 40.89, 99.37) --
	( 43.64, 99.37);

\path[] ( 40.89,126.97) --
	( 43.64,126.97);

\path[] ( 40.89,154.58) --
	( 43.64,154.58);

\path[] ( 40.89,182.18) --
	( 43.64,182.18);
\end{scope}
  \end{tikzpicture}}%
{%
	INE y MAGA} %

%#########################21########################

\cajita{%
	Disponibilidad per cápita de carnes}%
{%
<<<<<<< HEAD
En el 2013 se contaba mayormente con disponibilidad de carne de ganado vacuno (6.7 kilogramos per cápita), seguido de la carne de aves (5.4), carne de ganado porcino (2.1) y embutidos (1.8).}%
{%
	Disponibilidad per cápita de carnes } %
{%
	República de Guatemala, año 2013, en kilogramos } %
=======
}%
{%
	Disponibilidad per cápita de carnes } %
{%
	República de Guatemala, 2013 , en kilogramos } %
>>>>>>> origin/master
{%
	\begin{tikzpicture}[x=1pt,y=1pt]  % Created by tikzDevice version 0.9 on 2016-03-03 04:15:57
% !TEX encoding = UTF-8 Unicode
\definecolor{fillColor}{RGB}{255,255,255}
\path[use as bounding box,fill=fillColor,fill opacity=0.00] (0,0) rectangle (289.08,198.74);
\begin{scope}
\path[clip] (  0.00,  0.00) rectangle (289.08,198.74);

\path[] (  0.00,  0.00) rectangle (289.08,198.74);
\end{scope}
\begin{scope}
\path[clip] (  0.00,  0.00) rectangle (289.08,198.74);

\path[] ( 61.28,  0.00) rectangle (271.48,198.74);

\path[] ( 61.28, 19.23) --
	(271.48, 19.23);

\path[] ( 61.28, 51.29) --
	(271.48, 51.29);

\path[] ( 61.28, 83.34) --
	(271.48, 83.34);

\path[] ( 61.28,115.40) --
	(271.48,115.40);

\path[] ( 61.28,147.45) --
	(271.48,147.45);

\path[] ( 61.28,179.51) --
	(271.48,179.51);
\definecolor{drawColor}{RGB}{0,0,255}
\definecolor{fillColor}{RGB}{0,0,255}

\path[draw=drawColor,line width= 0.6pt,line join=round,fill=fillColor] ( 61.28,  9.62) rectangle (117.75, 28.85);

\path[draw=drawColor,line width= 0.6pt,line join=round,fill=fillColor] ( 61.28, 41.67) rectangle (230.70, 60.90);

\path[draw=drawColor,line width= 0.6pt,line join=round,fill=fillColor] ( 61.28, 73.73) rectangle ( 67.55, 92.96);

\path[draw=drawColor,line width= 0.6pt,line join=round,fill=fillColor] ( 61.28,105.78) rectangle (127.16,125.02);

\path[draw=drawColor,line width= 0.6pt,line join=round,fill=fillColor] ( 61.28,137.84) rectangle ( 89.52,157.07);

\path[draw=drawColor,line width= 0.6pt,line join=round,fill=fillColor] ( 61.28,169.89) rectangle (271.48,189.13);
\definecolor{drawColor}{RGB}{0,0,0}

\path[draw=drawColor,line width= 0.1pt,line join=round] ( 61.28,  0.00) -- ( 61.28,198.74);

\node[text=drawColor,anchor=base west,inner sep=0pt, outer sep=0pt, scale=  1.02] at (119.99, 15.26) {1.8};

\node[text=drawColor,anchor=base west,inner sep=0pt, outer sep=0pt, scale=  1.02] at (232.93, 47.32) {5.4};

\node[text=drawColor,anchor=base west,inner sep=0pt, outer sep=0pt, scale=  1.02] at ( 69.79, 79.37) {0.2};

\node[text=drawColor,anchor=base west,inner sep=0pt, outer sep=0pt, scale=  1.02] at (129.40,111.43) {2.1};

\node[text=drawColor,anchor=base west,inner sep=0pt, outer sep=0pt, scale=  1.02] at ( 91.75,143.48) {0.9};

\node[text=drawColor,anchor=base west,inner sep=0pt, outer sep=0pt, scale=  1.02] at (273.72,175.54) {6.7};

\path[] ( 61.28,  0.00) rectangle (271.48,198.74);
\end{scope}
\begin{scope}
\path[clip] (  0.00,  0.00) rectangle (289.08,198.74);

\path[] ( 61.28,  0.00) --
	( 61.28,198.74);
\end{scope}
\begin{scope}
\path[clip] (  0.00,  0.00) rectangle (289.08,198.74);
\definecolor{drawColor}{RGB}{0,0,0}

\node[text=drawColor,anchor=base east,inner sep=0pt, outer sep=0pt, scale=  1.00] at ( 58.53, 15.32) {Embutidos };

<<<<<<< HEAD
\node[text=drawColor,anchor=base east,inner sep=0pt, outer sep=0pt, scale=  1.00] at ( 58.53, 47.38) {Carne de ave};
=======
\node[text=drawColor,anchor=base east,inner sep=0pt, outer sep=0pt, scale=  1.00] at ( 58.53, 47.38) {carne de ave};
>>>>>>> origin/master

\node[text=drawColor,anchor=base east,inner sep=0pt, outer sep=0pt, scale=  1.00] at ( 58.53, 79.44) {Vísceras cerdos};

\node[text=drawColor,anchor=base east,inner sep=0pt, outer sep=0pt, scale=  1.00] at ( 58.53,111.49) {Carne cerdo};

\node[text=drawColor,anchor=base east,inner sep=0pt, outer sep=0pt, scale=  1.00] at ( 58.53,143.55) {Visceras vacunos};

\node[text=drawColor,anchor=base east,inner sep=0pt, outer sep=0pt, scale=  1.00] at ( 58.53,175.60) {Carne Vacunos};
\end{scope}
\begin{scope}
\path[clip] (  0.00,  0.00) rectangle (289.08,198.74);

\path[] ( 58.53, 19.23) --
	( 61.28, 19.23);

\path[] ( 58.53, 51.29) --
	( 61.28, 51.29);

\path[] ( 58.53, 83.34) --
	( 61.28, 83.34);

\path[] ( 58.53,115.40) --
	( 61.28,115.40);

\path[] ( 58.53,147.45) --
	( 61.28,147.45);

\path[] ( 58.53,179.51) --
	( 61.28,179.51);
\end{scope}
  \end{tikzpicture}}%
{%
	INE y MAGA} %

%#########################22########################

\cajita{%
<<<<<<< HEAD
	Disponibilidad de huevos, pescado y mariscos}%
{%
En cuanto a la disponibilidad de otros productos, en el 2013 se contaba con 7.6 kilogramos de huevos por persona, 2.5 kilogramos de pescado y menos de .1 gramo de camarón
=======
	Disponibilidad per cápita de huevo, pescado y mariscos}%
{%
>>>>>>> origin/master
}%
{%
	Disponibilidad per cápita de huevo, pescado y mariscos } %
{%
<<<<<<< HEAD
	República de Guatemala, año 2013, en kilogramos } %
=======
	República de Guatemala, 2013 , en kilogramos } %
>>>>>>> origin/master
{%
	\begin{tikzpicture}[x=1pt,y=1pt]  \input{graficas/2_22.tex}  \end{tikzpicture}}%
{%
	INE y MAGA} %

%#########################23########################

\cajita{%
<<<<<<< HEAD
	Disponibilidad de productos lácteos}%
{%
En el 2013 se contaba mayormente con disponibilidad de leche fluida entera pasteurizada (17.5 kilogramos per cápita), seguido de la leche fluida cruda de vaca (2.2), quesos (1.6) y leche fluida semidescremada (1.6).}%
{%
	Disponibilidad per cápita de productos lácteos} %
{%
	República de Guatemala, año 2013, en kilogramos } %
=======
	Disponibilidad per cápita de productos lácteos}%
{%
}%
{%
	Disponibilidad per cápita de productos lácteos} %
{%
	República de Guatemala, 2013 , en kilogramos } %
>>>>>>> origin/master
{%
	\begin{tikzpicture}[x=1pt,y=1pt]  % Created by tikzDevice version 0.9 on 2016-03-03 04:16:23
% !TEX encoding = UTF-8 Unicode
\definecolor{fillColor}{RGB}{255,255,255}
\path[use as bounding box,fill=fillColor,fill opacity=0.00] (0,0) rectangle (289.08,198.74);
\begin{scope}
\path[clip] (  0.00,  0.00) rectangle (289.08,198.74);

\path[] (  0.00,  0.00) rectangle (289.08,198.74);
\end{scope}
\begin{scope}
\path[clip] (  0.00,  0.00) rectangle (289.08,198.74);

\path[] (164.00,  0.00) rectangle (267.09,198.74);

\path[] (164.00, 11.69) --
	(267.09, 11.69);

\path[] (164.00, 31.18) --
	(267.09, 31.18);

\path[] (164.00, 50.66) --
	(267.09, 50.66);

\path[] (164.00, 70.14) --
	(267.09, 70.14);

\path[] (164.00, 89.63) --
	(267.09, 89.63);

\path[] (164.00,109.11) --
	(267.09,109.11);

\path[] (164.00,128.60) --
	(267.09,128.60);

\path[] (164.00,148.08) --
	(267.09,148.08);

\path[] (164.00,167.57) --
	(267.09,167.57);

\path[] (164.00,187.05) --
	(267.09,187.05);
\definecolor{drawColor}{RGB}{0,0,255}
\definecolor{fillColor}{RGB}{0,0,255}

\path[draw=drawColor,line width= 0.6pt,line join=round,fill=fillColor] (164.00,  5.85) rectangle (165.77, 17.54);

\path[draw=drawColor,line width= 0.6pt,line join=round,fill=fillColor] (164.00, 25.33) rectangle (166.95, 37.02);

\path[draw=drawColor,line width= 0.6pt,line join=round,fill=fillColor] (164.00, 44.81) rectangle (173.43, 56.51);

\path[draw=drawColor,line width= 0.6pt,line join=round,fill=fillColor] (164.00, 64.30) rectangle (167.53, 75.99);

\path[draw=drawColor,line width= 0.6pt,line join=round,fill=fillColor] (164.00, 83.78) rectangle (164.59, 95.47);

\path[draw=drawColor,line width= 0.6pt,line join=round,fill=fillColor] (164.00,103.27) rectangle (164.59,114.96);

\path[draw=drawColor,line width= 0.6pt,line join=round,fill=fillColor] (164.00,122.75) rectangle (164.00,134.44);

\path[draw=drawColor,line width= 0.6pt,line join=round,fill=fillColor] (164.00,142.24) rectangle (173.43,153.93);

\path[draw=drawColor,line width= 0.6pt,line join=round,fill=fillColor] (164.00,161.72) rectangle (267.09,173.41);

\path[draw=drawColor,line width= 0.6pt,line join=round,fill=fillColor] (164.00,181.21) rectangle (176.96,192.90);
\definecolor{drawColor}{RGB}{0,0,0}

\path[draw=drawColor,line width= 0.1pt,line join=round] (164.00,  0.00) -- (164.00,198.74);

\node[text=drawColor,anchor=base west,inner sep=0pt, outer sep=0pt, scale=  1.02] at (168.01,  7.72) {0.3};

\node[text=drawColor,anchor=base west,inner sep=0pt, outer sep=0pt, scale=  1.02] at (169.18, 27.20) {0.5};

\node[text=drawColor,anchor=base west,inner sep=0pt, outer sep=0pt, scale=  1.02] at (175.66, 46.69) {1.6};

\node[text=drawColor,anchor=base west,inner sep=0pt, outer sep=0pt, scale=  1.02] at (169.77, 66.17) {0.6};

\node[text=drawColor,anchor=base west,inner sep=0pt, outer sep=0pt, scale=  1.02] at (166.83, 85.66) {0.1};

\node[text=drawColor,anchor=base west,inner sep=0pt, outer sep=0pt, scale=  1.02] at (166.83,105.14) {0.1};

\node[text=drawColor,anchor=base west,inner sep=0pt, outer sep=0pt, scale=  1.02] at (166.24,124.63) {0.0};

\node[text=drawColor,anchor=base west,inner sep=0pt, outer sep=0pt, scale=  1.02] at (175.66,144.11) {1.6};

\node[text=drawColor,anchor=base west,inner sep=0pt, outer sep=0pt, scale=  1.02] at (270.21,163.60) {17.5};

\node[text=drawColor,anchor=base west,inner sep=0pt, outer sep=0pt, scale=  1.02] at (179.20,183.08) {2.2};

\path[] (164.00,  0.00) rectangle (267.09,198.74);
\end{scope}
\begin{scope}
\path[clip] (  0.00,  0.00) rectangle (289.08,198.74);

\path[] (164.00,  0.00) --
	(164.00,198.74);
\end{scope}
\begin{scope}
\path[clip] (  0.00,  0.00) rectangle (289.08,198.74);
\definecolor{drawColor}{RGB}{0,0,0}

\node[text=drawColor,anchor=base east,inner sep=0pt, outer sep=0pt, scale=  1.00] at (161.25,  7.78) {Leche pasteurizada/yogur};

\node[text=drawColor,anchor=base east,inner sep=0pt, outer sep=0pt, scale=  1.00] at (161.25, 27.27) {Leche pasteurizada/crema de leche};

\node[text=drawColor,anchor=base east,inner sep=0pt, outer sep=0pt, scale=  1.00] at (161.25, 46.75) {Quesos};

\node[text=drawColor,anchor=base east,inner sep=0pt, outer sep=0pt, scale=  1.00] at (161.25, 66.24) {Leche en polvo entera };

\node[text=drawColor,anchor=base east,inner sep=0pt, outer sep=0pt, scale=  1.00] at (161.25, 85.72) {Lecha en polvo descremada};

\node[text=drawColor,anchor=base east,inner sep=0pt, outer sep=0pt, scale=  1.00] at (161.25,105.21) {Leche pasteurizada/ Leche fluida descremada};

\node[text=drawColor,anchor=base east,inner sep=0pt, outer sep=0pt, scale=  1.00] at (161.25,124.69) {Leche en polvo semidescremada};

\node[text=drawColor,anchor=base east,inner sep=0pt, outer sep=0pt, scale=  1.00] at (161.25,144.17) {Leche fluida semidescremada};

\node[text=drawColor,anchor=base east,inner sep=0pt, outer sep=0pt, scale=  1.00] at (161.25,163.66) {Leche fluida entera pasteurizada};

\node[text=drawColor,anchor=base east,inner sep=0pt, outer sep=0pt, scale=  1.00] at (161.25,183.14) {Lecha fluida cruda de vaca};
\end{scope}
\begin{scope}
\path[clip] (  0.00,  0.00) rectangle (289.08,198.74);

\path[] (161.25, 11.69) --
	(164.00, 11.69);

\path[] (161.25, 31.18) --
	(164.00, 31.18);

\path[] (161.25, 50.66) --
	(164.00, 50.66);

\path[] (161.25, 70.14) --
	(164.00, 70.14);

\path[] (161.25, 89.63) --
	(164.00, 89.63);

\path[] (161.25,109.11) --
	(164.00,109.11);

\path[] (161.25,128.60) --
	(164.00,128.60);

\path[] (161.25,148.08) --
	(164.00,148.08);

\path[] (161.25,167.57) --
	(164.00,167.57);

\path[] (161.25,187.05) --
	(164.00,187.05);
\end{scope}
  \end{tikzpicture}}%
{%
	INE y MAGA} %


%#########################24########################

\cajita{%
<<<<<<< HEAD
	Disponibilidad de aceites y grasas}%
{%
En cuanto a la disponibilidad de productos de grasas y aceites, en el 2013 se contaba con 4.0 kilogramos de aceite refinado de soya, 1.4 kilogramos aceite refinado de palma africana y 0.8 kilogramos de aceite refinado de semilla de girasol.}%
{%
	Disponibilidad per cápita de aceites y grasas} %
{%
	República de Guatemala, año 2013, en kilogramos } %
=======
	Disponibilidad per cápita de aceites y grasas}%
{%
}%
{%
	Disponibilidad per cápita de aceites y grasas} %
{%
	República de Guatemala, 2013 , en kilogramos } %
>>>>>>> origin/master
{%
	\begin{tikzpicture}[x=1pt,y=1pt]  % Created by tikzDevice version 0.9 on 2016-03-03 04:16:46
% !TEX encoding = UTF-8 Unicode
\definecolor{fillColor}{RGB}{255,255,255}
\path[use as bounding box,fill=fillColor,fill opacity=0.00] (0,0) rectangle (289.08,198.74);
\begin{scope}
\path[clip] (  0.00,  0.00) rectangle (289.08,198.74);

\path[] ( -0.00,  0.00) rectangle (289.08,198.74);
\end{scope}
\begin{scope}
\path[clip] (  0.00,  0.00) rectangle (289.08,198.74);

\path[] (127.86,  0.00) rectangle (271.48,198.74);

\path[] (127.86, 14.54) --
	(271.48, 14.54);

\path[] (127.86, 38.78) --
	(271.48, 38.78);

\path[] (127.86, 63.02) --
	(271.48, 63.02);

\path[] (127.86, 87.25) --
	(271.48, 87.25);

\path[] (127.86,111.49) --
	(271.48,111.49);

\path[] (127.86,135.73) --
	(271.48,135.73);

\path[] (127.86,159.96) --
	(271.48,159.96);

\path[] (127.86,184.20) --
	(271.48,184.20);
\definecolor{drawColor}{RGB}{0,0,255}
\definecolor{fillColor}{RGB}{0,0,255}

\path[draw=drawColor,line width= 0.6pt,line join=round,fill=fillColor] (127.86,  7.27) rectangle (131.45, 21.81);

\path[draw=drawColor,line width= 0.6pt,line join=round,fill=fillColor] (127.86, 31.51) rectangle (131.45, 46.05);

\path[draw=drawColor,line width= 0.6pt,line join=round,fill=fillColor] (127.86, 55.74) rectangle (131.45, 70.29);

\path[draw=drawColor,line width= 0.6pt,line join=round,fill=fillColor] (127.86, 79.98) rectangle (127.86, 94.52);

\path[draw=drawColor,line width= 0.6pt,line join=round,fill=fillColor] (127.86,104.22) rectangle (156.58,118.76);

\path[draw=drawColor,line width= 0.6pt,line join=round,fill=fillColor] (127.86,128.46) rectangle (127.86,143.00);

\path[draw=drawColor,line width= 0.6pt,line join=round,fill=fillColor] (127.86,152.69) rectangle (271.48,167.23);

\path[draw=drawColor,line width= 0.6pt,line join=round,fill=fillColor] (127.86,176.93) rectangle (178.13,191.47);
\definecolor{drawColor}{RGB}{0,0,0}

\path[draw=drawColor,line width= 0.1pt,line join=round] (127.86,  0.00) -- (127.86,198.74);

\node[text=drawColor,anchor=base west,inner sep=0pt, outer sep=0pt, scale=  1.02] at (133.69, 10.57) {0.1};

\node[text=drawColor,anchor=base west,inner sep=0pt, outer sep=0pt, scale=  1.02] at (133.69, 34.81) {0.1};

\node[text=drawColor,anchor=base west,inner sep=0pt, outer sep=0pt, scale=  1.02] at (133.69, 59.04) {0.1};

\node[text=drawColor,anchor=base west,inner sep=0pt, outer sep=0pt, scale=  1.02] at (130.10, 83.28) {0.0};

\node[text=drawColor,anchor=base west,inner sep=0pt, outer sep=0pt, scale=  1.02] at (158.82,107.52) {0.8};

\node[text=drawColor,anchor=base west,inner sep=0pt, outer sep=0pt, scale=  1.02] at (130.10,131.76) {0.0};

\node[text=drawColor,anchor=base west,inner sep=0pt, outer sep=0pt, scale=  1.02] at (273.72,155.99) {4.0};

\node[text=drawColor,anchor=base west,inner sep=0pt, outer sep=0pt, scale=  1.02] at (180.36,180.23) {1.4};

\path[] (127.86,  0.00) rectangle (271.48,198.74);
\end{scope}
\begin{scope}
\path[clip] (  0.00,  0.00) rectangle (289.08,198.74);

\path[] (127.86,  0.00) --
	(127.86,198.74);
\end{scope}
\begin{scope}
\path[clip] (  0.00,  0.00) rectangle (289.08,198.74);
\definecolor{drawColor}{RGB}{0,0,0}

\node[text=drawColor,anchor=base east,inner sep=0pt, outer sep=0pt, scale=  1.00] at (125.11, 10.63) {cerdos faenados/manteca de cerdo};

\node[text=drawColor,anchor=base east,inner sep=0pt, outer sep=0pt, scale=  1.00] at (125.11, 34.87) {Vacunos faenados/grasa de res};

\node[text=drawColor,anchor=base east,inner sep=0pt, outer sep=0pt, scale=  1.00] at (125.11, 59.11) {Leche pasteurizada/mantequilla};

\node[text=drawColor,anchor=base east,inner sep=0pt, outer sep=0pt, scale=  1.00] at (125.11, 83.34) {Aceite refinado/aceituna (oliva)};

\node[text=drawColor,anchor=base east,inner sep=0pt, outer sep=0pt, scale=  1.00] at (125.11,107.58) {Aceite refinado/semilla de girasol};

\node[text=drawColor,anchor=base east,inner sep=0pt, outer sep=0pt, scale=  1.00] at (125.11,131.82) {Aceite refinado/semilla de algodón};

\node[text=drawColor,anchor=base east,inner sep=0pt, outer sep=0pt, scale=  1.00] at (125.11,156.06) {Aceite refinado/soya};

\node[text=drawColor,anchor=base east,inner sep=0pt, outer sep=0pt, scale=  1.00] at (125.11,180.29) {Aceite refinado/palma africana};
\end{scope}
\begin{scope}
\path[clip] (  0.00,  0.00) rectangle (289.08,198.74);

\path[] (125.11, 14.54) --
	(127.86, 14.54);

\path[] (125.11, 38.78) --
	(127.86, 38.78);

\path[] (125.11, 63.02) --
	(127.86, 63.02);

\path[] (125.11, 87.25) --
	(127.86, 87.25);

\path[] (125.11,111.49) --
	(127.86,111.49);

\path[] (125.11,135.73) --
	(127.86,135.73);

\path[] (125.11,159.96) --
	(127.86,159.96);

\path[] (125.11,184.20) --
	(127.86,184.20);
\end{scope}
  \end{tikzpicture}}%
{%
	INE y MAGA} %

%#########################25########################

\cajita{%
	Disponibilidad per cápita de alimentos gratificantes}%
{%
<<<<<<< HEAD
De los alimentos denominados como gratificantes, en el 2013 se contaba mayormente con disponibilidad de azúcar o bebidas gaseosas (68.2 kilogramos per cápita), seguido de la cerveza (11.9)}%
{%
	Disponibilidad per cápita de alimentos gratificantes} %
{%
	República de Guatemala, año 2013, en kilogramos } %
=======
}%
{%
	Disponibilidad per cápita de alimentos gratificantes} %
{%
	República de Guatemala, 2013 , en kilogramos } %
>>>>>>> origin/master
{%
	\begin{tikzpicture}[x=1pt,y=1pt]  % Created by tikzDevice version 0.9 on 2016-03-03 04:17:04
% !TEX encoding = UTF-8 Unicode
\definecolor{fillColor}{RGB}{255,255,255}
\path[use as bounding box,fill=fillColor,fill opacity=0.00] (0,0) rectangle (289.08,198.74);
\begin{scope}
\path[clip] (  0.00,  0.00) rectangle (289.08,198.74);

\path[] (  0.00,  0.00) rectangle (289.08,198.74);
\end{scope}
\begin{scope}
\path[clip] (  0.00,  0.00) rectangle (289.08,198.74);

\path[] ( 90.78,  0.00) rectangle (267.09,198.74);

\path[] ( 90.78, 37.26) --
	(267.09, 37.26);

\path[] ( 90.78, 99.37) --
	(267.09, 99.37);

\path[] ( 90.78,161.48) --
	(267.09,161.48);
\definecolor{drawColor}{RGB}{0,0,255}
\definecolor{fillColor}{RGB}{0,0,255}

\path[draw=drawColor,line width= 0.6pt,line join=round,fill=fillColor] ( 90.78, 18.63) rectangle (267.09, 55.90);

\path[draw=drawColor,line width= 0.6pt,line join=round,fill=fillColor] ( 90.78, 80.74) rectangle ( 99.05,118.00);

\path[draw=drawColor,line width= 0.6pt,line join=round,fill=fillColor] ( 90.78,142.85) rectangle (121.54,180.11);
\definecolor{drawColor}{RGB}{0,0,0}

\path[draw=drawColor,line width= 0.1pt,line join=round] ( 90.78,  0.00) -- ( 90.78,198.74);

\node[text=drawColor,anchor=base west,inner sep=0pt, outer sep=0pt, scale=  1.02] at (270.21, 33.29) {68.2};

\node[text=drawColor,anchor=base west,inner sep=0pt, outer sep=0pt, scale=  1.02] at (101.29, 95.40) {3.2};

\node[text=drawColor,anchor=base west,inner sep=0pt, outer sep=0pt, scale=  1.02] at (124.67,157.51) {11.9};

\path[] ( 90.78,  0.00) rectangle (267.09,198.74);
\end{scope}
\begin{scope}
\path[clip] (  0.00,  0.00) rectangle (289.08,198.74);

\path[] ( 90.78,  0.00) --
	( 90.78,198.74);
\end{scope}
\begin{scope}
\path[clip] (  0.00,  0.00) rectangle (289.08,198.74);
\definecolor{drawColor}{RGB}{0,0,0}

\node[text=drawColor,anchor=base east,inner sep=0pt, outer sep=0pt, scale=  1.00] at ( 88.03, 33.36) {azúcar/bebidas gaseosas};

\node[text=drawColor,anchor=base east,inner sep=0pt, outer sep=0pt, scale=  1.00] at ( 88.03, 95.46) {melazas/licores};

\node[text=drawColor,anchor=base east,inner sep=0pt, outer sep=0pt, scale=  1.00] at ( 88.03,157.57) {Cerveza};
\end{scope}
\begin{scope}
\path[clip] (  0.00,  0.00) rectangle (289.08,198.74);

\path[] ( 88.03, 37.26) --
	( 90.78, 37.26);

\path[] ( 88.03, 99.37) --
	( 90.78, 99.37);

\path[] ( 88.03,161.48) --
	( 90.78,161.48);
\end{scope}
  \end{tikzpicture}}%
{%
	INE y MAGA} %
		\INEchaptercarta{Acceso a los alimentos}{Como su nombre lo indica, esta dimensión se refiere a la garantía para el acceso a una alimentación adecuada (FAO, 2015). La misma se compone de seis indicadores y once sub-indicadores.
			
			Para esta dimensión, únicamente se encontró información para los indicadores de Nivel de Ingreso y Precio de los Alimentos, mientras que no se encontró información para el Nivel de Autoconsumo, Mercados, Comunicaciones y Fortificación de alimentos. 
			 
			Para el indicador \textbf{3.1 Nivel de Ingreso}, se encontró información para los siguientes sub-indicadores: salario mínimo, salario de cotizantes al IGSS e ingresos de Encuesta Nacional de Empleo e Ingresos (ENEI). No obstante, no se encontró información sobre subsidios. 
			
			Por otro lado, para el indicador \textbf{3.2 Precio de los Alimentos}, se encontró información para el sub-indicador de Canasta Básica Alimentaria (CBA).}
		%#########################1########################

 \cajita{%
Evolución del salario mínimo }%
{%
<<<<<<< HEAD
El salario mínimo diario  }%
=======
 }%
>>>>>>> origin/master
{%
 Salario mínimo diario según rama de actividad económica} %
{%
 República de Guatemala, serie histórica, en quetzales } %
{%
 \begin{tikzpicture}[x=1pt,y=1pt]  % Created by tikzDevice version 0.9 on 2016-03-03 04:33:50
% !TEX encoding = UTF-8 Unicode
\definecolor{fillColor}{RGB}{255,255,255}
\path[use as bounding box,fill=fillColor,fill opacity=0.00] (0,0) rectangle (289.08,198.74);
\begin{scope}
\path[clip] (  0.00,  0.00) rectangle (289.08,198.74);

\path[] (  0.00,  0.00) rectangle (289.08,198.74);
\end{scope}
\begin{scope}
\path[clip] (  0.00,  0.00) rectangle (289.08,198.74);

\path[] (  0.00, 18.46) rectangle (289.08,164.23);

\path[] ( 33.36, 18.46) --
	( 33.36,164.23);

\path[] ( 88.95, 18.46) --
	( 88.95,164.23);

\path[] (144.54, 18.46) --
	(144.54,164.23);

\path[] (200.13, 18.46) --
	(200.13,164.23);

\path[] (255.72, 18.46) --
	(255.72,164.23);
\definecolor{drawColor}{RGB}{0,0,255}
\definecolor{fillColor}{RGB}{0,0,255}

\path[draw=drawColor,line width= 0.6pt,line join=round,fill=fillColor] (  9.27, 18.46) rectangle ( 24.09,136.42);
\definecolor{drawColor}{RGB}{157,187,255}
\definecolor{fillColor}{RGB}{157,187,255}

\path[draw=drawColor,line width= 0.6pt,line join=round,fill=fillColor] ( 25.94, 18.46) rectangle ( 40.77,136.42);
\definecolor{drawColor}{RGB}{200,200,200}
\definecolor{fillColor}{RGB}{200,200,200}

\path[draw=drawColor,line width= 0.6pt,line join=round,fill=fillColor] ( 42.62, 18.46) rectangle ( 57.45,128.55);
\definecolor{drawColor}{RGB}{0,0,255}
\definecolor{fillColor}{RGB}{0,0,255}

\path[draw=drawColor,line width= 0.6pt,line join=round,fill=fillColor] ( 64.86, 18.46) rectangle ( 79.68,144.38);
\definecolor{drawColor}{RGB}{157,187,255}
\definecolor{fillColor}{RGB}{157,187,255}

\path[draw=drawColor,line width= 0.6pt,line join=round,fill=fillColor] ( 81.54, 18.46) rectangle ( 96.36,144.38);
\definecolor{drawColor}{RGB}{200,200,200}
\definecolor{fillColor}{RGB}{200,200,200}

\path[draw=drawColor,line width= 0.6pt,line join=round,fill=fillColor] ( 98.21, 18.46) rectangle (113.04,134.20);
\definecolor{drawColor}{RGB}{0,0,255}
\definecolor{fillColor}{RGB}{0,0,255}

\path[draw=drawColor,line width= 0.6pt,line join=round,fill=fillColor] (120.45, 18.46) rectangle (135.27,150.68);
\definecolor{drawColor}{RGB}{157,187,255}
\definecolor{fillColor}{RGB}{157,187,255}

\path[draw=drawColor,line width= 0.6pt,line join=round,fill=fillColor] (137.13, 18.46) rectangle (151.95,150.68);
\definecolor{drawColor}{RGB}{200,200,200}
\definecolor{fillColor}{RGB}{200,200,200}

\path[draw=drawColor,line width= 0.6pt,line join=round,fill=fillColor] (153.81, 18.46) rectangle (168.63,139.99);
\definecolor{drawColor}{RGB}{0,0,255}
\definecolor{fillColor}{RGB}{0,0,255}

\path[draw=drawColor,line width= 0.6pt,line join=round,fill=fillColor] (176.04, 18.46) rectangle (190.87,157.29);
\definecolor{drawColor}{RGB}{157,187,255}
\definecolor{fillColor}{RGB}{157,187,255}

\path[draw=drawColor,line width= 0.6pt,line join=round,fill=fillColor] (192.72, 18.46) rectangle (207.54,157.29);
\definecolor{drawColor}{RGB}{200,200,200}
\definecolor{fillColor}{RGB}{200,200,200}

\path[draw=drawColor,line width= 0.6pt,line join=round,fill=fillColor] (209.40, 18.46) rectangle (224.22,146.07);
\definecolor{drawColor}{RGB}{0,0,255}
\definecolor{fillColor}{RGB}{0,0,255}

\path[draw=drawColor,line width= 0.6pt,line join=round,fill=fillColor] (231.63, 18.46) rectangle (246.46,164.23);
\definecolor{drawColor}{RGB}{157,187,255}
\definecolor{fillColor}{RGB}{157,187,255}

\path[draw=drawColor,line width= 0.6pt,line join=round,fill=fillColor] (248.31, 18.46) rectangle (263.14,164.23);
\definecolor{drawColor}{RGB}{200,200,200}
\definecolor{fillColor}{RGB}{200,200,200}

\path[draw=drawColor,line width= 0.6pt,line join=round,fill=fillColor] (264.99, 18.46) rectangle (279.81,152.45);
\definecolor{drawColor}{RGB}{0,0,0}

\path[draw=drawColor,line width= 0.6pt,line join=round] (  0.00, 18.46) -- (289.08, 18.46);

\node[text=drawColor,rotate= 90.00,anchor=base west,inner sep=0pt, outer sep=0pt, scale=  0.83] at ( 19.91,138.97) {63.7};

\node[text=drawColor,rotate= 90.00,anchor=base west,inner sep=0pt, outer sep=0pt, scale=  0.83] at ( 36.59,138.97) {63.7};

\node[text=drawColor,rotate= 90.00,anchor=base west,inner sep=0pt, outer sep=0pt, scale=  0.83] at ( 53.27,131.10) {59.5};

\node[text=drawColor,rotate= 90.00,anchor=base west,inner sep=0pt, outer sep=0pt, scale=  0.83] at ( 75.50,146.93) {68.0};

\node[text=drawColor,rotate= 90.00,anchor=base west,inner sep=0pt, outer sep=0pt, scale=  0.83] at ( 92.18,146.93) {68.0};

\node[text=drawColor,rotate= 90.00,anchor=base west,inner sep=0pt, outer sep=0pt, scale=  0.83] at (108.86,136.74) {62.5};

\node[text=drawColor,rotate= 90.00,anchor=base west,inner sep=0pt, outer sep=0pt, scale=  0.83] at (131.10,153.23) {71.4};

\node[text=drawColor,rotate= 90.00,anchor=base west,inner sep=0pt, outer sep=0pt, scale=  0.83] at (147.77,153.23) {71.4};

\node[text=drawColor,rotate= 90.00,anchor=base west,inner sep=0pt, outer sep=0pt, scale=  0.83] at (164.45,142.54) {65.6};

\node[text=drawColor,rotate= 90.00,anchor=base west,inner sep=0pt, outer sep=0pt, scale=  0.83] at (186.69,159.84) {75.0};

\node[text=drawColor,rotate= 90.00,anchor=base west,inner sep=0pt, outer sep=0pt, scale=  0.83] at (203.37,159.84) {75.0};

\node[text=drawColor,rotate= 90.00,anchor=base west,inner sep=0pt, outer sep=0pt, scale=  0.83] at (220.04,148.61) {68.9};

\node[text=drawColor,rotate= 90.00,anchor=base west,inner sep=0pt, outer sep=0pt, scale=  0.83] at (242.28,166.78) {78.7};

\node[text=drawColor,rotate= 90.00,anchor=base west,inner sep=0pt, outer sep=0pt, scale=  0.83] at (258.96,166.78) {78.7};

\node[text=drawColor,rotate= 90.00,anchor=base west,inner sep=0pt, outer sep=0pt, scale=  0.83] at (275.64,155.00) {72.4};

\path[] (  0.00, 18.46) rectangle (289.08,164.23);
\end{scope}
\begin{scope}
\path[clip] (  0.00,  0.00) rectangle (289.08,198.74);

\path[] (  0.00, 18.46) --
	(289.08, 18.46);
\end{scope}
\begin{scope}
\path[clip] (  0.00,  0.00) rectangle (289.08,198.74);

\path[] ( 33.36, 15.71) --
	( 33.36, 18.46);

\path[] ( 88.95, 15.71) --
	( 88.95, 18.46);

\path[] (144.54, 15.71) --
	(144.54, 18.46);

\path[] (200.13, 15.71) --
	(200.13, 18.46);

\path[] (255.72, 15.71) --
	(255.72, 18.46);
\end{scope}
\begin{scope}
\path[clip] (  0.00,  0.00) rectangle (289.08,198.74);
\definecolor{drawColor}{RGB}{0,0,0}

\node[text=drawColor,anchor=base,inner sep=0pt, outer sep=0pt, scale=  1.00] at ( 33.36,  5.69) {2011};

\node[text=drawColor,anchor=base,inner sep=0pt, outer sep=0pt, scale=  1.00] at ( 88.95,  5.69) {2012};

\node[text=drawColor,anchor=base,inner sep=0pt, outer sep=0pt, scale=  1.00] at (144.54,  5.69) {2013};

\node[text=drawColor,anchor=base,inner sep=0pt, outer sep=0pt, scale=  1.00] at (200.13,  5.69) {2014};

\node[text=drawColor,anchor=base,inner sep=0pt, outer sep=0pt, scale=  1.00] at (255.72,  5.69) {2015};
\end{scope}
\begin{scope}
\path[clip] (  0.00,  0.00) rectangle (289.08,198.74);
\coordinate (apoyo1) at (31.76,188.79);
\coordinate (apoyo2) at (118.73,188.79);
\coordinate (apoyo3) at (215.96,188.79);
\coordinate (longitudFicticia) at (7.11,9.95);
\coordinate (longitud) at (7.11,7.11);
\coordinate (desY) at (0,1.42);
\definecolor[named]{ct1}{HTML}{
0000FF
}
\definecolor[named]{ct2}{HTML}{
9DBBFF
}
\definecolor[named]{ct3}{HTML}{
C8C8C8
}
\definecolor[named]{ctb1}{HTML}{
0000FF
}
\definecolor[named]{ctb2}{HTML}{
9DBBFF
}
\definecolor[named]{ctb3}{HTML}{
C8C8C8
}
\path [fill=none] (apoyo1) rectangle ($(apoyo1)+(longitudFicticia)$)
node [xshift=0.3cm,inner sep=0pt, outer sep=0pt,text width=1.33333333333333in,midway,right,scale = 0.9]{Agrícola};
\draw [color= ctb1, fill=ct1] ( $(apoyo1)  + (desY) $) rectangle ($(apoyo1)+ (desY) +(longitud)$);
\path [fill=none] (apoyo2) rectangle ($(apoyo2)+(longitudFicticia)$)
node [xshift=0.3cm,inner sep=0pt, outer sep=0pt,text width=1.33333333333333in,midway,right,scale = 0.9]{No agrícola};
\draw [color = ctb2, fill=ct2] ( $(apoyo2)  + (desY) $) rectangle ($(apoyo2)+ (desY) +(longitud)$);
\path [fill=none] (apoyo3) rectangle ($(apoyo3)+(longitudFicticia)$)
node [xshift=0.3cm,inner sep=0pt, outer sep=0pt,text width=1.33333333333333in,midway,right,scale = 0.9]{Maquila};
\path [color = ctb3, fill=ct3] ( $(apoyo3)  + (desY) $) rectangle ($(apoyo3)+ (desY) +(longitud)$);
\end{scope}
  \end{tikzpicture}}%
{%
 Instituto Nacional de Estadística} %


%#########################2########################

\cajita{%
	Ingreso laboral mensual }%
{%
}%
{%
	Promedio del ingreso laboral mensual} %
{%
	República de Guatemala, serie histórica de ENEI, en quetzales } %
{%
	\begin{tikzpicture}[x=1pt,y=1pt]  % Created by tikzDevice version 0.9 on 2016-03-03 04:33:56
% !TEX encoding = UTF-8 Unicode
\definecolor{fillColor}{RGB}{255,255,255}
\path[use as bounding box,fill=fillColor,fill opacity=0.00] (0,0) rectangle (289.08,198.74);
\begin{scope}
\path[clip] (  0.00,  0.00) rectangle (289.08,198.74);

\path[] (  0.00,  0.00) rectangle (289.08,198.74);
\end{scope}
\begin{scope}
\path[clip] (  0.00,  0.00) rectangle (289.08,198.74);

\path[] (  8.24, 15.61) rectangle (280.54,191.48);

\path[] (  8.24, 46.37) --
	(280.54, 46.37);

\path[] (  8.24,105.27) --
	(280.54,105.27);

\path[] (  8.24,164.17) --
	(280.54,164.17);

\path[] (  8.24, 16.92) --
	(280.54, 16.92);

\path[] (  8.24, 75.82) --
	(280.54, 75.82);

\path[] (  8.24,134.72) --
	(280.54,134.72);

\path[] ( 39.66, 15.61) --
	( 39.66,191.48);

\path[] ( 92.03, 15.61) --
	( 92.03,191.48);

\path[] (144.39, 15.61) --
	(144.39,191.48);

\path[] (196.76, 15.61) --
	(196.76,191.48);

\path[] (249.12, 15.61) --
	(249.12,191.48);
\definecolor{drawColor}{RGB}{0,0,255}

\path[draw=drawColor,line width= 1.7pt,line join=round] ( 39.66, 60.50) --
	( 92.03, 72.05) --
	(144.39,109.51) --
	(196.76,154.27) --
	(249.12,183.49);
\definecolor{drawColor}{RGB}{0,0,0}

\node[text=drawColor,anchor=base,inner sep=0pt, outer sep=0pt, scale=  1.02] at ( 39.66, 48.59) {1,685};

\node[text=drawColor,anchor=base east,inner sep=0pt, outer sep=0pt, scale=  1.02] at ( 88.01, 72.05) {1,734};

\node[text=drawColor,anchor=base east,inner sep=0pt, outer sep=0pt, scale=  1.02] at (140.37,109.51) {1,893};

\node[text=drawColor,anchor=base east,inner sep=0pt, outer sep=0pt, scale=  1.02] at (192.74,154.27) {2,083};

\node[text=drawColor,anchor=base,inner sep=0pt, outer sep=0pt, scale=  1.02] at (249.12,187.46) {2,207};

\path[draw=drawColor,line width= 0.1pt,line join=round] (  8.24, 23.61) -- (280.54, 23.61);

\path[] (  8.24, 15.61) rectangle (280.54,191.48);
\end{scope}
\begin{scope}
\path[clip] (  0.00,  0.00) rectangle (289.08,198.74);

\path[] (  8.24, 15.61) --
	(  8.24,191.48);
\end{scope}
\begin{scope}
\path[clip] (  0.00,  0.00) rectangle (289.08,198.74);
\definecolor{drawColor}{RGB}{255,255,255}

\node[text=drawColor,text opacity=0.00,anchor=base east,inner sep=0pt, outer sep=0pt, scale=  1.00] at (  3.29, 13.01) {1500};

\node[text=drawColor,text opacity=0.00,anchor=base east,inner sep=0pt, outer sep=0pt, scale=  1.00] at (  3.29, 71.91) {1750};

\node[text=drawColor,text opacity=0.00,anchor=base east,inner sep=0pt, outer sep=0pt, scale=  1.00] at (  3.29,130.81) {2000};
\end{scope}
\begin{scope}
\path[clip] (  0.00,  0.00) rectangle (289.08,198.74);

\path[] (  5.49, 16.92) --
	(  8.24, 16.92);

\path[] (  5.49, 75.82) --
	(  8.24, 75.82);

\path[] (  5.49,134.72) --
	(  8.24,134.72);
\end{scope}
\begin{scope}
\path[clip] (  0.00,  0.00) rectangle (289.08,198.74);

\path[] (  8.24, 15.61) --
	(280.54, 15.61);
\end{scope}
\begin{scope}
\path[clip] (  0.00,  0.00) rectangle (289.08,198.74);

\path[] ( 39.66, 12.86) --
	( 39.66, 15.61);

\path[] ( 92.03, 12.86) --
	( 92.03, 15.61);

\path[] (144.39, 12.86) --
	(144.39, 15.61);

\path[] (196.76, 12.86) --
	(196.76, 15.61);

\path[] (249.12, 12.86) --
	(249.12, 15.61);
\end{scope}
\begin{scope}
\path[clip] (  0.00,  0.00) rectangle (289.08,198.74);
\definecolor{drawColor}{RGB}{0,0,0}

\node[text=drawColor,anchor=base,inner sep=0pt, outer sep=0pt, scale=  1.00] at ( 39.66,  2.85) {2011};

\node[text=drawColor,anchor=base,inner sep=0pt, outer sep=0pt, scale=  1.00] at ( 92.03,  2.85) {2012};

\node[text=drawColor,anchor=base,inner sep=0pt, outer sep=0pt, scale=  1.00] at (144.39,  2.85) {2013};

\node[text=drawColor,anchor=base,inner sep=0pt, outer sep=0pt, scale=  1.00] at (196.76,  2.85) {2014};

\node[text=drawColor,anchor=base,inner sep=0pt, outer sep=0pt, scale=  1.00] at (249.12,  2.85) {2015};
\end{scope}
  \end{tikzpicture}}%
{%
	ENEI 2014, 2013, 2012 y 2011} %


%#########################3########################

\cajita{%
	Ingreso laboral mensual según dominio de estudio}%
{%
}%
{%
	Promedio del ingreso laboral mensual según dominio de estudio} %
{%
	República de Guatemala, serie histórica de ENEI, en quetzales } %
{%
	\begin{tikzpicture}[x=1pt,y=1pt]  % Created by tikzDevice version 0.9 on 2016-03-03 04:34:05
% !TEX encoding = UTF-8 Unicode
\definecolor{fillColor}{RGB}{255,255,255}
\path[use as bounding box,fill=fillColor,fill opacity=0.00] (0,0) rectangle (289.08,198.74);
\begin{scope}
\path[clip] (  0.00,  0.00) rectangle (289.08,198.74);

\path[] (  0.00,  0.00) rectangle (289.08,198.74);
\end{scope}
\begin{scope}
\path[clip] (  0.00,  0.00) rectangle (289.08,198.74);

\path[] (  0.00, 18.46) rectangle (289.08,159.90);

\path[] ( 41.30, 18.46) --
	( 41.30,159.90);

\path[] (110.13, 18.46) --
	(110.13,159.90);

\path[] (178.95, 18.46) --
	(178.95,159.90);

\path[] (247.78, 18.46) --
	(247.78,159.90);
\definecolor{drawColor}{RGB}{0,0,255}
\definecolor{fillColor}{RGB}{0,0,255}

\path[draw=drawColor,line width= 0.6pt,line join=round,fill=fillColor] ( 11.47, 18.46) rectangle ( 29.83,146.72);
\definecolor{drawColor}{RGB}{157,187,255}
\definecolor{fillColor}{RGB}{157,187,255}

\path[draw=drawColor,line width= 0.6pt,line join=round,fill=fillColor] ( 32.12, 18.46) rectangle ( 50.47,101.44);
\definecolor{drawColor}{RGB}{200,200,200}
\definecolor{fillColor}{RGB}{200,200,200}

\path[draw=drawColor,line width= 0.6pt,line join=round,fill=fillColor] ( 52.77, 18.46) rectangle ( 71.12, 77.50);
\definecolor{drawColor}{RGB}{0,0,255}
\definecolor{fillColor}{RGB}{0,0,255}

\path[draw=drawColor,line width= 0.6pt,line join=round,fill=fillColor] ( 80.30, 18.46) rectangle ( 98.65,145.32);
\definecolor{drawColor}{RGB}{157,187,255}
\definecolor{fillColor}{RGB}{157,187,255}

\path[draw=drawColor,line width= 0.6pt,line join=round,fill=fillColor] (100.95, 18.46) rectangle (119.30,107.14);
\definecolor{drawColor}{RGB}{200,200,200}
\definecolor{fillColor}{RGB}{200,200,200}

\path[draw=drawColor,line width= 0.6pt,line join=round,fill=fillColor] (121.60, 18.46) rectangle (139.95, 79.19);
\definecolor{drawColor}{RGB}{0,0,255}
\definecolor{fillColor}{RGB}{0,0,255}

\path[draw=drawColor,line width= 0.6pt,line join=round,fill=fillColor] (149.13, 18.46) rectangle (167.48,149.47);
\definecolor{drawColor}{RGB}{157,187,255}
\definecolor{fillColor}{RGB}{157,187,255}

\path[draw=drawColor,line width= 0.6pt,line join=round,fill=fillColor] (169.78, 18.46) rectangle (188.13,110.52);
\definecolor{drawColor}{RGB}{200,200,200}
\definecolor{fillColor}{RGB}{200,200,200}

\path[draw=drawColor,line width= 0.6pt,line join=round,fill=fillColor] (190.43, 18.46) rectangle (208.78, 89.81);
\definecolor{drawColor}{RGB}{0,0,255}
\definecolor{fillColor}{RGB}{0,0,255}

\path[draw=drawColor,line width= 0.6pt,line join=round,fill=fillColor] (217.96, 18.46) rectangle (236.31,159.90);
\definecolor{drawColor}{RGB}{157,187,255}
\definecolor{fillColor}{RGB}{157,187,255}

\path[draw=drawColor,line width= 0.6pt,line join=round,fill=fillColor] (238.61, 18.46) rectangle (256.96,147.40);
\definecolor{drawColor}{RGB}{200,200,200}
\definecolor{fillColor}{RGB}{200,200,200}

\path[draw=drawColor,line width= 0.6pt,line join=round,fill=fillColor] (259.25, 18.46) rectangle (277.61, 89.42);
\definecolor{drawColor}{RGB}{0,0,0}

\path[draw=drawColor,line width= 0.6pt,line join=round] (  0.00, 18.46) -- (289.08, 18.46);

\node[text=drawColor,rotate= 90.00,anchor=base west,inner sep=0pt, outer sep=0pt, scale=  0.83] at ( 23.88,149.99) {2,657};

\node[text=drawColor,rotate= 90.00,anchor=base west,inner sep=0pt, outer sep=0pt, scale=  0.83] at ( 44.53,104.71) {1,719};

\node[text=drawColor,rotate= 90.00,anchor=base west,inner sep=0pt, outer sep=0pt, scale=  0.83] at ( 65.18, 80.77) {1,223};

\node[text=drawColor,rotate= 90.00,anchor=base west,inner sep=0pt, outer sep=0pt, scale=  0.83] at ( 92.71,148.59) {2,628};

\node[text=drawColor,rotate= 90.00,anchor=base west,inner sep=0pt, outer sep=0pt, scale=  0.83] at (113.36,110.41) {1,837};

\node[text=drawColor,rotate= 90.00,anchor=base west,inner sep=0pt, outer sep=0pt, scale=  0.83] at (134.01, 82.46) {1,258};

\node[text=drawColor,rotate= 90.00,anchor=base west,inner sep=0pt, outer sep=0pt, scale=  0.83] at (161.54,152.75) {2,714};

\node[text=drawColor,rotate= 90.00,anchor=base west,inner sep=0pt, outer sep=0pt, scale=  0.83] at (182.19,113.79) {1,907};

\node[text=drawColor,rotate= 90.00,anchor=base west,inner sep=0pt, outer sep=0pt, scale=  0.83] at (202.84, 93.08) {1,478};

\node[text=drawColor,rotate= 90.00,anchor=base west,inner sep=0pt, outer sep=0pt, scale=  0.83] at (230.37,163.17) {2,930};

\node[text=drawColor,rotate= 90.00,anchor=base west,inner sep=0pt, outer sep=0pt, scale=  0.83] at (251.02,150.67) {2,671};

\node[text=drawColor,rotate= 90.00,anchor=base west,inner sep=0pt, outer sep=0pt, scale=  0.83] at (271.67, 92.69) {1,470};

\path[] (  0.00, 18.46) rectangle (289.08,159.90);
\end{scope}
\begin{scope}
\path[clip] (  0.00,  0.00) rectangle (289.08,198.74);

\path[] (  0.00, 18.46) --
	(289.08, 18.46);
\end{scope}
\begin{scope}
\path[clip] (  0.00,  0.00) rectangle (289.08,198.74);

\path[] ( 41.30, 15.71) --
	( 41.30, 18.46);

\path[] (110.13, 15.71) --
	(110.13, 18.46);

\path[] (178.95, 15.71) --
	(178.95, 18.46);

\path[] (247.78, 15.71) --
	(247.78, 18.46);
\end{scope}
\begin{scope}
\path[clip] (  0.00,  0.00) rectangle (289.08,198.74);
\definecolor{drawColor}{RGB}{0,0,0}

\node[text=drawColor,anchor=base,inner sep=0pt, outer sep=0pt, scale=  1.00] at ( 41.30,  5.69) {2011};

\node[text=drawColor,anchor=base,inner sep=0pt, outer sep=0pt, scale=  1.00] at (110.13,  5.69) {2012};

\node[text=drawColor,anchor=base,inner sep=0pt, outer sep=0pt, scale=  1.00] at (178.95,  5.69) {2013};

\node[text=drawColor,anchor=base,inner sep=0pt, outer sep=0pt, scale=  1.00] at (247.78,  5.69) {2014};
\end{scope}
\begin{scope}
\path[clip] (  0.00,  0.00) rectangle (289.08,198.74);
\coordinate (apoyo1) at (8.27,188.85);
\coordinate (apoyo2) at (115.05,188.85);
\coordinate (apoyo3) at (220.86,188.85);
\coordinate (longitudFicticia) at (7.11,9.89);
\coordinate (longitud) at (7.11,7.11);
\coordinate (desY) at (0,1.39);
\definecolor[named]{ct1}{HTML}{
0000FF
}
\definecolor[named]{ct2}{HTML}{
9DBBFF
}
\definecolor[named]{ct3}{HTML}{
C8C8C8
}
\definecolor[named]{ctb1}{HTML}{
0000FF
}
\definecolor[named]{ctb2}{HTML}{
9DBBFF
}
\definecolor[named]{ctb3}{HTML}{
C8C8C8
}
\path [fill=none] (apoyo1) rectangle ($(apoyo1)+(longitudFicticia)$)
node [xshift=0.3cm,inner sep=0pt, outer sep=0pt,text width=1.33333333333333in,midway,right,scale = 0.9]{Urbano Metropolitano};
\draw [color= ctb1, fill=ct1] ( $(apoyo1)  + (desY) $) rectangle ($(apoyo1)+ (desY) +(longitud)$);
\path [fill=none] (apoyo2) rectangle ($(apoyo2)+(longitudFicticia)$)
node [xshift=0.3cm,inner sep=0pt, outer sep=0pt,text width=1.33333333333333in,midway,right,scale = 0.9]{Resto Urbano};
\draw [color = ctb2, fill=ct2] ( $(apoyo2)  + (desY) $) rectangle ($(apoyo2)+ (desY) +(longitud)$);
\path [fill=none] (apoyo3) rectangle ($(apoyo3)+(longitudFicticia)$)
node [xshift=0.3cm,inner sep=0pt, outer sep=0pt,text width=1.33333333333333in,midway,right,scale = 0.9]{Rural};
\path [color = ctb3, fill=ct3] ( $(apoyo3)  + (desY) $) rectangle ($(apoyo3)+ (desY) +(longitud)$);
\end{scope}
  \end{tikzpicture}}%
{%
	ENEI 2014, 2013, 2012 y 2011} %

%#########################4########################

\cajita{%
	Ingreso laboral mensual según etnia}%
{%
}%
{%
	Promedio del ingreso laboral mensual según etnia} %
{%
	República de Guatemala, serie histórica de ENEI, en quetzales } %
{%
	\begin{tikzpicture}[x=1pt,y=1pt]  % Created by tikzDevice version 0.9 on 2016-03-03 04:34:30
% !TEX encoding = UTF-8 Unicode
\definecolor{fillColor}{RGB}{255,255,255}
\path[use as bounding box,fill=fillColor,fill opacity=0.00] (0,0) rectangle (289.08,198.74);
\begin{scope}
\path[clip] (  0.00,  0.00) rectangle (289.08,198.74);

\path[] (  0.00,  0.00) rectangle (289.08,198.74);
\end{scope}
\begin{scope}
\path[clip] (  0.00,  0.00) rectangle (289.08,198.74);

\path[] (  0.00, 18.46) rectangle (289.08,159.84);

\path[] ( 41.30, 18.46) --
	( 41.30,159.84);

\path[] (110.13, 18.46) --
	(110.13,159.84);

\path[] (178.95, 18.46) --
	(178.95,159.84);

\path[] (247.78, 18.46) --
	(247.78,159.84);
\definecolor{drawColor}{RGB}{0,0,255}
\definecolor{fillColor}{RGB}{0,0,255}

\path[draw=drawColor,line width= 0.6pt,line join=round,fill=fillColor] ( 12.05, 18.46) rectangle ( 39.58, 85.19);
\definecolor{drawColor}{RGB}{157,187,255}
\definecolor{fillColor}{RGB}{157,187,255}

\path[draw=drawColor,line width= 0.6pt,line join=round,fill=fillColor] ( 43.02, 18.46) rectangle ( 70.55,129.20);
\definecolor{drawColor}{RGB}{0,0,255}
\definecolor{fillColor}{RGB}{0,0,255}

\path[draw=drawColor,line width= 0.6pt,line join=round,fill=fillColor] ( 80.87, 18.46) rectangle (108.40, 86.08);
\definecolor{drawColor}{RGB}{157,187,255}
\definecolor{fillColor}{RGB}{157,187,255}

\path[draw=drawColor,line width= 0.6pt,line join=round,fill=fillColor] (111.85, 18.46) rectangle (139.38,132.99);
\definecolor{drawColor}{RGB}{0,0,255}
\definecolor{fillColor}{RGB}{0,0,255}

\path[draw=drawColor,line width= 0.6pt,line join=round,fill=fillColor] (149.70, 18.46) rectangle (177.23, 94.61);
\definecolor{drawColor}{RGB}{157,187,255}
\definecolor{fillColor}{RGB}{157,187,255}

\path[draw=drawColor,line width= 0.6pt,line join=round,fill=fillColor] (180.68, 18.46) rectangle (208.21,143.01);
\definecolor{drawColor}{RGB}{0,0,255}
\definecolor{fillColor}{RGB}{0,0,255}

\path[draw=drawColor,line width= 0.6pt,line join=round,fill=fillColor] (218.53, 18.46) rectangle (246.06,105.41);
\definecolor{drawColor}{RGB}{157,187,255}
\definecolor{fillColor}{RGB}{157,187,255}

\path[draw=drawColor,line width= 0.6pt,line join=round,fill=fillColor] (249.50, 18.46) rectangle (277.03,159.84);
\definecolor{drawColor}{RGB}{0,0,0}

\path[draw=drawColor,line width= 0.6pt,line join=round] (  0.00, 18.46) -- (289.08, 18.46);

\node[text=drawColor,rotate= 90.00,anchor=base west,inner sep=0pt, outer sep=0pt, scale=  0.83] at ( 29.05, 88.47) {1,198};

\node[text=drawColor,rotate= 90.00,anchor=base west,inner sep=0pt, outer sep=0pt, scale=  0.83] at ( 60.02,132.47) {1,988};

\node[text=drawColor,rotate= 90.00,anchor=base west,inner sep=0pt, outer sep=0pt, scale=  0.83] at ( 97.87, 89.36) {1,214};

\node[text=drawColor,rotate= 90.00,anchor=base west,inner sep=0pt, outer sep=0pt, scale=  0.83] at (128.85,136.26) {2,056};

\node[text=drawColor,rotate= 90.00,anchor=base west,inner sep=0pt, outer sep=0pt, scale=  0.83] at (166.70, 97.88) {1,367};

\node[text=drawColor,rotate= 90.00,anchor=base west,inner sep=0pt, outer sep=0pt, scale=  0.83] at (197.68,146.29) {2,236};

\node[text=drawColor,rotate= 90.00,anchor=base west,inner sep=0pt, outer sep=0pt, scale=  0.83] at (235.53,108.69) {1,561};

\node[text=drawColor,rotate= 90.00,anchor=base west,inner sep=0pt, outer sep=0pt, scale=  0.83] at (266.50,163.11) {2,538};

\path[] (  0.00, 18.46) rectangle (289.08,159.84);
\end{scope}
\begin{scope}
\path[clip] (  0.00,  0.00) rectangle (289.08,198.74);

\path[] (  0.00, 18.46) --
	(289.08, 18.46);
\end{scope}
\begin{scope}
\path[clip] (  0.00,  0.00) rectangle (289.08,198.74);

\path[] ( 41.30, 15.71) --
	( 41.30, 18.46);

\path[] (110.13, 15.71) --
	(110.13, 18.46);

\path[] (178.95, 15.71) --
	(178.95, 18.46);

\path[] (247.78, 15.71) --
	(247.78, 18.46);
\end{scope}
\begin{scope}
\path[clip] (  0.00,  0.00) rectangle (289.08,198.74);
\definecolor{drawColor}{RGB}{0,0,0}

\node[text=drawColor,anchor=base,inner sep=0pt, outer sep=0pt, scale=  1.00] at ( 41.30,  5.69) {2011};

\node[text=drawColor,anchor=base,inner sep=0pt, outer sep=0pt, scale=  1.00] at (110.13,  5.69) {2012};

\node[text=drawColor,anchor=base,inner sep=0pt, outer sep=0pt, scale=  1.00] at (178.95,  5.69) {2013};

\node[text=drawColor,anchor=base,inner sep=0pt, outer sep=0pt, scale=  1.00] at (247.78,  5.69) {2014};
\end{scope}
\begin{scope}
\path[clip] (  0.00,  0.00) rectangle (289.08,198.74);
\coordinate (apoyo) at (55.19,188.79);
\coordinate (longitudFicticia) at (7.11,9.95);
\coordinate (longitud) at (7.11,7.11);
\coordinate (desX) at (133.24,0);
\coordinate (desY) at (0,1.42);
\definecolor[named]{ct1}{HTML}{
0000FF
}
\definecolor[named]{ct2}{HTML}{
9DBBFF
}
\definecolor[named]{ctb1}{HTML}{
0000FF
}
\definecolor[named]{ctb2}{HTML}{
9DBBFF
}
\path [fill=none] (apoyo) rectangle ($(apoyo)+(longitudFicticia)$)
node [xshift=0.3cm,inner sep=0pt, outer sep=0pt,midway,right,scale = 0.9]{Indígena};
\draw [color = ctb1,fill=ct1] ( $(apoyo)  + (desY) $) rectangle ($(apoyo)+ (desY) +(longitud)$);
\path [fill=none] ($(apoyo)+(desX)$) rectangle ($(apoyo)+(desX)+(longitudFicticia)$)
node [xshift=0.3cm,inner sep=0pt, outer sep=0pt,midway,right,scale = 0.9]{No indígena};
\draw [color = ctb2 ,fill=ct2] ( $(apoyo)  + (desY) + (desX) $) rectangle ($(apoyo)+ (desY)+ (desX) +(longitud)$);
\end{scope}
  \end{tikzpicture}}%
{%
	ENEI 2014, 2013, 2012 y 2011} %


%#########################5########################

\cajita{%
	Ingreso de afiliados al IGSS}%
{%
}%
{%
	Promedio del ingreso laboral de trabajadores con seguro social } %
{%
	República de Guatemala, serie histórica, en quetzales } %
{%
	\begin{tikzpicture}[x=1pt,y=1pt]  % Created by tikzDevice version 0.9 on 2016-03-03 04:34:47
% !TEX encoding = UTF-8 Unicode
\definecolor{fillColor}{RGB}{255,255,255}
\path[use as bounding box,fill=fillColor,fill opacity=0.00] (0,0) rectangle (289.08,198.74);
\begin{scope}
\path[clip] (  0.00,  0.00) rectangle (289.08,198.74);

\path[] (  0.00,  0.00) rectangle (289.08,198.74);
\end{scope}
\begin{scope}
\path[clip] (  0.00,  0.00) rectangle (289.08,198.74);

\path[] (  8.24, 15.61) rectangle (280.54,191.48);

\path[] (  8.24, 35.67) --
	(280.54, 35.67);

\path[] (  8.24, 84.97) --
	(280.54, 84.97);

\path[] (  8.24,134.27) --
	(280.54,134.27);

\path[] (  8.24,183.58) --
	(280.54,183.58);

\path[] (  8.24, 60.32) --
	(280.54, 60.32);

\path[] (  8.24,109.62) --
	(280.54,109.62);

\path[] (  8.24,158.93) --
	(280.54,158.93);

\path[] ( 47.14, 15.61) --
	( 47.14,191.48);

\path[] (111.97, 15.61) --
	(111.97,191.48);

\path[] (176.81, 15.61) --
	(176.81,191.48);

\path[] (241.64, 15.61) --
	(241.64,191.48);
\definecolor{drawColor}{RGB}{0,0,255}

\path[draw=drawColor,line width= 1.7pt,line join=round] ( 47.14, 60.50) --
	(111.97,111.26) --
	(176.81,140.28) --
	(241.64,183.49);
\definecolor{drawColor}{RGB}{0,0,0}

\node[text=drawColor,anchor=base,inner sep=0pt, outer sep=0pt, scale=  1.02] at ( 47.14, 48.59) {3,250.9};

\node[text=drawColor,anchor=base east,inner sep=0pt, outer sep=0pt, scale=  1.02] at (106.61,111.26) {3,508.3};

\node[text=drawColor,anchor=base east,inner sep=0pt, outer sep=0pt, scale=  1.02] at (171.44,140.28) {3,655.4};

\node[text=drawColor,anchor=base,inner sep=0pt, outer sep=0pt, scale=  1.02] at (241.64,187.46) {3,874.6};

\path[draw=drawColor,line width= 0.1pt,line join=round] (  8.24, 23.61) -- (280.54, 23.61);

\path[] (  8.24, 15.61) rectangle (280.54,191.48);
\end{scope}
\begin{scope}
\path[clip] (  0.00,  0.00) rectangle (289.08,198.74);

\path[] (  8.24, 15.61) --
	(  8.24,191.48);
\end{scope}
\begin{scope}
\path[clip] (  0.00,  0.00) rectangle (289.08,198.74);
\definecolor{drawColor}{RGB}{255,255,255}

\node[text=drawColor,text opacity=0.00,anchor=base east,inner sep=0pt, outer sep=0pt, scale=  1.00] at (  3.29, 56.41) {3250};

\node[text=drawColor,text opacity=0.00,anchor=base east,inner sep=0pt, outer sep=0pt, scale=  1.00] at (  3.29,105.71) {3500};

\node[text=drawColor,text opacity=0.00,anchor=base east,inner sep=0pt, outer sep=0pt, scale=  1.00] at (  3.29,155.02) {3750};
\end{scope}
\begin{scope}
\path[clip] (  0.00,  0.00) rectangle (289.08,198.74);

\path[] (  5.49, 60.32) --
	(  8.24, 60.32);

\path[] (  5.49,109.62) --
	(  8.24,109.62);

\path[] (  5.49,158.93) --
	(  8.24,158.93);
\end{scope}
\begin{scope}
\path[clip] (  0.00,  0.00) rectangle (289.08,198.74);

\path[] (  8.24, 15.61) --
	(280.54, 15.61);
\end{scope}
\begin{scope}
\path[clip] (  0.00,  0.00) rectangle (289.08,198.74);

\path[] ( 47.14, 12.86) --
	( 47.14, 15.61);

\path[] (111.97, 12.86) --
	(111.97, 15.61);

\path[] (176.81, 12.86) --
	(176.81, 15.61);

\path[] (241.64, 12.86) --
	(241.64, 15.61);
\end{scope}
\begin{scope}
\path[clip] (  0.00,  0.00) rectangle (289.08,198.74);
\definecolor{drawColor}{RGB}{0,0,0}

\node[text=drawColor,anchor=base,inner sep=0pt, outer sep=0pt, scale=  1.00] at ( 47.14,  2.85) {2011};

\node[text=drawColor,anchor=base,inner sep=0pt, outer sep=0pt, scale=  1.00] at (111.97,  2.85) {2012};

\node[text=drawColor,anchor=base,inner sep=0pt, outer sep=0pt, scale=  1.00] at (176.81,  2.85) {2013};

\node[text=drawColor,anchor=base,inner sep=0pt, outer sep=0pt, scale=  1.00] at (241.64,  2.85) {2014};
\end{scope}
  \end{tikzpicture}}%
{%
	Instituto Guatemalteco de Seguridad Social} %


%#########################6########################

\cajita{%
	Ingreso de afiliados al IGSS según rama de actividad}%
{%
}%
{%
	Promedio del ingreso laboral de trabajadores con seguro social según rama de actividad económica} %
{%
	República de Guatemala, 2014, en quetzales } %
{%
	\begin{tikzpicture}[x=1pt,y=1pt]  % Created by tikzDevice version 0.9 on 2016-03-03 04:35:00
% !TEX encoding = UTF-8 Unicode
\definecolor{fillColor}{RGB}{255,255,255}
\path[use as bounding box,fill=fillColor,fill opacity=0.00] (0,0) rectangle (289.08,198.74);
\begin{scope}
\path[clip] (  0.00,  0.00) rectangle (289.08,198.74);

\path[] (  0.00,  0.00) rectangle (289.08,198.74);
\end{scope}
\begin{scope}
\path[clip] (  0.00,  0.00) rectangle (289.08,198.74);

\path[] ( 60.20,  0.00) rectangle (256.03,198.74);

\path[] ( 60.20, 14.54) --
	(256.03, 14.54);

\path[] ( 60.20, 38.78) --
	(256.03, 38.78);

\path[] ( 60.20, 63.02) --
	(256.03, 63.02);

\path[] ( 60.20, 87.25) --
	(256.03, 87.25);

\path[] ( 60.20,111.49) --
	(256.03,111.49);

\path[] ( 60.20,135.73) --
	(256.03,135.73);

\path[] ( 60.20,159.96) --
	(256.03,159.96);

\path[] ( 60.20,184.20) --
	(256.03,184.20);
\definecolor{drawColor}{RGB}{0,0,255}
\definecolor{fillColor}{RGB}{0,0,255}

\path[draw=drawColor,line width= 0.6pt,line join=round,fill=fillColor] ( 60.20,  7.27) rectangle (156.23, 21.81);

\path[draw=drawColor,line width= 0.6pt,line join=round,fill=fillColor] ( 60.20, 31.51) rectangle (165.88, 46.05);

\path[draw=drawColor,line width= 0.6pt,line join=round,fill=fillColor] ( 60.20, 55.74) rectangle (158.08, 70.29);

\path[draw=drawColor,line width= 0.6pt,line join=round,fill=fillColor] ( 60.20, 79.98) rectangle (228.85, 94.52);

\path[draw=drawColor,line width= 0.6pt,line join=round,fill=fillColor] ( 60.20,104.22) rectangle (133.25,118.76);

\path[draw=drawColor,line width= 0.6pt,line join=round,fill=fillColor] ( 60.20,128.46) rectangle (156.32,143.00);

\path[draw=drawColor,line width= 0.6pt,line join=round,fill=fillColor] ( 60.20,152.69) rectangle (256.03,167.23);

\path[draw=drawColor,line width= 0.6pt,line join=round,fill=fillColor] ( 60.20,176.93) rectangle (118.90,191.47);
\definecolor{drawColor}{RGB}{0,0,0}

\path[draw=drawColor,line width= 0.1pt,line join=round] ( 60.20,  0.00) -- ( 60.20,198.74);

\node[text=drawColor,anchor=base west,inner sep=0pt, outer sep=0pt, scale=  1.02] at (161.60, 10.57) {3,996.9};

\node[text=drawColor,anchor=base west,inner sep=0pt, outer sep=0pt, scale=  1.02] at (171.25, 34.81) {4,398.5};

\node[text=drawColor,anchor=base west,inner sep=0pt, outer sep=0pt, scale=  1.02] at (163.45, 59.04) {4,073.9};

\node[text=drawColor,anchor=base west,inner sep=0pt, outer sep=0pt, scale=  1.02] at (234.21, 83.28) {7,019.2};

\node[text=drawColor,anchor=base west,inner sep=0pt, outer sep=0pt, scale=  1.02] at (138.62,107.52) {3,040.5};

\node[text=drawColor,anchor=base west,inner sep=0pt, outer sep=0pt, scale=  1.02] at (161.68,131.76) {4,000.5};

\node[text=drawColor,anchor=base west,inner sep=0pt, outer sep=0pt, scale=  1.02] at (261.40,155.99) {8,150.6};

\node[text=drawColor,anchor=base west,inner sep=0pt, outer sep=0pt, scale=  1.02] at (124.27,180.23) {2,443.3};

\path[] ( 60.20,  0.00) rectangle (256.03,198.74);
\end{scope}
\begin{scope}
\path[clip] (  0.00,  0.00) rectangle (289.08,198.74);

\path[] ( 60.20,  0.00) --
	( 60.20,198.74);
\end{scope}
\begin{scope}
\path[clip] (  0.00,  0.00) rectangle (289.08,198.74);
\definecolor{drawColor}{RGB}{0,0,0}

\node[text=drawColor,anchor=base east,inner sep=0pt, outer sep=0pt, scale=  1.00] at ( 57.45, 10.63) {Servicios};

\node[text=drawColor,anchor=base east,inner sep=0pt, outer sep=0pt, scale=  1.00] at ( 57.45, 34.87) {Comunicaciones};

\node[text=drawColor,anchor=base east,inner sep=0pt, outer sep=0pt, scale=  1.00] at ( 57.45, 59.11) {Comercio};

\node[text=drawColor,anchor=base east,inner sep=0pt, outer sep=0pt, scale=  1.00] at ( 57.45, 83.34) {Electricidad};

\node[text=drawColor,anchor=base east,inner sep=0pt, outer sep=0pt, scale=  1.00] at ( 57.45,107.58) {Construcción};

\node[text=drawColor,anchor=base east,inner sep=0pt, outer sep=0pt, scale=  1.00] at ( 57.45,131.82) {Manufactura};

\node[text=drawColor,anchor=base east,inner sep=0pt, outer sep=0pt, scale=  1.00] at ( 57.45,156.06) {Minería};

\node[text=drawColor,anchor=base east,inner sep=0pt, outer sep=0pt, scale=  1.00] at ( 57.45,180.29) {Agricultura};
\end{scope}
\begin{scope}
\path[clip] (  0.00,  0.00) rectangle (289.08,198.74);

\path[] ( 57.45, 14.54) --
	( 60.20, 14.54);

\path[] ( 57.45, 38.78) --
	( 60.20, 38.78);

\path[] ( 57.45, 63.02) --
	( 60.20, 63.02);

\path[] ( 57.45, 87.25) --
	( 60.20, 87.25);

\path[] ( 57.45,111.49) --
	( 60.20,111.49);

\path[] ( 57.45,135.73) --
	( 60.20,135.73);

\path[] ( 57.45,159.96) --
	( 60.20,159.96);

\path[] ( 57.45,184.20) --
	( 60.20,184.20);
\end{scope}
  \end{tikzpicture}}%
{%
	Instituto Guatemalteco de Seguridad Social} %

  %%%%%%%%%%%%%%%%%%%%%%%%%%%%%%%%%%7%%%%%%%%%%%%%%%%%%%%%%%%
  
  \cajota{%
  	Ingreso de afiliados al IGSS por departamento }%
  {%
  }%
  {%
  	Promedio del ingreso laboral de trabajadores con seguro social por departamento de residencia
  } %
  {%
  	República de Guatemala, por departamentos 2014, en quetzales} %
  {%
  	\includegraphics[width=52\cuadri]{graficas/3_06.pdf}}%
  {%
  	Instituto Guatemalteco de Seguridad Social} %

%#########################8########################

\cajita{%
	Costo de la canasta básica}%
{%
}%
{%
Costo promedio diario de la canasta básica } %
{%
	República de Guatemala, serie histórica, en quetzales } %
{%
	\begin{tikzpicture}[x=1pt,y=1pt]  % Created by tikzDevice version 0.9 on 2016-03-03 04:35:21
% !TEX encoding = UTF-8 Unicode
\definecolor{fillColor}{RGB}{255,255,255}
\path[use as bounding box,fill=fillColor,fill opacity=0.00] (0,0) rectangle (289.08,198.74);
\begin{scope}
\path[clip] (  0.00,  0.00) rectangle (289.08,198.74);

\path[] (  0.00,  0.00) rectangle (289.08,198.74);
\end{scope}
\begin{scope}
\path[clip] (  0.00,  0.00) rectangle (289.08,198.74);

\path[] (  3.86, 15.61) rectangle (280.54,191.48);

\path[] (  3.86, 18.78) --
	(280.54, 18.78);

\path[] (  3.86, 52.84) --
	(280.54, 52.84);

\path[] (  3.86, 86.90) --
	(280.54, 86.90);

\path[] (  3.86,120.96) --
	(280.54,120.96);

\path[] (  3.86,155.02) --
	(280.54,155.02);

\path[] (  3.86,189.08) --
	(280.54,189.08);

\path[] (  3.86, 35.81) --
	(280.54, 35.81);

\path[] (  3.86, 69.87) --
	(280.54, 69.87);

\path[] (  3.86,103.93) --
	(280.54,103.93);

\path[] (  3.86,137.99) --
	(280.54,137.99);

\path[] (  3.86,172.05) --
	(280.54,172.05);

\path[] ( 35.79, 15.61) --
	( 35.79,191.48);

\path[] ( 88.99, 15.61) --
	( 88.99,191.48);

\path[] (142.20, 15.61) --
	(142.20,191.48);

\path[] (195.41, 15.61) --
	(195.41,191.48);

\path[] (248.62, 15.61) --
	(248.62,191.48);
\definecolor{drawColor}{RGB}{0,0,255}

\path[draw=drawColor,line width= 1.7pt,line join=round] ( 35.79, 60.50) --
	( 88.99, 86.42) --
	(142.20,114.46) --
	(195.41,144.18) --
	(248.62,183.49);
\definecolor{drawColor}{RGB}{0,0,0}

\node[text=drawColor,anchor=base,inner sep=0pt, outer sep=0pt, scale=  1.02] at ( 35.79, 48.59) {77.2};

\node[text=drawColor,anchor=base east,inner sep=0pt, outer sep=0pt, scale=  1.02] at ( 85.87, 86.42) {84.9};

\node[text=drawColor,anchor=base east,inner sep=0pt, outer sep=0pt, scale=  1.02] at (139.07,114.46) {93.1};

\node[text=drawColor,anchor=base east,inner sep=0pt, outer sep=0pt, scale=  1.02] at (191.39,144.18) {101.8};

\node[text=drawColor,anchor=base,inner sep=0pt, outer sep=0pt, scale=  1.02] at (248.62,187.46) {113.4};

\path[draw=drawColor,line width= 0.1pt,line join=round] (  3.86, 23.61) -- (280.54, 23.61);

\path[] (  3.86, 15.61) rectangle (280.54,191.48);
\end{scope}
\begin{scope}
\path[clip] (  0.00,  0.00) rectangle (289.08,198.74);

\path[] (  3.86, 15.61) --
	(  3.86,191.48);
\end{scope}
\begin{scope}
\path[clip] (  0.00,  0.00) rectangle (289.08,198.74);

\path[] (  1.11, 35.81) --
	(  3.86, 35.81);

\path[] (  1.11, 69.87) --
	(  3.86, 69.87);

\path[] (  1.11,103.93) --
	(  3.86,103.93);

\path[] (  1.11,137.99) --
	(  3.86,137.99);

\path[] (  1.11,172.05) --
	(  3.86,172.05);
\end{scope}
\begin{scope}
\path[clip] (  0.00,  0.00) rectangle (289.08,198.74);

\path[] (  3.86, 15.61) --
	(280.54, 15.61);
\end{scope}
\begin{scope}
\path[clip] (  0.00,  0.00) rectangle (289.08,198.74);

\path[] ( 35.79, 12.86) --
	( 35.79, 15.61);

\path[] ( 88.99, 12.86) --
	( 88.99, 15.61);

\path[] (142.20, 12.86) --
	(142.20, 15.61);

\path[] (195.41, 12.86) --
	(195.41, 15.61);

\path[] (248.62, 12.86) --
	(248.62, 15.61);
\end{scope}
\begin{scope}
\path[clip] (  0.00,  0.00) rectangle (289.08,198.74);
\definecolor{drawColor}{RGB}{0,0,0}

\node[text=drawColor,anchor=base,inner sep=0pt, outer sep=0pt, scale=  1.00] at ( 35.79,  2.85) {2011};

\node[text=drawColor,anchor=base,inner sep=0pt, outer sep=0pt, scale=  1.00] at ( 88.99,  2.85) {2012};

\node[text=drawColor,anchor=base,inner sep=0pt, outer sep=0pt, scale=  1.00] at (142.20,  2.85) {2013};

\node[text=drawColor,anchor=base,inner sep=0pt, outer sep=0pt, scale=  1.00] at (195.41,  2.85) {2014};

\node[text=drawColor,anchor=base,inner sep=0pt, outer sep=0pt, scale=  1.00] at (248.62,  2.85) {2015};
\end{scope}
  \end{tikzpicture}}%
{%
	Instituto Nacional de Estadística} %

%#########################9########################

\cajita{%
	Costo de la canasta básica ampliada}%
{%
}%
{%
	Costo promedio diario de la canasta básica ampliada } %
{%
	República de Guatemala, serie histórica, en quetzales } %
{%
	\begin{tikzpicture}[x=1pt,y=1pt]  % Created by tikzDevice version 0.9 on 2016-03-03 04:35:29
% !TEX encoding = UTF-8 Unicode
\definecolor{fillColor}{RGB}{255,255,255}
\path[use as bounding box,fill=fillColor,fill opacity=0.00] (0,0) rectangle (289.08,198.74);
\begin{scope}
\path[clip] (  0.00,  0.00) rectangle (289.08,198.74);

\path[] (  0.00,  0.00) rectangle (289.08,198.74);
\end{scope}
\begin{scope}
\path[clip] (  0.00,  0.00) rectangle (289.08,198.74);

\path[] (  3.86, 15.61) rectangle (280.54,191.48);

\path[] (  3.86, 39.89) --
	(280.54, 39.89);

\path[] (  3.86, 77.26) --
	(280.54, 77.26);

\path[] (  3.86,114.62) --
	(280.54,114.62);

\path[] (  3.86,151.99) --
	(280.54,151.99);

\path[] (  3.86,189.36) --
	(280.54,189.36);

\path[] (  3.86, 21.20) --
	(280.54, 21.20);

\path[] (  3.86, 58.57) --
	(280.54, 58.57);

\path[] (  3.86, 95.94) --
	(280.54, 95.94);

\path[] (  3.86,133.31) --
	(280.54,133.31);

\path[] (  3.86,170.67) --
	(280.54,170.67);

\path[] ( 35.79, 15.61) --
	( 35.79,191.48);

\path[] ( 88.99, 15.61) --
	( 88.99,191.48);

\path[] (142.20, 15.61) --
	(142.20,191.48);

\path[] (195.41, 15.61) --
	(195.41,191.48);

\path[] (248.62, 15.61) --
	(248.62,191.48);
\definecolor{drawColor}{RGB}{0,0,255}

\path[draw=drawColor,line width= 1.7pt,line join=round] ( 35.79, 60.50) --
	( 88.99, 86.30) --
	(142.20,114.41) --
	(195.41,144.13) --
	(248.62,183.49);
\definecolor{drawColor}{RGB}{0,0,0}

\node[text=drawColor,anchor=base,inner sep=0pt, outer sep=0pt, scale=  1.02] at ( 35.79, 48.59) {141.0};

\node[text=drawColor,anchor=base east,inner sep=0pt, outer sep=0pt, scale=  1.02] at ( 84.98, 86.30) {154.8};

\node[text=drawColor,anchor=base east,inner sep=0pt, outer sep=0pt, scale=  1.02] at (138.18,114.41) {169.9};

\node[text=drawColor,anchor=base east,inner sep=0pt, outer sep=0pt, scale=  1.02] at (191.39,144.13) {185.8};

\node[text=drawColor,anchor=base,inner sep=0pt, outer sep=0pt, scale=  1.02] at (248.62,187.46) {206.9};

\path[draw=drawColor,line width= 0.1pt,line join=round] (  3.86, 23.61) -- (280.54, 23.61);

\path[] (  3.86, 15.61) rectangle (280.54,191.48);
\end{scope}
\begin{scope}
\path[clip] (  0.00,  0.00) rectangle (289.08,198.74);

\path[] (  3.86, 15.61) --
	(  3.86,191.48);
\end{scope}
\begin{scope}
\path[clip] (  0.00,  0.00) rectangle (289.08,198.74);

\path[] (  1.11, 21.20) --
	(  3.86, 21.20);

\path[] (  1.11, 58.57) --
	(  3.86, 58.57);

\path[] (  1.11, 95.94) --
	(  3.86, 95.94);

\path[] (  1.11,133.31) --
	(  3.86,133.31);

\path[] (  1.11,170.67) --
	(  3.86,170.67);
\end{scope}
\begin{scope}
\path[clip] (  0.00,  0.00) rectangle (289.08,198.74);

\path[] (  3.86, 15.61) --
	(280.54, 15.61);
\end{scope}
\begin{scope}
\path[clip] (  0.00,  0.00) rectangle (289.08,198.74);

\path[] ( 35.79, 12.86) --
	( 35.79, 15.61);

\path[] ( 88.99, 12.86) --
	( 88.99, 15.61);

\path[] (142.20, 12.86) --
	(142.20, 15.61);

\path[] (195.41, 12.86) --
	(195.41, 15.61);

\path[] (248.62, 12.86) --
	(248.62, 15.61);
\end{scope}
\begin{scope}
\path[clip] (  0.00,  0.00) rectangle (289.08,198.74);
\definecolor{drawColor}{RGB}{0,0,0}

\node[text=drawColor,anchor=base,inner sep=0pt, outer sep=0pt, scale=  1.00] at ( 35.79,  2.85) {2011};

\node[text=drawColor,anchor=base,inner sep=0pt, outer sep=0pt, scale=  1.00] at ( 88.99,  2.85) {2012};

\node[text=drawColor,anchor=base,inner sep=0pt, outer sep=0pt, scale=  1.00] at (142.20,  2.85) {2013};

\node[text=drawColor,anchor=base,inner sep=0pt, outer sep=0pt, scale=  1.00] at (195.41,  2.85) {2014};

\node[text=drawColor,anchor=base,inner sep=0pt, outer sep=0pt, scale=  1.00] at (248.62,  2.85) {2015};
\end{scope}
  \end{tikzpicture}}%
{%
	Instituto Nacional de Estadística} %

	\INEchaptercarta{Consumo de alimentos}{Esta dimensión se refiere a lo que consumen los integrantes de cada hogar, ya sea proveniente de la autoproducción o del intercambio, ayudas o adquisiciones en los mercados, así como preparación y distribución intrafamiliar (Coneval, 2010). Esta dimensión se compone de tres indicadores y once sub-indicadores. 
		
		El indicador \textbf{4.1 Hábitos/costumbres}, se compone de los sub-indicadores: a) Encuesta Nacional de Condiciones de Vida (Encovi), b) Encuesta Nacional de Ingresos y Gastos Familiares (Enigfam), c) investigaciones, instituciones no gubernamentales, y d) Encuesta Nacional de Salud Materno Infantil (Ensmi). De estos sub-indicadores, no se presenta información sobre la Enigfam, ni sobre investigaciones, e instituciones no gubernamentales.
		
		El indicador \textbf{4.2 Educación}, se compone de los siguientes sub-indicadores: a) Educación alimentaria/nutricional por grado, b) existencia de huertos escolares, c) escolaridad de la madre, y d) analfabetismo. De estos sub-indicadores, no se presenta información sobre la educación alimentaria/nutricional por grado y huertos escolares.
		
		Por último, el indicador \textbf{4.3 Orientación al Consumidor}, se compone de los sub-indicadores: a) Programas de orientación por medios masivos, e b) información de etiquetado de alimentos. Para este indicador, se incluyó la información disponible más cercana a los sub-indicadores mencionados.}
	%#########################1########################

 \cajita{%
Lactancia en niños menores de 5 años }%
{%
De acuerdo a los datos de la Encuesta Nacional de Salud Materno-Infantil, en el período de análisis 2008/2009, mas del 90\% de los niños menores de 5 años habían recibido lactancia materna alguna vez.

Sin embargo, en el área urbana el porcentaje fue del 94.4\%, 2.7 puntos porcentuales menor a lo encontrado en niños del área rural.

La región con el mayor porcentaje de niños que no recibió lactancia materna fue el Suroriente, donde fue del 6.3\%.}%
{%
 Proporción de niños menores de cinco años que recibieron lactancia alguna vez, según varias características} %
{%
 República de Guatemala, 2008/2009, en porcentaje} %
{%
\small
%\ra{1.2}$\ $\\
\begin{tabular}{p{4cm}c}\hline
	\rowcolor{color2!0!white}&\\[-3mm]
	{\Bold   Característica}  & \textbf{Alguna vez recibió lactancia}\\
	\hline
	\rowcolor{color1!30!white} &\\[-4mm]
	\rowcolor{color1!30!white}\textbf{Área geográfica}	&		\\
	\rowcolor{color1!0!white}Urbana	&	94.4	\\
	Rural	&	97.1	\\
	\rowcolor{color1!30!white}\textbf{Región}	&		\\
	\rowcolor{color1!0!white}Metropolitana	&	94.2	\\
	Norte	&	95.9	\\
	Nororiente	&	95.4	\\
	Suroriente	&	93.7	\\
	Central	&	96.5	\\
	Suroccidente	&	97.1	\\
	Noroccidente	&	97.3	\\
	Petén	&	97.6	\\
	\rowcolor{color1!30!white}\textbf{Categoría étnica de la madre}	&		\\
	\rowcolor{color1!0!white}Indígena	&	97.1	\\
	Ladino	&	95.2	\\
	\hline
	
	\rowcolor{color1!0!white}	&\\[0.05cm]
\end{tabular}}%
{%
 ENSMI 2008/2009} %


%#########################2########################

\cajita{%
	Duración de la lactancia según área de residencia}%
{%
En cuanto a la duración de la lactancia materna en cualquiera de sus tipos\llamada, según el área geográfica de residencia, esta fue de 19.6 meses para los niños menores de 2 años en el área urbana y de 21.8 meses para el área rural.

\textollamada{Lactancia exclusiva, donde solo reciben pecho. Lactancia completa, pecho y agua.}}%
{%
	Duración de cualquier tipo de lactancia en niños menores de 2 años según área de residencia de la madre} %
{%
	República de Guatemala, 2008/2009, en meses} %
{%
	\begin{tikzpicture}[x=1pt,y=1pt]  \input{graficas/4_02.tex}  \end{tikzpicture}
}%
{%
	ENSMI 2008/2009} %


%#########################3########################

\cajita{%
	Duración de la lactancia según etnia}%
{%
En cuanto a la duración de la lactancia materna en cualquiera de sus tipos\llamada en niños menores de 2 años, según la etnicidad de la madre, esta fue de 22.6 meses para los niños con madres que se autoidentifican como indígenas y de 17.9 para las que se autoidentifican como ladinas.

\textollamada{Lactancia exclusiva, donde solo reciben pecho. Lactancia completa, pecho y agua.}}%
{%
	Duración de cualquier tipo de lactancia en niños menores de 2 años según etnia de la madre} %
{%
	República de Guatemala, 2008/2009, en meses} %
{%
	\begin{tikzpicture}[x=1pt,y=1pt]  \input{graficas/4_03.tex}  \end{tikzpicture}
}%
{%
	ENSMI 2008/2009} %

%#########################4########################

\cajita{%
	Duración de la lactancia por educación}%
{%
De acuerdo a la duración de la lactancia de cualquier tipo en niños menores de 2 años, esta presenta diferencias según el nivel educativo de la madre.

Para los niños con madres sin educación, la duración promedio de lactancia fue de 21.9 meses y para los que tenían madre con educación superior esta fue de 8.1 meses.}%
{%
	Duración de cualquier tipo de lactancia en niños menores de 2 años según el nivel de educación de la madre} %
{%
	República de Guatemala, 2008/2009, en meses} %
{%
	\begin{tikzpicture}[x=1pt,y=1pt]  \input{graficas/4_04.tex}  \end{tikzpicture}
}%
{%
	ENSMI 2008/2009} %


%#########################5########################

\cajita{%
Niños de 0 a 3 meses según lactancia}%
{%
La proporción de niños de 0 a 3 meses del área urbana, según el tipo de lactancia recibida, muestra que el 5.5\% no estaba lactando, el 37.9\% recibía lactancia exclusiva, en el 24.4\% el estado era predominantemente lactancia. El resto de niños del área urbana recibía otros tipos de lactancia\llamada. \textollamada{Se incluyen los niños que lactan y consumen líquidos no lácteos, lacta y consume otra leche y lacta y consume alimentos complementarios.}

Por otro lado, en el área urbana, recibían lactancia exclusiva el 67.4\% de los niños y lactancia predominante el 18.6\%. El 3.3\% no lactaba, 2.2 puntos porcentuales por debajo de los niños en el área urbana.}%
{%
	Proporción de niños de 0 a 3 meses por área de residencia, según principales situaciones de la lactancia
	} %
{%
	República de Guatemala, 2008/2009, en porcentaje} %
{%
	\begin{tikzpicture}[x=1pt,y=1pt]  \input{graficas/4_05.tex}  \end{tikzpicture}
}%
{%
	ENSMI 2008/2009} %


%#########################6########################

\cajita{%
	Niños de 0 a 3 meses según lactancia por etnia}%
{%
En cuanto a los niños de 0 a 3 meses según la etnicidad de la madre, 7 de cada 10 los niños con madre indígena recibían lactancia materna exclusiva y el 14.4\% recibían lactancia predominante. 

%La proporción de niños de 0 a 3 meses del área urbana, según el tipo de lactancia recibida, muestra que el 5.5\% no estaba lactando, el 37.9\% recibía lactancia exclusiva, en el 24.4\% el estado era predominantemente lactancia. El resto de niños del área urbana recibía otros tipos de lactancia\llamada. \textollamada{Acá se incluyen los niños que lactan y consumen líquidos no lácteos, lacta y consume otra leche y lacta y consume alimentos complementarios.}

Por otro lado, los niños con madre con etnicidad no indígena, el 4.5\% no estaba lactando, el 41.3\%  recibía lactancia exclusiva.}%
{%
 Proporción de niños de 0 a 3 meses según etnicidad de la madre, por tipo de lactancia recibido
} %
{%
	República de Guatemala, 2008/2009, en porcentaje} %
{%
	\begin{tikzpicture}[x=1pt,y=1pt]  \input{graficas/4_06.tex}  \end{tikzpicture}
}%
{%
	ENSMI 2008/2009} %


%#########################7########################

\cajita{%
	Niños de 0 a 3 meses según lactancia y educación de la madre}%
{%
De los niños de 0 a 3 meses cuyas madres no tenían educación formal, el 67.3\% recibió lactancia exclusiva, el 22.5\% lactancia predominante y el 3.6\% no lactaba; el resto recibía otros tipos de lactancia.

De las madres con educación superior, el 12.\% daba lactancia de forma predominante, el 5.7\% lactancia exclusiva y el 3.9\% no lactaba. El resto de porcentaje se distribuyen entre los tipos de lactancia que incluyen fórmula, que consumen líquidos no lácteos y otros.}%
{%
	Proporción de niños de 0 a 3 meses por educación de la madre, según situación de lactancia
} %
{%
	República de Guatemala, 2008/2009, en porcentaje} %
{%
	\begin{tikzpicture}[x=1pt,y=1pt]  \input{graficas/4_07.tex}  \end{tikzpicture}
}%
{%
	ENSMI 2008/2009} %

%#########################8########################

\cajita{%
	Niños de 0 a 5 meses según lactancia y área}%
{%
De acuerdo al área geográfica de residencia, el 32.5\% de los niños de 0 a 5 meses cuya madre vivía en área urbana recibía lactancia exclusiva, el 22.6\% lactancia predominante y el 9.9\% no estaba lactando.

Por otro lado, de los niños de 0 a 5 cuya madre residía en el área rural, el 60.4\% recibía lactancia exclusiva.}%
{%
	Proporción de niños de 0 a 5 meses según área geográfica de residencia por situación de lactancia
} %
{%
	República de Guatemala, 2008/2009, en porcentaje} %
{%
	\begin{tikzpicture}[x=1pt,y=1pt]  % Created by tikzDevice version 0.9 on 2016-03-03 04:58:32
% !TEX encoding = UTF-8 Unicode
\definecolor{fillColor}{RGB}{255,255,255}
\path[use as bounding box,fill=fillColor,fill opacity=0.00] (0,0) rectangle (289.08,198.74);
\begin{scope}
\path[clip] (  0.00,  0.00) rectangle (289.08,198.74);

\path[] (  0.00,  0.00) rectangle (289.08,198.74);
\end{scope}
\begin{scope}
\path[clip] (  0.00,  0.00) rectangle (289.08,198.74);

\path[] (  0.00, 18.46) rectangle (289.08,166.57);

\path[] ( 54.20, 18.46) --
	( 54.20,166.57);

\path[] (144.54, 18.46) --
	(144.54,166.57);

\path[] (234.88, 18.46) --
	(234.88,166.57);
\definecolor{drawColor}{RGB}{0,0,255}
\definecolor{fillColor}{RGB}{0,0,255}

\path[draw=drawColor,line width= 0.6pt,line join=round,fill=fillColor] ( 15.81, 18.46) rectangle ( 51.94, 42.73);
\definecolor{drawColor}{RGB}{157,187,255}
\definecolor{fillColor}{RGB}{157,187,255}

\path[draw=drawColor,line width= 0.6pt,line join=round,fill=fillColor] ( 56.46, 18.46) rectangle ( 92.60, 26.80);
\definecolor{drawColor}{RGB}{0,0,255}
\definecolor{fillColor}{RGB}{0,0,255}

\path[draw=drawColor,line width= 0.6pt,line join=round,fill=fillColor] (106.15, 18.46) rectangle (142.28, 98.16);
\definecolor{drawColor}{RGB}{157,187,255}
\definecolor{fillColor}{RGB}{157,187,255}

\path[draw=drawColor,line width= 0.6pt,line join=round,fill=fillColor] (146.80, 18.46) rectangle (182.93,166.57);
\definecolor{drawColor}{RGB}{0,0,255}
\definecolor{fillColor}{RGB}{0,0,255}

\path[draw=drawColor,line width= 0.6pt,line join=round,fill=fillColor] (196.48, 18.46) rectangle (232.62, 73.88);
\definecolor{drawColor}{RGB}{157,187,255}
\definecolor{fillColor}{RGB}{157,187,255}

\path[draw=drawColor,line width= 0.6pt,line join=round,fill=fillColor] (237.14, 18.46) rectangle (273.27, 61.13);
\definecolor{drawColor}{RGB}{0,0,0}

\path[draw=drawColor,line width= 0.6pt,line join=round] (  0.00, 18.46) -- (289.08, 18.46);

\node[text=drawColor,rotate= 90.00,anchor=base west,inner sep=0pt, outer sep=0pt, scale=  0.83] at ( 37.11, 44.56) {9.9};

\node[text=drawColor,rotate= 90.00,anchor=base west,inner sep=0pt, outer sep=0pt, scale=  0.83] at ( 77.76, 28.62) {3.4};

\node[text=drawColor,rotate= 90.00,anchor=base west,inner sep=0pt, outer sep=0pt, scale=  0.83] at (127.45,100.70) {32.5};

\node[text=drawColor,rotate= 90.00,anchor=base west,inner sep=0pt, outer sep=0pt, scale=  0.83] at (168.10,169.12) {60.4};

\node[text=drawColor,rotate= 90.00,anchor=base west,inner sep=0pt, outer sep=0pt, scale=  0.83] at (217.79, 76.43) {22.6};

\node[text=drawColor,rotate= 90.00,anchor=base west,inner sep=0pt, outer sep=0pt, scale=  0.83] at (258.44, 63.68) {17.4};

\path[] (  0.00, 18.46) rectangle (289.08,166.57);
\end{scope}
\begin{scope}
\path[clip] (  0.00,  0.00) rectangle (289.08,198.74);

\path[] (  0.00, 18.46) --
	(289.08, 18.46);
\end{scope}
\begin{scope}
\path[clip] (  0.00,  0.00) rectangle (289.08,198.74);

\path[] ( 54.20, 15.71) --
	( 54.20, 18.46);

\path[] (144.54, 15.71) --
	(144.54, 18.46);

\path[] (234.88, 15.71) --
	(234.88, 18.46);
\end{scope}
\begin{scope}
\path[clip] (  0.00,  0.00) rectangle (289.08,198.74);
\definecolor{drawColor}{RGB}{0,0,0}

\node[text=drawColor,anchor=base,inner sep=0pt, outer sep=0pt, scale=  1.00] at ( 54.20,  5.69) {No lactando};

\node[text=drawColor,anchor=base,inner sep=0pt, outer sep=0pt, scale=  1.00] at (144.54,  5.69) {Lactancia Exclusiva};

\node[text=drawColor,anchor=base,inner sep=0pt, outer sep=0pt, scale=  1.00] at (234.88,  5.69) {Lactancia Predominante};
\end{scope}
\begin{scope}
\path[clip] (  0.00,  0.00) rectangle (289.08,198.74);
\coordinate (apoyo) at (57.27,191.13);
\coordinate (longitudFicticia) at (7.11,7.61);
\coordinate (longitud) at (7.11,7.11);
\coordinate (desX) at (142.24,0);
\coordinate (desY) at (0,0.25);
\definecolor[named]{ct1}{HTML}{
0000FF
}
\definecolor[named]{ct2}{HTML}{
9DBBFF
}
\definecolor[named]{ctb1}{HTML}{
0000FF
}
\definecolor[named]{ctb2}{HTML}{
9DBBFF
}
\path [fill=none] (apoyo) rectangle ($(apoyo)+(longitudFicticia)$)
node [xshift=0.3cm,inner sep=0pt, outer sep=0pt,midway,right,scale = 0.9]{Urbana};
\draw [color = ctb1,fill=ct1] ( $(apoyo)  + (desY) $) rectangle ($(apoyo)+ (desY) +(longitud)$);
\path [fill=none] ($(apoyo)+(desX)$) rectangle ($(apoyo)+(desX)+(longitudFicticia)$)
node [xshift=0.3cm,inner sep=0pt, outer sep=0pt,midway,right,scale = 0.9]{Rural};
\draw [color = ctb2 ,fill=ct2] ( $(apoyo)  + (desY) + (desX) $) rectangle ($(apoyo)+ (desY)+ (desX) +(longitud)$);
\end{scope}
  \end{tikzpicture}
}%
{%
	ENSMI 2008/2009} %


%#########################9########################

\cajita{%
	Niños de 0 a 5 meses según lactancia por etnia}%
{%
De acuerdo a la etnia de la madre, el 66.4\% de los niños de 0 a 5 meses cuya madre se autoidenficiaba como indígena recibía lactancia exclusiva, el 15.5\% lactancia predominante y el 3.2\% no estaba lactando.

Por otro lado, de los niños de 0 a 5 de madre no indígena, el 34.4\% recibía lactancia exclusiva.}%
{%
	Proporción de niños de 0 a 5 meses según etnicidad de la madre, por situación de lactancia
} %
{%
	República de Guatemala, 2008/2009, en porcentaje} %
{%
	\begin{tikzpicture}[x=1pt,y=1pt]  % Created by tikzDevice version 0.9 on 2016-03-03 04:58:38
% !TEX encoding = UTF-8 Unicode
\definecolor{fillColor}{RGB}{255,255,255}
\path[use as bounding box,fill=fillColor,fill opacity=0.00] (0,0) rectangle (289.08,198.74);
\begin{scope}
\path[clip] (  0.00,  0.00) rectangle (289.08,198.74);

\path[] (  0.00,  0.00) rectangle (289.08,198.74);
\end{scope}
\begin{scope}
\path[clip] (  0.00,  0.00) rectangle (289.08,198.74);

\path[] (  0.00, 18.46) rectangle (289.08,164.23);

\path[] ( 54.20, 18.46) --
	( 54.20,164.23);

\path[] (144.54, 18.46) --
	(144.54,164.23);

\path[] (234.88, 18.46) --
	(234.88,164.23);
\definecolor{drawColor}{RGB}{0,0,255}
\definecolor{fillColor}{RGB}{0,0,255}

\path[draw=drawColor,line width= 0.6pt,line join=round,fill=fillColor] ( 15.81, 18.46) rectangle ( 51.94, 25.48);
\definecolor{drawColor}{RGB}{157,187,255}
\definecolor{fillColor}{RGB}{157,187,255}

\path[draw=drawColor,line width= 0.6pt,line join=round,fill=fillColor] ( 56.46, 18.46) rectangle ( 92.60, 36.90);
\definecolor{drawColor}{RGB}{0,0,255}
\definecolor{fillColor}{RGB}{0,0,255}

\path[draw=drawColor,line width= 0.6pt,line join=round,fill=fillColor] (106.15, 18.46) rectangle (142.28,164.23);
\definecolor{drawColor}{RGB}{157,187,255}
\definecolor{fillColor}{RGB}{157,187,255}

\path[draw=drawColor,line width= 0.6pt,line join=round,fill=fillColor] (146.80, 18.46) rectangle (182.93, 93.98);
\definecolor{drawColor}{RGB}{0,0,255}
\definecolor{fillColor}{RGB}{0,0,255}

\path[draw=drawColor,line width= 0.6pt,line join=round,fill=fillColor] (196.48, 18.46) rectangle (232.62, 52.49);
\definecolor{drawColor}{RGB}{157,187,255}
\definecolor{fillColor}{RGB}{157,187,255}

\path[draw=drawColor,line width= 0.6pt,line join=round,fill=fillColor] (237.14, 18.46) rectangle (273.27, 68.95);
\definecolor{drawColor}{RGB}{0,0,0}

\path[draw=drawColor,line width= 0.6pt,line join=round] (  0.00, 18.46) -- (289.08, 18.46);

\node[text=drawColor,rotate= 90.00,anchor=base west,inner sep=0pt, outer sep=0pt, scale=  0.83] at ( 37.11, 27.31) {3.2};

\node[text=drawColor,rotate= 90.00,anchor=base west,inner sep=0pt, outer sep=0pt, scale=  0.83] at ( 77.76, 38.72) {8.4};

\node[text=drawColor,rotate= 90.00,anchor=base west,inner sep=0pt, outer sep=0pt, scale=  0.83] at (127.45,166.78) {66.4};

\node[text=drawColor,rotate= 90.00,anchor=base west,inner sep=0pt, outer sep=0pt, scale=  0.83] at (168.10, 96.53) {34.4};

\node[text=drawColor,rotate= 90.00,anchor=base west,inner sep=0pt, outer sep=0pt, scale=  0.83] at (217.79, 55.03) {15.5};

\node[text=drawColor,rotate= 90.00,anchor=base west,inner sep=0pt, outer sep=0pt, scale=  0.83] at (258.44, 71.50) {23.0};

\path[] (  0.00, 18.46) rectangle (289.08,164.23);
\end{scope}
\begin{scope}
\path[clip] (  0.00,  0.00) rectangle (289.08,198.74);

\path[] (  0.00, 18.46) --
	(289.08, 18.46);
\end{scope}
\begin{scope}
\path[clip] (  0.00,  0.00) rectangle (289.08,198.74);

\path[] ( 54.20, 15.71) --
	( 54.20, 18.46);

\path[] (144.54, 15.71) --
	(144.54, 18.46);

\path[] (234.88, 15.71) --
	(234.88, 18.46);
\end{scope}
\begin{scope}
\path[clip] (  0.00,  0.00) rectangle (289.08,198.74);
\definecolor{drawColor}{RGB}{0,0,0}

\node[text=drawColor,anchor=base,inner sep=0pt, outer sep=0pt, scale=  1.00] at ( 54.20,  5.69) {No lactando};

\node[text=drawColor,anchor=base,inner sep=0pt, outer sep=0pt, scale=  1.00] at (144.54,  5.69) {Lactancia Exclusiva};

\node[text=drawColor,anchor=base,inner sep=0pt, outer sep=0pt, scale=  1.00] at (234.88,  5.69) {Lactancia Predominante};
\end{scope}
\begin{scope}
\path[clip] (  0.00,  0.00) rectangle (289.08,198.74);
\coordinate (apoyo) at (55.19,188.79);
\coordinate (longitudFicticia) at (7.11,9.95);
\coordinate (longitud) at (7.11,7.11);
\coordinate (desX) at (133.24,0);
\coordinate (desY) at (0,1.42);
\definecolor[named]{ct1}{HTML}{
0000FF
}
\definecolor[named]{ct2}{HTML}{
9DBBFF
}
\definecolor[named]{ctb1}{HTML}{
0000FF
}
\definecolor[named]{ctb2}{HTML}{
9DBBFF
}
\path [fill=none] (apoyo) rectangle ($(apoyo)+(longitudFicticia)$)
node [xshift=0.3cm,inner sep=0pt, outer sep=0pt,midway,right,scale = 0.9]{Indígena};
\draw [color = ctb1,fill=ct1] ( $(apoyo)  + (desY) $) rectangle ($(apoyo)+ (desY) +(longitud)$);
\path [fill=none] ($(apoyo)+(desX)$) rectangle ($(apoyo)+(desX)+(longitudFicticia)$)
node [xshift=0.3cm,inner sep=0pt, outer sep=0pt,midway,right,scale = 0.9]{No indígena};
\draw [color = ctb2 ,fill=ct2] ( $(apoyo)  + (desY) + (desX) $) rectangle ($(apoyo)+ (desY)+ (desX) +(longitud)$);
\end{scope}
  \end{tikzpicture}
}%
{%
	ENSMI 2008/2009} %


%#########################10########################

\cajita{%
	Niños de 0 a 5 meses según lactancia y educación de la madre}%
{%
De acuerdo a la situación de la lactancia en niños de 0 a 5 meses, ésta presenta diferencias según el nivel educativo de la madre.

Para los niños con madres sin educación, el 64.0\% recibía lactancia exclusiva, el 19.9\% lactancia predominante y el 3.0\% no estaba lactando. 

Por otro lado, los niños de 0 a 5 meses de madre con educación superior, el 9.9\% recibía lactancia predominante, el 8.9\% no lactaba y el 4.7\% lactancia exclusiva. El resto de niños con madre con educación superior se distribuía entre los estados de: recibir fórmula, combinación con líquidos no lácteos y otros.}%
{%
	Proporción de niños de 0 a 5 meses según educación de la madre, por situación lactancia
} %
{%
	República de Guatemala, 2008/2009, en porcentaje} %
{%
	\begin{tikzpicture}[x=1pt,y=1pt]  % Created by tikzDevice version 0.9 on 2016-03-03 04:58:43
% !TEX encoding = UTF-8 Unicode
\definecolor{fillColor}{RGB}{255,255,255}
\path[use as bounding box,fill=fillColor,fill opacity=0.00] (0,0) rectangle (289.08,198.74);
\begin{scope}
\path[clip] (  0.00,  0.00) rectangle (289.08,198.74);

\path[] (  0.00,  0.00) rectangle (289.08,198.74);
\end{scope}
\begin{scope}
\path[clip] (  0.00,  0.00) rectangle (289.08,198.74);

\path[] (  0.00, 18.46) rectangle (289.08,164.65);

\path[] ( 54.20, 18.46) --
	( 54.20,164.65);

\path[] (144.54, 18.46) --
	(144.54,164.65);

\path[] (234.88, 18.46) --
	(234.88,164.65);
\definecolor{drawColor}{RGB}{0,0,255}
\definecolor{fillColor}{RGB}{0,0,255}

\path[draw=drawColor,line width= 0.6pt,line join=round,fill=fillColor] ( 15.81, 18.46) rectangle ( 51.94, 25.31);
\definecolor{drawColor}{RGB}{157,187,255}
\definecolor{fillColor}{RGB}{157,187,255}

\path[draw=drawColor,line width= 0.6pt,line join=round,fill=fillColor] ( 56.46, 18.46) rectangle ( 92.60, 38.79);
\definecolor{drawColor}{RGB}{0,0,255}
\definecolor{fillColor}{RGB}{0,0,255}

\path[draw=drawColor,line width= 0.6pt,line join=round,fill=fillColor] (106.15, 18.46) rectangle (142.28,164.65);
\definecolor{drawColor}{RGB}{157,187,255}
\definecolor{fillColor}{RGB}{157,187,255}

\path[draw=drawColor,line width= 0.6pt,line join=round,fill=fillColor] (146.80, 18.46) rectangle (182.93, 29.19);
\definecolor{drawColor}{RGB}{0,0,255}
\definecolor{fillColor}{RGB}{0,0,255}

\path[draw=drawColor,line width= 0.6pt,line join=round,fill=fillColor] (196.48, 18.46) rectangle (232.62, 63.91);
\definecolor{drawColor}{RGB}{157,187,255}
\definecolor{fillColor}{RGB}{157,187,255}

\path[draw=drawColor,line width= 0.6pt,line join=round,fill=fillColor] (237.14, 18.46) rectangle (273.27, 41.07);
\definecolor{drawColor}{RGB}{0,0,0}

\path[draw=drawColor,line width= 0.6pt,line join=round] (  0.00, 18.46) -- (289.08, 18.46);

\node[text=drawColor,rotate= 90.00,anchor=base west,inner sep=0pt, outer sep=0pt, scale=  0.83] at ( 37.11, 27.13) {3.0};

\node[text=drawColor,rotate= 90.00,anchor=base west,inner sep=0pt, outer sep=0pt, scale=  0.83] at ( 77.76, 40.61) {8.9};

\node[text=drawColor,rotate= 90.00,anchor=base west,inner sep=0pt, outer sep=0pt, scale=  0.83] at (127.45,167.20) {64.0};

\node[text=drawColor,rotate= 90.00,anchor=base west,inner sep=0pt, outer sep=0pt, scale=  0.83] at (168.10, 31.02) {4.7};

\node[text=drawColor,rotate= 90.00,anchor=base west,inner sep=0pt, outer sep=0pt, scale=  0.83] at (217.79, 66.46) {19.9};

\node[text=drawColor,rotate= 90.00,anchor=base west,inner sep=0pt, outer sep=0pt, scale=  0.83] at (258.44, 42.89) {9.9};

\path[] (  0.00, 18.46) rectangle (289.08,164.65);
\end{scope}
\begin{scope}
\path[clip] (  0.00,  0.00) rectangle (289.08,198.74);

\path[] (  0.00, 18.46) --
	(289.08, 18.46);
\end{scope}
\begin{scope}
\path[clip] (  0.00,  0.00) rectangle (289.08,198.74);

\path[] ( 54.20, 15.71) --
	( 54.20, 18.46);

\path[] (144.54, 15.71) --
	(144.54, 18.46);

\path[] (234.88, 15.71) --
	(234.88, 18.46);
\end{scope}
\begin{scope}
\path[clip] (  0.00,  0.00) rectangle (289.08,198.74);
\definecolor{drawColor}{RGB}{0,0,0}

\node[text=drawColor,anchor=base,inner sep=0pt, outer sep=0pt, scale=  1.00] at ( 54.20,  5.69) {No lactando};

\node[text=drawColor,anchor=base,inner sep=0pt, outer sep=0pt, scale=  1.00] at (144.54,  5.69) {Lactancia Exclusiva};

\node[text=drawColor,anchor=base,inner sep=0pt, outer sep=0pt, scale=  1.00] at (234.88,  5.69) {Lactancia Predominante};
\end{scope}
\begin{scope}
\path[clip] (  0.00,  0.00) rectangle (289.08,198.74);
\coordinate (apoyo) at (46.95,189.2);
\coordinate (longitudFicticia) at (7.11,9.54);
\coordinate (longitud) at (7.11,7.11);
\coordinate (desX) at (147.1,0);
\coordinate (desY) at (0,1.21);
\definecolor[named]{ct1}{HTML}{
0000FF
}
\definecolor[named]{ct2}{HTML}{
9DBBFF
}
\definecolor[named]{ctb1}{HTML}{
0000FF
}
\definecolor[named]{ctb2}{HTML}{
9DBBFF
}
\path [fill=none] (apoyo) rectangle ($(apoyo)+(longitudFicticia)$)
node [xshift=0.3cm,inner sep=0pt, outer sep=0pt,midway,right,scale = 0.9]{Sin educación};
\draw [color = ctb1,fill=ct1] ( $(apoyo)  + (desY) $) rectangle ($(apoyo)+ (desY) +(longitud)$);
\path [fill=none] ($(apoyo)+(desX)$) rectangle ($(apoyo)+(desX)+(longitudFicticia)$)
node [xshift=0.3cm,inner sep=0pt, outer sep=0pt,midway,right,scale = 0.9]{Superior};
\draw [color = ctb2 ,fill=ct2] ( $(apoyo)  + (desY) + (desX) $) rectangle ($(apoyo)+ (desY)+ (desX) +(longitud)$);
\end{scope}
  \end{tikzpicture}
}%
{%
	ENSMI 2008/2009} %
	\INEchaptercarta{Utilización biológica de los alimentos}{Se refiere a que la desnutrición tiende a estar más concentrada en las poblaciones más vulnerables. Asimismo, las políticas y programas de nutrición deben ser lo suficientemente amplios para enfrentar el retraso en el crecimiento y la deficiencia en micronutrientes (FAO, 2015). Además, el aprovechamiento biológico de los alimentos depende de las condiciones de salud del individuo (Coneval, 2010). Esta dimensión se compone de cinco indicadores y diez sub-indicadores.
	
		
		El indicador \textbf{5.1 Morbilidad Relacionada}, se compone del sub-indicador de Consultas por diarrea e infecciones respiratorias agudas.
		
		 El indicador \textbf{5.2 Acceso a Servicios de Salud}, se compone de los sub-indicadores: a) entrega de micronutrientes, b) entrega de alimentación complementaria, c) cobertura de atención pre-nata y post-natal e d) inmunizaciones.
		 
		El indicador \textbf{5.3 Acceso a servicios básicos}, se compone de los sub-indicadores: a) Existencia de agua potable y b) existencia adecuada de disposición de excretas. De este último, sin información.  
		
		Para el indicador \textbf{5.4 Educación Sanitaria}, sin información disponible. 
		
		Por último, el indicador \textbf{5.5 Vivienda}, se compone de los indicadores de: a) hacinamiento, y b) material predominante de la vivienda (suelo), de los cuales no se encontró información para hacinamiento.
		
	\scriptsize{ Para esta dimensión, se sugiere enriquecer los cuadros con la información de la Ensmi 2014 cuando esté disponible.}
		}
	
%#########################12########################
\stepcounter{section}
\begin{center}
	\textbf{\color{color2}\LARGE \thesection} \quad  \textbf{\LARGE Morbilidad relacionada} \addcontentsline{toc}{section}{\numberline{\thesection} Morbilidad relacionada}
\end{center}
$\ $ \\[-2.3cm]
\cajita{%
	Diarrea}%
{%
	Los casos de diarrea han ido en aumento desde el año 2011, reportándose un total de 331,874 para el 2015.
}%
{%
	Niños menores de cinco años que recibieron atención médica por diarrea} %
{%
	República de Guatemala, serie histórica, número de niños} %
{%
	\begin{tikzpicture}[x=1pt,y=1pt]  % Created by tikzDevice version 0.9 on 2016-03-03 05:26:29
% !TEX encoding = UTF-8 Unicode
\definecolor{fillColor}{RGB}{255,255,255}
\path[use as bounding box,fill=fillColor,fill opacity=0.00] (0,0) rectangle (289.08,198.74);
\begin{scope}
\path[clip] (  0.00,  0.00) rectangle (289.08,198.74);

\path[] (  0.00,  0.00) rectangle (289.08,198.74);
\end{scope}
\begin{scope}
\path[clip] (  0.00,  0.00) rectangle (289.08,198.74);

\path[] ( 17.00, 15.61) rectangle (280.54,191.48);

\path[] ( 17.00, 46.23) --
	(280.54, 46.23);

\path[] ( 17.00, 89.73) --
	(280.54, 89.73);

\path[] ( 17.00,133.22) --
	(280.54,133.22);

\path[] ( 17.00,176.71) --
	(280.54,176.71);

\path[] ( 17.00, 24.49) --
	(280.54, 24.49);

\path[] ( 17.00, 67.98) --
	(280.54, 67.98);

\path[] ( 17.00,111.47) --
	(280.54,111.47);

\path[] ( 17.00,154.97) --
	(280.54,154.97);

\path[] ( 47.41, 15.61) --
	( 47.41,191.48);

\path[] ( 98.09, 15.61) --
	( 98.09,191.48);

\path[] (148.77, 15.61) --
	(148.77,191.48);

\path[] (199.45, 15.61) --
	(199.45,191.48);

\path[] (250.13, 15.61) --
	(250.13,191.48);
\definecolor{drawColor}{RGB}{0,0,255}

\path[draw=drawColor,line width= 1.7pt,line join=round] ( 47.41, 60.50) --
	( 98.09,147.99) --
	(148.77,183.49) --
	(199.45,179.50) --
	(250.13, 95.71);
\definecolor{drawColor}{RGB}{0,0,0}

\node[text=drawColor,anchor=base,inner sep=0pt, outer sep=0pt, scale=  1.02] at ( 47.41, 48.59) {291,404};

\node[text=drawColor,anchor=base east,inner sep=0pt, outer sep=0pt, scale=  1.02] at ( 92.29,147.99) {391,975};

\node[text=drawColor,anchor=base,inner sep=0pt, outer sep=0pt, scale=  1.02] at (148.77,187.46) {432,789};

\node[text=drawColor,anchor=base west,inner sep=0pt, outer sep=0pt, scale=  1.02] at (199.45,183.47) {428,206};

\node[text=drawColor,anchor=base,inner sep=0pt, outer sep=0pt, scale=  1.02] at (250.13, 83.79) {331,874};

\path[draw=drawColor,line width= 0.1pt,line join=round] ( 17.00, 23.61) -- (280.54, 23.61);

\path[] ( 17.00, 15.61) rectangle (280.54,191.48);
\end{scope}
\begin{scope}
\path[clip] (  0.00,  0.00) rectangle (289.08,198.74);

\path[] ( 17.00, 15.61) --
	( 17.00,191.48);
\end{scope}
\begin{scope}
\path[clip] (  0.00,  0.00) rectangle (289.08,198.74);
\definecolor{drawColor}{RGB}{255,255,255}

\node[text=drawColor,text opacity=0.00,anchor=base east,inner sep=0pt, outer sep=0pt, scale=  1.00] at ( 12.05, 20.58) {250000};

\node[text=drawColor,text opacity=0.00,anchor=base east,inner sep=0pt, outer sep=0pt, scale=  1.00] at ( 12.05, 64.07) {300000};

\node[text=drawColor,text opacity=0.00,anchor=base east,inner sep=0pt, outer sep=0pt, scale=  1.00] at ( 12.05,107.56) {350000};

\node[text=drawColor,text opacity=0.00,anchor=base east,inner sep=0pt, outer sep=0pt, scale=  1.00] at ( 12.05,151.06) {400000};
\end{scope}
\begin{scope}
\path[clip] (  0.00,  0.00) rectangle (289.08,198.74);

\path[] ( 14.25, 24.49) --
	( 17.00, 24.49);

\path[] ( 14.25, 67.98) --
	( 17.00, 67.98);

\path[] ( 14.25,111.47) --
	( 17.00,111.47);

\path[] ( 14.25,154.97) --
	( 17.00,154.97);
\end{scope}
\begin{scope}
\path[clip] (  0.00,  0.00) rectangle (289.08,198.74);

\path[] ( 17.00, 15.61) --
	(280.54, 15.61);
\end{scope}
\begin{scope}
\path[clip] (  0.00,  0.00) rectangle (289.08,198.74);

\path[] ( 47.41, 12.86) --
	( 47.41, 15.61);

\path[] ( 98.09, 12.86) --
	( 98.09, 15.61);

\path[] (148.77, 12.86) --
	(148.77, 15.61);

\path[] (199.45, 12.86) --
	(199.45, 15.61);

\path[] (250.13, 12.86) --
	(250.13, 15.61);
\end{scope}
\begin{scope}
\path[clip] (  0.00,  0.00) rectangle (289.08,198.74);
\definecolor{drawColor}{RGB}{0,0,0}

\node[text=drawColor,anchor=base,inner sep=0pt, outer sep=0pt, scale=  1.00] at ( 47.41,  2.85) {2011};

\node[text=drawColor,anchor=base,inner sep=0pt, outer sep=0pt, scale=  1.00] at ( 98.09,  2.85) {2012};

\node[text=drawColor,anchor=base,inner sep=0pt, outer sep=0pt, scale=  1.00] at (148.77,  2.85) {2013};

\node[text=drawColor,anchor=base,inner sep=0pt, outer sep=0pt, scale=  1.00] at (199.45,  2.85) {2014};

\node[text=drawColor,anchor=base,inner sep=0pt, outer sep=0pt, scale=  1.00] at (250.13,  2.85) {2015};
\end{scope}
  \end{tikzpicture}}%
{%
	Sigsa} %


%#########################13########################

\cajita{%
	Diarrea según sexo}%
{%
	Los casos de diarrea para el 2015 se presentaron en cantidades similares en niños y niñas, teniendo menor ocurrencia en las niñas. 
	}%
{%
	Niños menores de cinco años que recibieron atención médica por diarrea, según sexo} %
{%
	República de Guatemala, 2015, número de niños} %
{%
	\begin{tikzpicture}[x=1pt,y=1pt]  \input{graficas/5_13.tex}  \end{tikzpicture}}%
{%
	Sigsa} %


%#########################14########################

\cajota{%
	Casos de diarrea por departamento}%
{%
	Los departamentos que presentaron más casos de atención por diarrea fueron Huehuetenango (35,154 casos), San Marcos (34,588 casos), y Quiché (34,192 casos), mientras que los que tuvieron menos casos de atención médica por diarrea fueron Sacatepéquez (5,528 casos), Zacapa (5,468 casos), y El Progreso (4,055 casos). 
}%
{%
	Niños menores de cinco años que recibieron atención médica por diarrea por departamento} %
{%
	República de Guatemala, departamental, número de niños} %
{%
	\includegraphics[width=52\cuadri]{graficas/5_14.pdf}}%
{%
	Sigsa} %


%#########################15########################

\cajita{%
	Infecciones Respiratorias Agudas (IRA)}%
{%
	En el período de 2011  a 2015 la menor cantidad de infecciones respiratorias agudas que fueron atendidas por personal médico se registró en 2015, siendo un total de 1,184,756.
}%
{%
	Niños menores de cinco años que recibieron atención médica por infecciones respiratorias agudas} %
{%
	República de Guatemala, serie histórica, número de niños} %
{%
	\begin{tikzpicture}[x=1pt,y=1pt]  % Created by tikzDevice version 0.9 on 2016-03-03 05:26:40
% !TEX encoding = UTF-8 Unicode
\definecolor{fillColor}{RGB}{255,255,255}
\path[use as bounding box,fill=fillColor,fill opacity=0.00] (0,0) rectangle (289.08,198.74);
\begin{scope}
\path[clip] (  0.00,  0.00) rectangle (289.08,198.74);

\path[] (  0.00,  0.00) rectangle (289.08,198.74);
\end{scope}
\begin{scope}
\path[clip] (  0.00,  0.00) rectangle (289.08,198.74);

\path[] ( 21.38, 15.61) rectangle (280.54,191.48);

\path[] ( 21.38, 36.84) --
	(280.54, 36.84);

\path[] ( 21.38, 92.67) --
	(280.54, 92.67);

\path[] ( 21.38,148.50) --
	(280.54,148.50);

\path[] ( 21.38, 64.76) --
	(280.54, 64.76);

\path[] ( 21.38,120.58) --
	(280.54,120.58);

\path[] ( 21.38,176.41) --
	(280.54,176.41);

\path[] ( 51.28, 15.61) --
	( 51.28,191.48);

\path[] (101.12, 15.61) --
	(101.12,191.48);

\path[] (150.96, 15.61) --
	(150.96,191.48);

\path[] (200.80, 15.61) --
	(200.80,191.48);

\path[] (250.64, 15.61) --
	(250.64,191.48);
\definecolor{drawColor}{RGB}{0,0,255}

\path[draw=drawColor,line width= 1.7pt,line join=round] ( 51.28, 66.18) --
	(101.12,107.14) --
	(150.96,183.49) --
	(200.80,147.27) --
	(250.64, 60.50);
\definecolor{drawColor}{RGB}{0,0,0}

\node[text=drawColor,anchor=base,inner sep=0pt, outer sep=0pt, scale=  1.02] at ( 51.28, 54.27) {1,205,111};

\node[text=drawColor,anchor=base east,inner sep=0pt, outer sep=0pt, scale=  1.02] at ( 93.97,107.14) {1,351,837};

\node[text=drawColor,anchor=base,inner sep=0pt, outer sep=0pt, scale=  1.02] at (150.96,187.46) {1,625,358};

\node[text=drawColor,anchor=base west,inner sep=0pt, outer sep=0pt, scale=  1.02] at (200.80,151.24) {1,495,601};

\node[text=drawColor,anchor=base,inner sep=0pt, outer sep=0pt, scale=  1.02] at (250.64, 48.59) {1,184,756};

\path[draw=drawColor,line width= 0.1pt,line join=round] ( 21.38, 23.61) -- (280.54, 23.61);

\path[] ( 21.38, 15.61) rectangle (280.54,191.48);
\end{scope}
\begin{scope}
\path[clip] (  0.00,  0.00) rectangle (289.08,198.74);

\path[] ( 21.38, 15.61) --
	( 21.38,191.48);
\end{scope}
\begin{scope}
\path[clip] (  0.00,  0.00) rectangle (289.08,198.74);
\definecolor{drawColor}{RGB}{255,255,255}

\node[text=drawColor,text opacity=0.00,anchor=base east,inner sep=0pt, outer sep=0pt, scale=  1.00] at ( 16.43, 60.85) {1200000};

\node[text=drawColor,text opacity=0.00,anchor=base east,inner sep=0pt, outer sep=0pt, scale=  1.00] at ( 16.43,116.68) {1400000};

\node[text=drawColor,text opacity=0.00,anchor=base east,inner sep=0pt, outer sep=0pt, scale=  1.00] at ( 16.43,172.50) {1600000};
\end{scope}
\begin{scope}
\path[clip] (  0.00,  0.00) rectangle (289.08,198.74);

\path[] ( 18.63, 64.76) --
	( 21.38, 64.76);

\path[] ( 18.63,120.58) --
	( 21.38,120.58);

\path[] ( 18.63,176.41) --
	( 21.38,176.41);
\end{scope}
\begin{scope}
\path[clip] (  0.00,  0.00) rectangle (289.08,198.74);

\path[] ( 21.38, 15.61) --
	(280.54, 15.61);
\end{scope}
\begin{scope}
\path[clip] (  0.00,  0.00) rectangle (289.08,198.74);

\path[] ( 51.28, 12.86) --
	( 51.28, 15.61);

\path[] (101.12, 12.86) --
	(101.12, 15.61);

\path[] (150.96, 12.86) --
	(150.96, 15.61);

\path[] (200.80, 12.86) --
	(200.80, 15.61);

\path[] (250.64, 12.86) --
	(250.64, 15.61);
\end{scope}
\begin{scope}
\path[clip] (  0.00,  0.00) rectangle (289.08,198.74);
\definecolor{drawColor}{RGB}{0,0,0}

\node[text=drawColor,anchor=base,inner sep=0pt, outer sep=0pt, scale=  1.00] at ( 51.28,  2.85) {2011};

\node[text=drawColor,anchor=base,inner sep=0pt, outer sep=0pt, scale=  1.00] at (101.12,  2.85) {2012};

\node[text=drawColor,anchor=base,inner sep=0pt, outer sep=0pt, scale=  1.00] at (150.96,  2.85) {2013};

\node[text=drawColor,anchor=base,inner sep=0pt, outer sep=0pt, scale=  1.00] at (200.80,  2.85) {2014};

\node[text=drawColor,anchor=base,inner sep=0pt, outer sep=0pt, scale=  1.00] at (250.64,  2.85) {2015};
\end{scope}
  \end{tikzpicture}}%
{%
	Sigsa} %

%#########################16########################

\cajita{%
	Infecciones respiratorias agudas según sexo}%
{%
	Al igual que en el caso de la atención por diarreas, la cantidad de casos de IRA atendidos por personal médico es muy similar entre hombres y mujeres. 
}%
{%
	Niños menores de cinco años que recibieron atención médica por infecciones respiratorias agudas, según sexo} %
{%
	República de Guatemala, 2015, número de niños} %
{%
	\begin{tikzpicture}[x=1pt,y=1pt]  \input{graficas/5_16.tex}  \end{tikzpicture}}%
{%
	Sigsa} %

%#########################17########################

\cajota{%
	Casos de IRA por departamento}%
{%
	Los departamentos que registraron mayor cantidad de atención médica de casos de IRA fueron Guatemala (124,186), San Marcos (108,781) y Petén (98,928). 	Los departamentos con la menor cantidad de registros médicos de atención a casos de IRA fueron Zacapa (21,997), Sacatepéquez (20,614) y El Progreso (15,350). 
}%
{%
	Niños menores de cinco años que recibieron atención médica por infecciones respiratorias agudas por departamento} %
{%
	República de Guatemala, departamental, número de niños} %
{%
	\includegraphics[width=52\cuadri]{graficas/5_17.pdf}}%
{%
	Sigsa} %



%#########################9########################
\stepcounter{section}
\begin{center}
\textbf{\color{color2}\LARGE \thesection} \quad  \textbf{\LARGE Acceso a servicios de Salud} \addcontentsline{toc}{section}{\numberline{\thesection} Acceso a servicios de Salud}
\end{center}
$ \ $ \\[-5cm]
\cajita{%
	Asistencia prenatal}%
{%

En todos los grupos de edad, el porcentaje de mujeres que recibieron atención médica prenatal de un proveedor calificado es al menos del 90\%. 
}%
{%
	Mujeres que recibieron atención prenatal de un proveedor calificado, según grupos de edad} %
{%
	República de Guatemala, 2014, en porcentaje} %
{%
	\begin{tikzpicture}[x=1pt,y=1pt]  \input{graficas/5_09.tex}  \end{tikzpicture}}%
{%
	Ensmi, 2014} %

%#########################10########################

\cajita{%
	Asistencia en el parto}%
{%
	Siete de cada diez mujeres menores de 34 años reciben algún tipo de asistencia médica a la hora del parto, mientras que para mujeres mayores de 34 años este indicador baja a seis de cada diez. 
}%
{%
	Mujeres que recibieron atención a la hora del parto  de un proveedor calificado, según grupos de edad} %
{%
	República de Guatemala, 2014, en porcentaje} %
{%
	\begin{tikzpicture}[x=1pt,y=1pt]  \input{graficas/5_10.tex}  \end{tikzpicture}}%
{%
	Ensmi, 2014} %



%#########################11########################

\cajita{%
	Niños que reciben alguna vacuna o desparasitante}%
{%
	El número de niños que recibe alguna vacuna o desparasitante aumentó en 2015 respecto al 2011. 
	
	El año que presentó mayor cantidad de niños desparasitados o vacunados fue  2012.
}%
{%
	Niños que reciben vacunas o desparasitantes} %
{%
	República de Guatemala, serie histórica, número de niños} %
{%
	\begin{tikzpicture}[x=1pt,y=1pt]  % Created by tikzDevice version 0.9 on 2016-03-03 05:26:22
% !TEX encoding = UTF-8 Unicode
\definecolor{fillColor}{RGB}{255,255,255}
\path[use as bounding box,fill=fillColor,fill opacity=0.00] (0,0) rectangle (289.08,198.74);
\begin{scope}
\path[clip] (  0.00,  0.00) rectangle (289.08,198.74);

\path[] (  0.00,  0.00) rectangle (289.08,198.74);
\end{scope}
\begin{scope}
\path[clip] (  0.00,  0.00) rectangle (289.08,198.74);

\path[] ( 21.38, 15.61) rectangle (280.54,191.48);

\path[] ( 21.38, 44.20) --
	(280.54, 44.20);

\path[] ( 21.38, 99.35) --
	(280.54, 99.35);

\path[] ( 21.38,154.49) --
	(280.54,154.49);

\path[] ( 21.38, 16.63) --
	(280.54, 16.63);

\path[] ( 21.38, 71.77) --
	(280.54, 71.77);

\path[] ( 21.38,126.92) --
	(280.54,126.92);

\path[] ( 21.38,182.06) --
	(280.54,182.06);

\path[] ( 51.28, 15.61) --
	( 51.28,191.48);

\path[] (101.12, 15.61) --
	(101.12,191.48);

\path[] (150.96, 15.61) --
	(150.96,191.48);

\path[] (200.80, 15.61) --
	(200.80,191.48);

\path[] (250.64, 15.61) --
	(250.64,191.48);
\definecolor{drawColor}{RGB}{0,0,255}

\path[draw=drawColor,line width= 1.7pt,line join=round] ( 51.28, 60.50) --
	(101.12,183.49) --
	(150.96,102.33) --
	(200.80, 99.78) --
	(250.64, 99.78);
\definecolor{drawColor}{RGB}{0,0,0}

\node[text=drawColor,anchor=base,inner sep=0pt, outer sep=0pt, scale=  1.02] at ( 51.28, 48.59) {2,897,786};

\node[text=drawColor,anchor=base,inner sep=0pt, outer sep=0pt, scale=  1.02] at (101.12,187.46) {4,012,987};

\node[text=drawColor,anchor=base west,inner sep=0pt, outer sep=0pt, scale=  1.02] at (150.96,106.30) {3,277,046};

\node[text=drawColor,anchor=base west,inner sep=0pt, outer sep=0pt, scale=  1.02] at (200.80,103.76) {3,253,982};

\node[text=drawColor,anchor=base,inner sep=0pt, outer sep=0pt, scale=  1.02] at (250.64, 87.87) {3,253,951};

\path[draw=drawColor,line width= 0.1pt,line join=round] ( 21.38, 23.61) -- (280.54, 23.61);

\path[] ( 21.38, 15.61) rectangle (280.54,191.48);
\end{scope}
\begin{scope}
\path[clip] (  0.00,  0.00) rectangle (289.08,198.74);

\path[] ( 21.38, 15.61) --
	( 21.38,191.48);
\end{scope}
\begin{scope}
\path[clip] (  0.00,  0.00) rectangle (289.08,198.74);
\definecolor{drawColor}{RGB}{255,255,255}

\node[text=drawColor,text opacity=0.00,anchor=base east,inner sep=0pt, outer sep=0pt, scale=  1.00] at ( 16.43, 12.73) {2500000};

\node[text=drawColor,text opacity=0.00,anchor=base east,inner sep=0pt, outer sep=0pt, scale=  1.00] at ( 16.43, 67.87) {3000000};

\node[text=drawColor,text opacity=0.00,anchor=base east,inner sep=0pt, outer sep=0pt, scale=  1.00] at ( 16.43,123.01) {3500000};

\node[text=drawColor,text opacity=0.00,anchor=base east,inner sep=0pt, outer sep=0pt, scale=  1.00] at ( 16.43,178.15) {4000000};
\end{scope}
\begin{scope}
\path[clip] (  0.00,  0.00) rectangle (289.08,198.74);

\path[] ( 18.63, 16.63) --
	( 21.38, 16.63);

\path[] ( 18.63, 71.77) --
	( 21.38, 71.77);

\path[] ( 18.63,126.92) --
	( 21.38,126.92);

\path[] ( 18.63,182.06) --
	( 21.38,182.06);
\end{scope}
\begin{scope}
\path[clip] (  0.00,  0.00) rectangle (289.08,198.74);

\path[] ( 21.38, 15.61) --
	(280.54, 15.61);
\end{scope}
\begin{scope}
\path[clip] (  0.00,  0.00) rectangle (289.08,198.74);

\path[] ( 51.28, 12.86) --
	( 51.28, 15.61);

\path[] (101.12, 12.86) --
	(101.12, 15.61);

\path[] (150.96, 12.86) --
	(150.96, 15.61);

\path[] (200.80, 12.86) --
	(200.80, 15.61);

\path[] (250.64, 12.86) --
	(250.64, 15.61);
\end{scope}
\begin{scope}
\path[clip] (  0.00,  0.00) rectangle (289.08,198.74);
\definecolor{drawColor}{RGB}{0,0,0}

\node[text=drawColor,anchor=base,inner sep=0pt, outer sep=0pt, scale=  1.00] at ( 51.28,  2.85) {2011};

\node[text=drawColor,anchor=base,inner sep=0pt, outer sep=0pt, scale=  1.00] at (101.12,  2.85) {2012};

\node[text=drawColor,anchor=base,inner sep=0pt, outer sep=0pt, scale=  1.00] at (150.96,  2.85) {2013};

\node[text=drawColor,anchor=base,inner sep=0pt, outer sep=0pt, scale=  1.00] at (200.80,  2.85) {2014};

\node[text=drawColor,anchor=base,inner sep=0pt, outer sep=0pt, scale=  1.00] at (250.64,  2.85) {2015};
\end{scope}
  \end{tikzpicture}}%
{%
	Sigsa} %

%#########################7########################

\cajita{%
	Tasa Global de Fecundidad (TGF)}%
{%	
	La Tasa Global de Fecundidad (TGF) ha ido en descenso, ya que en 1995 se ubicó en 5.1, mientras que para el 2014 llegó a 3.1 niños por mujer. 
}%
{%
	Evolución de la tasa global de fecundidad} %
{%
	República de Guatemala, serie histórica, niños por mujer} %
{%
	\begin{tikzpicture}[x=1pt,y=1pt]  % Created by tikzDevice version 0.9 on 2016-03-03 05:26:08
% !TEX encoding = UTF-8 Unicode
\definecolor{fillColor}{RGB}{255,255,255}
\path[use as bounding box,fill=fillColor,fill opacity=0.00] (0,0) rectangle (289.08,198.74);
\begin{scope}
\path[clip] (  0.00,  0.00) rectangle (289.08,198.74);

\path[] (  0.00,  0.00) rectangle (289.08,198.74);
\end{scope}
\begin{scope}
\path[clip] (  0.00,  0.00) rectangle (289.08,198.74);

\path[] (  1.74, 15.61) rectangle (280.54,191.48);

\path[] (  1.74, 38.98) --
	(280.54, 38.98);

\path[] (  1.74, 69.73) --
	(280.54, 69.73);

\path[] (  1.74,100.47) --
	(280.54,100.47);

\path[] (  1.74,131.22) --
	(280.54,131.22);

\path[] (  1.74,161.97) --
	(280.54,161.97);

\path[] (  1.74, 23.61) --
	(280.54, 23.61);

\path[] (  1.74, 54.35) --
	(280.54, 54.35);

\path[] (  1.74, 85.10) --
	(280.54, 85.10);

\path[] (  1.74,115.85) --
	(280.54,115.85);

\path[] (  1.74,146.59) --
	(280.54,146.59);

\path[] (  1.74,177.34) --
	(280.54,177.34);

\path[] ( 41.57, 15.61) --
	( 41.57,191.48);

\path[] (107.95, 15.61) --
	(107.95,191.48);

\path[] (174.33, 15.61) --
	(174.33,191.48);

\path[] (240.71, 15.61) --
	(240.71,191.48);
\definecolor{drawColor}{RGB}{0,0,255}

\path[draw=drawColor,line width= 1.7pt,line join=round] ( 41.57,183.49) --
	(107.95,140.44) --
	(174.33, 91.25) --
	(240.71, 60.50);
\definecolor{drawColor}{RGB}{0,0,0}

\node[text=drawColor,anchor=base,inner sep=0pt, outer sep=0pt, scale=  1.02] at ( 41.57,187.46) {5.1};

\node[text=drawColor,anchor=base west,inner sep=0pt, outer sep=0pt, scale=  1.02] at (107.95,144.41) {4.4};

\node[text=drawColor,anchor=base west,inner sep=0pt, outer sep=0pt, scale=  1.02] at (174.33, 95.22) {3.6};

\node[text=drawColor,anchor=base,inner sep=0pt, outer sep=0pt, scale=  1.02] at (240.71, 48.59) {3.1};

\path[draw=drawColor,line width= 0.1pt,line join=round] (  1.74, 23.61) -- (280.54, 23.61);

\path[] (  1.74, 15.61) rectangle (280.54,191.48);
\end{scope}
\begin{scope}
\path[clip] (  0.00,  0.00) rectangle (289.08,198.74);

\path[] (  1.74, 15.61) --
	(  1.74,191.48);
\end{scope}
\begin{scope}
\path[clip] (  0.00,  0.00) rectangle (289.08,198.74);

\path[] (  0.00, 23.61) --
	(  1.74, 23.61);

\path[] (  0.00, 54.35) --
	(  1.74, 54.35);

\path[] (  0.00, 85.10) --
	(  1.74, 85.10);

\path[] (  0.00,115.85) --
	(  1.74,115.85);

\path[] (  0.00,146.59) --
	(  1.74,146.59);

\path[] (  0.00,177.34) --
	(  1.74,177.34);
\end{scope}
\begin{scope}
\path[clip] (  0.00,  0.00) rectangle (289.08,198.74);

\path[] (  1.74, 15.61) --
	(280.54, 15.61);
\end{scope}
\begin{scope}
\path[clip] (  0.00,  0.00) rectangle (289.08,198.74);

\path[] ( 41.57, 12.86) --
	( 41.57, 15.61);

\path[] (107.95, 12.86) --
	(107.95, 15.61);

\path[] (174.33, 12.86) --
	(174.33, 15.61);

\path[] (240.71, 12.86) --
	(240.71, 15.61);
\end{scope}
\begin{scope}
\path[clip] (  0.00,  0.00) rectangle (289.08,198.74);
\definecolor{drawColor}{RGB}{0,0,0}

\node[text=drawColor,anchor=base,inner sep=0pt, outer sep=0pt, scale=  1.00] at ( 41.57,  2.85) {1995};

\node[text=drawColor,anchor=base,inner sep=0pt, outer sep=0pt, scale=  1.00] at (107.95,  2.85) {2002};

\node[text=drawColor,anchor=base,inner sep=0pt, outer sep=0pt, scale=  1.00] at (174.33,  2.85) {2008-2009};

\node[text=drawColor,anchor=base,inner sep=0pt, outer sep=0pt, scale=  1.00] at (240.71,  2.85) {2014-2015};
\end{scope}
  \end{tikzpicture}}%
{%
	Ensmi, 2014, 2008, 2002 y 1995} %

%#########################8########################

\cajita{%
	Tasa Global de Fecundidad, por área de residencia}%
{%
	Al analizar la Tasa Global de Fecundidad por área de residencia, se observa que las mujeres en el área urbana tienen menos hijos que las del área rural, esto es 2.5 contra 3.7 niños por mujer respectivamente. 
}%
{%
	Tasa global de fecundidad por área de residencia} %
{%
	República de Guatemala, 2014, niños por mujer} %
{%
	\begin{tikzpicture}[x=1pt,y=1pt]  \input{graficas/5_08.tex}  \end{tikzpicture}}%
{%
	Ensmi, 2014} %




%%%%%%%%%%%%%%%%%%%%%%%%%%%%%%%%%%13%%%%%%%%%%%%%%%%%%%%%%%%
\newpage
\stepcounter{section}
\begin{center}
\textbf{\color{color2}\LARGE \thesection} \quad  \textbf{\LARGE Acceso a servicios básicos} \addcontentsline{toc}{section}{\numberline{\thesection} Acceso a servicios básicos}
\end{center}
$\ $ \\[-2.3cm]
\cajota{%
	Acceso a agua}%
{%
	
	Para el 2011, los departamentos en los cuales más del 50\% de los hogares rurales no contaban con la disponibilidad de agua potable fueron Alta Verapaz y Retalhuleu.}%
{%
	Proporción de hogares del área rural que no poseían acceso a agua potable
} %
{%
	Por departamento, año 2011, en porcentaje} %
{%
	\includegraphics[width=52\cuadri]{graficas/1_14.pdf}}%
{%
	Instituto Nacional de Estadística (INE), Censos Municipales 2008 - 2011.} %




%%%%%%%%%%%%%%%%%%%%%%%%%%%%%%%%%%14%%%%%%%%%%%%%%%%%%%%%%%%

\cajota{%
	Acceso a servicios de saneamiento}%
{%
	Esta necesidad básica clasificada en el acceso a servicios sanitarios, toma las variables de la Encuesta Nacional de Condiciones de Vida (ENCOVI), la cual  recoge datos sobre la disponibilidad de servicios sanitarios y sistemas de eliminación de excretas.
	
	Para el 2011, el 40.4\% de los hogares rurales en Chiquimula no contaba con acceso a servicios de saneamiento, y en Jutiapa, Petén y Jalapa, tres de cada diez hogares del área rural también carecían de este servicio.   }%
{%
	Proporción de hogares del área rural sin acceso a servicios de saneamiento }
{%
	Por departamento, año 2011, en porcentaje} %
{%
	\includegraphics[width=52\cuadri]{graficas/1_15.pdf}}%
{%
	Instituto Nacional de Estadística (INE), Censos Municipales 2008 - 2011. } %


%#########################5########################
\stepcounter{section}
\begin{center}
\textbf{\color{color2}\LARGE \thesection} \quad  \textbf{\LARGE Vivienda} \addcontentsline{toc}{section}{\numberline{\thesection} Vivienda}
\end{center}
$\ $ \\[-5cm]
\cajita{%
	Viviendas y acceso al agua}%
{%
	Para el 2014, el 78.1\% de los hogares contaba con acceso a agua. A pesar de esto se presentan diferencias significativas entre el área rural y el área urbana, teniendo menor acceso esta última.
}%
{%
	Viviendas que tienen acceso a agua, según área de residencia} %
{%
	República de Guatemala, 2014, en porcentaje} %
{%
	\begin{tikzpicture}[x=1pt,y=1pt]  \input{graficas/5_05.tex}  \end{tikzpicture}}%
{%
	Encovi, 2014} %
	

%#########################6########################

\cajita{%
	Viviendas y acceso a drenajes}%
{%
	
	Para el 2014 en la República de Guatemala, el acceso a drenajes no alcanzaba el 50\%. Además, en el área rural apenas el 11.6\% cuenta con drenajes. 
	
}%
{%
	Viviendas que tienen acceso a drenajes según, área	 de residencia} %
{%
	República de Guatemala, 2014, en porcentaje} %
{%
	\begin{tikzpicture}[x=1pt,y=1pt]  \input{graficas/5_06.tex}  \end{tikzpicture}}%
{%
	Encovi, 2014} %



%#########################1########################

 \cajita{%
Viviendas formales }%
{%
	En la República, el 91.4\% de los hogares que no son pobres tiene una vivienda formal, mientras que cerca del 20\%  de los hogares que está en pobreza extrema no cuenta con una vivienda formal.  
 }%
{%
 Hogares con viviendas formales, según tipo de pobreza} %
{%
 República de Guatemala, 2014, en porcentaje} %
{%
 \begin{tikzpicture}[x=1pt,y=1pt]  % Created by tikzDevice version 0.9 on 2016-03-03 05:25:42
% !TEX encoding = UTF-8 Unicode
\definecolor{fillColor}{RGB}{255,255,255}
\path[use as bounding box,fill=fillColor,fill opacity=0.00] (0,0) rectangle (289.08,198.74);
\begin{scope}
\path[clip] (  0.00,  0.00) rectangle (289.08,198.74);

\path[] (  0.00,  0.00) rectangle (289.08,198.74);
\end{scope}
\begin{scope}
\path[clip] (  0.00,  0.00) rectangle (289.08,198.74);

\path[] ( -0.52, 15.61) rectangle (280.54,191.48);

\path[] (  0.00, 23.49) --
	(280.54, 23.49);

\path[] (  0.00, 71.25) --
	(280.54, 71.25);

\path[] (  0.00,119.01) --
	(280.54,119.01);

\path[] (  0.00,166.77) --
	(280.54,166.77);

\path[] (  0.00, 47.37) --
	(280.54, 47.37);

\path[] (  0.00, 95.13) --
	(280.54, 95.13);

\path[] (  0.00,142.89) --
	(280.54,142.89);

\path[] (  0.00,190.65) --
	(280.54,190.65);

\path[] ( 52.18, 15.61) --
	( 52.18,191.48);

\path[] (140.01, 15.61) --
	(140.01,191.48);

\path[] (227.84, 15.61) --
	(227.84,191.48);
\definecolor{drawColor}{RGB}{0,0,255}

\path[draw=drawColor,line width= 1.7pt,line join=round] ( 52.18, 60.50) --
	(140.01,141.70) --
	(227.84,183.49);
\definecolor{drawColor}{RGB}{0,0,0}

\node[text=drawColor,anchor=base,inner sep=0pt, outer sep=0pt, scale=  1.02] at ( 52.18, 48.59) {81.1};

\node[text=drawColor,anchor=base east,inner sep=0pt, outer sep=0pt, scale=  1.02] at (136.88,141.70) {87.9};

\node[text=drawColor,anchor=base,inner sep=0pt, outer sep=0pt, scale=  1.02] at (227.84,187.46) {91.4};

\path[draw=drawColor,line width= 0.1pt,line join=round] (  0.00, 23.61) -- (280.54, 23.61);

\path[] ( -0.52, 15.61) rectangle (280.54,191.48);
\end{scope}
\begin{scope}
\path[clip] (  0.00,  0.00) rectangle (289.08,198.74);

\path[] (  0.00, 15.61) --
	(280.54, 15.61);
\end{scope}
\begin{scope}
\path[clip] (  0.00,  0.00) rectangle (289.08,198.74);

\path[] ( 52.18, 12.86) --
	( 52.18, 15.61);

\path[] (140.01, 12.86) --
	(140.01, 15.61);

\path[] (227.84, 12.86) --
	(227.84, 15.61);
\end{scope}
\begin{scope}
\path[clip] (  0.00,  0.00) rectangle (289.08,198.74);
\definecolor{drawColor}{RGB}{0,0,0}

\node[text=drawColor,anchor=base,inner sep=0pt, outer sep=0pt, scale=  1.00] at ( 52.18,  2.85) {Pobreza Extrema};

\node[text=drawColor,anchor=base,inner sep=0pt, outer sep=0pt, scale=  1.00] at (140.01,  2.85) {Pobreza no extrema};

\node[text=drawColor,anchor=base,inner sep=0pt, outer sep=0pt, scale=  1.00] at (227.84,  2.85) {No pobreza};
\end{scope}
  \end{tikzpicture}}%
{%
 Encovi, 2014} %



%%%%%%%%%%%%%%%%%%%%%%%%%%%%%%%%%%12%%%%%%%%%%%%%%%%%%%%%%%%

\cajota{%
	Hogares que viven en hacinamiento en los departamentos}%
{% 
	El hacinamiento toma las variables de la Encuesta Nacional de Condiciones de Vida (ENCOVI), las cuales  recogen información sobre el número de personas en el hogar y el número de cuartos de la vivienda. 	Los departamentos con mayor porcentaje de hogares en el área rural que vivían en hacinamiento en el 2011 fueron: Alta Verapaz (64.8\%), Quiché (59.9\%), Huehuetenango (54.6\%), San Marcos (54.6\%), Suchitepéquez (52.7\%), Izabal (52.5\%), Petén (51.4\%), y Jalapa (51.4\%). }%
{%
	Proporción de hogares del área rural que vive en hacinamiento
} %
{%
	Por departamento, año 2011, en porcentaje} %
{%
	\includegraphics[width=52\cuadri]{graficas/1_13.pdf}}%
{%
 Encovi, 2014} %


%#########################2########################

\cajita{%
	Viviendas con pared de block }%
{%
	Para el 2014, apenas el 22\% de los hogares en pobreza extrema tenía una vivienda con pared de block, mientras que en los hogares que  no son pobres este indicador se ubicaba en 75.7\%. 
}%
{%
	Viviendas con pared de block, según tipo de pobreza} %
{%
	República de Guatemala, 2014, en porcentaje} %
{%
	\begin{tikzpicture}[x=1pt,y=1pt]  % Created by tikzDevice version 0.9 on 2016-03-03 05:25:45
% !TEX encoding = UTF-8 Unicode
\definecolor{fillColor}{RGB}{255,255,255}
\path[use as bounding box,fill=fillColor,fill opacity=0.00] (0,0) rectangle (289.08,198.74);
\begin{scope}
\path[clip] (  0.00,  0.00) rectangle (289.08,198.74);

\path[] (  0.00,  0.00) rectangle (289.08,198.74);
\end{scope}
\begin{scope}
\path[clip] (  0.00,  0.00) rectangle (289.08,198.74);

\path[] ( -0.52, 15.61) rectangle (280.54,191.48);

\path[] (  0.00, 33.09) --
	(280.54, 33.09);

\path[] (  0.00, 78.90) --
	(280.54, 78.90);

\path[] (  0.00,124.70) --
	(280.54,124.70);

\path[] (  0.00,170.51) --
	(280.54,170.51);

\path[] (  0.00, 55.99) --
	(280.54, 55.99);

\path[] (  0.00,101.80) --
	(280.54,101.80);

\path[] (  0.00,147.61) --
	(280.54,147.61);

\path[] ( 52.18, 15.61) --
	( 52.18,191.48);

\path[] (140.01, 15.61) --
	(140.01,191.48);

\path[] (227.84, 15.61) --
	(227.84,191.48);
\definecolor{drawColor}{RGB}{0,0,255}

\path[draw=drawColor,line width= 1.7pt,line join=round] ( 52.18, 60.50) --
	(140.01,121.30) --
	(227.84,183.49);
\definecolor{drawColor}{RGB}{0,0,0}

\node[text=drawColor,anchor=base,inner sep=0pt, outer sep=0pt, scale=  1.02] at ( 52.18, 48.59) {22.0};

\node[text=drawColor,anchor=base east,inner sep=0pt, outer sep=0pt, scale=  1.02] at (136.88,121.30) {48.5};

\node[text=drawColor,anchor=base,inner sep=0pt, outer sep=0pt, scale=  1.02] at (227.84,187.46) {75.7};

\path[draw=drawColor,line width= 0.1pt,line join=round] (  0.00, 23.61) -- (280.54, 23.61);

\path[] ( -0.52, 15.61) rectangle (280.54,191.48);
\end{scope}
\begin{scope}
\path[clip] (  0.00,  0.00) rectangle (289.08,198.74);

\path[] (  0.00, 15.61) --
	(280.54, 15.61);
\end{scope}
\begin{scope}
\path[clip] (  0.00,  0.00) rectangle (289.08,198.74);

\path[] ( 52.18, 12.86) --
	( 52.18, 15.61);

\path[] (140.01, 12.86) --
	(140.01, 15.61);

\path[] (227.84, 12.86) --
	(227.84, 15.61);
\end{scope}
\begin{scope}
\path[clip] (  0.00,  0.00) rectangle (289.08,198.74);
\definecolor{drawColor}{RGB}{0,0,0}

\node[text=drawColor,anchor=base,inner sep=0pt, outer sep=0pt, scale=  1.00] at ( 52.18,  2.85) {Pobreza Extrema};

\node[text=drawColor,anchor=base,inner sep=0pt, outer sep=0pt, scale=  1.00] at (140.01,  2.85) {Pobreza no extrema};

\node[text=drawColor,anchor=base,inner sep=0pt, outer sep=0pt, scale=  1.00] at (227.84,  2.85) {No pobreza};
\end{scope}
  \end{tikzpicture}}%
{%
	Encovi, 2014} %

%#########################3########################

\cajita{%
	Viviendas con techo de lámina }%
{%
Los hogares que no son pobres utilizan en menor medida la lámina para la construcción del techo, con una diferencia cercana a los 25 puntos respecto a los hogares en pobreza extrema. 
	}%
{%
	Viviendas con techo de lámina,  según tipo de pobreza} %
{%
	República de Guatemala, 2014, en porcentaje} %
{%
	\begin{tikzpicture}[x=1pt,y=1pt]  % Created by tikzDevice version 0.9 on 2016-03-03 05:25:49
% !TEX encoding = UTF-8 Unicode
\definecolor{fillColor}{RGB}{255,255,255}
\path[use as bounding box,fill=fillColor,fill opacity=0.00] (0,0) rectangle (289.08,198.74);
\begin{scope}
\path[clip] (  0.00,  0.00) rectangle (289.08,198.74);

\path[] (  0.00,  0.00) rectangle (289.08,198.74);
\end{scope}
\begin{scope}
\path[clip] (  0.00,  0.00) rectangle (289.08,198.74);

\path[] ( -0.52, 15.61) rectangle (280.54,191.48);

\path[] (  0.00, 34.39) --
	(280.54, 34.39);

\path[] (  0.00, 62.22) --
	(280.54, 62.22);

\path[] (  0.00, 90.06) --
	(280.54, 90.06);

\path[] (  0.00,117.89) --
	(280.54,117.89);

\path[] (  0.00,145.72) --
	(280.54,145.72);

\path[] (  0.00,173.56) --
	(280.54,173.56);

\path[] (  0.00, 20.47) --
	(280.54, 20.47);

\path[] (  0.00, 48.31) --
	(280.54, 48.31);

\path[] (  0.00, 76.14) --
	(280.54, 76.14);

\path[] (  0.00,103.97) --
	(280.54,103.97);

\path[] (  0.00,131.81) --
	(280.54,131.81);

\path[] (  0.00,159.64) --
	(280.54,159.64);

\path[] (  0.00,187.47) --
	(280.54,187.47);

\path[] ( 52.18, 15.61) --
	( 52.18,191.48);

\path[] (140.01, 15.61) --
	(140.01,191.48);

\path[] (227.84, 15.61) --
	(227.84,191.48);
\definecolor{drawColor}{RGB}{0,0,255}

\path[draw=drawColor,line width= 1.7pt,line join=round] ( 52.18,183.49) --
	(140.01,172.52) --
	(227.84, 60.50);
\definecolor{drawColor}{RGB}{0,0,0}

\node[text=drawColor,anchor=base,inner sep=0pt, outer sep=0pt, scale=  1.02] at ( 52.18,187.46) {84.3};

\node[text=drawColor,anchor=base west,inner sep=0pt, outer sep=0pt, scale=  1.02] at (140.01,176.49) {82.3};

\node[text=drawColor,anchor=base,inner sep=0pt, outer sep=0pt, scale=  1.02] at (227.84, 48.59) {62.2};

\path[draw=drawColor,line width= 0.1pt,line join=round] (  0.00, 23.61) -- (280.54, 23.61);

\path[] ( -0.52, 15.61) rectangle (280.54,191.48);
\end{scope}
\begin{scope}
\path[clip] (  0.00,  0.00) rectangle (289.08,198.74);

\path[] (  0.00, 15.61) --
	(280.54, 15.61);
\end{scope}
\begin{scope}
\path[clip] (  0.00,  0.00) rectangle (289.08,198.74);

\path[] ( 52.18, 12.86) --
	( 52.18, 15.61);

\path[] (140.01, 12.86) --
	(140.01, 15.61);

\path[] (227.84, 12.86) --
	(227.84, 15.61);
\end{scope}
\begin{scope}
\path[clip] (  0.00,  0.00) rectangle (289.08,198.74);
\definecolor{drawColor}{RGB}{0,0,0}

\node[text=drawColor,anchor=base,inner sep=0pt, outer sep=0pt, scale=  1.00] at ( 52.18,  2.85) {Pobreza Extrema};

\node[text=drawColor,anchor=base,inner sep=0pt, outer sep=0pt, scale=  1.00] at (140.01,  2.85) {Pobreza no extrema};

\node[text=drawColor,anchor=base,inner sep=0pt, outer sep=0pt, scale=  1.00] at (227.84,  2.85) {No pobreza};
\end{scope}
  \end{tikzpicture}}%
{%
	Encovi, 2014} %

%#########################4########################

\cajita{%
	Viviendas y material del piso}%
{%
	Los hogares en   pobreza extrema son los que presentan una menor proporción en el uso de torta de cemento (25.3\% contra un 67\%). Para los hogares que no son pobres predomina el uso de torta de cemento, ya que el 40.2\% de los hogares lo usa, contra un 10.4\% que usa piso de tierra. 
}%
{%
	Viviendas según material del piso y tipo de pobreza} %
{%
	República de Guatemala, 2014, en porcentaje} %
{%
	\begin{tikzpicture}[x=1pt,y=1pt]  % Created by tikzDevice version 0.9 on 2016-03-03 05:25:52
% !TEX encoding = UTF-8 Unicode
\definecolor{fillColor}{RGB}{255,255,255}
\path[use as bounding box,fill=fillColor,fill opacity=0.00] (0,0) rectangle (289.08,198.74);
\begin{scope}
\path[clip] (  0.00,  0.00) rectangle (289.08,198.74);

\path[] (  0.00,  0.00) rectangle (289.08,198.74);
\end{scope}
\begin{scope}
\path[clip] (  0.00,  0.00) rectangle (289.08,198.74);

\path[] (  0.00, 18.46) rectangle (289.08,166.57);

\path[] ( 54.20, 18.46) --
	( 54.20,166.57);

\path[] (144.54, 18.46) --
	(144.54,166.57);

\path[] (234.88, 18.46) --
	(234.88,166.57);
\definecolor{drawColor}{RGB}{0,0,255}
\definecolor{fillColor}{RGB}{0,0,255}

\path[draw=drawColor,line width= 0.6pt,line join=round,fill=fillColor] ( 15.81, 18.46) rectangle ( 51.94, 74.33);
\definecolor{drawColor}{RGB}{157,187,255}
\definecolor{fillColor}{RGB}{157,187,255}

\path[draw=drawColor,line width= 0.6pt,line join=round,fill=fillColor] ( 56.46, 18.46) rectangle ( 92.60,166.57);
\definecolor{drawColor}{RGB}{0,0,255}
\definecolor{fillColor}{RGB}{0,0,255}

\path[draw=drawColor,line width= 0.6pt,line join=round,fill=fillColor] (106.15, 18.46) rectangle (142.28,120.72);
\definecolor{drawColor}{RGB}{157,187,255}
\definecolor{fillColor}{RGB}{157,187,255}

\path[draw=drawColor,line width= 0.6pt,line join=round,fill=fillColor] (146.80, 18.46) rectangle (182.93, 95.86);
\definecolor{drawColor}{RGB}{0,0,255}
\definecolor{fillColor}{RGB}{0,0,255}

\path[draw=drawColor,line width= 0.6pt,line join=round,fill=fillColor] (196.48, 18.46) rectangle (232.62,107.27);
\definecolor{drawColor}{RGB}{157,187,255}
\definecolor{fillColor}{RGB}{157,187,255}

\path[draw=drawColor,line width= 0.6pt,line join=round,fill=fillColor] (237.14, 18.46) rectangle (273.27, 41.36);
\definecolor{drawColor}{RGB}{0,0,0}

\path[draw=drawColor,line width= 0.6pt,line join=round] (  0.00, 18.46) -- (289.08, 18.46);

\node[text=drawColor,rotate= 90.00,anchor=base west,inner sep=0pt, outer sep=0pt, scale=  0.83] at ( 37.11, 76.88) {25.3};

\node[text=drawColor,rotate= 90.00,anchor=base west,inner sep=0pt, outer sep=0pt, scale=  0.83] at ( 77.76,169.12) {67.0};

\node[text=drawColor,rotate= 90.00,anchor=base west,inner sep=0pt, outer sep=0pt, scale=  0.83] at (127.45,123.26) {46.3};

\node[text=drawColor,rotate= 90.00,anchor=base west,inner sep=0pt, outer sep=0pt, scale=  0.83] at (168.10, 98.40) {35.0};

\node[text=drawColor,rotate= 90.00,anchor=base west,inner sep=0pt, outer sep=0pt, scale=  0.83] at (217.79,109.81) {40.2};

\node[text=drawColor,rotate= 90.00,anchor=base west,inner sep=0pt, outer sep=0pt, scale=  0.83] at (258.44, 43.91) {10.4};

\path[] (  0.00, 18.46) rectangle (289.08,166.57);
\end{scope}
\begin{scope}
\path[clip] (  0.00,  0.00) rectangle (289.08,198.74);

\path[] (  0.00, 18.46) --
	(289.08, 18.46);
\end{scope}
\begin{scope}
\path[clip] (  0.00,  0.00) rectangle (289.08,198.74);

\path[] ( 54.20, 15.71) --
	( 54.20, 18.46);

\path[] (144.54, 15.71) --
	(144.54, 18.46);

\path[] (234.88, 15.71) --
	(234.88, 18.46);
\end{scope}
\begin{scope}
\path[clip] (  0.00,  0.00) rectangle (289.08,198.74);
\definecolor{drawColor}{RGB}{0,0,0}

\node[text=drawColor,anchor=base,inner sep=0pt, outer sep=0pt, scale=  1.00] at ( 54.20,  5.69) {Pobreza Extrema};

\node[text=drawColor,anchor=base,inner sep=0pt, outer sep=0pt, scale=  1.00] at (144.54,  5.69) {Pobreza no extrema};

\node[text=drawColor,anchor=base,inner sep=0pt, outer sep=0pt, scale=  1.00] at (234.88,  5.69) {No pobreza};
\end{scope}
\begin{scope}
\path[clip] (  0.00,  0.00) rectangle (289.08,198.74);
\coordinate (apoyo) at (41.26,191.13);
\coordinate (longitudFicticia) at (7.11,7.61);
\coordinate (longitud) at (7.11,7.11);
\coordinate (desX) at (157.45,0);
\coordinate (desY) at (0,0.25);
\definecolor[named]{ct1}{HTML}{
0000FF
}
\definecolor[named]{ct2}{HTML}{
9DBBFF
}
\definecolor[named]{ctb1}{HTML}{
0000FF
}
\definecolor[named]{ctb2}{HTML}{
9DBBFF
}
\path [fill=none] (apoyo) rectangle ($(apoyo)+(longitudFicticia)$)
node [xshift=0.3cm,inner sep=0pt, outer sep=0pt,midway,right,scale = 0.9]{Torta de cemento};
\draw [color = ctb1,fill=ct1] ( $(apoyo)  + (desY) $) rectangle ($(apoyo)+ (desY) +(longitud)$);
\path [fill=none] ($(apoyo)+(desX)$) rectangle ($(apoyo)+(desX)+(longitudFicticia)$)
node [xshift=0.3cm,inner sep=0pt, outer sep=0pt,midway,right,scale = 0.9]{Tierra};
\draw [color = ctb2 ,fill=ct2] ( $(apoyo)  + (desY) + (desX) $) rectangle ($(apoyo)+ (desY)+ (desX) +(longitud)$);
\end{scope}
  \end{tikzpicture}}%
{%
	Encovi, 2014} %


	\INEchaptercarta{Situación y atención a la desnutrición o malnutrición}{Se refiere a la desnutrición aguda y la atención que se la da a la misma. La desnutrición aguda, también conocida como emaciación o bajo peso para la talla, es causada por una ingesta insuficiente de alimentos y por la presencia de infecciones graves durante períodos prolongados. Por otro lado, la desnutrición crónica o de baja talla para la edad, es el retardo en el crecimiento lineal de los niños debido a una mala alimentación e infecciones agudas repetidas (Coneval, 2010). Esta dimensión está compuesta por tres indicadores y siete sub-indicadores.
		
		El indicador \textbf{6.1 Situación Nutricional de la Madre}, se compone de los sub-indicadores: a) adecuación de peso de la madre durante el embarazo, b) talla de las mujeres en edad fértil, c) deficiencia de micronutrientes en mujeres en edad fértil, y d) sobrepeso y obesidad en mujeres en edad fértil.
		
		El indicador \textbf{6.2 Situación Nutricional del Niño}, se compone de los sub-indicadores: a) peso al nacer, b) desnutrición crónica en menores de 5 años – retardo en talla, c) desnutrición global en menores de 5 años – retardo en peso, d) desnutrición aguda en menores de 5 años – retardo en peso para la talla, e) deficiencia de micronutrientes en niños menores de 5 años, y f) sobrepeso y obesidad en niños. 
		
		Por último, el indicador \textbf{6.3 Atención a Desnutridos}, se compone del sub-indicador de tratamiento y recuperación de desnutridos, institucional y ambulatorio. Para este indicador no se encontró información disponible.
		}
	\stepcounter{section}
\begin{center}
	\textbf{\color{color2}\LARGE \thesection} \quad  \textbf{\LARGE Situación nutricional de la madre} \addcontentsline{toc}{section}{\numberline{\thesection} Situación nutricional de la madre}
\end{center}
$ \ $ \\[-5cm]%#########################1########################

 \cajita{%
Mujeres con anemia }%
{Para el año 2008 se registró que del total de las madres embarazadas el 29.1\% padecía anemia, mientras que en las mujeres que no se encontraban en gestación el porcentaje de anemia bajó en 7.7 puntos porcentuales. 
 }%
{%
 Mujeres con anemia, según estado de gestación} %
{%
 República de Guatemala, 2008-2009, en porcentaje} %
{%
 \begin{tikzpicture}[x=1pt,y=1pt]  \input{graficas/6_01.tex}  \end{tikzpicture}}%
{%
Encuesta Nacional de Salud Materno Infantil (Ensmi), 2008/2009 } %

%#########################2########################

\cajita{%
	Índice de Masa Corporal (IMC) de mujeres  }%
{%
	La mayoría de las mujeres guatemaltecas que no se encontraban embarazadas en el 2008 presentaba un IMC normal, mientras que solo un 1.6\% tenía IMC bajo. 
}%
{%
	Índice de masa corporal de mujeres no embarazadas} %
{%
	República de Guatemala, 2008-2009, en porcentaje} %
{%
	\begin{tikzpicture}[x=1pt,y=1pt]  % Created by tikzDevice version 0.9 on 2016-03-03 06:06:28
% !TEX encoding = UTF-8 Unicode
\definecolor{fillColor}{RGB}{255,255,255}
\path[use as bounding box,fill=fillColor,fill opacity=0.00] (0,0) rectangle (289.08,198.74);
\begin{scope}
\path[clip] (  0.00,  0.00) rectangle (289.08,198.74);

\path[] (  0.00,  0.00) rectangle (289.08,198.74);
\end{scope}
\begin{scope}
\path[clip] (  0.00,  0.00) rectangle (289.08,198.74);

\path[] (  0.00, 12.77) rectangle (289.08,181.67);

\path[] ( 41.30, 12.77) --
	( 41.30,181.67);

\path[] (110.13, 12.77) --
	(110.13,181.67);

\path[] (178.95, 12.77) --
	(178.95,181.67);

\path[] (247.78, 12.77) --
	(247.78,181.67);
\definecolor{drawColor}{RGB}{0,0,255}
\definecolor{fillColor}{RGB}{0,0,255}

\path[draw=drawColor,line width= 0.6pt,line join=round,fill=fillColor] ( 24.09, 20.44) rectangle ( 58.50, 25.57);

\path[draw=drawColor,line width= 0.6pt,line join=round,fill=fillColor] ( 92.92, 20.44) rectangle (127.33,173.99);

\path[draw=drawColor,line width= 0.6pt,line join=round,fill=fillColor] (161.75, 20.44) rectangle (196.16,132.96);

\path[draw=drawColor,line width= 0.6pt,line join=round,fill=fillColor] (230.58, 20.44) rectangle (264.99, 69.81);
\definecolor{drawColor}{RGB}{0,0,0}

\path[draw=drawColor,line width= 0.1pt,line join=round] (  0.00, 20.44) -- (289.08, 20.44);

\node[text=drawColor,anchor=base,inner sep=0pt, outer sep=0pt, scale=  1.02] at ( 41.30, 29.54) {1.6};

\node[text=drawColor,anchor=base,inner sep=0pt, outer sep=0pt, scale=  1.02] at (110.13,177.96) {47.9};

\node[text=drawColor,anchor=base,inner sep=0pt, outer sep=0pt, scale=  1.02] at (178.95,136.93) {35.1};

\node[text=drawColor,anchor=base,inner sep=0pt, outer sep=0pt, scale=  1.02] at (247.78, 73.78) {15.4};

\path[] (  0.00, 12.77) rectangle (289.08,181.67);
\end{scope}
\begin{scope}
\path[clip] (  0.00,  0.00) rectangle (289.08,198.74);

\path[] (  0.00, 12.77) --
	(289.08, 12.77);
\end{scope}
\begin{scope}
\path[clip] (  0.00,  0.00) rectangle (289.08,198.74);

\path[] ( 41.30, 10.02) --
	( 41.30, 12.77);

\path[] (110.13, 10.02) --
	(110.13, 12.77);

\path[] (178.95, 10.02) --
	(178.95, 12.77);

\path[] (247.78, 10.02) --
	(247.78, 12.77);
\end{scope}
\begin{scope}
\path[clip] (  0.00,  0.00) rectangle (289.08,198.74);
\definecolor{drawColor}{RGB}{0,0,0}

\node[text=drawColor,anchor=base,inner sep=0pt, outer sep=0pt, scale=  1.00] at ( 41.30, 3.00) {Bajo};

\node[text=drawColor,anchor=base,inner sep=0pt, outer sep=0pt, scale=  1.00] at (110.13, 3.00) {Normal};

\node[text=drawColor,anchor=base,inner sep=0pt, outer sep=0pt, scale=  1.00] at (178.95, 3.00) {Sobrepeso};

\node[text=drawColor,anchor=base,inner sep=0pt, outer sep=0pt, scale=  1.00] at (247.78, 3.00) {Obesidad};
\end{scope}
  \end{tikzpicture}}%
{%
	Encuesta Nacional de Salud Materno Infantil (Ensmi), 2008/2009} %


%#########################3########################

\cajita{%
	IMC de mujeres, según área de residencia }%
{%
	En el área urbana se presenta mayor proporción de mujeres con sobrepeso y obesidad (37.5\% y 20.3\%) que en el área rural (33.4\% y 12.1\%). 
	Además, el área rural presenta mayor proporción de mujeres con IMC normal que el área urbana. 
}%
{%
	Índice de Masa Corporal (IMC) de mujeres no embarazadas, según área de residencia} %
{%
	República de Guatemala, 2008-2009, en porcentaje} %
{%
	\begin{tikzpicture}[x=1pt,y=1pt]  % Created by tikzDevice version 0.9 on 2016-03-03 06:06:32
% !TEX encoding = UTF-8 Unicode
\definecolor{fillColor}{RGB}{255,255,255}
\path[use as bounding box,fill=fillColor,fill opacity=0.00] (0,0) rectangle (289.08,198.74);
\begin{scope}
\path[clip] (  0.00,  0.00) rectangle (289.08,198.74);

\path[] (  0.00,  0.00) rectangle (289.08,198.74);
\end{scope}
\begin{scope}
\path[clip] (  0.00,  0.00) rectangle (289.08,198.74);

\path[] (  0.00, 18.46) rectangle (289.08,166.57);

\path[] ( 41.30, 18.46) --
	( 41.30,166.57);

\path[] (110.13, 18.46) --
	(110.13,166.57);

\path[] (178.95, 18.46) --
	(178.95,166.57);

\path[] (247.78, 18.46) --
	(247.78,166.57);
\definecolor{drawColor}{RGB}{0,0,255}
\definecolor{fillColor}{RGB}{0,0,255}

\path[draw=drawColor,line width= 0.6pt,line join=round,fill=fillColor] ( 12.05, 18.46) rectangle ( 39.58, 23.49);
\definecolor{drawColor}{RGB}{157,187,255}
\definecolor{fillColor}{RGB}{157,187,255}

\path[draw=drawColor,line width= 0.6pt,line join=round,fill=fillColor] ( 43.02, 18.46) rectangle ( 70.55, 22.65);
\definecolor{drawColor}{RGB}{0,0,255}
\definecolor{fillColor}{RGB}{0,0,255}

\path[draw=drawColor,line width= 0.6pt,line join=round,fill=fillColor] ( 80.87, 18.46) rectangle (108.40,131.64);
\definecolor{drawColor}{RGB}{157,187,255}
\definecolor{fillColor}{RGB}{157,187,255}

\path[draw=drawColor,line width= 0.6pt,line join=round,fill=fillColor] (111.85, 18.46) rectangle (139.38,166.57);
\definecolor{drawColor}{RGB}{0,0,255}
\definecolor{fillColor}{RGB}{0,0,255}

\path[draw=drawColor,line width= 0.6pt,line join=round,fill=fillColor] (149.70, 18.46) rectangle (177.23,123.26);
\definecolor{drawColor}{RGB}{157,187,255}
\definecolor{fillColor}{RGB}{157,187,255}

\path[draw=drawColor,line width= 0.6pt,line join=round,fill=fillColor] (180.68, 18.46) rectangle (208.21,111.80);
\definecolor{drawColor}{RGB}{0,0,255}
\definecolor{fillColor}{RGB}{0,0,255}

\path[draw=drawColor,line width= 0.6pt,line join=round,fill=fillColor] (218.53, 18.46) rectangle (246.06, 75.19);
\definecolor{drawColor}{RGB}{157,187,255}
\definecolor{fillColor}{RGB}{157,187,255}

\path[draw=drawColor,line width= 0.6pt,line join=round,fill=fillColor] (249.50, 18.46) rectangle (277.03, 52.27);
\definecolor{drawColor}{RGB}{0,0,0}

\path[draw=drawColor,line width= 0.6pt,line join=round] (  0.00, 18.46) -- (289.08, 18.46);

\node[text=drawColor,rotate= 90.00,anchor=base west,inner sep=0pt, outer sep=0pt, scale=  0.83] at ( 29.05, 25.31) {1.8};

\node[text=drawColor,rotate= 90.00,anchor=base west,inner sep=0pt, outer sep=0pt, scale=  0.83] at ( 60.02, 24.47) {1.5};

\node[text=drawColor,rotate= 90.00,anchor=base west,inner sep=0pt, outer sep=0pt, scale=  0.83] at ( 97.87,134.19) {40.5};

\node[text=drawColor,rotate= 90.00,anchor=base west,inner sep=0pt, outer sep=0pt, scale=  0.83] at (128.85,169.12) {53.0};

\node[text=drawColor,rotate= 90.00,anchor=base west,inner sep=0pt, outer sep=0pt, scale=  0.83] at (166.70,125.81) {37.5};

\node[text=drawColor,rotate= 90.00,anchor=base west,inner sep=0pt, outer sep=0pt, scale=  0.83] at (197.68,114.35) {33.4};

\node[text=drawColor,rotate= 90.00,anchor=base west,inner sep=0pt, outer sep=0pt, scale=  0.83] at (235.53, 77.74) {20.3};

\node[text=drawColor,rotate= 90.00,anchor=base west,inner sep=0pt, outer sep=0pt, scale=  0.83] at (266.50, 54.82) {12.1};

\path[] (  0.00, 18.46) rectangle (289.08,166.57);
\end{scope}
\begin{scope}
\path[clip] (  0.00,  0.00) rectangle (289.08,198.74);

\path[] (  0.00, 18.46) --
	(289.08, 18.46);
\end{scope}
\begin{scope}
\path[clip] (  0.00,  0.00) rectangle (289.08,198.74);

\path[] ( 41.30, 15.71) --
	( 41.30, 18.46);

\path[] (110.13, 15.71) --
	(110.13, 18.46);

\path[] (178.95, 15.71) --
	(178.95, 18.46);

\path[] (247.78, 15.71) --
	(247.78, 18.46);
\end{scope}
\begin{scope}
\path[clip] (  0.00,  0.00) rectangle (289.08,198.74);
\definecolor{drawColor}{RGB}{0,0,0}

\node[text=drawColor,anchor=base,inner sep=0pt, outer sep=0pt, scale=  1.00] at ( 41.30,  5.69) {Bajo};

\node[text=drawColor,anchor=base,inner sep=0pt, outer sep=0pt, scale=  1.00] at (110.13,  5.69) {Normal};

\node[text=drawColor,anchor=base,inner sep=0pt, outer sep=0pt, scale=  1.00] at (178.95,  5.69) {Sobre peso};

\node[text=drawColor,anchor=base,inner sep=0pt, outer sep=0pt, scale=  1.00] at (247.78,  5.69) {Obesidad};
\end{scope}
\begin{scope}
\path[clip] (  0.00,  0.00) rectangle (289.08,198.74);
\coordinate (apoyo) at (57.27,191.13);
\coordinate (longitudFicticia) at (7.11,7.61);
\coordinate (longitud) at (7.11,7.11);
\coordinate (desX) at (142.24,0);
\coordinate (desY) at (0,0.25);
\definecolor[named]{ct1}{HTML}{
0000FF
}
\definecolor[named]{ct2}{HTML}{
9DBBFF
}
\definecolor[named]{ctb1}{HTML}{
0000FF
}
\definecolor[named]{ctb2}{HTML}{
9DBBFF
}
\path [fill=none] (apoyo) rectangle ($(apoyo)+(longitudFicticia)$)
node [xshift=0.3cm,inner sep=0pt, outer sep=0pt,midway,right,scale = 0.9]{Urbana};
\draw [color = ctb1,fill=ct1] ( $(apoyo)  + (desY) $) rectangle ($(apoyo)+ (desY) +(longitud)$);
\path [fill=none] ($(apoyo)+(desX)$) rectangle ($(apoyo)+(desX)+(longitudFicticia)$)
node [xshift=0.3cm,inner sep=0pt, outer sep=0pt,midway,right,scale = 0.9]{Rural};
\draw [color = ctb2 ,fill=ct2] ( $(apoyo)  + (desY) + (desX) $) rectangle ($(apoyo)+ (desY)+ (desX) +(longitud)$);
\end{scope}
  \end{tikzpicture}}%
{%
	Encuesta Nacional de Salud Materno Infantil (Ensmi), 2008/2009} %


%#########################4########################

\cajita{%
	Talla promedio según área de residencia }%
{%
	Para el 2008 la talla\footnote{Hace referencia a la estatura de una persona, y en este caso se presenta en centímetros.} para madres del área urbana de niños menores de 5 años es apenas mayor que la de las madres del área rural. 
}%
{%
	Talla promedio de madres  de niños menores de 5 años, según área de residencia} %
{%
	República de Guatemala, 2008-2009, en centímetros} %
{%
	\begin{tikzpicture}[x=1pt,y=1pt]  \input{graficas/6_04.tex}  \end{tikzpicture}}%
{%
	Encuesta Nacional de Salud Materno Infantil (Ensmi), 2008/2009} %

%#########################5########################
\stepcounter{section}
\begin{center}
	\textbf{\color{color2}\LARGE \thesection} \quad  \textbf{\LARGE Situación nutricional del niño} \addcontentsline{toc}{section}{\numberline{\thesection} Situación nutricional del niño}
\end{center} $ \ $\\[-5cm]
\cajita{%
	Peso de niños recién nacidos }%
{%
	Uno de los datos más importantes que debe ser medido en un neonato es el peso.
	Para el 2008, el 88.1\% de los recién nacidos presentó un peso no menor de  2.5 kilogramos. 
}%
{%
	Distribución de niños recién nacidos, según peso (en kilogramos), reportado por la madre} %
{%
	República de Guatemala, 2008-2009, en porcentaje} %
{%
	\begin{tikzpicture}[x=1pt,y=1pt]  \input{graficas/6_041.tex}  \end{tikzpicture}}%
{%
	Encuesta Nacional de Salud Materno Infantil (Ensmi), 2008/2009} %

%#########################6########################

\cajita{%
	Desnutrición global }%
{%
	La tasa de desnutricion global total para el 2008 fue de 13.1\%, siendo el 2.1\% desnutrición global severa. 
}%
{%
	Niños menores de 5 años con desnutrición global} %
{%
	República de Guatemala, 2008-2009, en porcentaje} %
{%
	\begin{tikzpicture}[x=1pt,y=1pt]  \input{graficas/6_05.tex}  \end{tikzpicture}}%
{%
	Encuesta Nacional de Salud Materno Infantil (Ensmi), 2008/2009} %

%#########################7########################

\cajita{%
	Desnutrición aguda }%
{%
	La desnutrición aguda (bajo peso para la talla) se ubicó en 1.4\% para el 2008.
}%
{%
	Niños menores de 5 años con desnutrición aguda} %
{%
	República de Guatemala, 2008-2009, en porcentaje} %
{%
	\begin{tikzpicture}[x=1pt,y=1pt]  \input{graficas/6_06.tex}  \end{tikzpicture}}%
{%
	Encuesta Nacional de Salud Materno Infantil (Ensmi), 2008/2009} %


%#########################8########################

\cajita{%
	Desnutrición crónica }%
{%
	La desnutrición crónica es el retraso del crecimiento esperado para una edad. 
	
	Para niños menores de 5 años se observó en 2008 que el 49.8\% de estos presenta una talla menor a la esperada para su edad, es decir que se encontraban con desnutrición crónica. 
}%
{%
	Niños menores de 5 años con desnutrición crónica} %
{%
	República de Guatemala, 2008-2009, en porcentaje} %
{%
	\begin{tikzpicture}[x=1pt,y=1pt]  \input{graficas/6_07.tex}  \end{tikzpicture}}%
{%
	Encuesta Nacional de Salud Materno Infantil (Ensmi), 2008/2009} %


%#########################9########################

\cajita{%
	Niños con anemia }%
{%
	El área rural presenta mayor porcentaje de niños con anemia, ubicándose en 48.6\%, esto está 2.4 puntos por arriba del indicador para el área urbana. 
	}%
{%
	Niños entre 6 y 59 meses con anemia, según área de residencia} %
{%
	República de Guatemala, 2008-2009, en porcentaje} %
{%
	\begin{tikzpicture}[x=1pt,y=1pt]  \input{graficas/6_08.tex}  \end{tikzpicture}}%
{%
	Encuesta Nacional de Salud Materno Infantil (Ensmi), 2008/2009} %

%#########################9########################

\cajota{%
	Niños con anemia por departamento }%
{%
	El departamento de Suchitepéquez es el que presenta menor porcentaje de niños con anemia (37.7\%), seguido de El Progreso (37.8\%), y Quetzaltenango (40.2\%). Los departamentos con mayor porcentaje de anemia en niños son Totonicapán (62.2\%), Sololá (56.1\%), y Chiquimula (55.5\%).
}%
{%
	Niños entre 6 y 59 meses con anemia por departamento} %
{%
	República de Guatemala, 2008-2009, en porcentaje} %
{%
	\includegraphics[width=52\cuadri]{graficas/6_09.pdf}}%
{%
	Encuesta Nacional de Salud Materno Infantil (Ensmi), 2008/2009} %
	\INEchaptercarta{Inversión pública en SAN }{La dimensión sobre Inversión Pública en Seguridad Alimentaria y Nutricional tiene como finalidad recabar información sobre los logros alcanzados en materia de SAN en los últimos años para Guatemala. Esta dimensión se compone por un indicador, el cual reúne tres sub-indicadores.
		
		El indicador 7.1, se compone de los sub-indicadores: a) integración del presupuesto a nivel nacional, b) gasto del Plan del Pacto Hambre Cero, y c) ventana de los mil días.		}
	%#########################1########################

 \cajita{%
Asignación presupuestaria a la SESAN }%
{%
	La SESAN es la Secretaría de Seguridad Alimentaria y Nutricional de la presidencia de la república. Es importante conocer el porcentaje de asignación que se hace a esta secretaria respecto al presupuesto de ingresos e ingresos de la nación. En 2015 está asignación fue del 7.7\%\footnote{La asignación presupuestaria no siempre logra ser ejecutada en su totalidad por las instituciones de gobierno}.  
 }%
{%
 Tasa de asignación presupuestaria a la SESAN respecto del presupuesto general de la nación } %
{%
 República de Guatemala, serie histórica, en porcentaje} %
{%
 \begin{tikzpicture}[x=1pt,y=1pt]  % Created by tikzDevice version 0.9 on 2016-03-03 06:36:56
% !TEX encoding = UTF-8 Unicode
\definecolor{fillColor}{RGB}{255,255,255}
\path[use as bounding box,fill=fillColor,fill opacity=0.00] (0,0) rectangle (289.08,198.74);
\begin{scope}
\path[clip] (  0.00,  0.00) rectangle (289.08,198.74);

\path[] (  0.00,  0.00) rectangle (289.08,198.74);
\end{scope}
\begin{scope}
\path[clip] (  0.00,  0.00) rectangle (289.08,198.74);

\path[] (  1.74, 15.61) rectangle (280.54,191.48);

\path[] (  1.74, 37.77) --
	(280.54, 37.77);

\path[] (  1.74, 86.42) --
	(280.54, 86.42);

\path[] (  1.74,135.08) --
	(280.54,135.08);

\path[] (  1.74,183.73) --
	(280.54,183.73);

\path[] (  1.74, 62.10) --
	(280.54, 62.10);

\path[] (  1.74,110.75) --
	(280.54,110.75);

\path[] (  1.74,159.40) --
	(280.54,159.40);

\path[] ( 54.01, 15.61) --
	( 54.01,191.48);

\path[] (141.14, 15.61) --
	(141.14,191.48);

\path[] (228.27, 15.61) --
	(228.27,191.48);
\definecolor{drawColor}{RGB}{0,0,255}

\path[draw=drawColor,line width= 1.7pt,line join=round] ( 54.01,183.49) --
	(141.14,144.70) --
	(228.27, 60.50);
\definecolor{drawColor}{RGB}{0,0,0}

\node[text=drawColor,anchor=base,inner sep=0pt, outer sep=0pt, scale=  1.02] at ( 54.01,187.46) {7.9};

\node[text=drawColor,anchor=base west,inner sep=0pt, outer sep=0pt, scale=  1.02] at (141.14,148.68) {7.9};

\node[text=drawColor,anchor=base,inner sep=0pt, outer sep=0pt, scale=  1.02] at (228.27, 48.59) {7.7};

\path[draw=drawColor,line width= 0.1pt,line join=round] (  1.74, 23.61) -- (280.54, 23.61);

\path[] (  1.74, 15.61) rectangle (280.54,191.48);
\end{scope}
\begin{scope}
\path[clip] (  0.00,  0.00) rectangle (289.08,198.74);

\path[] (  1.74, 15.61) --
	(  1.74,191.48);
\end{scope}
\begin{scope}
\path[clip] (  0.00,  0.00) rectangle (289.08,198.74);

\path[] (  0.00, 62.10) --
	(  1.74, 62.10);

\path[] (  0.00,110.75) --
	(  1.74,110.75);

\path[] (  0.00,159.40) --
	(  1.74,159.40);
\end{scope}
\begin{scope}
\path[clip] (  0.00,  0.00) rectangle (289.08,198.74);

\path[] (  1.74, 15.61) --
	(280.54, 15.61);
\end{scope}
\begin{scope}
\path[clip] (  0.00,  0.00) rectangle (289.08,198.74);

\path[] ( 54.01, 12.86) --
	( 54.01, 15.61);

\path[] (141.14, 12.86) --
	(141.14, 15.61);

\path[] (228.27, 12.86) --
	(228.27, 15.61);
\end{scope}
\begin{scope}
\path[clip] (  0.00,  0.00) rectangle (289.08,198.74);
\definecolor{drawColor}{RGB}{0,0,0}

\node[text=drawColor,anchor=base,inner sep=0pt, outer sep=0pt, scale=  1.00] at ( 54.01,  2.85) {2012};

\node[text=drawColor,anchor=base,inner sep=0pt, outer sep=0pt, scale=  1.00] at (141.14,  2.85) {2014};

\node[text=drawColor,anchor=base,inner sep=0pt, outer sep=0pt, scale=  1.00] at (228.27,  2.85) {2015};
\end{scope}
  \end{tikzpicture}}%
{%
 SICOIN} %


 \cajita{%
 	Ejecución presupuestaria de la SESAN }%
 {%
 	El porcentaje de ejecución presupuestaria es la razón entre el monto total erogado por la SESAN respecto de su asignación presupuestaria. 
 	
 	La ejecución presupuestaria de la SESAN entre el 2012 y 2015 ha variado entre 66.6 y 89.2 por ciento, teniendo su ejecuación más baja en el 2015.
 }%
 {%
 	Tasa de ejecución de la SESAN respecto al presupuesto asignado } %
 {%
 	República de Guatemala, serie histórica, en porcentaje} %
 {%
 	\begin{tikzpicture}[x=1pt,y=1pt]  % Created by tikzDevice version 0.9 on 2016-03-03 06:36:58
% !TEX encoding = UTF-8 Unicode
\definecolor{fillColor}{RGB}{255,255,255}
\path[use as bounding box,fill=fillColor,fill opacity=0.00] (0,0) rectangle (289.08,198.74);
\begin{scope}
\path[clip] (  0.00,  0.00) rectangle (289.08,198.74);

\path[] (  0.00,  0.00) rectangle (289.08,198.74);
\end{scope}
\begin{scope}
\path[clip] (  0.00,  0.00) rectangle (289.08,198.74);

\path[] ( -0.52, 15.61) rectangle (280.54,191.48);

\path[] (  0.00, 51.79) --
	(280.54, 51.79);

\path[] (  0.00,106.22) --
	(280.54,106.22);

\path[] (  0.00,160.66) --
	(280.54,160.66);

\path[] (  0.00, 24.58) --
	(280.54, 24.58);

\path[] (  0.00, 79.01) --
	(280.54, 79.01);

\path[] (  0.00,133.44) --
	(280.54,133.44);

\path[] (  0.00,187.87) --
	(280.54,187.87);

\path[] ( 39.63, 15.61) --
	( 39.63,191.48);

\path[] (106.55, 15.61) --
	(106.55,191.48);

\path[] (173.47, 15.61) --
	(173.47,191.48);

\path[] (240.39, 15.61) --
	(240.39,191.48);
\definecolor{drawColor}{RGB}{0,0,255}

\path[draw=drawColor,line width= 1.7pt,line join=round] ( 39.63,183.49) --
	(106.55,106.32) --
	(173.47,161.99) --
	(240.39, 60.50);
\definecolor{drawColor}{RGB}{0,0,0}

\node[text=drawColor,anchor=base,inner sep=0pt, outer sep=0pt, scale=  1.02] at ( 39.63,187.46) {89.2};

\node[text=drawColor,anchor=base,inner sep=0pt, outer sep=0pt, scale=  1.02] at (106.55, 94.41) {75.0};

\node[text=drawColor,anchor=base,inner sep=0pt, outer sep=0pt, scale=  1.02] at (173.47,165.96) {85.2};

\node[text=drawColor,anchor=base,inner sep=0pt, outer sep=0pt, scale=  1.02] at (240.39, 48.59) {66.6};

\path[draw=drawColor,line width= 0.1pt,line join=round] (  0.00, 23.61) -- (280.54, 23.61);

\path[] ( -0.52, 15.61) rectangle (280.54,191.48);
\end{scope}
\begin{scope}
\path[clip] (  0.00,  0.00) rectangle (289.08,198.74);

\path[] (  0.00, 15.61) --
	(280.54, 15.61);
\end{scope}
\begin{scope}
\path[clip] (  0.00,  0.00) rectangle (289.08,198.74);

\path[] ( 39.63, 12.86) --
	( 39.63, 15.61);

\path[] (106.55, 12.86) --
	(106.55, 15.61);

\path[] (173.47, 12.86) --
	(173.47, 15.61);

\path[] (240.39, 12.86) --
	(240.39, 15.61);
\end{scope}
\begin{scope}
\path[clip] (  0.00,  0.00) rectangle (289.08,198.74);
\definecolor{drawColor}{RGB}{0,0,0}

\node[text=drawColor,anchor=base,inner sep=0pt, outer sep=0pt, scale=  1.00] at ( 39.63,  2.85) {2012};

\node[text=drawColor,anchor=base,inner sep=0pt, outer sep=0pt, scale=  1.00] at (106.55,  2.85) {2013};

\node[text=drawColor,anchor=base,inner sep=0pt, outer sep=0pt, scale=  1.00] at (173.47,  2.85) {2014};

\node[text=drawColor,anchor=base,inner sep=0pt, outer sep=0pt, scale=  1.00] at (240.39,  2.85) {2015};
\end{scope}
  \end{tikzpicture}}%
 {%
 	SICOIN} %
 
 
  \cajita{%
  	Detalle de gasto de la SESAN }%
  {%
  	Para atender la desnutrición se implementó el plan hambre cero\footnote{El Plan hambre cero representa una estrategia conjunta de atención a la desnutrición crónica, la
  		desnutrición aguda y la inseguridad alimentaria, que afectan principalmente a la niñez
  		guatemalteca menor de cinco años, que vive en condiciones de pobreza y pobreza
  		extrema. Está focalizado especialmente en el área rural y urbana marginal del país, y
  		promueve la creación de condiciones y medios necesarios para la generación en el
  		mediano y largo plazo, de una seguridad alimentaria y nutricional efectiva y sostenible,
  		con el propósito de disminuir en forma significativa la desnutrición crónica y la
  		desnutrición aguda que afecta a la niñez guatemalteca.
  		}, siendo la SESAN parte importante en la ejecución del mismo. 
  		
  		En Plan Hambre Cero está compuesto por componentes directos, de viabilidad y el eje transversal\footnote{La transversalidad se refiere a aquellos temas cuyo contenido debe ser aplicado en
  			forma intrínseca, integral y apropiada en todos los componente del Plan Hambre Cero.
  			}, siendo este último el aspecto en la que más se invirtió dinero por parte de la SESAN para el 2015
  	
  	
  }%
  {%
  	Detalle de ejecución de la SESAN } %
  {%
  	República de Guatemala, 2014, en quetzales} %
  {%
  	\begin{tikzpicture}[x=1pt,y=1pt]  \input{graficas/7_03.tex}  \end{tikzpicture}}%
  {%
  	SICOIN} %
  
  
  
  \cajita{%
  	Asignación presupuestaria a la ventana de los mil días}%
  {%
  	La ventana de los mil días\footnote{es el período que transcurre desde el embarazo -270 días promedio- hasta los dos años de vida del niño -730 días- } es una estrategia lanzada por el Ministerio de Salud Pública y Asistencia Social y funciona como un paquete de atención en salud y nutrición cuyo objetivo principal es reducir y prevenir la desnutrición crónica en Guatemala
  	
  	En la serie histórica se observa que ha habido un aumento en la inversión para el plan la ventana de los mil días. En el 2015 se hizo una asignación de 705,840,558 quetzales. 
  }%
  {%
  Asignación presupuestaria al programa ventana de los mi días para el año 2015} %
  {%
  	República de Guatemala, serie histórica, en quetzales} %
  {%
  	\begin{tikzpicture}[x=1pt,y=1pt]  % Created by tikzDevice version 0.9 on 2016-03-03 06:37:00
% !TEX encoding = UTF-8 Unicode
\definecolor{fillColor}{RGB}{255,255,255}
\path[use as bounding box,fill=fillColor,fill opacity=0.00] (0,0) rectangle (289.08,198.74);
\begin{scope}
\path[clip] (  0.00,  0.00) rectangle (289.08,198.74);

\path[] (  0.00,  0.00) rectangle (289.08,198.74);
\end{scope}
\begin{scope}
\path[clip] (  0.00,  0.00) rectangle (289.08,198.74);

\path[] ( 12.53, 15.61) rectangle (280.54,191.48);

\path[] ( 12.53, 54.04) --
	(280.54, 54.04);

\path[] ( 12.53, 96.77) --
	(280.54, 96.77);

\path[] ( 12.53,139.51) --
	(280.54,139.51);

\path[] ( 12.53,182.24) --
	(280.54,182.24);

\path[] ( 12.53, 32.67) --
	(280.54, 32.67);

\path[] ( 12.53, 75.41) --
	(280.54, 75.41);

\path[] ( 12.53,118.14) --
	(280.54,118.14);

\path[] ( 12.53,160.87) --
	(280.54,160.87);

\path[] ( 62.78, 15.61) --
	( 62.78,191.48);

\path[] (146.54, 15.61) --
	(146.54,191.48);

\path[] (230.29, 15.61) --
	(230.29,191.48);
\definecolor{drawColor}{RGB}{0,0,255}

\path[draw=drawColor,line width= 1.7pt,line join=round] ( 62.78, 60.50) --
	(146.54,113.90) --
	(230.29,183.49);
\definecolor{drawColor}{RGB}{0,0,0}

\node[text=drawColor,anchor=base,inner sep=0pt, outer sep=0pt, scale=  1.02] at ( 62.78, 48.59) {130,246,108};

\node[text=drawColor,anchor=base east,inner sep=0pt, outer sep=0pt, scale=  1.02] at (137.61,113.90) {380,172,611};

\node[text=drawColor,anchor=base,inner sep=0pt, outer sep=0pt, scale=  1.02] at (230.29,187.46) {705,840,558};

\path[draw=drawColor,line width= 0.1pt,line join=round] ( 12.53, 23.61) -- (280.54, 23.61);

\path[] ( 12.53, 15.61) rectangle (280.54,191.48);
\end{scope}
\begin{scope}
\path[clip] (  0.00,  0.00) rectangle (289.08,198.74);

\path[] ( 12.53, 15.61) --
	( 12.53,191.48);
\end{scope}
\begin{scope}
\path[clip] (  0.00,  0.00) rectangle (289.08,198.74);
\definecolor{drawColor}{RGB}{255,255,255}

\node[text=drawColor,text opacity=0.00,anchor=base east,inner sep=0pt, outer sep=0pt, scale=  1.00] at (  7.58, 28.76) {0e+00};

\node[text=drawColor,text opacity=0.00,anchor=base east,inner sep=0pt, outer sep=0pt, scale=  1.00] at (  7.58, 71.50) {2e+08};

\node[text=drawColor,text opacity=0.00,anchor=base east,inner sep=0pt, outer sep=0pt, scale=  1.00] at (  7.58,114.23) {4e+08};

\node[text=drawColor,text opacity=0.00,anchor=base east,inner sep=0pt, outer sep=0pt, scale=  1.00] at (  7.58,156.97) {6e+08};
\end{scope}
\begin{scope}
\path[clip] (  0.00,  0.00) rectangle (289.08,198.74);

\path[] (  9.78, 32.67) --
	( 12.53, 32.67);

\path[] (  9.78, 75.41) --
	( 12.53, 75.41);

\path[] (  9.78,118.14) --
	( 12.53,118.14);

\path[] (  9.78,160.87) --
	( 12.53,160.87);
\end{scope}
\begin{scope}
\path[clip] (  0.00,  0.00) rectangle (289.08,198.74);

\path[] ( 12.53, 15.61) --
	(280.54, 15.61);
\end{scope}
\begin{scope}
\path[clip] (  0.00,  0.00) rectangle (289.08,198.74);

\path[] ( 62.78, 12.86) --
	( 62.78, 15.61);

\path[] (146.54, 12.86) --
	(146.54, 15.61);

\path[] (230.29, 12.86) --
	(230.29, 15.61);
\end{scope}
\begin{scope}
\path[clip] (  0.00,  0.00) rectangle (289.08,198.74);
\definecolor{drawColor}{RGB}{0,0,0}

\node[text=drawColor,anchor=base,inner sep=0pt, outer sep=0pt, scale=  1.00] at ( 62.78,  2.85) {2013};

\node[text=drawColor,anchor=base,inner sep=0pt, outer sep=0pt, scale=  1.00] at (146.54,  2.85) {2014};

\node[text=drawColor,anchor=base,inner sep=0pt, outer sep=0pt, scale=  1.00] at (230.29,  2.85) {2015};
\end{scope}
  \end{tikzpicture}}%
  {%
  	SICOIN} %
  
  
    \cajita{%
    	Gasto programa ventana de los mil días}%
    {%
    	El programa la ventana de los mil días tiene varios rubros de inversión, que responden a necesidades que la población necesita satisfacer.
    	
    	Para el 2015 el gasto se centralizó en la compra de vacunas, seguido de la compra de Vitamina A.
    }%
    {%
    	Detalle del gasto en el programa ventana de los mil días} %
    {%
    	República de Guatemala, 2015, en miles de millones quetzales} %
    {%
    	\begin{tikzpicture}[x=1pt,y=1pt]  % Created by tikzDevice version 0.9 on 2016-03-03 06:46:08
% !TEX encoding = UTF-8 Unicode
\definecolor{fillColor}{RGB}{255,255,255}
\path[use as bounding box,fill=fillColor,fill opacity=0.00] (0,0) rectangle (289.08,198.74);
\begin{scope}
\path[clip] (  0.00,  0.00) rectangle (289.08,198.74);

\path[] (  0.00,  0.00) rectangle (289.08,198.74);
\end{scope}
\begin{scope}
\path[clip] (  0.00,  0.00) rectangle (289.08,198.74);

\path[] (  0.00, 24.65) rectangle (289.08,181.67);

\path[] ( 24.09, 24.65) --
	( 24.09,181.67);

\path[] ( 64.24, 24.65) --
	( 64.24,181.67);

\path[] (104.39, 24.65) --
	(104.39,181.67);

\path[] (144.54, 24.65) --
	(144.54,181.67);

\path[] (184.69, 24.65) --
	(184.69,181.67);

\path[] (224.84, 24.65) --
	(224.84,181.67);

\path[] (264.99, 24.65) --
	(264.99,181.67);
\definecolor{drawColor}{RGB}{0,0,255}
\definecolor{fillColor}{RGB}{0,0,255}

\path[draw=drawColor,line width= 0.6pt,line join=round,fill=fillColor] ( 12.04, 31.78) rectangle ( 36.14, 43.15);

\path[draw=drawColor,line width= 0.6pt,line join=round,fill=fillColor] ( 52.20, 31.78) rectangle ( 76.29, 34.07);

\path[draw=drawColor,line width= 0.6pt,line join=round,fill=fillColor] ( 92.35, 31.78) rectangle (116.43, 57.18);

\path[draw=drawColor,line width= 0.6pt,line join=round,fill=fillColor] (132.50, 31.78) rectangle (156.59,174.53);

\path[draw=drawColor,line width= 0.6pt,line join=round,fill=fillColor] (172.64, 31.78) rectangle (196.73, 35.67);

\path[draw=drawColor,line width= 0.6pt,line join=round,fill=fillColor] (212.80, 31.78) rectangle (236.88, 45.97);

\path[draw=drawColor,line width= 0.6pt,line join=round,fill=fillColor] (252.94, 31.78) rectangle (277.03, 31.82);
\definecolor{drawColor}{RGB}{0,0,0}

\path[draw=drawColor,line width= 0.1pt,line join=round] (  0.00, 31.78) -- (289.08, 31.78);

\node[text=drawColor,anchor=base,inner sep=0pt, outer sep=0pt, scale=  1.02] at ( 24.09, 47.12) {44,701.7};

\node[text=drawColor,anchor=base,inner sep=0pt, outer sep=0pt, scale=  1.02] at ( 64.24, 38.04) {9,003.1};

\node[text=drawColor,anchor=base,inner sep=0pt, outer sep=0pt, scale=  1.02] at (104.39, 61.15) {99,898.7};

\node[text=drawColor,anchor=base,inner sep=0pt, outer sep=0pt, scale=  1.02] at (144.54,178.50) {561,531.9};

\node[text=drawColor,anchor=base,inner sep=0pt, outer sep=0pt, scale=  1.02] at (184.69, 39.64) {15,274.6};

\node[text=drawColor,anchor=base,inner sep=0pt, outer sep=0pt, scale=  1.02] at (224.84, 49.95) {55,819.2};

\node[text=drawColor,anchor=base,inner sep=0pt, outer sep=0pt, scale=  1.02] at (264.99, 35.79) {129.7};

\path[] (  0.00, 24.65) rectangle (289.08,181.67);
\end{scope}
\begin{scope}
\path[clip] (  0.00,  0.00) rectangle (289.08,198.74);

\path[] (  0.00, 24.65) --
	(289.08, 24.65);
\end{scope}
\begin{scope}
\path[clip] (  0.00,  0.00) rectangle (289.08,198.74);

\path[] ( 24.09, 21.90) --
	( 24.09, 24.65);

\path[] ( 64.24, 21.90) --
	( 64.24, 24.65);

\path[] (104.39, 21.90) --
	(104.39, 24.65);

\path[] (144.54, 21.90) --
	(144.54, 24.65);

\path[] (184.69, 21.90) --
	(184.69, 24.65);

\path[] (224.84, 21.90) --
	(224.84, 24.65);

\path[] (264.99, 21.90) --
	(264.99, 24.65);
\end{scope}
\begin{scope}
\path[clip] (  0.00,  0.00) rectangle (289.08,198.74);
\definecolor{drawColor}{RGB}{0,0,0}

\node[text=drawColor,anchor=base,inner sep=0pt, outer sep=0pt, scale=  0.750] at ( 24.09, 11.88) {Lactancia };

\node[text=drawColor,anchor=base,inner sep=0pt, outer sep=0pt, scale=  0.750] at ( 24.09,  0.00) {  materna};

\node[text=drawColor,anchor=base,inner sep=0pt, outer sep=0pt, scale=  0.750] at ( 64.24, 11.88) {Atención a };

\node[text=drawColor,anchor=base,inner sep=0pt, outer sep=0pt, scale=  0.750] at ( 64.24,  0.00) { diarrea};

\node[text=drawColor,anchor=base,inner sep=0pt, outer sep=0pt, scale=  0.750] at (104.39,  5.94) {Vitamina A};

\node[text=drawColor,anchor=base,inner sep=0pt, outer sep=0pt, scale=  0.750] at (144.54,  5.94) {Vacunas};

\node[text=drawColor,anchor=base,inner sep=0pt, outer sep=0pt, scale=  0.750] at (184.69, 11.88) {Suplementación};

\node[text=drawColor,anchor=base,inner sep=0pt, outer sep=0pt, scale=  0.750] at (184.69,  0.00) {a la  mujer fértil};

\node[text=drawColor,anchor=base,inner sep=0pt, outer sep=0pt, scale=  0.750] at (230.84, 11.88) {Alimento};

\node[text=drawColor,anchor=base,inner sep=0pt, outer sep=0pt, scale=  0.750] at (229.84,  0.00) { 6 meses};

\node[text=drawColor,anchor=base,inner sep=0pt, outer sep=0pt, scale=  0.650] at (264.99,  5.94) {Micronutrientes};
\end{scope}
  \end{tikzpicture}}%
    {%
    	SICOIN} %
  
		
		\appendix

%

\newpage

\titulo{Notas metodológicas}


\noindent\textbf{Recopilación:}

La recopilación de los datos estadísticos de Seguridad Alimentaria y Nutricional (SAN) para el Compendio Estadístico SAN 2014 se llevó a cabo siguiendo los siguientes pasos:
\begin{itemize}
	\item[a.]	{\textbf{Identificación de indicadores y sub-indicadores} El primer paso para elegir los indicadores y sub-indicadores del Compendio fue llevar a cabo una revisión bibliográfica de la información sobre SAN disponible (INE, 2013; FAO, 2014; DevTech Systems, Inc., \& Iarna/URL, 2015; URL, Iarna, IICA, \& McGill University, 2015). Como resultado se obtuvo una lista de indicadores bastante amplio; sin embargo, se decidió reducir el número de indicadores para que la información fuese manejable. Esto se hizo con la ayuda de expertos en SAN, lo que dio como resultado el listado de indicadores y sub-indicadores final (Figura 4,  Figura 6, Figura 7, Figura 8, Figura 9 y Figura 10). El indicador y sub-indicadores de la dimensión 7 se agregaron con el fin de integrar los logros en SAN hasta la fecha. }

\item[b.]	\textbf{Cuestionarios en línea} El Iarna-URL, junto con el apoyo del personal del INE, elaboró cinco cuestionarios en el programa Surveymonkey. Los mismos contenían preguntas para llevar a cabo un diagnóstico de la información con la que las organizaciones productoras y recopiladoras de estadísticas de SAN contaban. Cada cuestionario corresponde a las dimensiones/pilares en las que se basa el Compendio. En un principio, se enviaron los cuestionarios únicamente a las instituciones miembros de la Oficina Coordinadora Sectorial de Estadísticas de Seguridad Alimentaria y Nutricional (Ocsesan) que contaran con información relacionada con los indicadores y sub-indicadores elegidos. No obstante, se envió el listado de indicadores y sub-indicadores a todas las instituciones miembros de la Ocsesan para que fueran ellas quienes determinaran si podían aportar información o no. Asimismo, se enviaron los cuestionarios a otras instituciones sugeridas por las instituciones miembros.


\item[c.]	\textbf{Diagnóstico de información por organización} Con base en los resultados del cuestionario, se realizaron cuadros resumen para conocer con qué información contaba cada institución, para así crear cuadros de salida tentativos y recopilar la información final del compendio.

\item[d.]	\textbf{Diseño de cuadros de salida} Para diseñar los cuadros de salida del Compendio, se llevaron a cabo cuatro reuniones extraordinarias de la Ocsesan, en las que participó personal del Iarna-URL, el INE y la Sesan. Se diseñaron cuadros para cada uno de los pilares/dimensiones del Compendio y se socializaron en la octava reunión de la Ocsesan celebrada en octubre de 2015. Los asistentes a dicha reunión evaluaron los cuadros y sugirieron cambios que luego se hicieron en otra reunión extraordinaria.

\item[e.]	\textbf{Recopilación de información en instituciones} La recopilación de información de datos fue llevada a cabo por dos técnicos contratados por el Iarna-URL, y personal de la Sesan, el INE y el Iarna-URL, quienes utilizaron las fuentes más recientes disponibles, y llenaron los cuadros de salida previamente elaborados. Durante este proceso fue que realmente se verificó la disponibilidad y calidad de la información. Los técnicos cambiaron en alguna medida los cuadros de salida, eliminando información en algunos casos, y enriqueciendo los cuadros en otros, dependiendo de la disponibilidad y calidad de la información estadística. A pesar de que la información para algunos de los indicadores no estaba disponible, o no presentaba la calidad requerida para formar parte del Compendio, se decidió dejar el listado de indicadores intacto, con el fin de conseguir la información en futuras versiones de este documento.
\end{itemize}


\noindent\textbf{Procesamiento y control de calidad:}

Se debe tomar en cuenta que la información presentada en cada una de las dimensiones es responsabilidad de cada fuente de información. Asimismo, se tomó en cuenta la información disponible más actualizada. La fuente de la información está indicada en la parte inferior de cada cuadro. 

\noindent\textbf{Integración:}

Se integraron los cuadros ya llenos en los respectivos capítulos del Compendio. 

\noindent\textbf{Edición y diagramación:}


Se editó el borrador final y se realizó la diagramación correspondiente para obtener un documento de calidad. Se tomaron datos de los cuadros integrados para la generación de gráficas.

\noindent\textbf{Revisión editorial, aprobación e impresión:}

Se integró un equipo para la revisión final del documento, conformado por personal del Iarna-URL y el INE. El informe se sometió a la aprobación de la Sección de Socieconómicas y Ambientales del INE, y finalmente se envió a imprenta.



%

\newpage

\titulo{Importancia de la seguridad alimentaria y nutricional}


\noindent\textbf{1. Seguridad alimentaria y nutricional a nivel mundial y regional:}

En el ámbito mundial y de acuerdo con estimaciones de la Organización de las Naciones Unidas para la Alimentación y la Agricultura (FAO, por sus siglas en inglés), se han logrado importantes avances para erradicar el hambre, incluyendo a los países en desarrollo, los cuales representan la gran mayoría de la subalimentación mundial. Se calcula que en en el período de 2012 a 2014, la población en países en desarrollo que padecía hambre crónica era de 791 millones de personas, es decir, 203 millones menos que en en el período de 1990 a 1992 (FAO, FIDA \& PMA, 2015). 

Según la FAO (2015a) «La seguridad alimentaria se da cuando todas las personas tienen acceso físico, social y económico permanente a alimentos seguros, nutritivos y en cantidad suficiente para satisfacer sus requerimientos nutricionales y preferencias alimentarias, y así poder llevar una vida activa y saludable».

La Cumbre Mundial sobre la Alimentación (CMA), se convocó como respuesta a la persistencia de una desnutrición generalizada y a la preocupación por la capacidad de la agricultura para cubrir las necesidades futuras de alimentación. En 1974, los gobiernos participantes en la Conferencia Mundial de la Alimentación proclamaron que «todos los hombres, mujeres y niños tienen derecho inalienable a no padecer de hambre y malnutrición a fin de poder desarrollarse plenamente y conservar sus facultades físicas y mentales». Además, la Conferencia se fijó el objetivo de erradicar el hambre, la inseguridad alimentaria y la malnutrición en el plazo de un decenio, objetivo que no se alcanzó (FAO, s.f.).  

Posteriormente, se llevó a cabo la Cumbre Mundial sobre la Alimentación, celebrada del 13 al 17 de noviembre de 1996. Los participantes fueron los representantes de 185 países y de la Comunidad Europea del Este. La misma constituyó un foro para el debate sobre una de las cuestiones más importantes con que se enfrentan los dirigentes mundiales: La erradicación del hambre (FAO, s.f).

El hambre, la inseguridad alimentaria y la nutrición son problemas complejos que no pueden ser resueltos por un solo sector. Es necesario tomar una serie de medidas en diversos sectores que aborden las causas inmediatas y subyacentes del hambre. Entre estos sectores se encuentran: la producción y productividad agraria, el desarrollo rural, la silvicultura, la pesca, la protección social, y el comercio y los mercados. Asimismo, la gobernanza de la seguridad alimentaria se refiere a fomentar un entorno propicio que cree incentivos para que los sectores involucrados mejoren su repercusión sobre el hambre, la malnutrición y la inseguridad alimentaria (FAO, FIDA \& PMA, 2015) (Figura \ref{figura1})

\begin{figure}
		\centering
	\includegraphics[width=0.7\textwidth]{figura1}
	\caption{Cinco dimensiones fundamentales de un entorno propicio para la gobernanza de la seguridad alimentaria y nutricional. \textbf{Elaborado por:} Iarna-URL, 2015 con base en FAO, FIDA \& PMA, 2015.}
	\label{figura1}
%	\centering
\end{figure}
%$\ $\\[-1cm]
Los estados latinoamericanos y caribeños tomaron la decisión de acabar con el hambre para 2025, fundamento para la acción nacional y regional de promover la seguridad alimentaria. La región en conjunto es la única que alcanzó la meta del primer Objetivo de Desarrollo del Milenio (ODM) relacionada con el hambre. Asimismo, América Latina alcanzó el objetivo de la Cumbre Mundial sobre la Alimentación (FAO, FIDA \& PMA, 2015). 


\noindent\textbf{2.	Seguridad alimentaria y nutricional para Guatemala:}

El Índice del Hambre Global (GHI, por sus siglas en inglés), mide integralmente el hambre en los ámbitos global, regional y de país. Cada año, el Instituto de Investigación de Política Alimentaria (Ifpri, por sus siglas en ingles), hace un cálculo del punto de GHI para evaluar el progreso, o la falta del mismo, en la disminución del hambre (Ifpri, 2015). Según dicho índice, Guatemala se encuentra entre los países de Latinoamérica con menos progreso, solo por encima de Haití (Figura \ref{figura2}). 


\begin{figure}
		\centering
	\includegraphics[width=0.7\textwidth]{figura2}
	\caption{Índice del Hambre Global para Guatemala. \textbf{Fuente:} Modificado de Ifpri, 2015.}
	\label{figura2}
\end{figure}


Como un movimiento nacional y un compromiso del Estado de Guatemala para afrontar el problema del hambre en el país, el Gobierno de Guatemala y representantes de todos los sectores firmaron el Pacto Hambre Cero en 2012 (Gobierno de Guatemala, 2012). 

Los objetivos del Pacto Hambre Cero son: «a) Reducir en 10\% la prevalencia de la desnutrición crónica infantil, para finales del 2015, promoviendo el desarrollo infantil temprano; b) prevenir el hambre estacional y reducir la mortalidad en la niñez menor de 5 años, por la desnutrición aguda; c) promover la seguridad alimentaria y nutricional, fundamento del desarrollo integral de toda la población guatemalteca, y d) prevenir y atender las emergencias alimentarias, relacionadas con el cambio climático y los desastres naturales» (Gobierno de Guatemala, 2012). 


\noindent\textbf{3.	Importancia de la estadística en seguridad alimentaria y nutricional}

A nivel mundial, existe una iniciativa para el desarrollo de la Estrategia Global para el mejoramiento de las estadísticas agrícolas y rurales (Gsars) liderada por FAO, la cual surge como una respuesta para abordar la falta de capacidad de los países en desarrollo de proporcionar datos estadísticos confiables sobre alimentación y agricultura, así como ofrecer un modelo de sistemas estadísticos agrícolas sostenible a largo plazo. Dicha estrategia está conformada por tres pilares: a) establecer un conjunto mínimo de datos básicos; b) integrar la agricultura en el sistema nacional de estadística, y c) mejorar la gobernanza y la creación de capacidad estadística (Global Strategy, 2015).

		
\INEchaptercarta{Proyecciones de población}{}
%%%cuadro 1


\hoja{
	{\Bold\Large 1.1 Población}\\[-.5cm]
	\begin{center}\fontsize{3.3mm}{1.45em}\selectfont \setlength{\arrayrulewidth}{0.7pt}\addtocounter{Cuadro}{1}
		$\!$\begin{tabular}{lrrrrrrrr}
			\multicolumn{9}{l}{$\ $}\\[0.15cm]
			\multicolumn{9}{l}{\Bold\color{color1!80!black}{\normalsize Cuadro \theCuadro $\,-$   Proyecciones de población por departamento y por año.}}\\
			\multicolumn{9}{l}{\normalsize (Personas)}
			\\[0.4cm]
			\multicolumn{1}{l}{$\ $} &  \multicolumn{8}{c}{$\ $} \\[-0.28cm]\hline
	\multicolumn{1}{l}{\multirow{3}[0]{*}{\Bold{\raisebox{0.2cm}{Departamento}}}} & \multicolumn{8}{c}{\Bold{Año}} \\\cline{2-9}
				\multicolumn{1}{l}{$\ $} &  \multicolumn{8}{c}{$\ $} \\[-0.28cm] 
			\rowcolor{color1!0!white} { } & \multicolumn{1}{c}{2008} & \multicolumn{1}{c}{2009} & \multicolumn{1}{c}{2010} & \multicolumn{1}{c}{2011} & \multicolumn{1}{c}{2012} & \multicolumn{1}{c}{2013} & \multicolumn{1}{c}{2014} & \multicolumn{1}{c}{2015} \\ \hline
%			\rowcolor{color1!40!white} &&&&&&&& \\[-0.28cm]
			\rowcolor{color1!40!white} {\Bold{Total}}	&	 \Bold{13,677,815} 	&	 \Bold{14,017,057} 	&	 \Bold{14,361,666 }	&	 \Bold{14,713,763} 	&	 \Bold{15,073,375} 	&	 \Bold{15,438,384} 	&	 \Bold{15,806,675} 	&	 \Bold{16,176,133} 	\\\hline
%				&&&&&&&& \\[-0.58cm]
			\multicolumn{1}{l}{Guatemala}	&	 2,994,047 	&	 3,049,601 	&	 3,103,685 	&	 3,156,284 	&	 3,207,587 	&	 3,257,616 	&	 3,306,397 	&	 3,353,951 	\\
			\rowcolor{color1!10!white} \multicolumn{1}{l}{El Progreso}	&	 151,058 	&	 153,261 	&	 155,596 	&	 158,092 	&	 160,754 	&	 163,537 	&	 166,397 	&	 169,290 	\\
			\multicolumn{1}{l}{Sacatepéquez}	&	 296,890 	&	 303,459 	&	 310,037 	&	 316,638 	&	 323,283 	&	 329,947 	&	 336,606 	&	 343,236 	\\
			\rowcolor{color1!10!white} \multicolumn{1}{l}{Chimaltenango}	&	 562,555 	&	 578,976 	&	 595,769 	&	 612,973 	&	 630,609 	&	 648,617 	&	 666,938 	&	 685,513 	\\
			\multicolumn{1}{l}{Escuintla}	&	 655,189 	&	 670,570 	&	 685,830 	&	 701,016 	&	 716,204 	&	 731,326 	&	 746,309 	&	 761,085 	\\
			\rowcolor{color1!10!white} \multicolumn{1}{l}{Santa Rosa}	&	 329,433 	&	 334,720 	&	 340,381 	&	 346,590 	&	 353,261 	&	 360,288 	&	 367,569 	&	 375,001 	\\
			\multicolumn{1}{l}{Sololá}	&	 398,519 	&	 411,202 	&	 424,068 	&	 437,145 	&	 450,471 	&	 464,005 	&	 477,705 	&	 491,530 	\\
			\rowcolor{color1!10!white} \multicolumn{1}{l}{Totonicapán}	&	 433,749 	&	 447,651 	&	 461,838 	&	 476,369 	&	 491,298 	&	 506,537 	&	 521,995 	&	 537,584 	\\
			\multicolumn{1}{l}{Quetzaltenango}	&	 737,593 	&	 754,457 	&	 771,674 	&	 789,358 	&	 807,571 	&	 826,143 	&	 844,906 	&	 863,689 	\\
		\rowcolor{color1!10!white} \multicolumn{1}{l}{Suchitepéquez}	&	 481,047 	&	 492,481 	&	 504,267 	&	 516,467 	&	 529,096 	&	 542,059 	&	 555,261 	&	 568,608 	\\
			\multicolumn{1}{l}{Retalhuleu}	&	 284,359 	&	 290,796 	&	 297,385 	&	 304,168 	&	 311,167 	&	 318,319 	&	 325,556 	&	 332,815 	\\
			\rowcolor{color1!10!white} \multicolumn{1}{l}{San Marcos}	&	 950,592 	&	 972,781 	&	 995,742 	&	 1,019,719 	&	 1,044,667 	&	 1,070,215 	&	 1,095,997 	&	 1,121,644 	\\
			\multicolumn{1}{l}{Huehuetenango}	&	 1,056,566 	&	 1,085,357 	&	 1,114,389 	&	 1,143,887 	&	 1,173,977 	&	 1,204,324 	&	 1,234,593 	&	 1,264,449 	\\
			\rowcolor{color1!10!white} \multicolumn{1}{l}{Quiché}	&	 861,089 	&	 890,764 	&	 921,390 	&	 953,027 	&	 985,690 	&	 1,019,290 	&	 1,053,737 	&	 1,088,942 	\\
			\multicolumn{1}{l}{Baja Verapaz}	&	 252,047 	&	 257,876 	&	 264,019 	&	 270,521 	&	 277,380 	&	 284,530 	&	 291,903 	&	 299,432 	\\
			\rowcolor{color1!10!white} \multicolumn{1}{l}{Alta Verapaz}	&	 1,014,419 	&	 1,046,185 	&	 1,078,942 	&	 1,112,781 	&	 1,147,593 	&	 1,183,241 	&	 1,219,585 	&	 1,256,486 	\\
			\multicolumn{1}{l}{Petén}	&	 563,832 	&	 588,860 	&	 613,693 	&	 638,296 	&	 662,779 	&	 687,192 	&	 711,585 	&	 736,010 	\\
			\rowcolor{color1!10!white} \multicolumn{1}{l}{Izabal}	&	 383,636 	&	 393,345 	&	 403,256 	&	 413,399 	&	 423,788 	&	 434,378 	&	 445,125 	&	 455,982 	\\
			\multicolumn{1}{l}{Zacapa}	&	 213,313 	&	 215,752 	&	 218,510 	&	 221,646 	&	 225,108 	&	 228,810 	&	 232,667 	&	 236,593 	\\
			\rowcolor{color1!10!white} \multicolumn{1}{l}{Chiquimula}	&	 347,960 	&	 355,223 	&	 362,826 	&	 370,891 	&	 379,359 	&	 388,155 	&	 397,202 	&	 406,422 	\\
			\multicolumn{1}{l}{Jalapa}	&	 293,926 	&	 301,755 	&	 309,908 	&	 318,420 	&	 327,297 	&	 336,484 	&	 345,926 	&	 355,566 	\\
			\rowcolor{color1!10!white} \multicolumn{1}{l}{Jutiapa}	&	 415,996 	&	 421,984 	&	 428,462 	&	 436,076 	&	 444,434 	&	 453,369 	&	 462,714 	&	 472,304 	\\
			\hline
			&&&&&&&&\\[-0.28cm]
			\multicolumn{9}{l}{\footnotesize Fuente:  INE. Proyecciones de Población con base en el XI Censo de Población y VI de Habitación.}
		\end{tabular}\addtocounter{Cuadro}{1}
	\end{center}}


\hoja{
	{\Bold\color{color1!80!black}{\normalsize Cuadro \theCuadro $\,-$ Densidad poblacional por departamento y por año.}}\\
	{\Bold\color{color1!80!black}{\normalsize República de Guatemala, 2008-2015}}\\
	{\normalsize (Personas por kilómetro cuadrado)}
	\begin{center}\fontsize{3.6mm}{1.45em}\selectfont \setlength{\arrayrulewidth}{0.7pt}
		$\!$\begin{tabular}{lrrrrrrrr}
			\multicolumn{9}{l}{$\ $}\\[0.15cm]
%			\multicolumn{9}{l}{\Bold\color{color1!80!black}{\normalsize Cuadro \theCuadro $\,-$ Densidad poblacional por departamento y por año.}}\\
%						\multicolumn{9}{l}{\Bold\color{color1!80!black}{\normalsize República de Guatemala, 2008-2015}}\\[-0.01cm]
%			\multicolumn{9}{l}{\normalsize (Personas por kilómetro cuadrado)}
%			\\[0.4cm]
			\multicolumn{1}{l}{$\ $} &  \multicolumn{8}{c}{$\ $} \\[-0.28cm]	
	\multicolumn{1}{l}{\multirow{3}[0]{*}{\Bold{\raisebox{-0.6cm}{ }}}} & \multicolumn{8}{c}{\Bold{Año}} \\\cline{2-9}
	\multicolumn{1}{l}{$\ $} &  \multicolumn{8}{c}{$\ $} \\[-0.28cm]
			%			\rowcolor{color1!0!white} &&&&&&&& \\[-0.28cm]     
			\rowcolor{color1!0!white} { } & \multicolumn{1}{c}{2008} & \multicolumn{1}{c}{2009} & \multicolumn{1}{c}{2010} & \multicolumn{1}{c}{2011} & \multicolumn{1}{c}{2012} & \multicolumn{1}{c}{2013} & \multicolumn{1}{c}{2014} & \multicolumn{1}{c}{2015} \\ \hline
			%			\rowcolor{color1!40!white} &&&&&&&& \\[-0.28cm]
			\rowcolor{color1!40!white} {\Bold{Total República}}&	\Bold{126}	&	\Bold{129}	&	\Bold{132}	&	\Bold{135}	&	\Bold{138}	&	\Bold{142}	&	\Bold{145}	&	\Bold{149}	\\\hline
			%				&&&&&&&& \\[-0.58cm]
			 \multicolumn{1}{l}{Guatemala	}&	1408	&	1434	&	1460	&	1485	&	1509	&	1532	&	1555	&	1578	\\
			\rowcolor{color1!10!white} \multicolumn{1}{l}{El Progreso	}&	79	&	80	&	81	&	82	&	84	&	85	&	87	&	88	\\
			 \multicolumn{1}{l}{Sacatepéquez	}&	638	&	653	&	667	&	681	&	695	&	710	&	724	&	738	\\
			\rowcolor{color1!10!white} \multicolumn{1}{l}{Chimaltenango	}&	284	&	293	&	301	&	310	&	319	&	328	&	337	&	346	\\
			 \multicolumn{1}{l}{Escuintla	}&	149	&	153	&	156	&	160	&	163	&	167	&	170	&	174	\\
			\rowcolor{color1!10!white} \multicolumn{1}{l}{Santa Rosa	}&	111	&	113	&	115	&	117	&	120	&	122	&	124	&	127	\\
			 \multicolumn{1}{l}{Sololá	}&	376	&	388	&	400	&	412	&	425	&	437	&	450	&	463	\\
			\rowcolor{color1!10!white} \multicolumn{1}{l}{Totonicapán	}&	409	&	422	&	435	&	449	&	463	&	477	&	492	&	507	\\
			 \multicolumn{1}{l}{Quetzaltenango	}&	378	&	387	&	396	&	405	&	414	&	423	&	433	&	443	\\
			\rowcolor{color1!10!white} \multicolumn{1}{l}{Suchitepéquez	}&	192	&	196	&	201	&	206	&	211	&	216	&	221	&	227	\\
			 \multicolumn{1}{l}{Retalhuleu	}&	153	&	157	&	160	&	164	&	168	&	172	&	175	&	179	\\
			\rowcolor{color1!10!white} \multicolumn{1}{l}{San Marcos	}&	251	&	257	&	263	&	269	&	276	&	282	&	289	&	296	\\
			 \multicolumn{1}{l}{Huehuetenango	}&	143	&	147	&	151	&	155	&	159	&	163	&	167	&	171	\\
			\rowcolor{color1!10!white} \multicolumn{1}{l}{Quiché	}&	103	&	106	&	110	&	114	&	118	&	122	&	126	&	130	\\
			 \multicolumn{1}{l}{Baja Verapaz	}&	81	&	83	&	85	&	87	&	89	&	91	&	93	&	96	\\
			\rowcolor{color1!10!white} \multicolumn{1}{l}{Alta Verapaz	}&	117	&	120	&	124	&	128	&	132	&	136	&	140	&	145	\\
			 \multicolumn{1}{l}{Petén	}&	16	&	16	&	17	&	18	&	18	&	19	&	20	&	21	\\
			\rowcolor{color1!10!white} \multicolumn{1}{l}{Izabal	}&	42	&	44	&	45	&	46	&	47	&	48	&	49	&	50	\\
			 \multicolumn{1}{l}{Zacapa	}&	79	&	80	&	81	&	82	&	84	&	85	&	86	&	88	\\
			\rowcolor{color1!10!white} \multicolumn{1}{l}{Chiquimula	}&	146	&	150	&	153	&	156	&	160	&	163	&	167	&	171	\\
			 \multicolumn{1}{l}{Jalapa	}&	142	&	146	&	150	&	154	&	159	&	163	&	168	&	172	\\
			\rowcolor{color1!10!white} \multicolumn{1}{l}{Jutiapa	}&	129	&	131	&	133	&	135	&	138	&	141	&	144	&	147	\\

			\hline
			&&&&&&&&\\[-0.28cm]
			\multicolumn{9}{l}{\footnotesize Fuente:  INE. Proyecciones de Población con base en el XI Censo de Población y VI de Habitación.}
		\end{tabular}\addtocounter{Cuadro}{1}
	\end{center}}


%%%%%%%%%%%%%%%%%%%%%%%%%%%%%%111
\hoja{{\Bold\color{color1!80!black}{\normalsize Cuadro \theCuadro $\,-$ Número de habitantes total y por sexo; según grupos quinquenales de edad. }}\\
	{\Bold\color{color1!80!black}{\normalsize República de Guatemala, 2008-2015}}\\
	{\normalsize (Personas)}\\[-10mm]
	\begin{center}\fontsize{2.6mm}{1.1em}\selectfont \setlength{\arrayrulewidth}{0.7pt}
		$\!$\begin{tabular}{lrrrrrrrr}
			\multicolumn{9}{l}{$\ $}\\[0.15cm]
%			\multicolumn{9}{l}{\Bold\color{color1!80!black}{\normalsize Cuadro \theCuadro $\,-$ Número de habitantes total y por sexo; según grupos quinquenales de edad. }}\\
%			\multicolumn{9}{l}{\Bold\color{color1!80!black}{\normalsize República de Guatemala, 2008-2015}}\\[-0.01cm]
%			\multicolumn{9}{l}{\normalsize (Personas)}
%			\\[-0.01cm]
			\multicolumn{1}{l}{$\ $} &  \multicolumn{8}{c}{$\ $} \\[-0.48cm]\hline
			\multicolumn{1}{l}{\multirow{3}[0]{*}{\Bold{\raisebox{0.06cm}{Rango}}}} & \multicolumn{8}{c}{\Bold{Año}} \\\cline{2-9}
			\multicolumn{1}{l}{$\ $} &  \multicolumn{8}{c}{$\ $} \\[-0.28cm]
			%			\rowcolor{color1!0!white} &&&&&&&& \\[-0.28cm]     
		& \multicolumn{1}{c}{2008} & \multicolumn{1}{c}{2009} & \multicolumn{1}{c}{2010} & \multicolumn{1}{c}{2011} & \multicolumn{1}{c}{2012} & \multicolumn{1}{c}{2013} & \multicolumn{1}{c}{2014} & \multicolumn{1}{c}{2015} \\ \hline
			%			\rowcolor{color1!40!white} &&&&&&&& \\[-0.28cm]
			%				&&&&&&&& \\[-0.58cm]
\rowcolor{color1!40!white} {\Bold{Total	}}&	\Bold{13,677,815}	&	\Bold{14,017,057}	&	\Bold{14,361,666}	&	\Bold{14,713,763}	&	\Bold{15,073,375}	&	\Bold{15,438,384}	&	\Bold{15,806,675}	&	\Bold{16,176,133}	\\
\hline
\multicolumn{1}{l}{0- 4	}&	2,118,563	&	2,142,733	&	2,165,745	&	2,187,869	&	2,208,844	&	2,229,167	&	2,246,402	&	2,262,514	\\
\rowcolor{color1!10!white} \multicolumn{1}{l}{5- 9	}&	1,934,742	&	1,971,063	&	2,004,670	&	2,035,555	&	2,064,591	&	2,092,393	&	2,117,797	&	2,142,308	\\
\multicolumn{1}{l}{10-14	}&	1,728,741	&	1,762,864	&	1,798,262	&	1,836,052	&	1,875,512	&	1,914,769	&	1,953,293	&	1,988,541	\\
\rowcolor{color1!10!white} \multicolumn{1}{l}{15-19	}&	1,503,455	&	1,548,213	&	1,590,147	&	1,628,787	&	1,665,901	&	1,702,408	&	1,738,858	&	1,776,352	\\
\multicolumn{1}{l}{20-24	}&	1,259,936	&	1,288,381	&	1,322,125	&	1,363,226	&	1,409,899	&	1,458,875	&	1,508,001	&	1,553,450	\\
\rowcolor{color1!10!white} \multicolumn{1}{l}{25-29	}&	1,058,598	&	1,094,515	&	1,128,960	&	1,160,398	&	1,189,507	&	1,218,461	&	1,250,250	&	1,286,639	\\
\multicolumn{1}{l}{30-34	}&	845,487	&	878,933	&	913,192	&	949,092	&	986,818	&	1,025,113	&	1,062,983	&	1,099,039	\\
\rowcolor{color1!10!white} \multicolumn{1}{l}{35-39	}&	670,838	&	697,509	&	725,691	&	755,662	&	787,422	&	820,315	&	854,804	&	889,673	\\
\multicolumn{1}{l}{40-44	}&	541,392	&	559,931	&	580,303	&	602,581	&	626,480	&	651,879	&	678,855	&	707,191	\\
\rowcolor{color1!10!white} \multicolumn{1}{l}{45-49	}&	445,635	&	460,246	&	475,449	&	491,013	&	507,211	&	524,383	&	543,038	&	563,431	\\
\multicolumn{1}{l}{50-54	}&	380,865	&	386,090	&	393,702	&	404,004	&	416,153	&	429,651	&	444,291	&	459,432	\\
\rowcolor{color1!10!white} \multicolumn{1}{l}{55-59	}&	336,368	&	343,414	&	350,124	&	355,719	&	360,033	&	364,368	&	369,601	&	377,242	\\
\multicolumn{1}{l}{60-64	}&	268,582	&	281,316	&	292,331	&	301,501	&	309,722	&	317,174	&	324,161	&	330,803	\\
\rowcolor{color1!10!white} \multicolumn{1}{l}{65 o más	}&	584,612	&	601,848	&	620,965	&	642,304	&	665,281	&	689,429	&	714,340	&	739,518	\\
\rowcolor{color1!40!white} {\Bold{Hombres	}}&\textbf{6,673,533}	&	\textbf{6,836,849}	&	\textbf{7,003,337}	&	\textbf{7,173,966}	&	\textbf{7,352,869}	&	\textbf{7,535,238}	&	\textbf{7,719,396}	&	\textbf{7,903,664}	\\
\multicolumn{1}{l}{0- 4	}&	1,079,536	&	1,091,882	&	1,103,521	&	1,114,514	&	1,125,419	&	1,136,241	&	1,144,931	&	1,153,297	\\
\rowcolor{color1!10!white} \multicolumn{1}{l}{5- 9	}&	980,907	&	999,773	&	1,017,180	&	1,033,112	&	1,048,670	&	1,063,755	&	1,077,256	&	1,090,294	\\
\multicolumn{1}{l}{10-14	}&	870,648	&	888,268	&	906,603	&	926,237	&	947,338	&	968,295	&	989,122	&	1,008,018	\\
\rowcolor{color1!10!white} \multicolumn{1}{l}{15-19	}&	749,719	&	772,779	&	794,459	&	814,513	&	834,337	&	853,985	&	873,579	&	893,687	\\
\multicolumn{1}{l}{20-24	}&	613,854	&	628,975	&	646,911	&	668,760	&	694,005	&	720,428	&	747,110	&	771,615	\\
\rowcolor{color1!10!white} \multicolumn{1}{l}{25-29	}&	501,091	&	519,984	&	538,214	&	555,029	&	571,128	&	587,243	&	604,937	&	624,841	\\
\multicolumn{1}{l}{30-34	}&	384,537	&	401,181	&	418,535	&	437,092	&	457,160	&	477,757	&	498,342	&	517,919	\\
\rowcolor{color1!10!white} \multicolumn{1}{l}{35-39	}&	299,300	&	310,518	&	323,010	&	336,935	&	352,290	&	368,458	&	385,960	&	403,769	\\
\multicolumn{1}{l}{40-44	}&	243,402	&	250,452	&	258,454	&	267,301	&	276,955	&	287,439	&	299,001	&	311,703	\\
\rowcolor{color1!10!white} \multicolumn{1}{l}{45-49	}&	204,460	&	209,746	&	215,304	&	220,989	&	227,026	&	233,512	&	240,738	&	248,840	\\
\multicolumn{1}{l}{50-54	}&	179,132	&	180,370	&	182,662	&	186,117	&	190,433	&	195,298	&	200,699	&	206,306	\\
\rowcolor{color1!10!white} \multicolumn{1}{l}{55-59	}&	160,655	&	163,446	&	165,910	&	167,663	&	168,770	&	169,821	&	171,112	&	173,501	\\
\multicolumn{1}{l}{60-64	}&	129,175	&	134,692	&	139,395	&	143,257	&	146,778	&	149,903	&	152,725	&	155,222	\\
\rowcolor{color1!10!white} \multicolumn{1}{l}{65 o más	}&	277,117	&	284,783	&	293,178	&	302,448	&	312,558	&	323,102	&	333,884	&	344,652	\\
\rowcolor{color1!40!white} {\Bold{Mujeres	}}&	\textbf{7,004,282}	&	\textbf{7,180,208}	&	\textbf{7,358,328}	&	\textbf{7,539,797}	&	\textbf{7,720,506}	&	\textbf{7,903,145}	&	\textbf{8,087,279}	&	\textbf{8,272,469}	\\
\multicolumn{1}{l}{0- 4	}&	1,039,027	&	1,050,850	&	1,062,224	&	1,073,355	&	1,083,425	&	1,092,926	&	1,101,471	&	1,109,217	\\
\rowcolor{color1!10!white} \multicolumn{1}{l}{5- 9	}&	953,835	&	971,290	&	987,490	&	1,002,443	&	1,015,920	&	1,028,637	&	1,040,541	&	1,052,014	\\
\multicolumn{1}{l}{10-14	}&	858,093	&	874,596	&	891,659	&	909,814	&	928,174	&	946,474	&	964,171	&	980,523	\\
\rowcolor{color1!10!white} \multicolumn{1}{l}{15-19	}&	753,736	&	775,434	&	795,688	&	814,274	&	831,565	&	848,423	&	865,280	&	882,665	\\
\multicolumn{1}{l}{20-24	}&	646,083	&	659,406	&	675,214	&	694,466	&	715,894	&	738,447	&	760,891	&	781,835	\\
\rowcolor{color1!10!white} \multicolumn{1}{l}{25-29	}&	557,507	&	574,531	&	590,746	&	605,369	&	618,379	&	631,218	&	645,313	&	661,798	\\
\multicolumn{1}{l}{30-34	}&	460,950	&	477,752	&	494,657	&	512,000	&	529,658	&	547,355	&	564,642	&	581,120	\\
\rowcolor{color1!10!white} \multicolumn{1}{l}{35-39	}&	371,538	&	386,992	&	402,681	&	418,727	&	435,132	&	451,857	&	468,844	&	485,904	\\
\multicolumn{1}{l}{40-44	}&	297,990	&	309,479	&	321,849	&	335,281	&	349,525	&	364,440	&	379,854	&	395,488	\\
\rowcolor{color1!10!white} \multicolumn{1}{l}{45-49	}&	241,175	&	250,500	&	260,145	&	270,025	&	280,185	&	290,871	&	302,301	&	314,591	\\
\multicolumn{1}{l}{50-54	}&	201,734	&	205,720	&	211,040	&	217,887	&	225,719	&	234,353	&	243,592	&	253,126	\\
\rowcolor{color1!10!white} \multicolumn{1}{l}{55-59	}&	175,712	&	179,969	&	184,214	&	188,056	&	191,263	&	194,547	&	198,488	&	203,741	\\
\multicolumn{1}{l}{60-64	}&	139,406	&	146,624	&	152,936	&	158,245	&	162,944	&	167,271	&	171,436	&	175,581	\\
\rowcolor{color1!10!white} \multicolumn{1}{l}{65 o más	}&	307,495	&	317,065	&	327,787	&	339,856	&	352,724	&	366,327	&	380,456	&	394,866	\\
	\hline
			&&&&&&&&\\[-0.28cm]
			\multicolumn{9}{l}{\footnotesize Fuente:  INE. Proyecciones de Población con base en el XI Censo de Población y VI de Habitación.}
		\end{tabular}\addtocounter{Cuadro}{1}
	\end{center}}


%%%%%%%%%%%%5

%
%
%\hoja{
%{\Bold\color{color1!80!black}{Cuadro \theCuadro $\,-$  Población por sexo y grupo quinquenal de edad. }}\\
%{\Bold\color{color1!80!black}{República de Guatemala. Año 2015. }}\\
%(Personas)\\
%\begin{center}
%%	$\ $\\[-7cm]
%	\begin{tabular}{lrr}
%%		\multicolumn{3}{l}{\Bold\color{color1!80!black}{Cuadro \theCuadro $\,-$   Pirámide poblacional. Año 2015.}}\\
%%		\multicolumn{3}{l}{(Personas)}
%%		\\[0.4cm]
%		\hline\rowcolor{color1!40!white} &&\\[-0.36cm] \rowcolor{color1!40!white}
%		\multicolumn{1}{x{2.1cm}}{\Bold{Rangos}} & \multicolumn{1}{x{2.1cm}}{\Bold{Hombres}} & \multicolumn{1}{x{2.2cm}}{\Bold{Mujeres}} \\[0.05cm]
%		\hline
%		&&\\[-0.35cm]		
%\multicolumn{1}{l}{0- 4	 }& 	 1,153,297 	 & 	1,153,297	 \\ 
%\rowcolor{color1!10!white} \multicolumn{1}{l}{5- 9	 }& 	 1,090,294 	 & 	1,090,294	 \\ 
%\multicolumn{1}{l}{10-14	 }& 	 1,008,018 	 & 	1,008,018	 \\ 
%\rowcolor{color1!10!white} \multicolumn{1}{l}{15-19	 }& 	 893,687 	 & 	893,687	 \\ 
%\multicolumn{1}{l}{20-24	 }& 	 771,615 	 & 	771,615	 \\ 
%\rowcolor{color1!10!white} \multicolumn{1}{l}{25-29	 }& 	 624,841 	 & 	624,841	 \\ 
%\multicolumn{1}{l}{30-34	 }& 	 517,919 	 & 	517,919	 \\ 
%\rowcolor{color1!10!white} \multicolumn{1}{l}{35-39	 }& 	 403,769 	 & 	403,769	 \\ 
%\multicolumn{1}{l}{40-44	 }& 	 311,703 	 & 	311,703	 \\ 
%\rowcolor{color1!10!white} \multicolumn{1}{l}{45-49	 }& 	 248,840 	 & 	248,840	 \\ 
%\multicolumn{1}{l}{50-54	 }& 	 206,306 	 & 	206,306	 \\ 
%\rowcolor{color1!10!white} \multicolumn{1}{l}{55-59	 }& 	 173,501 	 & 	173,501	 \\ 
%\multicolumn{1}{l}{60-64	 }& 	 155,222 	 & 	155,222	 \\ 
%\rowcolor{color1!10!white} \multicolumn{1}{l}{65 o más	 }& 	 344,652 	 & 	344,652	 \\ 
%%Tasa global de participación de la PEA $\quad $ & 65.2  & 79.7  & 52.5 \\
%%Tasa bruta de ocupación  $\quad $ & 61.5  & 75.8  & 49.0 \\
%%Tasa específica de ocupación $\quad $ & 94.3  & 95.0  & 93.4 \\
%%Tasa de desempleo abierto $\quad $ & 5.7   & 5.0   & 6.6 \\
%%Tasa de subempleo visible $\quad $ & 12.5  & 12.2  & 13.0 \\
%\hline
%&&\\[-0.36cm]
%\multicolumn{3}{l}{\footnotesize Fuente:  INE. Proyecciones de Población con base }\\
%\multicolumn{3}{l}{\footnotesize  en el XI Censo de Población y VI de Habitación.}\\[-0.36cm]
%\end{tabular}\\[1.8cm]\addtocounter{Cuadro}{1}
%\end{center}
%}



\hoja{{\Bold\color{color1!80!black}{\normalsize Cuadro \theCuadro $\,-$  Índice de desarrollo humano por componentes, según departamento. }}\\
	{\Bold\color{color1!80!black}{\normalsize 	República de Guatemala, años 2006 y 2011.}}\\
	(Adimensional)\\[-.5cm]
	\begin{center}\fontsize{3.8mm}{1.6em}\selectfont \setlength{\arrayrulewidth}{0.7pt}
		$\!$\begin{tabular}{lrrrrrrrr}
			\multicolumn{9}{l}{$\ $}\\[-.2cm]
			%			\multicolumn{9}{l}{\Bold\color{color1!80!black}{\normalsize Cuadro \theCuadro $\,-$ Número de habitantes total y por sexo; según grupos quinquenales de edad.}}\\
			%			\multicolumn{9}{l}{\normalsize (Personas)}
			%			\\[-0.1cm]
			\multicolumn{1}{l}{\multirow{3}[0]{*}{\Bold{\raisebox{-0.6cm}{Departamento}}}} & \multicolumn{8}{c}{\Bold{{Indice de Desarrollo Humano}}} \\\cline{2-9}
			&&&&&&&& \\[-0.6cm]
			\multicolumn{1}{l}{$\ $} &  \multicolumn{8}{c}{$\ $} \\[-0.48cm]
			\multicolumn{1}{c}{$\ $} & \multicolumn{2}{c}{\Bold{IDH}} & \multicolumn{2}{c}{\Bold{IDH Salud}} & \multicolumn{2}{c}{\Bold{IDH Educación}} & \multicolumn{2}{c}{\Bold{IDH Ingresos}} \\
			\multicolumn{1}{c}{} &  \multicolumn{1}{c}{\Bold{2006}} & \multicolumn{1}{c}{\Bold{2011}} &  \multicolumn{1}{c}{\Bold{2006}} & \multicolumn{1}{c}{\Bold{2011}} &  \multicolumn{1}{c}{\Bold{2006}} & \multicolumn{1}{c}{\Bold{2011}} &  \multicolumn{1}{c}{\Bold{2006}} & \multicolumn{1}{c}{\Bold{2011}} \\						     
			\hline
			\rowcolor{color1!40!white}				     &&&&&&&& \\[-0.5cm]
			%			\multicolumn{1}{l}{\multirow{3}[0]{*}{\Bold{\raisebox{-0.6cm}{ }}}} & \multicolumn{8}{c}{\Bold{Año}} \\\cline{2-9}
			%			\multicolumn{1}{l}{$\ $} &  \multicolumn{8}{c}{$\ $} \\[-0.28cm]
			%			\rowcolor{color1!0!white} &&&&&&&& \\[-0.28cm]     
			%			\rowcolor{color1!0!white} { } & \multicolumn{1}{c}{2008} & \multicolumn{1}{c}{2009} & \multicolumn{1}{c}{2010} & \multicolumn{1}{c}{2011} & \multicolumn{1}{c}{2012} & \multicolumn{1}{c}{2013} & \multicolumn{1}{c}{2014} & \multicolumn{1}{c}{2015} \\ \hline
			%			\rowcolor{color1!40!white} &&&&&&&& \\[-0.28cm]
			%				&&&&&&&& \\[-0.58cm]
			\rowcolor{color1!40!white} {\Bold{República}}&	\textbf{0.569}	&	\textbf{0.580}	&	\textbf{0.823}	&	\textbf{0.807}	&	\textbf{0.419}	&	\textbf{0.452}	&	\textbf{0.533}	&	\textbf{0.534}	\\
			\multicolumn{1}{l}{Guatemala}&	0.703	&	0.697	&	0.867	&	0.846	&	0.636	&	0.636	&	0.631	&	0.631	\\
			\rowcolor{color1!10!white} \multicolumn{1}{l}{El Progreso}&	0.578	&	0.593	&	0.829	&	0.816	&	0.454	&	0.489	&	0.513	&	0.524	\\
			\multicolumn{1}{l}{Sacatepéquez}&	0.618	&	0.623	&	0.848	&	0.832	&	0.511	&	0.525	&	0.544	&	0.553	\\
			\rowcolor{color1!10!white} \multicolumn{1}{l}{Chimaltenango}&	0.558	&	0.559	&	0.828	&	0.805	&	0.442	&	0.457	&	0.476	&	0.476	\\
			\multicolumn{1}{l}{Escuintla}&	0.557	&	0.615	&	0.835	&	0.826	&	0.404	&	0.505	&	0.510	&	0.557	\\
			\rowcolor{color1!10!white} \multicolumn{1}{l}{Santa Rosa}&	0.532	&	0.547	&	0.813	&	0.801	&	0.374	&	0.417	&	0.495	&	0.492	\\
			\multicolumn{1}{l}{Sololá}&	0.471	&	0.514	&	0.811	&	0.803	&	0.307	&	0.377	&	0.421	&	0.449	\\
			\rowcolor{color1!10!white} \multicolumn{1}{l}{Totonicapán}&	0.465	&	0.502	&	0.802	&	0.792	&	0.293	&	0.357	&	0.428	&	0.448	\\
			\multicolumn{1}{l}{Quetzaltenango}&	0.575	&	0.566	&	0.828	&	0.812	&	0.445	&	0.460	&	0.517	&	0.486	\\
			\rowcolor{color1!10!white} \multicolumn{1}{l}{Suchitepéquez}&	0.524	&	0.539	&	0.817	&	0.803	&	0.363	&	0.390	&	0.486	&	0.500	\\
			\multicolumn{1}{l}{Retalhuleu}&	0.550	&	0.540	&	0.817	&	0.798	&	0.405	&	0.419	&	0.503	&	0.472	\\
			\rowcolor{color1!10!white} \multicolumn{1}{l}{San Marcos}&	0.512	&	0.512	&	0.806	&	0.790	&	0.360	&	0.387	&	0.462	&	0.441	\\
			\multicolumn{1}{l}{Huehuetenango}&	0.467	&	0.498	&	0.803	&	0.789	&	0.279	&	0.337	&	0.453	&	0.465	\\
			\rowcolor{color1!10!white} \multicolumn{1}{l}{Quiché}&	0.416	&	0.470	&	0.794	&	0.785	&	0.229	&	0.287	&	0.397	&	0.460	\\
			\multicolumn{1}{l}{Baja Verapaz}&	0.494	&	0.556	&	0.801	&	0.799	&	0.338	&	0.412	&	0.447	&	0.522	\\
			\rowcolor{color1!10!white} \multicolumn{1}{l}{Alta Verapaz}&	0.474	&	0.507	&	0.784	&	0.768	&	0.288	&	0.342	&	0.473	&	0.495	\\
			\multicolumn{1}{l}{Petén}&	0.525	&	0.524	&	0.795	&	0.783	&	0.356	&	0.367	&	0.511	&	0.502	\\
			\rowcolor{color1!10!white} \multicolumn{1}{l}{Izabal}&	0.560	&	0.568	&	0.813	&	0.806	&	0.379	&	0.434	&	0.570	&	0.523	\\
			\multicolumn{1}{l}{Zacapa}&	0.564	&	0.572	&	0.827	&	0.808	&	0.409	&	0.445	&	0.530	&	0.521	\\
			\rowcolor{color1!10!white} \multicolumn{1}{l}{Chiquimula}&	0.521	&	0.541	&	0.806	&	0.793	&	0.351	&	0.413	&	0.500	&	0.484	\\
			\multicolumn{1}{l}{Jalapa}&	0.486	&	0.526	&	0.796	&	0.790	&	0.315	&	0.401	&	0.457	&	0.459	\\
			\rowcolor{color1!10!white} \multicolumn{1}{l}{Jutiapa}&	0.534	&	0.579	&	0.804	&	0.798	&	0.376	&	0.459	&	0.505	&	0.530	\\
			\hline
			&&&&&&&&\\[-0.28cm]
			%			\multicolumn{9}{l}{\footnotesize Fuente: Informe Nacional de Desarrollo Humano (PNUD), con base en las Encuestas Nacionales de Condiciones de Vida (Encovi).}
		\end{tabular}\addtocounter{Cuadro}{1}
	\end{center}
	{\footnotesize Fuente: Informe Nacional de Desarrollo Humano (PNUD), con base en las Encuestas Nacionales de Condiciones de Vida (Encovi).}}

%%%%%%%%%%%%%%%%%%%%1-5 (5)
\hoja{
	{\Bold\Large 1.2 Vivienda}\\[.5cm]
	{\Bold\color{color1!80!black}{\parbox{15cm}{Cuadro \theCuadro $\,-$  Necesidades Básicas Insatisfechas de los hogares rurales en  vivienda, hacinamiento, agua y saneamiento; según departamento. }}}\\[.1cm]
{\Bold\color{color1!80!black}{República de Guatemala. Año 2011.}}\\
{(Porcentaje de hogares)}\\[0.4cm]
\begin{center}\fontsize{3.8mm}{1.65em}\selectfont 
	\begin{tabular}{p{3cm}S[table-format=3]S[table-format=3]S[table-format=3]S[table-format=3]}
		
%		\multicolumn{5}{l}{\Bold\color{color1!80!black}{\parbox{15cm}{Cuadro \theCuadro $\,-$  Necesidades Básicas Insatisfechas de los hogares rurales en  vivienda, hacinamiento, agua y saneamiento; según departamento. }}}\\
%				\multicolumn{5}{l}{\Bold\color{color1!80!black}{República de Guatemala. Año 2011.}}\\
%		\multicolumn{5}{l}{(Porcentaje de hogares)}
%		\\[0.4cm]
		\hline\rowcolor{color1!40!white} &&&&\\[-0.4cm] \rowcolor{color1!40!white}
		\multicolumn{1}{x{3cm}}{\Bold{Departamento}} & \multicolumn{1}{x{2.cm}}{\Bold{Vivienda}} & \multicolumn{1}{x{2.cm}}{\Bold{Hacinamiento}} & \multicolumn{1}{x{2.cm}}{\Bold{Agua}} & \multicolumn{1}{x{2.cm}}{\Bold{Saneamiento}} \\[0.05cm]
		\hline\color{black}
		&&&&\\[-0.8cm]
		Guatemala	&	 33.9 	&	 36.3 	&	32.5	&	 7.1 	\\
		\rowcolor{color1!10!white}El Progreso	&	 33.5 	&	 38.0 	&	26.6	&	 17.9 	\\
		Sacatepéquez	&	 32.1 	&	 33.4 	&	9.2	&	 6.3 	\\
		\rowcolor{color1!10!white}Chimaltenango	&	 36.8 	&	 41.8 	&	25.5	&	 8.1 	\\
		Escuintla	&	 25.2 	&	 40.7 	&	31.5	&	 6.7 	\\
		\rowcolor{color1!10!white}Santa Rosa	&	 44.0 	&	 40.2 	&	31.3	&	 16.0 	\\
		Sololá	&	 42.0 	&	 46.7 	&	10.6	&	 17.0 	\\
		\rowcolor{color1!10!white}Totonicapán	&	 59.0 	&	 49.5 	&	21.3	&	 12.4 	\\
		Quetzaltenango	&	 33.3 	&	 45.2 	&	31.1	&	 7.6 	\\
		\rowcolor{color1!10!white}Suchitepéquez	&	 37.6 	&	 52.7 	&	38.9	&	 19.0 	\\
		Retalhuleu	&	 48.3 	&	 50.0 	&	60.5	&	 11.1 	\\
		\rowcolor{color1!10!white}San Marcos	&	 46.9 	&	 54.6 	&	33.1	&	 6.6 	\\
		Huehuetenango	&	 56.2 	&	 54.6 	&	32	&	 16.0 	\\
		\rowcolor{color1!10!white}Quiché	&	 70.2 	&	 59.9 	&	33.6	&	 22.4 	\\
		Baja Verapaz	&	 54.0 	&	 46.2 	&	30.9	&	 17.3 	\\
		\rowcolor{color1!10!white}Alta Verapaz	&	 77.2 	&	 64.8 	&	60.6	&	 13.4 	\\
		Petén	&	 54.4 	&	 51.4 	&	43.7	&	 30.7 	\\
		\rowcolor{color1!10!white}Izabal	&	 53.6 	&	 52.5 	&	37.8	&	 24.1 	\\
		Zacapa	&	 40.0 	&	 45.2 	&	23	&	 22.3 	\\
		\rowcolor{color1!10!white}Chiquimula	&	 60.6 	&	 45.9 	&	32.6	&	 40.4 	\\
		Jalapa	&	 70.1 	&	 51.4 	&	40.3	&	 29.6 	\\
		\rowcolor{color1!10!white}Jutiapa	&	 40.0 	&	 40.6 	&	24.7	&	 34.5 	\\
		\hline
		&&&&\\[-0.36cm]
		\multicolumn{5}{l}{\footnotesize Fuente: Instituto Nacional de Estadística (INE), Censos Municipales 2008 – 2011.}
	\end{tabular}\\[1.8cm] \addtocounter{Cuadro}{1}
\end{center}
}



%%% cuadro 1-6
\newpage
	{\Bold\Large 1.3 Pobreza}\\[-1.5cm]

\begin{center}\fontsize{3.2mm}{1.4em}\selectfont \setlength{\arrayrulewidth}{0.7pt}
	$\ $\\[-0.5cm]
	$\!$\begin{longtable}{lrrrrrr}
								\multicolumn{7}{l}{\Bold\color{color1!80!black}{\normalsize Cuadro \theCuadro $\,-$  Mapa de pobreza rural, por departamento y municipio; según tipo de pobreza.  }}\\
								\multicolumn{7}{l}{\Bold\color{color1!80!black}{\normalsize República de Guatemala, año 2011.}}\\[-.1cm]
								(Porcentaje de personas)\\
								\multicolumn{7}{l}{$\ $}\\[-.2cm]\hline
								\multicolumn{1}{x{5.5cm}}{\multirow{3}[0]{*}{\Bold{\raisebox{.3cm}{Departamento y municipio}}}} & \multicolumn{2}{c}{\Bold{{Pobreza extrema}}}& & \multicolumn{2}{c}{\Bold{{Pobreza total}}} &\\\cline{2-7}
								&&&&&& \\[-0.6cm]
								\multicolumn{1}{l}{$\ $} &  \multicolumn{6}{c}{$\ $} \\[-0.48cm]
								\multicolumn{1}{c}{} &  \multicolumn{1}{c}{\Bold{Incidencia}} & \multicolumn{1}{c}{\Bold{Error estándar}} &  \multicolumn{1}{c}{ } &\multicolumn{1}{c}{\Bold{Incidencia}} & \multicolumn{1}{c}{\Bold{Error estándar}} &  \multicolumn{1}{c}{ }\\						     
								\hline\endfirsthead
									\multicolumn{7}{l}{\Bold\color{color1!80!black}{\normalsize Cuadro \theCuadro $\,-$  Mapa de pobreza rural, por departamento y municipio; según tipo de pobreza.  }}\\
%									\multicolumn{7}{l}{\Bold\color{color1!80!black}{\normalsize República de Guatemala, año 2011.}}\\[-.1cm]
									(Continuación)\\
									\multicolumn{7}{l}{$\ $}\\[-.2cm]\hline
									\multicolumn{1}{l}{\multirow{3}[0]{*}{\Bold{\raisebox{.3cm}{Departamento y municipio}}}} & \multicolumn{2}{c}{\Bold{{Pobreza extrema}}}& & \multicolumn{2}{c}{\Bold{{Pobreza total}}} &\\\cline{2-7}
									&&&&&& \\[-0.6cm]
									\multicolumn{1}{l}{$\ $} &  \multicolumn{6}{c}{$\ $} \\[-0.48cm]
									\multicolumn{1}{c}{} &  \multicolumn{1}{c}{\Bold{Incidencia}} & \multicolumn{1}{c}{\Bold{Error estándar}} &  \multicolumn{1}{c}{ } &\multicolumn{1}{c}{\Bold{Incidencia}} & \multicolumn{1}{c}{\Bold{Error estándar}} &  \multicolumn{1}{c}{ }\\						     
									\hline\endhead
		\hline \multicolumn{7}{r}{\textit{Continúa en la siguiente página}} \\
		\endfoot
		&&&&&& \\[-0.7cm]
		\multicolumn{7}{l}{\footnotesize Fuente: INE y Banco Mundial, 2013.}\\
		\multicolumn{7}{l}{\parbox{15cm}{\footnotesize \textbf{Notas:} - Solo áreas rurales. La tasa de pobreza total se calculó usando la línea oficial de Q.8,283. La tasa de pobreza extrema se calculó usando la línea oficial de Q.4,380.}}\\[0.2cm]
		\multicolumn{7}{l}{\parbox{15cm}{\footnotesize (*) Dato no confiable porque el coeficiente de variación está por encima de 10\% para la tasa de pobreza total y 25\% para la tasa de pobreza extrema (valores de referencia observados en la Encovi para estas medidas).}}\\[0.1cm]
		\multicolumn{7}{l}{\parbox{15cm}{\footnotesize (+) Dato no confiable porque hay menos de 500 hogares en el  municipio.}}\\[0.1cm]
		\multicolumn{7}{l}{\parbox{15cm}{\footnotesize Instituto Nacional de Estadística (INE) y Banco Mundial. 2013. Mapas de Pobreza Rural en Guatemala 2011. Autor, Guatemala. En: \url{http://www.ine.gob.gt/sistema/uploads/2014/01/10/ifRRpEnf0cjUfRZGhyXD7RQjf7EQH2Er.pdf} (Consultado: febrero 2016).}}
		\endlastfoot
		\rowcolor{color1!40!white}				     &&&&&& \\[-0.5cm]
		\rowcolor{color1!40!white} {\Bold{	El Progreso	}}&	6.11	&	0.01	&		&	44.28	&	0.02	&		\\
		\multicolumn{1}{l}{	Guastatoya	}&	2.877	&	0.691	&		&	24.333	&	2.060	&		\\
		\rowcolor{color1!10!white} \multicolumn{1}{l}{	Morazán	}&	4.620	&	0.957	&		&	29.961	&	1.980	&		\\
		\multicolumn{1}{l}{	San Agustín Acasaguastlán	}&	12.667	&	1.140	&		&	53.889	&	1.420	&		\\
		\rowcolor{color1!10!white} \multicolumn{1}{l}{	San Cristóbal Acasaguastlán	}&	6.229	&	1.420	&		&	35.029	&	2.600	&		\\
		\multicolumn{1}{l}{	El Jícaro	}&	7.602	&	1.240	&		&	43.372	&	2.580	&		\\
		\rowcolor{color1!10!white} \multicolumn{1}{l}{	Sansare	}&	15.715	&	1.760	&		&	56.254	&	2.330	&		\\
		\multicolumn{1}{l}{	Sanarate	}&	6.714	&	0.922	&		&	38.473	&	1.880	&		\\
		\rowcolor{color1!10!white} \multicolumn{1}{l}{	San Antonio La Paz	}&	11.215	&	1.240	&		&	49.672	&	1.840	&		\\
		\rowcolor{color1!40!white} {\Bold{	Sacatepéquez	}}&	11.40	&	0.02	&		&	62.14	&	0.04	&		\\
		\multicolumn{1}{l}{	Antigua Guatemala	}&	9.270	&	2.400	&	(*)	&	54.048	&	4.750	&		\\
		\rowcolor{color1!10!white} \multicolumn{1}{l}{	Jocotenango	}&	7.045	&	3.600	&	(*)	&	55.163	&	8.360	&	(*)	\\
		\multicolumn{1}{l}{	Pastores	}&	4.638	&	1.940	&	(*)	&	46.423	&	6.420	&	(*)	\\
		\rowcolor{color1!10!white} \multicolumn{1}{l}{	Sumpango	}&	20.911	&	4.240	&		&	71.953	&	3.340	&		\\
		\multicolumn{1}{l}{	Santo Domingo Xenacoj	}&	21.251	&	6.870	&	(*)(+)	&	78.426	&	6.270	&	(+)	\\
		\rowcolor{color1!10!white} \multicolumn{1}{l}{	San Bartolomé Milpas Altas	}&	0.912	&	2.950	&	(*)(+)	&	26.063	&	16.500	&	(*)(+)	\\
		\multicolumn{1}{l}{	Magdalena Milpas Altas	}&	13.492	&	4.310	&	(*)	&	65.747	&	5.670	&		\\
		\rowcolor{color1!10!white} \multicolumn{1}{l}{	Santa Maria De Jesus	}&	21.103	&	9.920	&	(*)(+)	&	77.695	&	8.270	&	(*)(+)	\\
		\multicolumn{1}{l}{	Ciudad Vieja	}&	4.647	&	3.120	&	(*)	&	46.524	&	9.620	&	(*)	\\
		\rowcolor{color1!10!white} \multicolumn{1}{l}{	Alotenango	}&	4.054	&	3.310	&	(*)(+)	&	46.799	&	9.440	&	(*)(+)	\\
		\multicolumn{1}{l}{	   Santa Catarina Barahona	}&	9.776	&	9.880	&	(*)(+)	&	74.635	&	13.400	&	(*)(+)	\\
		\rowcolor{color1!40!white} {\Bold{	Chimaltenango	}}&	16.37	&	0.06	&		&	78.68	&	0.08	&		\\
		\multicolumn{1}{l}{	Chimaltenango	}&	19.762	&	1.720	&		&	74.308	&	1.440	&		\\
		\rowcolor{color1!10!white} \multicolumn{1}{l}{	San José Poaquil	}&	13.455	&	1.880	&		&	76.685	&	1.810	&		\\
		\multicolumn{1}{l}{	San Martín Jilotepeque	}&	17.298	&	1.420	&		&	79.329	&	1.200	&		\\
		\rowcolor{color1!10!white} \multicolumn{1}{l}{	Comalapa	}&	7.004	&	1.350	&		&	60.704	&	2.530	&		\\
		\multicolumn{1}{l}{	Santa Apolonia	}&	27.755	&	2.540	&		&	83.046	&	1.760	&		\\
		\rowcolor{color1!10!white} \multicolumn{1}{l}{	Tecpán Guatemala	}&	20.959	&	1.200	&		&	75.102	&	1.200	&		\\
		\multicolumn{1}{l}{	Patzún	}&	9.609	&	1.040	&		&	62.809	&	2.020	&		\\
		\rowcolor{color1!10!white} \multicolumn{1}{l}{	Pochuta	}&	31.057	&	2.560	&		&	77.620	&	2.840	&		\\
		\multicolumn{1}{l}{	Patzicia	}&	11.578	&	2.130	&		&	78.167	&	2.590	&		\\
		\rowcolor{color1!10!white} \multicolumn{1}{l}{	Santa Cruz Balanyá	}&	5.714	&	1.620	&	(*)(+)	&	61.588	&	4.980	&	(+)	\\
		\multicolumn{1}{l}{	Acatenango	}&	16.590	&	1.710	&		&	73.349	&	1.690	&		\\
		\rowcolor{color1!10!white} \multicolumn{1}{l}{	Yepocapa	}&	24.446	&	2.150	&		&	78.677	&	1.710	&		\\
		\multicolumn{1}{l}{	San Andrés Itzapa	}&	20.051	&	2.730	&		&	78.287	&	2.450	&		\\
		\rowcolor{color1!10!white} \multicolumn{1}{l}{	Parramos	}&	17.141	&	3.070	&		&	78.218	&	3.150	&		\\
		\multicolumn{1}{l}{	Zaragoza	}&	11.587	&	1.850	&		&	72.472	&	2.020	&		\\
		\rowcolor{color1!10!white} \multicolumn{1}{l}{	El Tejar	}&	21.735	&	3.170	&		&	67.279	&	2.990	&		\\
		\rowcolor{color1!40!white} {\Bold{	Escuintla	}}&	3.04	&	0.03	&		&	47.37	&	0.12	&		\\
		\multicolumn{1}{l}{	Escuintla	}&	5.490	&	0.537	&		&	53.507	&	1.350	&		\\
		\rowcolor{color1!10!white} \multicolumn{1}{l}{	Santa Lucía Cotzumalguapa	}&	4.913	&	0.526	&		&	51.734	&	1.470	&		\\
		\multicolumn{1}{l}{	La Democracia	}&	1.404	&	0.499	&	(*)	&	28.523	&	1.520	&		\\
		\rowcolor{color1!10!white} \multicolumn{1}{l}{	Siquinalá	}&	9.222	&	0.907	&		&	68.323	&	1.730	&		\\
		\multicolumn{1}{l}{	Masagua	}&	6.478	&	0.619	&		&	56.146	&	1.840	&		\\
		\rowcolor{color1!10!white} \multicolumn{1}{l}{	Tiquisate	}&	2.871	&	1.020	&	(*)	&	34.583	&	2.750	&		\\
		\multicolumn{1}{l}{	La Gomera	}&	3.529	&	1.480	&	(*)	&	34.573	&	2.910	&		\\
		\rowcolor{color1!10!white} \multicolumn{1}{l}{	Guanagazapa	}&	16.166	&	1.330	&		&	74.936	&	2.040	&		\\
		\multicolumn{1}{l}{	San José	}&	1.393	&	0.489	&	(*)	&	23.947	&	1.570	&		\\
		\rowcolor{color1!10!white} \multicolumn{1}{l}{	Iztapa	}&	2.211	&	1.020	&	(*)	&	23.854	&	2.950	&	(*)	\\
		\multicolumn{1}{l}{	Palín	}&	3.057	&	0.554	&		&	45.244	&	1.630	&		\\
		\rowcolor{color1!10!white} \multicolumn{1}{l}{	San Vicente Pacaya	}&	2.771	&	0.598	&		&	40.953	&	1.910	&		\\
		\multicolumn{1}{l}{	Nueva Concepción	}&	4.949	&	0.779	&		&	44.898	&	1.530	&		\\
		\rowcolor{color1!40!white} {\Bold{	Santa Rosa	}}&	14.27	&	0.05	&		&	62.61	&	0.07	&		\\
		\multicolumn{1}{l}{	Cuilapa	}&	11.051	&	1.390	&		&	61.367	&	0.916	&		\\
		\rowcolor{color1!10!white} \multicolumn{1}{l}{	Barberena	}&	6.033	&	1.310	&		&	54.204	&	0.967	&		\\
		\multicolumn{1}{l}{	Santa Rosa De Lima	}&	8.723	&	1.750	&		&	45.465	&	1.440	&		\\
		\rowcolor{color1!10!white} \multicolumn{1}{l}{	Casillas	}&	10.474	&	1.490	&		&	56.204	&	1.140	&		\\
		\multicolumn{1}{l}{	San Rafael Las Flores	}&	11.075	&	1.820	&		&	63.818	&	1.690	&		\\
		\rowcolor{color1!10!white} \multicolumn{1}{l}{	Oratorio	}&	15.967	&	1.780	&		&	61.340	&	1.220	&		\\
		\multicolumn{1}{l}{	San Juan Tecuaco	}&	19.638	&	2.570	&		&	70.858	&	1.800	&		\\
		\rowcolor{color1!10!white} \multicolumn{1}{l}{	Chiquimulilla	}&	15.937	&	1.170	&		&	62.745	&	0.735	&		\\
		\multicolumn{1}{l}{	Taxisco	}&	13.837	&	1.320	&		&	64.906	&	1.040	&		\\
		\rowcolor{color1!10!white} \multicolumn{1}{l}{	Santa Maráa Ixhuatán	}&	19.356	&	1.500	&		&	66.888	&	1.100	&		\\
		\multicolumn{1}{l}{	Guazacapán	}&	10.249	&	1.670	&		&	61.385	&	1.520	&		\\
		\rowcolor{color1!10!white} \multicolumn{1}{l}{	Santa Cruz Naranjo	}&	8.122	&	1.750	&		&	47.678	&	1.240	&		\\
		\multicolumn{1}{l}{	Pueblo Nuevo Viñas	}&	15.409	&	1.740	&		&	69.708	&	1.200	&		\\
		\rowcolor{color1!10!white} \multicolumn{1}{l}{	Nueva Santa Rosa	}&	13.172	&	1.390	&		&	58.042	&	0.977	&		\\
		\rowcolor{color1!40!white} {\Bold{	Sololá	}}&	14.57	&	0.05	&		&	84.48	&	0.08	&		\\
		\multicolumn{1}{l}{	Sololá	}&	17.400	&	2.430	&		&	84.911	&	2.030	&		\\
		\rowcolor{color1!10!white} \multicolumn{1}{l}{	San José Chacayá	}&	6.303	&	2.970	&	(*)(+)	&	69.552	&	4.790	&	(+)	\\
		\multicolumn{1}{l}{	Santa María Visitación	}&	13.454	&	5.800	&	(*)(+)	&	66.538	&	8.730	&	(*)(+)	\\
		\rowcolor{color1!10!white} \multicolumn{1}{l}{	Santa Lucía Utatlán	}&	17.294	&	2.310	&		&	80.668	&	2.130	&		\\
		\multicolumn{1}{l}{	Nahualá	}&	13.125	&	2.070	&		&	85.868	&	1.910	&		\\
		\rowcolor{color1!10!white} \multicolumn{1}{l}{	Santa Catarina Ixtahuacán	}&	12.836	&	2.210	&		&	83.515	&	2.120	&		\\
		\multicolumn{1}{l}{	Santa Clara La Laguna	}&	7.252	&	2.680	&	(*)(+)	&	62.028	&	7.030	&	(*)(+)	\\
		\rowcolor{color1!10!white} \multicolumn{1}{l}{	Concepción	}&	8.651	&	4.690	&	(*)(+)	&	86.452	&	4.420	&	(+)	\\
		\multicolumn{1}{l}{	San Andrés Semetabaj	}&	19.167	&	3.800	&		&	82.800	&	2.890	&		\\
		\rowcolor{color1!10!white} \multicolumn{1}{l}{	Panajachel	}&	11.148	&	2.980	&	(*)	&	57.987	&	5.500	&		\\
		\multicolumn{1}{l}{	Santa Catarina Palopó	}&	28.170	&	8.630	&	(*)(+)	&	89.467	&	3.360	&	(+)	\\
		\rowcolor{color1!10!white} \multicolumn{1}{l}{	San Antonio Palopó	}&	14.716	&	2.700	&		&	80.721	&	2.770	&		\\
		\multicolumn{1}{l}{	San Lucas Tolimán	}&	29.420	&	4.330	&		&	91.133	&	1.640	&		\\
		\rowcolor{color1!10!white} \multicolumn{1}{l}{	Santa Cruz La Laguna	}&	30.780	&	6.750	&		&	94.987	&	1.860	&		\\
		\multicolumn{1}{l}{	San Juan La Laguna	}&	9.723	&	4.460	&	(*)	&	78.044	&	5.470	&		\\
		\rowcolor{color1!10!white} \multicolumn{1}{l}{	San Pedro La Laguna	}&	8.222	&	5.450	&	(*)(+)	&	81.527	&	8.280	&	(*)(+)	\\
		\multicolumn{1}{l}{	Santiago Atitlán	}&	43.738	&	7.450	&		&	97.305	&	1.270	&		\\
		\rowcolor{color1!40!white} {\Bold{	Totonicapán	}}&	24.50	&	0.05	&		&	80.57	&	0.06	&		\\
		\multicolumn{1}{l}{	Totonicapán	}&	22.577	&	1.650	&		&	69.653	&	1.580	&		\\
		\rowcolor{color1!10!white} \multicolumn{1}{l}{	San Cristóbal Totonicapán	}&	18.888	&	2.030	&		&	68.403	&	2.360	&		\\
		\multicolumn{1}{l}{	San Francisco El Alto	}&	20.208	&	2.380	&		&	70.667	&	2.610	&		\\
		\rowcolor{color1!10!white} \multicolumn{1}{l}{	San Andrés Xecul	}&	22.155	&	2.760	&		&	73.414	&	3.310	&		\\
		\multicolumn{1}{l}{	Momostenango	}&	46.974	&	1.500	&		&	87.904	&	0.812	&		\\
		\rowcolor{color1!10!white} \multicolumn{1}{l}{	Santa María Chiquimula	}&	30.907	&	1.870	&		&	83.088	&	1.340	&		\\
		\multicolumn{1}{l}{	Santa Lucía La Reforma	}&	70.133	&	3.540	&		&	97.774	&	0.663	&		\\
		\rowcolor{color1!10!white} \multicolumn{1}{l}{	San Bartolo Aguas Calientes	}&	35.791	&	3.260	&		&	82.909	&	2.330	&		\\
		\rowcolor{color1!40!white} {\Bold{	Quetzaltenango	}}&	17.31	&	0.08	&		&	67.33	&	0.10	&		\\
		\multicolumn{1}{l}{	Quetzaltenango	}&	15.730	&	1.880	&		&	68.909	&	2.300	&		\\
		\rowcolor{color1!10!white} \multicolumn{1}{l}{	San Carlos Sija	}&	13.100	&	1.490	&		&	56.765	&	2.080	&		\\
		\multicolumn{1}{l}{	Cabrican	}&	15.569	&	2.000	&		&	64.841	&	2.680	&		\\
		\rowcolor{color1!10!white} \multicolumn{1}{l}{	Cajolá	}&	54.754	&	5.730	&		&	94.852	&	1.430	&		\\
		\multicolumn{1}{l}{	San Miguel Siguilá	}&	39.259	&	5.790	&		&	90.615	&	2.170	&		\\
		\rowcolor{color1!10!white} \multicolumn{1}{l}{	Ostuncalco	}&	39.471	&	2.500	&		&	85.577	&	1.340	&		\\
		\multicolumn{1}{l}{	Concepción Chiquirichapa	}&	11.428	&	2.240	&		&	61.205	&	3.960	&		\\
		\rowcolor{color1!10!white} \multicolumn{1}{l}{	San Martín Sacatepéquez	}&	21.053	&	2.200	&		&	73.600	&	2.420	&		\\
		\multicolumn{1}{l}{	Cantel	}&	22.697	&	3.000	&		&	76.326	&	2.680	&		\\
		\rowcolor{color1!10!white} \multicolumn{1}{l}{	Huitán	}&	16.521	&	3.430	&		&	76.465	&	3.270	&		\\
		\multicolumn{1}{l}{	Colomba	}&	13.324	&	1.480	&		&	52.576	&	2.370	&		\\
		\rowcolor{color1!10!white} \multicolumn{1}{l}{	San Francisco La Unión	}&	26.625	&	3.540	&		&	72.904	&	3.680	&		\\
		\multicolumn{1}{l}{	El Palmar	}&	8.952	&	1.710	&		&	49.310	&	3.090	&		\\
		\rowcolor{color1!10!white} \multicolumn{1}{l}{	Coatepeque	}&	15.713	&	1.030	&		&	56.655	&	2.030	&		\\
		\multicolumn{1}{l}{	Génova	}&	25.716	&	1.680	&		&	74.270	&	2.030	&		\\
		\rowcolor{color1!10!white} \multicolumn{1}{l}{	Flores Costa Cuca	}&	9.330	&	1.300	&		&	53.071	&	2.410	&		\\
		\multicolumn{1}{l}{	Palestina De Los Altos	}&	22.716	&	2.350	&		&	75.728	&	1.950	&		\\
		\rowcolor{color1!40!white} {\Bold{	Suchitepéquez	}}&	29.53	&	0.08	&		&	80.48	&	0.10	&		\\
		\multicolumn{1}{l}{	Mazatenango	}&	35.200	&	2.220	&		&	73.035	&	1.490	&		\\
		\rowcolor{color1!10!white} \multicolumn{1}{l}{	Cuyotenango	}&	23.763	&	2.310	&		&	73.667	&	1.880	&		\\
		\multicolumn{1}{l}{	San Francisco Zapotitlán	}&	18.661	&	3.840	&		&	78.833	&	4.790	&		\\
		\rowcolor{color1!10!white} \multicolumn{1}{l}{	San Bernardino	}&	5.464	&	3.270	&	(*)	&	49.224	&	5.480	&	(*)	\\
		\multicolumn{1}{l}{	San José El Ídolo	}&	23.160	&	3.820	&		&	75.858	&	2.730	&		\\
		\rowcolor{color1!10!white} \multicolumn{1}{l}{	Santo Domingo Suchitepéquez	}&	16.084	&	2.040	&		&	68.048	&	2.200	&		\\
		\multicolumn{1}{l}{	San Lorenzo	}&	50.522	&	4.160	&		&	93.181	&	1.530	&		\\
		\rowcolor{color1!10!white} \multicolumn{1}{l}{	Samayac	}&	17.116	&	3.430	&		&	75.058	&	3.470	&		\\
		\multicolumn{1}{l}{	San Pablo Jocopilas	}&	10.077	&	2.340	&		&	60.455	&	3.840	&		\\
		\rowcolor{color1!10!white} \multicolumn{1}{l}{	San Antonio Suchitepéquez	}&	11.317	&	2.030	&		&	61.237	&	3.380	&		\\
		\multicolumn{1}{l}{	San Miguel Panán	}&	11.577	&	4.740	&	(*)	&	64.362	&	8.550	&	(*)	\\
		\rowcolor{color1!10!white} \multicolumn{1}{l}{	San Gabriel	}&	36.498	&	6.610	&	(+)	&	77.086	&	4.640	&	(+)	\\
		\multicolumn{1}{l}{	Chicacao	}&	76.599	&	2.700	&		&	96.435	&	0.748	&		\\
		\rowcolor{color1!10!white} \multicolumn{1}{l}{	Patulul	}&	33.178	&	1.920	&		&	70.641	&	1.780	&		\\
		\multicolumn{1}{l}{	Santa Bárbara	}&	37.537	&	2.880	&		&	83.986	&	1.570	&		\\
		\rowcolor{color1!10!white} \multicolumn{1}{l}{	San Juan Bautista	}&	20.111	&	4.290	&		&	78.006	&	4.810	&		\\
		\multicolumn{1}{l}{	Santo Tomás La Unión	}&	26.891	&	4.950	&		&	80.707	&	4.240	&		\\
		\rowcolor{color1!10!white} \multicolumn{1}{l}{	Zunilito	}&	12.967	&	4.920	&	(*)	&	66.987	&	5.700	&		\\
		\multicolumn{1}{l}{	Pueblo Nuevo	}&	52.533	&	3.870	&		&	89.927	&	2.480	&		\\
		\rowcolor{color1!10!white} \multicolumn{1}{l}{	Rio Bravo	}&	37.569	&	4.040	&		&	88.675	&	1.820	&		\\
		\rowcolor{color1!40!white} {\Bold{	Retalhuleu	}}&	15.04	&	0.03	&		&	68.62	&	0.05	&		\\
		\multicolumn{1}{l}{	Retalhuleu	}&	13.900	&	1.340	&		&	64.735	&	1.940	&		\\
		\rowcolor{color1!10!white} \multicolumn{1}{l}{	Santa Cruz Muluá	}&	10.724	&	2.770	&	(*)	&	58.789	&	4.980	&		\\
		\multicolumn{1}{l}{	San Andres Villa Seca	}&	6.580	&	1.410	&		&	45.555	&	2.900	&		\\
		\rowcolor{color1!10!white} \multicolumn{1}{l}{	Champerico	}&	13.613	&	1.900	&		&	61.148	&	2.020	&		\\
		\multicolumn{1}{l}{	Nuevo San Carlos	}&	17.442	&	2.060	&		&	71.191	&	2.270	&		\\
		\rowcolor{color1!10!white} \multicolumn{1}{l}{	El Asintal	}&	24.611	&	2.690	&		&	76.255	&	2.430	&		\\
		\rowcolor{color1!40!white} {\Bold{	San Marcos	}}&	18.73	&	0.14	&		&	76.43	&	0.22	&		\\
		\rowcolor{color1!10!white} \multicolumn{1}{l}{	San Marcos	}&	2.365	&	0.501	&		&	42.286	&	1.420	&		\\
		\multicolumn{1}{l}{	San Pedro Sacatepéquez	}&	3.852	&	0.509	&		&	50.203	&	1.180	&		\\
		\rowcolor{color1!10!white} \multicolumn{1}{l}{	San Antonio Sacatepéquez	}&	4.232	&	0.869	&		&	56.487	&	1.730	&		\\
		\multicolumn{1}{l}{	Comitancillo	}&	26.617	&	2.060	&		&	89.893	&	1.070	&		\\
		\rowcolor{color1!10!white} \multicolumn{1}{l}{	San Miguel Ixtahuacán	}&	27.874	&	1.910	&		&	91.092	&	0.868	&		\\
		\multicolumn{1}{l}{	Concepción Tutuapa	}&	37.418	&	1.650	&		&	91.915	&	0.689	&		\\
		\rowcolor{color1!10!white} \multicolumn{1}{l}{	Tacaná	}&	37.849	&	1.250	&		&	88.758	&	0.738	&		\\
		\multicolumn{1}{l}{	Sibinal	}&	46.907	&	2.090	&		&	91.170	&	1.120	&		\\
		\rowcolor{color1!10!white} \multicolumn{1}{l}{	Tajumulco	}&	21.432	&	1.300	&		&	79.696	&	1.010	&		\\
		\multicolumn{1}{l}{	Tejutla	}&	40.983	&	1.910	&		&	89.156	&	1.220	&		\\
		\rowcolor{color1!10!white} \multicolumn{1}{l}{	San Rafael Pie De La Cuesta	}&	9.371	&	1.090	&		&	66.245	&	1.730	&		\\
		\multicolumn{1}{l}{	Nuevo Progreso	}&	24.054	&	1.240	&		&	79.201	&	1.070	&		\\
		\rowcolor{color1!10!white} \multicolumn{1}{l}{	El Tumbador	}&	14.342	&	1.040	&		&	72.995	&	1.240	&		\\
		\multicolumn{1}{l}{	El Rodeo	}&	30.257	&	1.700	&		&	84.028	&	1.380	&		\\
		\rowcolor{color1!10!white} \multicolumn{1}{l}{	Malacatán	}&	18.416	&	1.030	&		&	77.056	&	0.925	&		\\
		\multicolumn{1}{l}{	Catarina	}&	15.545	&	1.330	&		&	77.288	&	1.570	&		\\
		\rowcolor{color1!10!white} \multicolumn{1}{l}{	Ayutla	}&	1.400	&	0.454	&	(*)	&	37.003	&	2.540	&		\\
		\multicolumn{1}{l}{	Ocós	}&	10.440	&	1.060	&		&	74.942	&	1.820	&		\\
		\rowcolor{color1!10!white} \multicolumn{1}{l}{	San Pablo	}&	14.367	&	0.924	&		&	71.765	&	1.050	&		\\
		\multicolumn{1}{l}{	El Quetzal	}&	27.114	&	1.540	&		&	81.885	&	1.280	&		\\
		\rowcolor{color1!10!white} \multicolumn{1}{l}{	La Reforma	}&	30.849	&	1.550	&		&	86.429	&	1.230	&		\\
		\multicolumn{1}{l}{	Pajapita	}&	22.669	&	1.980	&		&	81.423	&	1.470	&		\\
		\rowcolor{color1!10!white} \multicolumn{1}{l}{	Ixchiguán	}&	27.573	&	1.700	&		&	80.289	&	1.120	&		\\
		\multicolumn{1}{l}{	San José Ojetenam	}&	53.852	&	2.310	&		&	93.207	&	0.999	&		\\
		\rowcolor{color1!10!white} \multicolumn{1}{l}{	San Cristóbal Cucho	}&	16.893	&	1.690	&		&	78.752	&	1.930	&		\\
		\multicolumn{1}{l}{	Sipacapa	}&	36.738	&	2.140	&		&	93.174	&	0.848	&		\\
		\rowcolor{color1!10!white} \multicolumn{1}{l}{	Esquipulas Palo Gordo	}&	11.562	&	1.610	&		&	74.815	&	2.110	&		\\
		\multicolumn{1}{l}{	Río Blanco	}&	7.084	&	1.840	&	(*)	&	60.243	&	2.150	&		\\
		\rowcolor{color1!10!white} \multicolumn{1}{l}{	San Lorenzo	}&	33.808	&	2.320	&		&	86.692	&	1.320	&		\\
		\rowcolor{color1!40!white} {\Bold{	Huehuetenango	}}&	11.27	&	0.11	&		&	67.59	&	0.20	&		\\
		\multicolumn{1}{l}{	Huehuetenango	}&	1.687	&	0.254	&		&	37.121	&	1.540	&		\\
		\rowcolor{color1!10!white} \multicolumn{1}{l}{	Chiantla	}&	8.987	&	1.230	&		&	64.646	&	1.050	&		\\
		\multicolumn{1}{l}{	Malacatancito	}&	9.154	&	1.040	&		&	69.718	&	1.350	&		\\
		\rowcolor{color1!10!white} \multicolumn{1}{l}{	Cuilco	}&	16.789	&	1.050	&		&	81.023	&	0.904	&		\\
		\multicolumn{1}{l}{	Nentón	}&	17.861	&	2.010	&		&	78.967	&	1.540	&		\\
		\rowcolor{color1!10!white} \multicolumn{1}{l}{	San Pedro Necta	}&	10.167	&	1.260	&		&	82.735	&	1.120	&		\\
		\multicolumn{1}{l}{	Jacaltenango	}&	8.517	&	1.510	&		&	58.212	&	1.470	&		\\
		\rowcolor{color1!10!white} \multicolumn{1}{l}{	Soloma	}&	9.911	&	1.300	&		&	71.805	&	1.340	&		\\
		\multicolumn{1}{l}{	San Ildefonso Ixtahuacán	}&	23.045	&	1.850	&		&	90.857	&	0.942	&		\\
		\rowcolor{color1!10!white} \multicolumn{1}{l}{	Santa Bárbara	}&	15.890	&	1.870	&		&	82.222	&	1.550	&		\\
		\multicolumn{1}{l}{	La Libertad	}&	14.284	&	1.160	&		&	75.451	&	1.160	&		\\
		\rowcolor{color1!10!white} \multicolumn{1}{l}{	La Democracia	}&	14.996	&	1.130	&		&	70.647	&	1.040	&		\\
		\multicolumn{1}{l}{	San Miguel Acatán	}&	22.469	&	1.450	&		&	73.085	&	2.090	&		\\
		\rowcolor{color1!10!white} \multicolumn{1}{l}{	San Rafael La Independencia	}&	31.164	&	1.900	&		&	84.797	&	2.620	&		\\
		\multicolumn{1}{l}{	Todos Santos Cuchumatán	}&	6.538	&	1.240	&		&	71.865	&	1.150	&		\\
		\rowcolor{color1!10!white} \multicolumn{1}{l}{	San Juan Atitán	}&	8.342	&	1.540	&		&	82.072	&	1.310	&		\\
		\multicolumn{1}{l}{	Santa Eulalia	}&	7.772	&	1.550	&		&	74.186	&	1.510	&		\\
		\rowcolor{color1!10!white} \multicolumn{1}{l}{	San Mateo Ixtatán	}&	9.720	&	1.190	&		&	76.833	&	1.290	&		\\
		\multicolumn{1}{l}{	Colotenango	}&	16.783	&	1.620	&		&	88.112	&	1.220	&		\\
		\rowcolor{color1!10!white} \multicolumn{1}{l}{	San Sebastían Huehuetenango	}&	8.994	&	1.440	&		&	77.542	&	1.230	&		\\
		\multicolumn{1}{l}{	Tectitán	}&	12.397	&	1.590	&		&	84.853	&	1.180	&		\\
		\rowcolor{color1!10!white} \multicolumn{1}{l}{	Concepción Huista	}&	16.661	&	1.420	&		&	76.905	&	1.470	&		\\
		\multicolumn{1}{l}{	San Juan Ixcoy	}&	18.924	&	1.570	&		&	82.093	&	1.390	&		\\
		\rowcolor{color1!10!white} \multicolumn{1}{l}{	San Antonio Huista	}&	13.856	&	1.380	&		&	64.914	&	1.610	&		\\
		\multicolumn{1}{l}{	San Sebastián Coatán	}&	6.411	&	1.430	&		&	74.154	&	1.240	&		\\
		\rowcolor{color1!10!white} \multicolumn{1}{l}{	Barillas	}&	5.759	&	0.961	&		&	72.900	&	1.110	&		\\
		\multicolumn{1}{l}{	Aguacatán	}&	2.435	&	0.593	&		&	52.986	&	1.460	&		\\
		\rowcolor{color1!10!white} \multicolumn{1}{l}{	San Rafael Petzal	}&	2.881	&	1.020	&	(*)	&	67.663	&	2.330	&		\\
		\multicolumn{1}{l}{	San Gaspar Ixchil	}&	33.063	&	2.270	&		&	89.381	&	1.610	&		\\
		\rowcolor{color1!10!white} \multicolumn{1}{l}{	Santiago Chimaltenango	}&	18.935	&	2.780	&		&	87.934	&	1.600	&		\\
		\multicolumn{1}{l}{	Santa Ana Huista	}&	9.931	&	1.550	&		&	61.195	&	1.840	&		\\
		\rowcolor{color1!10!white} \multicolumn{1}{l}{	Union Cantinil	}&	15.919	&	2.980	&		&	74.610	&	2.420	&		\\
		\rowcolor{color1!40!white} {\Bold{	Quiché	}}&	20.15	&	0.12	&		&	76.90	&	0.17	&		\\
		\multicolumn{1}{l}{	Santa Cruz Del Quiché	}&	15.486	&	1.300	&		&	65.234	&	1.680	&		\\
		\rowcolor{color1!10!white} \multicolumn{1}{l}{	Chiché	}&	10.735	&	1.640	&		&	61.833	&	2.740	&		\\
		\multicolumn{1}{l}{	Chinique	}&	31.929	&	3.290	&		&	80.973	&	2.110	&		\\
		\rowcolor{color1!10!white} \multicolumn{1}{l}{	Zacualpa	}&	23.440	&	2.170	&		&	77.070	&	2.150	&		\\
		\multicolumn{1}{l}{	Chajul	}&	26.779	&	2.650	&		&	83.441	&	2.220	&		\\
		\rowcolor{color1!10!white} \multicolumn{1}{l}{	Chichicastenango	}&	28.438	&	2.020	&		&	82.899	&	1.340	&		\\
		\multicolumn{1}{l}{	Patzité	}&	26.524	&	3.320	&		&	82.435	&	2.460	&		\\
		\rowcolor{color1!10!white} \multicolumn{1}{l}{	San Antonio Ilotenango	}&	9.942	&	1.370	&		&	60.301	&	2.300	&		\\
		\multicolumn{1}{l}{	San Pedro Jocopilas	}&	9.598	&	1.660	&		&	55.403	&	2.320	&		\\
		\rowcolor{color1!10!white} \multicolumn{1}{l}{	Cunén	}&	9.180	&	1.630	&		&	54.995	&	2.850	&		\\
		\multicolumn{1}{l}{	San Juan Cotzal	}&	67.340	&	7.760	&		&	97.784	&	1.230	&		\\
		\rowcolor{color1!10!white} \multicolumn{1}{l}{	Joyabaj	}&	32.157	&	2.220	&		&	81.484	&	1.490	&		\\
		\multicolumn{1}{l}{	Nebaj	}&	12.110	&	1.280	&		&	68.301	&	1.980	&		\\
		\rowcolor{color1!10!white} \multicolumn{1}{l}{	San Andrés Sajcabaja	}&	21.408	&	2.150	&		&	77.746	&	1.880	&		\\
		\multicolumn{1}{l}{	Uspantán	}&	19.295	&	1.710	&		&	75.997	&	1.860	&		\\
		\rowcolor{color1!10!white} \multicolumn{1}{l}{	Sacapulas	}&	22.711	&	2.480	&		&	75.775	&	2.310	&		\\
		\multicolumn{1}{l}{	San Bartolomé Jocotenango	}&	55.862	&	3.720	&		&	95.719	&	1.050	&		\\
		\rowcolor{color1!10!white} \multicolumn{1}{l}{	Canillá	}&	17.758	&	2.150	&		&	65.005	&	2.180	&		\\
		\multicolumn{1}{l}{	Chicamán	}&	21.486	&	2.290	&		&	77.529	&	1.930	&		\\
		\rowcolor{color1!10!white} \multicolumn{1}{l}{	Ixcán	}&	16.853	&	2.300	&		&	62.698	&	1.720	&		\\
		\multicolumn{1}{l}{	Pachalum	}&	55.402	&	11.500	&		&	92.950	&	5.160	&		\\
		\rowcolor{color1!40!white} {\Bold{	Baja Verapaz	}}&	27.30	&	0.04	&		&	72.54	&	0.05	&		\\
		\multicolumn{1}{l}{	Salamá	}&	16.818	&	0.861	&		&	60.820	&	0.798	&		\\
		\rowcolor{color1!10!white} \multicolumn{1}{l}{	San Miguel Chicaj	}&	25.355	&	1.070	&		&	76.956	&	1.010	&		\\
		\multicolumn{1}{l}{	Rabinal	}&	17.268	&	1.110	&		&	68.176	&	1.030	&		\\
		\rowcolor{color1!10!white} \multicolumn{1}{l}{	Cubulco	}&	16.777	&	0.967	&		&	67.625	&	0.910	&		\\
		\multicolumn{1}{l}{	Granados	}&	16.778	&	1.080	&		&	60.826	&	0.931	&		\\
		\rowcolor{color1!10!white} \multicolumn{1}{l}{	El Chol	}&	6.538	&	0.970	&		&	44.408	&	1.350	&		\\
		\multicolumn{1}{l}{	San Jerónimo	}&	16.103	&	0.956	&		&	62.120	&	0.977	&		\\
		\rowcolor{color1!10!white} \multicolumn{1}{l}{	Purulhá	}&	71.336	&	1.780	&		&	96.508	&	0.453	&		\\
		\rowcolor{color1!40!white} {\Bold{	Alta Verapaz	}}&	46.65	&	0.33	&		&	89.58	&	0.33	&		\\
		\multicolumn{1}{l}{	Cobán	}&	25.519	&	0.746	&		&	78.972	&	0.918	&		\\
		\rowcolor{color1!10!white} \multicolumn{1}{l}{	Santa Cruz Verapaz	}&	37.132	&	1.380	&		&	81.374	&	2.170	&		\\
		\multicolumn{1}{l}{	San Cristóbal Verapaz	}&	53.592	&	1.590	&		&	87.059	&	1.900	&		\\
		\rowcolor{color1!10!white} \multicolumn{1}{l}{	Tactic	}&	10.072	&	2.680	&	(*)	&	39.672	&	1.370	&		\\
		\multicolumn{1}{l}{	Tamahú	}&	50.309	&	1.940	&		&	84.047	&	2.750	&		\\
		\rowcolor{color1!10!white} \multicolumn{1}{l}{	Tucurú	}&	65.093	&	1.400	&		&	94.704	&	0.725	&		\\
		\multicolumn{1}{l}{	Panzós	}&	75.753	&	1.530	&		&	96.802	&	0.414	&		\\
		\rowcolor{color1!10!white} \multicolumn{1}{l}{	Senahú	}&	27.949	&	1.050	&		&	85.950	&	1.030	&		\\
		\multicolumn{1}{l}{	San Pedro Carchá	}&	45.588	&	0.752	&		&	89.402	&	1.080	&		\\
		\rowcolor{color1!10!white} \multicolumn{1}{l}{	San Juan Chamelco	}&	3.078	&	0.598	&		&	40.956	&	1.450	&		\\
		\multicolumn{1}{l}{	Lanquín	}&	28.753	&	1.690	&		&	85.459	&	1.500	&		\\
		\rowcolor{color1!10!white} \multicolumn{1}{l}{	Cahabón	}&	26.285	&	1.050	&		&	80.355	&	1.200	&		\\
		\multicolumn{1}{l}{	Chisec	}&	65.453	&	1.140	&		&	97.290	&	0.442	&		\\
		\rowcolor{color1!10!white} \multicolumn{1}{l}{	Chahal	}&	11.990	&	2.140	&		&	47.894	&	1.620	&		\\
		\multicolumn{1}{l}{	Fray Bartolomé De Las Casas	}&	39.809	&	0.849	&		&	84.539	&	2.990	&		\\
		\rowcolor{color1!10!white} \multicolumn{1}{l}{	Santa Catalina La Tinta	}&	61.235	&	1.350	&		&	96.420	&	0.526	&		\\
		\multicolumn{1}{l}{	Raxruha	}&	36.678	&	3.090	&		&	86.759	&	2.070	&		\\
		\rowcolor{color1!40!white} {\Bold{	Petén	}}&	19.79	&	0.09	&		&	75.14	&	0.14	&		\\
		\multicolumn{1}{l}{	Flores	}&	11.164	&	1.870	&		&	52.230	&	4.320	&		\\
		\rowcolor{color1!10!white} \multicolumn{1}{l}{	San José	}&	31.948	&	3.910	&		&	83.737	&	2.550	&		\\
		\multicolumn{1}{l}{	San Benito	}&	16.206	&	2.070	&		&	57.989	&	3.220	&		\\
		\rowcolor{color1!10!white} \multicolumn{1}{l}{	San Andrés	}&	19.817	&	1.930	&		&	73.367	&	2.200	&		\\
		\multicolumn{1}{l}{	La Libertad	}&	13.718	&	1.150	&		&	64.563	&	1.670	&		\\
		\rowcolor{color1!10!white} \multicolumn{1}{l}{	San Francisco	}&	17.437	&	2.080	&		&	64.915	&	2.790	&		\\
		\multicolumn{1}{l}{	Santa Ana	}&	12.457	&	1.540	&		&	59.956	&	3.120	&		\\
		\rowcolor{color1!10!white} \multicolumn{1}{l}{	Dolores	}&	13.073	&	1.090	&		&	53.736	&	2.250	&		\\
		\multicolumn{1}{l}{	San Luis	}&	45.038	&	2.500	&		&	86.028	&	1.180	&		\\
		\rowcolor{color1!10!white} \multicolumn{1}{l}{	Sayaxché	}&	29.360	&	2.220	&		&	76.329	&	1.730	&		\\
		\multicolumn{1}{l}{	Melchor De Mencos	}&	11.509	&	1.340	&		&	59.038	&	2.360	&		\\
		\rowcolor{color1!10!white} \multicolumn{1}{l}{	Poptún	}&	12.488	&	1.380	&		&	49.156	&	2.150	&		\\
		\rowcolor{color1!40!white} {\Bold{	Izabal	}}&	28.90	&	0.06	&		&	69.10	&	0.08	&		\\
		\multicolumn{1}{l}{	Puerto Barrios	}&	9.224	&	1.190	&		&	43.037	&	1.560	&		\\
		\rowcolor{color1!10!white} \multicolumn{1}{l}{	Livingston	}&	53.872	&	1.270	&		&	90.061	&	0.780	&		\\
		\multicolumn{1}{l}{	El Estor	}&	19.680	&	1.850	&		&	82.395	&	1.730	&		\\
		\rowcolor{color1!10!white} \multicolumn{1}{l}{	Morales	}&	22.137	&	0.917	&		&	62.444	&	1.010	&		\\
		\multicolumn{1}{l}{	Los Amates	}&	30.017	&	1.110	&		&	75.402	&	0.980	&		\\
		\rowcolor{color1!40!white} {\Bold{	Zacapa	}}&	36.72	&	0.03	&		&	71.64	&	0.03	&		\\
		\multicolumn{1}{l}{	Zacapa	}&	37.064	&	2.710	&		&	73.039	&	2.610	&		\\
		\rowcolor{color1!10!white} \multicolumn{1}{l}{	Estanzuela	}&	17.568	&	5.540	&	(*)(+)	&	54.364	&	8.040	&	(*)(+)	\\
		\multicolumn{1}{l}{	Río Hondo	}&	12.375	&	1.910	&		&	44.596	&	3.880	&		\\
		\rowcolor{color1!10!white} \multicolumn{1}{l}{	Gualán	}&	43.135	&	2.310	&		&	78.864	&	1.990	&		\\
		\multicolumn{1}{l}{	Teculután	}&	10.892	&	2.120	&		&	42.289	&	3.600	&		\\
		\rowcolor{color1!10!white} \multicolumn{1}{l}{	Usumatlán	}&	10.776	&	2.130	&		&	40.004	&	4.270	&	(*)	\\
		\multicolumn{1}{l}{	Cabañas	}&	16.329	&	3.420	&		&	50.577	&	4.210	&		\\
		\rowcolor{color1!10!white} \multicolumn{1}{l}{	San Diego	}&	20.408	&	4.090	&		&	64.787	&	5.330	&		\\
		\multicolumn{1}{l}{	La Unión	}&	66.227	&	3.280	&		&	93.527	&	1.410	&		\\
		\rowcolor{color1!10!white} \multicolumn{1}{l}{	Huité	}&	15.984	&	3.340	&		&	49.030	&	4.970	&	(*)	\\
		\rowcolor{color1!40!white} {\Bold{	Chiquimula	}}&	37.00	&	0.07	&		&	78.98	&	0.07	&		\\
		\multicolumn{1}{l}{	Chiquimula	}&	34.878	&	3.290	&		&	86.453	&	1.480	&		\\
		\rowcolor{color1!10!white} \multicolumn{1}{l}{	San José La Arada	}&	18.877	&	2.440	&		&	61.122	&	2.470	&		\\
		\multicolumn{1}{l}{	San Juan Ermita	}&	39.641	&	2.140	&		&	83.502	&	1.050	&		\\
		\rowcolor{color1!10!white} \multicolumn{1}{l}{	Jocotán	}&	59.840	&	1.920	&		&	93.537	&	0.604	&		\\
		\multicolumn{1}{l}{	Camotán	}&	41.384	&	2.020	&		&	85.842	&	0.930	&		\\
		\rowcolor{color1!10!white} \multicolumn{1}{l}{	Olopa	}&	39.577	&	2.270	&		&	87.743	&	0.987	&		\\
		\multicolumn{1}{l}{	Esquipulas	}&	39.445	&	1.580	&		&	78.116	&	1.030	&		\\
		\rowcolor{color1!10!white} \multicolumn{1}{l}{	Concepción Las Minas	}&	8.635	&	1.210	&		&	41.273	&	2.380	&		\\
		\multicolumn{1}{l}{	Quezaltepeque	}&	27.820	&	1.740	&		&	75.434	&	1.370	&		\\
		\rowcolor{color1!10!white} \multicolumn{1}{l}{	San Jacinto	}&	31.624	&	2.980	&		&	81.051	&	1.780	&		\\
		\multicolumn{1}{l}{	Ipala	}&	8.370	&	1.330	&		&	45.756	&	2.170	&		\\
		\rowcolor{color1!40!white} {\Bold{	Jalapa	}}&	22.75	&	0.05	&		&	77.34	&	0.07	&		\\
		\multicolumn{1}{l}{	Jalapa	}&	36.283	&	1.510	&		&	89.383	&	0.907	&		\\
		\rowcolor{color1!10!white} \multicolumn{1}{l}{	San Pedro Pinula	}&	31.537	&	1.590	&		&	87.753	&	1.130	&		\\
		\multicolumn{1}{l}{	San Luis Jilotepeque	}&	24.396	&	1.700	&		&	73.690	&	2.030	&		\\
		\rowcolor{color1!10!white} \multicolumn{1}{l}{	San Manuel Chaparrón	}&	8.502	&	1.480	&		&	54.348	&	2.500	&		\\
		\multicolumn{1}{l}{	San Carlos Alzatate	}&	33.656	&	2.980	&		&	91.316	&	1.350	&		\\
		\rowcolor{color1!10!white} \multicolumn{1}{l}{	Monjas	}&	18.929	&	1.390	&		&	62.426	&	1.930	&		\\
		\multicolumn{1}{l}{	Mataquescuintla	}&	10.987	&	1.340	&		&	58.429	&	1.860	&		\\
		\rowcolor{color1!40!white} {\Bold{	Jutiapa	}}&	16.27	&	0.05	&		&	60.17	&	0.08	&		\\
		\multicolumn{1}{l}{	Jutiapa	}&	5.735	&	0.802	&		&	43.570	&	1.700	&		\\
		\rowcolor{color1!10!white} \multicolumn{1}{l}{	El Progreso	}&	8.917	&	1.350	&		&	39.793	&	2.170	&		\\
		\multicolumn{1}{l}{	Santa Catarina Mita	}&	6.288	&	1.030	&		&	38.927	&	2.050	&		\\
		\rowcolor{color1!10!white} \multicolumn{1}{l}{	Agua Blanca	}&	10.353	&	1.460	&		&	50.619	&	2.660	&		\\
		\multicolumn{1}{l}{	Asunción Mita	}&	13.031	&	1.610	&		&	55.504	&	2.090	&		\\
		\rowcolor{color1!10!white} \multicolumn{1}{l}{	Yupiltepeque	}&	10.444	&	1.690	&		&	50.282	&	2.550	&		\\
		\multicolumn{1}{l}{	Atescatempa	}&	10.103	&	1.510	&		&	52.339	&	2.660	&		\\
		\rowcolor{color1!10!white} \multicolumn{1}{l}{	Jerez	}&	7.454	&	1.570	&		&	38.393	&	3.610	&		\\
		\multicolumn{1}{l}{	El Adelanto	}&	9.580	&	2.690	&	(*)	&	54.371	&	4.620	&		\\
		\rowcolor{color1!10!white} \multicolumn{1}{l}{	Zapotitlán	}&	27.462	&	2.930	&		&	75.903	&	2.780	&		\\
		\multicolumn{1}{l}{	Comapa	}&	28.176	&	1.960	&		&	71.590	&	1.960	&		\\
		\rowcolor{color1!10!white} \multicolumn{1}{l}{	Jalpatagua	}&	27.010	&	2.400	&		&	74.604	&	1.830	&		\\
		\multicolumn{1}{l}{	Conguaco	}&	57.036	&	3.250	&		&	87.127	&	2.640	&		\\
		\rowcolor{color1!10!white} \multicolumn{1}{l}{	Moyuta	}&	53.916	&	2.460	&		&	90.851	&	0.940	&		\\
		\multicolumn{1}{l}{	Pasaco	}&	73.113	&	2.770	&		&	97.345	&	0.650	&		\\
		\rowcolor{color1!10!white} \multicolumn{1}{l}{	San José Acatempa	}&	1.468	&	0.552	&	(*)	&	22.014	&	2.330	&	(*)	\\
		\multicolumn{1}{l}{	Quezada	}&	5.375	&	1.150	&		&	46.058	&	2.360	&		\\
		\hline
		&&&&&&\\[-0.28cm]
	\end{longtable}\addtocounter{Cuadro}{1}
\end{center}




%%%%%8... (1.7)



\hoja{	{\Bold\color{color1!80!black}{Cuadro \theCuadro $\,-$  Incidencia de pobreza (personas) por departamentos. }}\\
	{\Bold\color{color1!80!black}{\normalsize República de Guatemala, año 2014.}}\\[.1cm]
	{(Porcentaje de la población)}
	\\[0.4cm]
\begin{center}\fontsize{3.8mm}{1.6em}\selectfont
	\begin{tabular}{lS[table-format=2]S[table-format=2]S[table-format=2]S[table-format=2]}
%		\multicolumn{5}{l}{\Bold\color{color1!80!black}{Cuadro \theCuadro $\,-$  Incidencia de pobreza (personas) por departamentos. }}\\
%											\multicolumn{5}{l}{\Bold\color{color1!80!black}{\normalsize República de Guatemala, año 2014.}}\\[-.1cm]
%		\multicolumn{5}{l}{(Porcentaje de la población)}
%		\\[0.4cm]
		\hline  &&&&\\[-0.55cm]
		& 		\multicolumn{3}{c}{\Bold{Pobreza}}&	\\[.05cm]\cline{2-4}
		\multicolumn{1}{x{2.4cm}}{\Bold{\raisebox{.2cm}{Departamento}}} & \multicolumn{1}{x{2.4cm}}{\Bold{Total}} & \multicolumn{1}{x{2.4cm}}{\textbf{Extrema}} & \multicolumn{1}{x{2.4cm}}{\Bold{No extrema}}& \multicolumn{1}{x{2.4cm}}{\raisebox{.2cm}{\Bold{No pobreza}}} \\[0.05cm]
		\hline
		\rowcolor{color1!40!white} 	&&&&\\[-0.35cm]
		\rowcolor{color1!40!white} {\Bold{	Total	}}&	\textbf{59.3}	&	\textbf{23.4}	&	\textbf{35.9}&	\textbf{40.7}	\\[0.05cm]
		\multicolumn{1}{l}{	Guatemala	}&	 33 	&	 5 	&	 28 	&	 67 	\\
		\rowcolor{color1!10!white} \multicolumn{1}{l}{	El Progreso	}&	 53 	&	 13 	&	 40 	&	 47 	\\
		\multicolumn{1}{l}{	Sacatepéquez	}&	 41 	&	 8 	&	 33 	&	 59 	\\
		\rowcolor{color1!10!white} \multicolumn{1}{l}{	Chimaltenango	}&	 66 	&	 23 	&	 43 	&	 34 	\\
		\multicolumn{1}{l}{	Escuintla	}&	 53 	&	 11 	&	 42 	&	 47 	\\
		\rowcolor{color1!10!white} \multicolumn{1}{l}{	Santa Rosa	}&	 54 	&	 13 	&	 41 	&	 46 	\\
		\multicolumn{1}{l}{	Sololá	}&	 81 	&	 40 	&	 41 	&	 19 	\\
		\rowcolor{color1!10!white} \multicolumn{1}{l}{	Totonicapán	}&	 77 	&	 41 	&	 36 	&	 23 	\\
		\multicolumn{1}{l}{	Quetzaltenango	}&	 56 	&	 17 	&	 39 	&	 44 	\\
		\rowcolor{color1!10!white} \multicolumn{1}{l}{	Suchitepéquez	}&	 64 	&	 20 	&	 44 	&	 36 	\\
		\multicolumn{1}{l}{	Retalhuleu	}&	 56 	&	 15 	&	 41 	&	 44 	\\
		\rowcolor{color1!10!white} \multicolumn{1}{l}{	San Marcos	}&	 60 	&	 22 	&	 38 	&	 40 	\\
		\multicolumn{1}{l}{	Huehuetenango	}&	 74 	&	 29 	&	 45 	&	 26 	\\
		\rowcolor{color1!10!white} \multicolumn{1}{l}{	Quiché	}&	 75 	&	 42 	&	 33 	&	 25 	\\
		\multicolumn{1}{l}{	Baja Verapaz	}&	 66 	&	 25 	&	 42 	&	 34 	\\
		\rowcolor{color1!10!white} \multicolumn{1}{l}{	Alta Verapaz	}&	 83 	&	 54 	&	 29 	&	 17 	\\
		\multicolumn{1}{l}{	Petén	}&	 61 	&	 20 	&	 41 	&	 39 	\\
		\rowcolor{color1!10!white} \multicolumn{1}{l}{	Izabal	}&	 60 	&	 35 	&	 25 	&	 40 	\\
		\multicolumn{1}{l}{	Zacapa	}&	 56 	&	 21 	&	 35 	&	 44 	\\
		\rowcolor{color1!10!white} \multicolumn{1}{l}{	Chiquimula	}&	 71 	&	 41 	&	 29 	&	 29 	\\
		\multicolumn{1}{l}{	Jalapa	}&	 67 	&	 22 	&	 45 	&	 33 	\\
		\rowcolor{color1!10!white} \multicolumn{1}{l}{	Jutiapa	}&	 63 	&	 24 	&	 39 	&	 37 	\\
		\hline
		&&&&\\[-0.36cm]
		\multicolumn{5}{l}{\footnotesize Fuente: INE, Encuesta Nacional de Condiciones de Vida (Encovi) 2014.}
	\end{tabular}\\[1.8cm] \addtocounter{Cuadro}{1}
\end{center}
}


\newpage
	{\Bold\Large 1.4 Trabajo}\\[.5cm]
	{\Bold\color{color1!80!black}{Cuadro \theCuadro $\,-$  Principales poblaciones del mercado laboral.}}\\
	{\Bold\color{color1!80!black}{República de Guatemala, varios años.}}\\
	{(Población de 15 o más años de edad)}	\\[-0.2cm]
\begin{center}\fontsize{3.8mm}{1.6em}\selectfont
	\begin{tabular}{lccc}
%		\multicolumn{4}{l}{\Bold\color{color1!80!black}{Cuadro \theCuadro $\,-$  Principales poblaciones del mercado laboral.}}\\
%			\multicolumn{4}{l}{\Bold\color{color1!80!black}{República de Guatemala, varios años.}}\\
%		\multicolumn{4}{l}{(Población de 15 o más años de edad)}
%		\\[0.4cm]
		\hline
		&&&\\[-0.35cm]
		\multicolumn{1}{l}{\Bold{Característica}} & \multicolumn{1}{x{2.4cm}}{\Bold{ENEI 2012}} & \multicolumn{1}{x{2.4cm}}{\Bold{ENEI 2-2013}} & \multicolumn{1}{x{2.4cm}}{\Bold{ENEI 2-2014}} \\[.1cm]
		\hline
		\rowcolor{color1!40!white}	&&&\\[-0.55cm]
		\rowcolor{color1!40!white} {\Bold{	Población en edad de trabajar	 }}& 	 9,531,370 	 & 	 9,894,951 	 & 	 10,498,289 	 \\%[.1cm]
		\multicolumn{1}{l}{\Bold{	Por sexo	}}&		 & 		 & 		 \\ 
		\multicolumn{1}{l}{	Hombre	}&	 4,479,049 	 & 	 4,664,729 	 & 	 4,965,724 	 \\ 
		\multicolumn{1}{l}{	Mujer	}&	 5,052,321 	 & 	 5,230,222 	 & 	 5,532,565 	 \\ 
		\rowcolor{color1!10!white} \multicolumn{1}{l}{{\Bold{	Por dominio de estudio	}}}&		 & 		 & 		 \\ 
		\rowcolor{color1!10!white} \multicolumn{1}{l}{	Urbano Metropolitano	}&	 1,925,399 	 & 	 1,986,084 	 & 	 2,052,166 	 \\ 
		\rowcolor{color1!10!white} \multicolumn{1}{l}{	Resto Urbano	}&	 2,990,735 	 & 	 3,133,478 	 & 	 3,518,609 	 \\ 
		\rowcolor{color1!10!white} \multicolumn{1}{l}{	Rural nacional	}&	 4,615,236 	 & 	 4,775,389 	 & 	 4,927,514 	 \\ 
			\rowcolor{color1!40!white}	&&&\\[-0.55cm]
		\rowcolor{color1!40!white} {\Bold{	Población económicamente activa	 }}& 	 6,235,064 	 & 	 5,990,436 	 & 	 6,316,005 	 \\
		\multicolumn{1}{l}{\Bold{	Por sexo	}}&		 & 		 & 		 \\ 
		\multicolumn{1}{l}{	Hombre	}&	 3,924,339 	 & 	 3,868,166 	 & 	 4,107,605 	 \\ 
		\multicolumn{1}{l}{	Mujer	}&	 2,310,725 	 & 	 2,122,270 	 & 	 2,208,400 	 \\ 
		\rowcolor{color1!10!white} \multicolumn{1}{l}{\Bold{	Por dominio de estudio	}}&		 & 		 & 		 \\ 
		\rowcolor{color1!10!white} \multicolumn{1}{l}{	Urbano Metropolitano	}&	 1,240,741 	 & 	 1,311,235 	 & 	 1,344,195 	 \\ 
		\rowcolor{color1!10!white} \multicolumn{1}{l}{	Resto Urbano	}&	 1,976,966 	 & 	 1,881,020 	 & 	 2,116,184 	 \\ 
		\rowcolor{color1!10!white} \multicolumn{1}{l}{	Rural nacional	}&	 3,017,357 	 & 	 2,798,181 	 & 	 2,855,626 	 \\ 
			\rowcolor{color1!40!white}	&&&\\[-0.55cm]
		\rowcolor{color1!40!white} {\Bold{	Población ocupada	 }}& 	 6,055,826 	 & 	 5,811,193 	 & 	 6,131,995 	 \\ %[.1cm]
		\multicolumn{1}{l}{\Bold{	Por sexo	}}&		 & 		 & 		 \\ 
		\multicolumn{1}{l}{	Hombre	}&	 3,829,175 	 & 	 3,750,099 	 & 	 3,995,886 	 \\ 
		\multicolumn{1}{l}{	Mujer	}&	 2,226,651 	 & 	 2,061,094 	 & 	 2,136,109 	 \\ 
		\rowcolor{color1!10!white} \multicolumn{1}{l}{\Bold{	Por dominio de estudio	}}&		 & 		 & 		 \\ 
		\rowcolor{color1!10!white} \multicolumn{1}{l}{	Urbano Metropolitano	}&	 1,157,608 	 & 	 1,243,210 	 & 	 1,281,013 	 \\ 
		\rowcolor{color1!10!white} \multicolumn{1}{l}{	Resto Urbano	}&	 1,930,524 	 & 	 1,829,287 	 & 	 2,047,652 	 \\ 
		\rowcolor{color1!10!white} \multicolumn{1}{l}{	Rural nacional	}&	 2,967,694 	 & 	 2,738,696 	 & 	 2,803,330 	 \\ 
		\hline
		&&&\\[-0.36cm]
		\multicolumn{4}{l}{\footnotesize  INE, Encuesta Nacional de Empleo e Ingresos (Enei). }
	\end{tabular}\\[1.8cm] \addtocounter{Cuadro}{1}
\end{center}

%		
		
\addtocounter{Cuadro}{1}

\begin{center}
	\begin{tabular}{lS[table-format=8]S[table-format=8]S[table-format=8]}
		\multicolumn{4}{l}{\Bold\color{color1!80!black}{Cuadro \theCuadro $\,-$   Maíz (Zea mays) por área cosechada, producción y rendimiento}}\\
		\multicolumn{4}{l}{\Bold\color{color1!80!black}{según año agrícola. República de Guatemala, años varios.}}\\
		%		\multicolumn{4}{l}{(Población de 15 o más años de edad)}
		\\[0.4cm]
		\hline &&&\\[-0.36cm]  
		\multicolumn{1}{x{2.7cm}}{ } &	\multicolumn{3}{c}{\Bold{Maíz en grano}}\\[0.05cm]\cline{2-4}
		\multicolumn{1}{x{2.7cm}}{\Bold{Año agrícola 1/}} &	\multicolumn{1}{x{2.7cm}}{\Bold{Área cosechada}} & \multicolumn{1}{x{2.7cm}}{\Bold{Producción }} & \multicolumn{1}{x{2.4cm}}{\Bold{Rendimiento}}\\[0.05cm]
		\multicolumn{1}{x{2.7cm}}{} &	\multicolumn{1}{x{2.7cm}}{\Bold{(manzanas)}} & \multicolumn{1}{x{2.7cm}}{\Bold{(quintales)}} & \multicolumn{1}{x{2.4cm}}{\Bold{(qq/mz)}}\\[0.05cm]
		\hline
		\rowcolor{color1!10!white}	&&&\\[-0.35cm]
		\rowcolor{color1!10!white}2009/2010	&	1,174,955	&	35,842,974	&	30.5	\\[0.05cm]
		2010/2011	&	1,175,255	&	36,117,212	&	30.7	\\[0.05cm]
		\rowcolor{color1!10!white}2011/2012	&	1,199,900	&	36,932,600	&	30.8	\\[0.05cm]
		2012/2013	&	1,211,900	&	37,995,900	&	31.4	\\[0.05cm]
		\rowcolor{color1!10!white}2013/2014 p/	&	1,233,300	&	39,576,500	&	32.1	\\[0.05cm]
		2014/2015 e/	&	1,247,100	&	40,724,100	&	32.7	\\
		
		\hline
		&&&\\[-0.36cm]
		\multicolumn{4}{l}{\footnotesize Fuente: Diplan-MAGA con datos de Banguat (MAGA, 2013).}\\
		\multicolumn{4}{l}{\footnotesize 1/ De mayo de un año a abril del siguiente.}\\
		\multicolumn{4}{l}{\footnotesize p/ Cifras preliminares.  e/ Cifras estimadas.}\\	
		%		\multicolumn{4}{m{15cm}}{\footnotesize	Ministerio de Agricultura, Ganadería y Alimentación (MAGA). 2013. El agro en cifras 2015. En: \url{http://web.maga.gob.gt/download/1agro-cifras2014.pdf}  (Consultado: febrero 2016).}
	\end{tabular}\addtocounter{Cuadro}{1}
\end{center}
{\footnotesize	Ministerio de Agricultura, Ganadería y Alimentación (MAGA). 2013. El agro en cifras 2015. En: \url{http://web.maga.gob.gt/download/1agro-cifras2014.pdf}  (Consultado: febrero 2016).}




%%% 2-2



\begin{center}
	\begin{tabular}{lS[table-format=8]S[table-format=8]S[table-format=3]}
		\multicolumn{4}{l}{\Bold\color{color1!80!black}{Cuadro \theCuadro $\,-$   Frijol (Phaseolus vulgaris) por área cosechada, producción y rendimiento}}\\
		\multicolumn{4}{l}{\Bold\color{color1!80!black}{según año agrícola. República de Guatemala, años varios.}}\\
		%		\multicolumn{4}{l}{(Población de 15 o más años de edad)}
		\\[0.4cm]
		\hline &&&\\[-0.36cm]  
		\multicolumn{1}{x{2.7cm}}{ } &	\multicolumn{3}{c}{\Bold{Frijol}}\\[0.05cm]\cline{2-4}
		\multicolumn{1}{x{2.7cm}}{\Bold{Año agrícola 1/}} &	\multicolumn{1}{x{2.7cm}}{\Bold{Área cosechada}} & \multicolumn{1}{x{2.7cm}}{\Bold{Producción }} & \multicolumn{1}{x{2.4cm}}{\Bold{Rendimiento}}\\[0.05cm]
		\multicolumn{1}{x{2.7cm}}{} &	\multicolumn{1}{x{2.7cm}}{\Bold{(manzanas)}} & \multicolumn{1}{x{2.7cm}}{\Bold{(quintales)}} & \multicolumn{1}{x{2.4cm}}{\Bold{(qq/mz)}}\\[0.05cm]
		\hline
		\rowcolor{color1!10!white}	&&&\\[-0.35cm]
		\rowcolor{color1!10!white}	2009/2010	&	336,500	&	4,367,660	&	13.0	\\[0.05cm]
		2010/2011	&	336,756	&	4,610,828	&	13.7	\\[0.05cm]
		\rowcolor{color1!10!white}	2011/2012	&	339,200	&	4,704,200	&	13.9	\\[0.05cm]
		2012/2013	&	345,400	&	4,845,500	&	14.0	\\[0.05cm]
		\rowcolor{color1!10!white}	2013/2014 p/	&	352,500	&	5,026,200	&	14.3	\\[0.05cm]
		2014/2015 e/	&	358,300	&	5,181,500	&	14.5	\\[0.05cm]
		\hline
		&&&\\[-0.36cm]
		\multicolumn{4}{l}{\footnotesize Fuente: Diplan-MAGA con datos de Banguat (MAGA, 2013).}\\
		\multicolumn{4}{l}{\footnotesize 1/ De mayo de un año a abril del siguiente.}\\
		\multicolumn{4}{l}{\footnotesize p/ Cifras preliminares.  e/ Cifras estimadas.}\\	
		%					\multicolumn{4}{m{15cm}}{\footnotesize	Ministerio de Agricultura, Ganadería y Alimentación (MAGA). 2013. El agro en cifras 2015. En: \url{http://web.maga.gob.gt/download/1agro-cifras2014.pdf}  (Consultado: febrero 2016).}
	\end{tabular}\addtocounter{Cuadro}{1}
\end{center}
{\footnotesize	Ministerio de Agricultura, Ganadería y Alimentación (MAGA). 2013. El agro en cifras 2015. En: \url{http://web.maga.gob.gt/download/1agro-cifras2014.pdf}  (Consultado: febrero 2016).}






%%%%%  2-3
\hoja{
	{\Bold\color{color1!80!black}{Cuadro \theCuadro $\,-$    Arroz (Oryza sativa), por área cosechada, producción y rendimiento}}\\
	{\Bold\color{color1!80!black}{según año agrícola. República de Guatemala, años varios.}}\\
	\begin{center}
		\begin{tabular}{lccc}
			%		\multicolumn{4}{l}{(Población de 15 o más años de edad)}
			\hline &&&\\[-0.36cm]  
			\multicolumn{1}{r}{ } &	\multicolumn{3}{c}{\Bold{Arroz}}\\%[0.05cm]
			&&&\\[-0.3cm]\cline{2-4}
			&&&\\[-0.15cm]
			\multicolumn{1}{c}{\Bold{Año agrícola 1/}} &	\multicolumn{1}{c}{\Bold{Área cosechada}} & \multicolumn{1}{c}{\Bold{Producción 2/}} & \multicolumn{1}{c}{\Bold{Rendimiento}}\\[0.05cm]
			\multicolumn{1}{c}{} &	\multicolumn{1}{c}{\Bold{(manzanas)}} & \multicolumn{1}{c}{\Bold{(quintales)}} & \multicolumn{1}{c}{\Bold{(qq/mz)}}\\[0.05cm]
			\hline
			\rowcolor{color1!10!white}	&&&\\[-0.35cm]
			\rowcolor{color1!10!white}	2009/2010	&	14,200	&	632,765	&	44.6	\\[0.05cm]
			2010/2011	&	15,012	&	653,140	&	43.5	\\[0.05cm]
			\rowcolor{color1!10!white}	2011/2012	&	15,200	&	670,300	&	44.1	\\[0.05cm]
			2012/2013	&	15,400	&	686,400	&	44.6	\\[0.05cm]
			\rowcolor{color1!10!white}	2013/2014 p/	&	15,700	&	710,900	&	45.3	\\[0.05cm]
			2014/2015 e/	&	16,000	&	732,900	&	45.8	\\[0.05cm]
			\hline
			&&&\\[-0.36cm]
			\multicolumn{4}{l}{\footnotesize Fuente: Diplan-MAGA con datos de Banguat (MAGA, 2013).}\\
			\multicolumn{4}{l}{\footnotesize 1/ De mayo de un año a abril del siguiente.}\\
			\multicolumn{4}{l}{\footnotesize 2/ Se refiere al grano en granza.}\\
			\multicolumn{4}{l}{\footnotesize p/ Cifras preliminares.  e/ Cifras estimadas.}\\	
			%					\multicolumn{4}{m{15cm}}{\footnotesize	Ministerio de Agricultura, Ganadería y Alimentación (MAGA). 2013. El agro en cifras 2015. En: \url{http://web.maga.gob.gt/download/1agro-cifras2014.pdf}  (Consultado: febrero 2016).}
		\end{tabular}\addtocounter{Cuadro}{1}
	\end{center}
	{\footnotesize	Ministerio de Agricultura, Ganadería y Alimentación (MAGA). 2013. El agro en cifras 2015. En:}\\ {\footnotesize\url{http://web.maga.gob.gt/download/1agro-cifras2014.pdf}} 
	{\footnotesize (Consultado: febrero 2016).}




	\begin{center}
		%%%%%%%%%%%
		$\ $\\[1cm]
		{\Bold\color{color1!80!black}{Cuadro \theCuadro $\,-$    Trigo (Triticum spp.), por área cosechada, producción y rendimiento}}\\
		{\Bold\color{color1!80!black}{según año agrícola. República de Guatemala, años varios.}}\\
		$\ $\\[-1cm]
		\begin{tabular}{lccc}
			&&&\\[0.6cm]
			\hline &&&\\[-0.36cm]  
			\multicolumn{1}{x{2.7cm}}{ } &	\multicolumn{3}{c}{\Bold{Trigo}}\\[0.01cm]\cline{2-4}
			\multicolumn{1}{x{2.7cm}}{\Bold{Año agrícola 1/}} &	\multicolumn{1}{x{2.7cm}}{\Bold{Área cosechada}} & \multicolumn{1}{x{2.7cm}}{\Bold{Producción /2}} & \multicolumn{1}{x{2.4cm}}{\Bold{Rendimiento}}\\[0.05cm]
			\multicolumn{1}{x{2.7cm}}{} &	\multicolumn{1}{x{2.7cm}}{\Bold{(manzanas)}} & \multicolumn{1}{x{2.7cm}}{\Bold{(quintales)}} & \multicolumn{1}{x{2.4cm}}{\Bold{(qq/mz)}}\\[0.05cm]
			\hline
			\rowcolor{color1!10!white}	&&&\\[-0.35cm]
			\rowcolor{color1!10!white}	2008/2009	&	1,005	&	35,633	&	35.5	\\[0.05cm]
			2009/2010	&	975	&	34,125	&	35.0	\\[0.05cm]
			\rowcolor{color1!10!white}	2010/2011	&	968	&	31,681	&	32.7	\\[0.05cm]
			2011/2012	&	1,000	&	31,600	&	31.6	\\[0.05cm]
			\rowcolor{color1!10!white}	2012/2013	&	1,000	&	33,400	&	33.4	\\[0.05cm]
			2013/2014 p/	&	1,100	&	34,300	&	31.2	\\[0.05cm]
			\rowcolor{color1!10!white}	2014/2015 e/	&	1,100	&	35,800	&	32.5	\\[0.05cm]
			\hline
			&&&\\[-0.36cm]
			\multicolumn{4}{l}{\footnotesize Fuente: Diplan-MAGA con datos de Banguat (MAGA, 2013).}\\
			\multicolumn{4}{l}{\footnotesize 1/ De mayo de un año a abril del siguiente.}\\
			%			\multicolumn{4}{l}{\footnotesize 2/ Se refiere al grano en granza.}\\
			\multicolumn{4}{l}{\footnotesize p/ Cifras preliminares.  e/ Cifras estimadas.}\\	
			%					\multicolumn{4}{m{15cm}}{\footnotesize	Ministerio de Agricultura, Ganadería y Alimentación (MAGA). 2013. El agro en cifras 2015. En: \url{http://web.maga.gob.gt/download/1agro-cifras2014.pdf}  (Consultado: febrero 2016).}
			\multicolumn{4}{l}{\footnotesize	Ministerio de Agricultura, Ganadería y Alimentación (MAGA). 2013. El agro en cifras 2015. En:}\\
			\multicolumn{4}{l}{\footnotesize \url{http://web.maga.gob.gt/download/1agro-cifras2014.pdf}}\\
			\multicolumn{4}{l}{\footnotesize  (Consultado: febrero 2016).}
		\end{tabular}\addtocounter{Cuadro}{1}
	\end{center} }
	
	\hoja{
		\begin{center}
			\begin{tabular}{lS[table-format=8]S[table-format=8]S[table-format=8]}
				\multicolumn{4}{l}{\Bold\color{color1!80!black}{Cuadro \theCuadro $\,-$    Ajonjolí (Sesamum indicum), por área cosechada, producción y rendimiento}}\\
				\multicolumn{4}{l}{\Bold\color{color1!80!black}{según año agrícola. República de Guatemala, años varios.}}\\
				%		\multicolumn{4}{l}{(Población de 15 o más años de edad)}
				%[0.4cm]
				\hline &&&\\[-0.36cm]  
				\multicolumn{1}{x{2.7cm}}{ } &	\multicolumn{3}{c}{\Bold{Ajonjolí}}\\[0.05cm]\cline{2-4}
				\multicolumn{1}{x{2.7cm}}{\Bold{Año agrícola 1/}} &	\multicolumn{1}{x{2.7cm}}{\Bold{Área cosechada}} & \multicolumn{1}{x{2.7cm}}{\Bold{Producción 2/}} & \multicolumn{1}{x{2.4cm}}{\Bold{Rendimiento}}\\[0.05cm]
				\multicolumn{1}{x{2.7cm}}{} &	\multicolumn{1}{x{2.7cm}}{\Bold{(manzanas)}} & \multicolumn{1}{x{2.7cm}}{\Bold{(quintales)}} & \multicolumn{1}{x{2.4cm}}{\Bold{(qq/mz)}}\\[0.05cm]
				\hline
				\rowcolor{color1!10!white}	&&&\\[-0.35cm]
				\rowcolor{color1!10!white}	2008/2009	&	48,900	&	838,000	&	17.1	\\[0.05cm]
				2009/2010	&	50,378	&	1,099,422	&	21.8	\\[0.05cm]
				\rowcolor{color1!10!white}	2010/2011	&	43,200	&	870,100	&	20.1	\\[0.05cm]
				2011/2012	&	54,000	&	1,077,600	&	20.0	\\[0.05cm]
				\rowcolor{color1!10!white}	2012/2013 p/	&	56,200	&	1,304,500	&	23.2	\\[0.05cm]
				2013/2014 e/	&	54,900	&	1,119,800	&	20.4	\\[0.05cm]
				\hline
				&&&\\[-0.36cm]
				\multicolumn{4}{l}{\footnotesize Fuente: Diplan-MAGA con datos de Banguat (MAGA, 2013).}\\
				\multicolumn{4}{l}{\footnotesize 1/ De octubre de un año a septiembre del siguiente.}\\
				%			\multicolumn{4}{l}{\footnotesize 2/ Se refiere al grano en granza.}\\
				\multicolumn{4}{l}{\footnotesize p/ Cifras preliminares.  e/ Cifras estimadas.}\\	
				%					\multicolumn{4}{m{15cm}}{\footnotesize	Ministerio de Agricultura, Ganadería y Alimentación (MAGA). 2013. El agro en cifras 2015. En: \url{http://web.maga.gob.gt/download/1agro-cifras2014.pdf}  (Consultado: febrero 2016).}
			\end{tabular}\addtocounter{Cuadro}{1}
		\end{center}
		{\footnotesize	Ministerio de Agricultura, Ganadería y Alimentación (MAGA). 2013. El agro en cifras 2015. En: \url{http://web.maga.gob.gt/download/1agro-cifras2014.pdf}  (Consultado: febrero 2016).}
		
		
		%%%%%%%%%%%%%%%%%%
		$\ $\\[1cm]
		
		{\Bold\color{color1!80!black}{Cuadro \theCuadro $\,-$    Comercio exterior de maíz blanco}}\\
		{\Bold\color{color1!80!black}{según año. República de Guatemala,  2005 - 2014.}}\\
		{(En toneladas métricas.)}\\[-.5cm]
		
		\begin{center}
			\begin{tabular}{cS[table-format=8]S[table-format=8]}
				
				%[0.4cm]
				\hline &&\\[-0.3cm]  
				\multicolumn{1}{x{2.7cm}}{ } &	\multicolumn{2}{c}{\Bold{Maíz blanco}}\\[0.05cm]\cline{2-3}
				\multicolumn{1}{x{2.7cm}}{\Bold{\raisebox{.4cm}{Año}}}&\multicolumn{1}{x{2.7cm}}{\Bold{Exportaciones}} &	\multicolumn{1}{x{2.7cm}}{\Bold{Importaciones}} \\[0.05cm]
				\hline
				\rowcolor{color1!10!white}	&&\\[-0.35cm]
				\rowcolor{color1!10!white}	2005	&	457.09	&	78,206.93	\\[0.05cm]
				2006	&	8.78	&	80,426.15	\\[0.05cm]
				\rowcolor{color1!10!white}	2007	&	4,094.29	&	58,143.62	\\[0.05cm]
				2008	&	11,977.73	&	19,558.90	\\[0.05cm]
				\rowcolor{color1!10!white}	2009	&	2,153.46	&	39,092.91	\\[0.05cm]
				2010	&	2,127.54	&	24,745.31	\\[0.05cm]
				\rowcolor{color1!10!white}	2011	&	14,164.00	&	41,547.83	\\[0.05cm]
				2012	&	2,568.63	&	36,393.62	\\[0.05cm]
				\rowcolor{color1!10!white}	2013	&	8,214.93	&	18,422.06	\\[0.05cm]
				2014*	&	1,994.95	&	23,081.27	\\[0.05cm]
				\hline
				&&\\[-0.36cm]
				\multicolumn{3}{l}{\footnotesize Fuente: Diplan-MAGA con datos de Banguat (MAGA, 2013).}\\
				\multicolumn{3}{l}{\footnotesize */ Cifras al mes de agosto.}\\
			\end{tabular}\addtocounter{Cuadro}{1}
		\end{center}
		{\footnotesize	Ministerio de Agricultura, Ganadería y Alimentación (MAGA). 2013. El agro en cifras 2015. En: \url{http://web.maga.gob.gt/download/1agro-cifras2014.pdf}  (Consultado: febrero 2016).}}
	
	
	
	
	
	\begin{center}
		\begin{tabular}{cS[table-format=8]S[table-format=8]S[table-format=8]}
			\multicolumn{4}{l}{\Bold\color{color1!80!black}{Cuadro \theCuadro $\,-$    Comercio exterior de maíz amarillo}}\\
			\multicolumn{4}{l}{\Bold\color{color1!80!black}{según año. República de Guatemala,  2005 - 2014.}}\\
			\multicolumn{4}{l}{(En toneladas métricas.)}\\
			%[0.4cm]
			\hline &&&\\[-0.36cm]  
			\multicolumn{1}{x{2.7cm}}{ } &	\multicolumn{3}{c}{\Bold{Maíz amarillo}}\\[0.05cm]\cline{2-4}
			\multicolumn{1}{x{2.7cm}}{\Bold{\raisebox{.4cm}{Año}}}&\multicolumn{1}{x{2.7cm}}{\Bold{Exportaciones}} &	\multicolumn{1}{x{2.7cm}}{\Bold{Importaciones}} &	\multicolumn{1}{x{2.7cm}}{\Bold{Balanza}} \\[0.05cm]
			\hline
			\rowcolor{color1!10!white}&	&&\\[-0.35cm]
			\rowcolor{color1!10!white}	2005	&	12.9	&	585,177.2	&	-585,164.4	\\[0.05cm]
			2006	&	0.0	&	686,018.6	&	-686,018.5	\\[0.05cm]
			\rowcolor{color1!10!white}	2007	&	62.1	&	641,780.6	&	-641,718.5	\\[0.05cm]
			2008	&	0.8	&	574,103.6	&	-574,102.8	\\[0.05cm]
			\rowcolor{color1!10!white}	2009	&	2,437.2	&	618,840.8	&	-616,403.6	\\[0.05cm]
			2010	&	782.3	&	602,003.1	&	-601,220.8	\\[0.05cm]
			\rowcolor{color1!10!white}	2011	&	0.0	&	666,495.0	&	-666,495.0	\\[0.05cm]
			2012	&	2.5	&	652,455.8	&	-652,453.3	\\[0.05cm]
			\rowcolor{color1!10!white}	2013	&	4.1	&	667,311.8	&	-667,307.8	\\[0.05cm]
			2014*	&	25.2	&	519,406.2	&	-519,381.0	\\[0.05cm]
			
			\hline
			&&&\\[-0.36cm]
			\multicolumn{4}{l}{\footnotesize Fuente: Diplan-MAGA con datos de Banguat (MAGA, 2013).}\\
			\multicolumn{4}{l}{\footnotesize */ Cifras al mes de agosto.}\\
		\end{tabular}\addtocounter{Cuadro}{1}
	\end{center}
	{\footnotesize	Ministerio de Agricultura, Ganadería y Alimentación (MAGA). 2013. El agro en cifras 2015. En: \url{http://web.maga.gob.gt/download/1agro-cifras2014.pdf}  (Consultado: febrero 2016).}
	
	
	
	
	
	
	
	\begin{center}
		\begin{tabular}{cS[table-format=8]S[table-format=8]S[table-format=8]}
			\multicolumn{4}{l}{\Bold\color{color1!80!black}{Cuadro \theCuadro $\,-$    Comercio exterior de frijol (Phaseolus vulgaris)}}\\
			\multicolumn{4}{l}{\Bold\color{color1!80!black}{según año. República de Guatemala,  2005 - 2014.}}\\
			\multicolumn{4}{l}{(En toneladas métricas.)}\\
			%[0.4cm]
			\hline &&&\\[-0.36cm]  
			\multicolumn{1}{x{2.7cm}}{ } &	\multicolumn{3}{c}{\Bold{Frijol}}\\[0.05cm]\cline{2-4}
			\multicolumn{1}{x{2.7cm}}{\Bold{\raisebox{.4cm}{Año}}}&\multicolumn{1}{x{2.7cm}}{\Bold{Exportaciones}} &	\multicolumn{1}{x{2.7cm}}{\Bold{Importaciones}} &	\multicolumn{1}{x{2.7cm}}{\Bold{Balanza}} \\[0.05cm]
			\hline
			\rowcolor{color1!10!white}&	&&\\[-0.35cm]
			\rowcolor{color1!10!white}	2005	&	1,093.8	&	4,778.9	&	-3,685.1	\\[0.05cm]
			2006	&	55.8	&	11,547.6	&	-11,491.8	\\[0.05cm]
			\rowcolor{color1!10!white}	2007	&	2,306.1	&	7,975.8	&	-5,669.7	\\[0.05cm]
			2008	&	2,187.1	&	5,148.7	&	-2,961.6	\\[0.05cm]
			\rowcolor{color1!10!white}	2009	&	478.1	&	8,134.5	&	-7,656.4	\\[0.05cm]
			2010	&	1,246.3	&	11,913.9	&	-10,667.6	\\[0.05cm]
			\rowcolor{color1!10!white}	2011	&	1,568.0	&	20,531.0	&	-18,963.0	\\[0.05cm]
			2012	&	216.9	&	10,980.1	&	-10,763.2	\\[0.05cm]
			\rowcolor{color1!10!white}	2013	&	1,508.1	&	6,414.1	&	-4,906.1	\\[0.05cm]
			2014*	&	2,222.7	&	3,839.2	&	-1,616.5	\\[0.05cm]
			\hline
			&&&\\[-0.36cm]
			\multicolumn{4}{l}{\footnotesize Fuente: Diplan-MAGA con datos de Banguat (MAGA, 2013).}\\
			\multicolumn{4}{l}{\footnotesize */ Cifras al mes de agosto.}\\
		\end{tabular}\addtocounter{Cuadro}{1}
	\end{center}
	{\footnotesize	Ministerio de Agricultura, Ganadería y Alimentación (MAGA). 2013. El agro en cifras 2015. En: \url{http://web.maga.gob.gt/download/1agro-cifras2014.pdf}  (Consultado: febrero 2016).}
	
	
	
	
	
	
	
	\begin{center}
		\begin{tabular}{cS[table-format=8]S[table-format=8]S[table-format=8]}
			\multicolumn{4}{l}{\Bold\color{color1!80!black}{Cuadro \theCuadro $\,-$    Comercio exterior de arroz (Oriza sativa)}}\\
			\multicolumn{4}{l}{\Bold\color{color1!80!black}{según año. República de Guatemala,  2005 - 2014.}}\\
			\multicolumn{4}{l}{(En toneladas métricas.)}\\
			%[0.4cm]
			\hline &&&\\[-0.36cm]  
			\multicolumn{1}{x{2.7cm}}{ } &	\multicolumn{3}{c}{\Bold{Arroz}}\\[0.05cm]\cline{2-4}
			\multicolumn{1}{x{2.7cm}}{\Bold{\raisebox{.4cm}{Año}}}&\multicolumn{1}{x{2.7cm}}{\Bold{Exportaciones}} &	\multicolumn{1}{x{2.7cm}}{\Bold{Importaciones}} &	\multicolumn{1}{x{2.7cm}}{\Bold{Balanza}} \\[0.05cm]
			\hline
			\rowcolor{color1!10!white}&	&&\\[-0.35cm]
			\rowcolor{color1!10!white}	2005	&	3,204.51	&	91,283.85	&	-88,079.34	\\[0.05cm]
			2006	&	4,720.84	&	108,190.15	&	-103,469.31	\\[0.05cm]
			\rowcolor{color1!10!white}	2007	&	6,351.59	&	96,218.06	&	-89,866.47	\\[0.05cm]
			2008	&	5,338.60	&	88,951.71	&	-83,613.11	\\[0.05cm]
			\rowcolor{color1!10!white}	2009	&	4,112.57	&	83,050.01	&	-78,937.44	\\[0.05cm]
			2010	&	2,445.67	&	71,041.94	&	-68,596.27	\\[0.05cm]
			\rowcolor{color1!10!white}	2011	&	1,472.00	&	77,464.00	&	-75,992.00	\\[0.05cm]
			2012	&	2,261.72	&	101,424.27	&	-99,162.55	\\[0.05cm]
			\rowcolor{color1!10!white}	2013	&	490.57	&	97,845.52	&	-97,354.95	\\[0.05cm]
			2014*	&	55.76	&	64,289.78	&	-64,234.02	\\[0.05cm]
			\hline
			&&&\\[-0.36cm]
			\multicolumn{4}{l}{\footnotesize Fuente: Diplan-MAGA con datos de Banguat (MAGA, 2013).}\\
			\multicolumn{4}{l}{\footnotesize */ Cifras al mes de agosto.}\\
		\end{tabular}\addtocounter{Cuadro}{1}
	\end{center}
	{\footnotesize	Ministerio de Agricultura, Ganadería y Alimentación (MAGA). 2013. El agro en cifras 2015. En: \url{http://web.maga.gob.gt/download/1agro-cifras2014.pdf}  (Consultado: febrero 2016).}
	
	
	
	
	
	
	\begin{center}
		\begin{tabular}{cS[table-format=8]S[table-format=8]S[table-format=8]}
			\multicolumn{4}{l}{\Bold\color{color1!80!black}{Cuadro \theCuadro $\,-$    Comercio exterior de trigo (Triticum spp.)}}\\
			\multicolumn{4}{l}{\Bold\color{color1!80!black}{según año. República de Guatemala,  2005 - 2014.}}\\
			\multicolumn{4}{l}{(En toneladas métricas.)}\\
			%[0.4cm]
			\hline &&&\\[-0.36cm]  
			\multicolumn{1}{x{2.7cm}}{ } &	\multicolumn{3}{c}{\Bold{Trigo}}\\[0.05cm]\cline{2-4}
			\multicolumn{1}{x{2.7cm}}{\Bold{\raisebox{.4cm}{Año}}}&\multicolumn{1}{x{2.7cm}}{\Bold{Exportaciones}} &	\multicolumn{1}{x{2.7cm}}{\Bold{Importaciones}} &	\multicolumn{1}{x{2.7cm}}{\Bold{Balanza}} \\[0.05cm]
			\hline
			\rowcolor{color1!10!white}&	&&\\[-0.35cm]
			\rowcolor{color1!10!white}	2005	&	2,041.0	&	487,422.7	&	-485,381.7	\\[0.05cm]
			2006	&	262.3	&	450,196.7	&	-449,934.4	\\[0.05cm]
			\rowcolor{color1!10!white}	2007	&	1,774.0	&	493,626.5	&	-491,852.5	\\[0.05cm]
			2008	&	1,709.8	&	473,767.1	&	-472,057.3	\\[0.05cm]
			\rowcolor{color1!10!white}	2009	&	215.7	&	444,051.5	&	-443,835.8	\\[0.05cm]
			2010	&	3.0	&	492,354.0	&	-492,351.0	\\[0.05cm]
			\rowcolor{color1!10!white}	2011	&	201.0	&	516,907.0	&	-516,706.0	\\[0.05cm]
			2012	&	252.9	&	514,445.6	&	-514,192.7	\\[0.05cm]
			\rowcolor{color1!10!white}	2013	&	1,092.6	&	462,758.5	&	-461,665.9	\\[0.05cm]
			2014*	&	706.0	&	364,685.9	&	-363,980.0	\\[0.05cm]
			\hline
			&&&\\[-0.36cm]
			\multicolumn{4}{l}{\footnotesize Fuente: Diplan-MAGA con datos de Banguat (MAGA, 2013).}\\
			\multicolumn{4}{l}{\footnotesize */ Cifras al mes de agosto.}\\
		\end{tabular}\addtocounter{Cuadro}{1}
	\end{center}
	{\footnotesize	Ministerio de Agricultura, Ganadería y Alimentación (MAGA). 2013. El agro en cifras 2015. En: \url{http://web.maga.gob.gt/download/1agro-cifras2014.pdf}  (Consultado: febrero 2016).}
	
	
	
	
	
	
	\begin{center}
		\begin{tabular}{cS[table-format=8]S[table-format=8]S[table-format=8]}
			\multicolumn{4}{l}{\Bold\color{color1!80!black}{Cuadro \theCuadro $\,-$    Comercio exterior de ajonjolí (Sesamum indicum)}}\\
			\multicolumn{4}{l}{\Bold\color{color1!80!black}{según año. República de Guatemala,  2005 - 2014.}}\\
			\multicolumn{4}{l}{(En toneladas métricas.)}\\
			%[0.4cm]
			\hline &&&\\[-0.36cm]  
			\multicolumn{1}{x{2.7cm}}{ } &	\multicolumn{3}{c}{\Bold{Ajonjolí}}\\[0.05cm]\cline{2-4}
			\multicolumn{1}{x{2.7cm}}{\Bold{\raisebox{.4cm}{Año}}}&\multicolumn{1}{x{2.7cm}}{\Bold{Exportaciones}} &	\multicolumn{1}{x{2.7cm}}{\Bold{Importaciones}} &	\multicolumn{1}{x{2.7cm}}{\Bold{Balanza}} \\[0.05cm]
			\hline
			\rowcolor{color1!10!white}&	&&\\[-0.35cm]
			\rowcolor{color1!10!white}	2005	&	28,295.3	&	14,523.6	&	13,771.7	\\[0.05cm]
			2006	&	22,306.7	&	11,313.0	&	10,993.7	\\[0.05cm]
			\rowcolor{color1!10!white}	2007	&	26,652.3	&	6,642.2	&	20,010.1	\\[0.05cm]
			2008	&	13,344.6	&	9,754.8	&	3,589.8	\\[0.05cm]
			\rowcolor{color1!10!white}	2009	&	19,261.9	&	9,591.2	&	9,670.6	\\[0.05cm]
			2010	&	23,143.9	&	8,764.1	&	14,379.8	\\[0.05cm]
			\rowcolor{color1!10!white}	2011	&	17,977.0	&	18,812.0	&	-835.0	\\[0.05cm]
			2012	&	24,812.0	&	9,282.0	&	15,530.1	\\[0.05cm]
			\rowcolor{color1!10!white}	2013	&	34,078.0	&	11,061.2	&	23,016.9	\\[0.05cm]
			2014*	&	17,799.0	&	14,645.6	&	3,153.4	\\[0.05cm]
			\hline
			&&&\\[-0.36cm]
			\multicolumn{4}{l}{\footnotesize Fuente: Diplan-MAGA con datos de Banguat (MAGA, 2013).}\\
			\multicolumn{4}{l}{\footnotesize */ Cifras al mes de agosto.}\\
		\end{tabular}\addtocounter{Cuadro}{1}
	\end{center}
	{\footnotesize	Ministerio de Agricultura, Ganadería y Alimentación (MAGA). 2013. El agro en cifras 2015. En: \url{http://web.maga.gob.gt/download/1agro-cifras2014.pdf}  (Consultado: febrero 2016).}
	






{\Bold\color{color1!80!black}{\normalsize Cuadro \theCuadro $\,-$  Disponibilidad per cápita, estimación de pérdidas de los productos de la hoja de balance de alimentos; según producto.}}\\
{\Bold\color{color1!80!black}{\normalsize República de Guatemala, año 2013.}}\\
%(Porcentaje de personas)\
\begin{center}\fontsize{3.8mm}{1.6em}\selectfont \setlength{\arrayrulewidth}{0.7pt}
	$\ $\\[-2.5cm]
	$\!$\begin{longtable}{x{0.65cm}m{4.3cm}x{2cm}x{3cm}x{2.2cm}x{2cm}}
		\multicolumn{6}{l}{$\ $}\\[-.2cm]
		%			\multicolumn{9}{l}{\Bold\color{color1!80!black}{\normalsize Cuadro \theCuadro $\,-$ Número de habitantes total y por sexo; según grupos quinquenales de edad.}}\\
		%			\multicolumn{9}{l}{\normalsize (Personas)}
		%			\\[-0.1cm]
		\hline
		\multicolumn{1}{l}{\small \Bold{No.}} & \multicolumn{1}{c}{\small \Bold{Productos}}& \multicolumn{1}{c}{\small \Bold{Precio de}} &\multicolumn{1}{c}{\small \Bold{Unidad de }}&\multicolumn{1}{c}{\small \Bold{Disponibilidad}}&\multicolumn{1}{c}{\small \Bold{Pérdidas}}\\
		\multicolumn{1}{l}{ } & \multicolumn{1}{c}{ }& \multicolumn{1}{c}{\small \Bold{productos a/}} &\multicolumn{1}{c}{\small \Bold{medida}}&\multicolumn{1}{c}{\small \Bold{per cápita (kg)}}&\multicolumn{1}{c}{\small \Bold{mermas (Tm)}}\\
		&&&&& \\[-0.6cm]
		\multicolumn{1}{l}{$\ $} &  \multicolumn{5}{c}{$\ $} \\[-0.48cm]
		%			\multicolumn{1}{c}{$\ $} & \multicolumn{1}{c}{\Bold{IDH}} & \multicolumn{2}{c}{\Bold{IDH Salud}} & \multicolumn{2}{c}{\Bold{IDH Educación}} & \multicolumn{2}{c}{\Bold{IDH Ingresos}} \\
		%		\multicolumn{1}{c}{} &  \multicolumn{1}{c}{\Bold{Incidencia}} & \multicolumn{1}{c}{\Bold{Error estándar}} &  \multicolumn{1}{c}{ } &\multicolumn{1}{c}{\Bold{Incidencia}} & \multicolumn{1}{c}{\Bold{Error estándar}} &  \multicolumn{1}{c}{ }\\						     
		\hline\endhead
		\hline \multicolumn{6}{r}{\textit{Continúa en la siguiente página}} \\
		\endfoot
		&&&&& \\[-0.9cm]
		\multicolumn{6}{l}{\footnotesize Fuente: INE, 2014 y MAGA 2015.}\\
		\multicolumn{6}{l}{\parbox{15cm}{\footnotesize \textbf{Notas:} - Solo áreas rurales. La tasa de pobreza total se calculó usando la línea oficial de Q.8,283.a/ los precios son promedios anuales. En la medida de lo posible se ha utilizado datos del IPC, donde no datos de precios a mayoristas.}}\\[0.1cm]
		\multicolumn{6}{l}{\parbox{15cm}{\footnotesize n.d: dato no disponible.}}\\[0.1cm]
		\multicolumn{6}{l}{\parbox{15cm}{\footnotesize n.a: no aplica}}\\[0.1cm]
		\multicolumn{6}{l}{\parbox{15cm}{\footnotesize Instituto Nacional de Estadística (INE). 2014. Hoja de Balance de Alimentos 2013. Guatemala. autor. }}\\[0.1cm]
		\multicolumn{6}{l}{\parbox{15cm}{\footnotesize  Ministerio de Agricultura, Ganadería y Alimentación (MAGA). 2015. El Agro en Cifras 2014. Índice de precios al consumidor, diciembre 2013.}}
		\endlastfoot
		\rowcolor{color1!40!white}				     &&&&& \\[-0.5cm]
		%						\multicolumn{1}{l}{\multirow{3}[0]{*}{\Bold{\raisebox{-0.6cm}{ }}}} & \multicolumn{7}{c}{\Bold{Año}} \\\cline{2-8}
		%						\multicolumn{1}{l}{$\ $} &  \multicolumn{7}{c}{$\ $} \\[-0.28cm]
		%						\rowcolor{color1!0!white} &&&&&& \\[-0.28cm]     
		%						\rowcolor{color1!0!white} { } & \multicolumn{1}{c}{2008} & \multicolumn{1}{c}{2009} & \multicolumn{1}{c}{2010} & \multicolumn{1}{c}{2011} & \multicolumn{1}{c}{2012} & \multicolumn{1}{c}{2013} & \multicolumn{1}{c}{2014}  \\ \hline
		%						\rowcolor{color1!40!white} &&&&&&& \\[-0.28cm]
		%							&&&&&& \\[-0.58cm]
		\rowcolor{color1!40!white} \multicolumn{6}{l}{\Bold{	1. Cereales	}}		\\
		\multicolumn{1}{l}{	1	}&		Maíz blanco	&	131.71	&	Quintal	&		&	58066	\\
		\rowcolor{color1!5!white}\multicolumn{1}{l}{	2	}&		Maíz amarillo	&	144.70	&	Quintal	&		&	9453	\\
		\multicolumn{1}{l}{	3	}&		Harina de maíz	&	n.d	&	n.d	&	9.9	&	20863	\\
		\rowcolor{color1!5!white}\multicolumn{1}{l}{	4	}&		Tortilla	&	6.27	&	460 gramos	&	173.1	&		\\
		\multicolumn{1}{l}{	5	}&		Trigo	&	n.d	&	n.d	&		&	47	\\
		\rowcolor{color1!5!white}\multicolumn{1}{l}{	6	}&		Harina de trigo	&	n.d	&	n.d	&	10.4	&	1673	\\
		\multicolumn{1}{l}{	7	}&		Pan y galleta	&	n.d	&	n.d	&	11.8	&		\\
		\rowcolor{color1!5!white}\multicolumn{1}{l}{	8	}&		Pastas alimenticias	&	n.d	&	n.d	&	1.3	&		\\
		\multicolumn{1}{l}{	9	}&		Avena 	&	n.d	&	n.d	&	0	&		\\
		\rowcolor{color1!5!white}\multicolumn{1}{l}{	10	}&		Arroz en granza	&	n.d	&	n.d	&		&	1090	\\
		\multicolumn{1}{l}{	11	}&		Arroz oro	&	322.96	&	Quintal	&	4.8	&	10032	\\
		\rowcolor{color1!5!white}\multicolumn{1}{l}{	12	}&		Maicillo (sorgo)	&	n.d	&	n.d	&		&	1752	\\
		\multicolumn{1}{l}{	13	}&		Tortilla (maicillo)	&	n.d	&	n.d	&	1.3	&		\\
		\rowcolor{color1!40!white} \multicolumn{6}{l}{\Bold{	2.  Leguminosas	}}		\\
		\rowcolor{color1!5!white} \multicolumn{1}{l}{	1	}&		Frijoles	&	5.74	&	460 gramos	&	11.8	&	34688	\\
		\rowcolor{color1!40!white} \multicolumn{6}{l}{\Bold{	3.  Azúcares	}}	\\
		\multicolumn{1}{l}{	1	}&		Caña de azúcar	&	n.d	&	n.d	&		&	269136	\\
		\multicolumn{1}{l}{	2	}&		Azúcar cruda	&	n.d	&	n.d	&		&	633	\\
		\rowcolor{color1!5!white}\multicolumn{1}{l}{	3	}&		Azúcar blanca y refinada	&	3.66	&	460 gramos	&	35.7	&	153	\\
		\multicolumn{1}{l}{	4	}&		Materiales azucarados	&	n.d	&	n.d	&	2	&	7	\\
		\rowcolor{color1!5!white}\multicolumn{1}{l}{	5	}&		Melazas  	&	n.d	&	n.d	&		&	35111	\\
		\rowcolor{color1!40!white} \multicolumn{6}{l}{\Bold{	4.  Tubérculos y raíces	}}	\\
		\multicolumn{1}{l}{	1	}&		Papa	&	6.72	&	460 gramos	&	24.6	&	57914	\\
		\rowcolor{color1!5!white} \multicolumn{1}{l}{	2	}&		Yuca	&	n.d	&	n.d	&	0.3	&	440	\\
		\rowcolor{color1!40!white} \multicolumn{6}{l}{\Bold{	5.  Hortalizas	}}	\\
		\multicolumn{1}{l}{	1	}&		Cebolla	&	5.70	&	460 gramos	&	7.8	&	12169	\\
		\rowcolor{color1!5!white}\multicolumn{1}{l}{	2	}&		Tomate	&	6.49	&	460 gramos	&	13.7	&	37788	\\
		\multicolumn{1}{l}{	3	}&		Zanahoria	&	31.69	&	Red (7 a 8 docenas)	&	2.2	&	8638	\\
		\rowcolor{color1!5!white}\multicolumn{1}{l}{	4	}&		Chile pimiento	&	91.72	&	Caja (90 a 100 unidades)	&	2.6	&	5570	\\
		\multicolumn{1}{l}{	5	}&		Güicoy	&	n.d	&	n.d	&	3.5	&	6502	\\
		\rowcolor{color1!5!white}\multicolumn{1}{l}{	6	}&		Otras hortalizas (lechuga, repollo y coliflor)	&	n.a	&	n.a	&	2.6	&	15786	\\
		\rowcolor{color1!40!white} \multicolumn{6}{l}{\Bold{	6.  Frutas	}}		\\
		\multicolumn{1}{l}{	1	}&		Plátano	&	4.56	&	460 gramos	&	1.9	&	17934	\\
		\rowcolor{color1!5!white} \multicolumn{1}{l}{	2	}&		Banano	&	4.53	&	460 gramos	&	51.2	&	160054	\\
		\multicolumn{1}{l}{	3	}&		Cítricos (limón, naranja)	&	n.d	&	n.d	&	18.1	&	42141	\\
		\rowcolor{color1!5!white} \multicolumn{1}{l}{	4	}&		Aguacate	&	209.74	&	Red (90 a 100 unidades)	&	4.9	&	19192	\\
		\multicolumn{1}{l}{	5	}&		Melón	&	501.11	&	Ciento	&	5.6	&	79652	\\
		\rowcolor{color1!5!white} \multicolumn{1}{l}{	6	}&		Piña	&	328.70	&	Ciento	&	11.2	&	48705	\\
		\multicolumn{1}{l}{	7	}&		Otras frutas (sandía, papaya, manzana y mango)	&	n.a	&	n.a	&	6.2	&	34072	\\
		\rowcolor{color1!40!white} \multicolumn{6}{l}{\Bold{	7.  Carnes	}}		\\
		\multicolumn{1}{l}{	1	}&		Carne con y sin hueso (vacunos)	&	14.18	&	Libra	&	6.7	&	9274	\\
		\rowcolor{color1!5!white}\multicolumn{1}{l}{	2	}&		Vísceras y menudos (vacunos)	&	n.d	&	n.d	&	0.9	&	1179	\\
		\multicolumn{1}{l}{	3	}&		Carne con y sin hueso (cerdos)	&	12.62	&	Libra	&	2.1	&	2722	\\
		\rowcolor{color1!5!white}\multicolumn{1}{l}{	4	}&		Vísceras y menudos (cerdos) 	&	n.d	&	n.d	&	0.2	&	203	\\
		\multicolumn{1}{l}{	5	}&		carne de ave	&	n.a	&	n.a	&	5.4	&	7725	\\
		\rowcolor{color1!5!white}\multicolumn{1}{l}{	6	}&		Embutidos de toda clase	&	n.d	&	n.d	&	1.8	&	2113	\\
		\rowcolor{color1!40!white} \multicolumn{6}{l}{\Bold{	8.  Huevos	}}	\\
		\rowcolor{color1!5!white} \multicolumn{1}{l}{	1	}&		Gallinas en postura (cabezas) / Huevos	&	17,33	&	648 gramos	&	7.6	&	11867	\\
		\rowcolor{color1!40!white} \multicolumn{6}{l}{\Bold{	9.  Pescado y mariscos	}}	\\
		\multicolumn{1}{l}{	1	}&		Pescado	&	n.d	&	n.d	&	2.5	&	7881	\\
		\rowcolor{color1!5!white}\multicolumn{1}{l}{	2	}&		Camarón	&	n.d	&	n.d	&	0.00	&	30	\\
		\rowcolor{color1!40!white} \multicolumn{6}{l}{\Bold{	10.  Productos lácteos	}}	\\
		\multicolumn{1}{l}{	1	}&		Leche fluida cruda de vaca	&	n.d	&	n.d	&	2.2	&	57815	\\
		\rowcolor{color1!5!white} \multicolumn{1}{l}{	2	}&		Leche fluida entera pasteurizada	&	11,84	&	Litro	&	17.5	&		\\
		\multicolumn{1}{l}{	3	}&		Leche fluida semidescremada	&	n.d	&	n.d	&	1.6	&		\\
		\rowcolor{color1!5!white} \multicolumn{1}{l}{	4	}&		Leche en polvo semidescremada	&	n.d	&	n.d	&	0	&		\\
		\multicolumn{1}{l}{	5	}&		Leche pasteurizada/ Leche fluida descremada	&	n.d	&	n.d	&	0.1	&		\\
		\rowcolor{color1!5!white} \multicolumn{1}{l}{	6	}&		Lecha en polvo descremada	&	28,11	&	Bolsa (360 gramos)	&	0.1	&		\\
		\multicolumn{1}{l}{	7	}&		Leche en polvo entera 	&	26,92	&	Bolsa (360 gramos)	&	0.6	&		\\
		\rowcolor{color1!5!white} \multicolumn{1}{l}{	8	}&		Quesos	&	33.98	&	250 ml	&	1.6	&		\\
		\multicolumn{1}{l}{	9	}&		Leche pasteurizada/crema de leche	&	8,76	&	250 ml	&	0.5	&		\\
		\rowcolor{color1!5!white} \multicolumn{1}{l}{	10	}&		Leche pasteurizada/yogur	&	n.d	&	n.d	&	0.3	&		\\
		\rowcolor{color1!40!white} \multicolumn{6}{l}{\Bold{	11.  Aceites y grasas	}}	\\
		\rowcolor{color1!15!white} \multicolumn{6}{l}{\Bold{	Aceites vegetales:	}}		\\
		\multicolumn{1}{l}{	1	}&		Aceite refinado/palma africana	&	n.d	&	n.d	&	1.4	&	2838	\\
		\rowcolor{color1!5!white}\multicolumn{1}{l}{	2	}&		Aceite refinado/soya	&	n.d	&	n.d	&	4	&	7991	\\
		\multicolumn{1}{l}{	3	}&		Aceite refinado/semilla de algodón	&	n.d	&	n.d	&	0.00	&	2	\\
		\rowcolor{color1!5!white}\multicolumn{1}{l}{	4	}&		Aceite refinado/semilla de girasol	&	n.d	&	n.d	&	0.8	&	1690	\\
		\multicolumn{1}{l}{	5	}&		Aceite refinado/aceituna (oliva)	&	n.d	&	n.d	&	0.00	&	100	\\
		\rowcolor{color1!15!white} \multicolumn{6}{l}{\Bold{	Grasas animales:	}}	\\
		\multicolumn{1}{l}{	1	}&		Leche pasteurizada/mantequilla	&	n.d	&	n.d	&	0.1	&	155	\\
		\rowcolor{color1!5!white} \multicolumn{1}{l}{	2	}&		Vacunos faenados/grasa de res	&	n.d	&	n.d	&	0.1	&	5069	\\
		\multicolumn{1}{l}{	3	}&		Cerdos faenados/manteca	&	n.d	&	n.d	&	0.1	&		\\
		\rowcolor{color1!40!white} \multicolumn{6}{l}{\Bold{	12. Alimentos gratificantes	}}	\\
		\multicolumn{1}{l}{	1	}&		Cerveza	&	n.d	&	n.d	&	11.9	&		\\
		\rowcolor{color1!5!white}\multicolumn{1}{l}{	2	}&		melazas/licores	&	n.d	&	n.d	&	3.2	&	805	\\
		\multicolumn{1}{l}{	3	}&		azúcar/bebidas gaseosas	&	10,39	&	1000 ml	&	68.2	&		\\
		\hline
		&&&&&\\[-0.28cm]
		%			\multicolumn{9}{l}{\footnotesize Fuente: Informe Nacional de Desarrollo Humano (PNUD), con base en las Encuestas Nacionales de Condiciones de Vida (Encovi).}
	\end{longtable}\addtocounter{Cuadro}{1}
\end{center}






		\hoja{
	{\Bold\color{color1!80!black}{Cuadro \theCuadro $\,-$  Salario mínimo por actividad de actividad económica. República de Guatemala, años 2008-2015.}}\\
	{\Bold\color{color1!80!black}{(Quetzales corrientes.)}}\\[-.5cm]
\begin{center}
	\begin{spacing}{1.2}
	\begin{tabular}{m{2.5cm}rrr}
		\hline &&&\\[-0.56cm]  
		\multicolumn{1}{r}{ } &	\multicolumn{3}{c}{\Bold{Actividad económica}}\\[0cm]\cline{2-4}
		 &&&\\[-0.36cm] 
		\multicolumn{1}{c}{\Bold{Año}} & \multicolumn{1}{c}{\Bold{Agrícolas}} &	\multicolumn{1}{c}{\Bold{No agrícolas}} & \multicolumn{1}{c}{\Bold{De exportación}} \\[0.05cm]
		\hline
		\rowcolor{color1!40!white}$\ $	&&&\\[-0.55cm]
\rowcolor{color1!40!white} \multicolumn{4}{c}{\Bold{Salario diario	}} 	\\
	\multicolumn{1}{c}{	2008	} &	Q. 47.00	&	Q. 48.50	&	Q47.75	\\
	\rowcolor{color1!5!white}\multicolumn{1}{c}{	2009	} &	Q. 52.00	&	Q. 52.00	&	Q47.75	\\
	\multicolumn{1}{c}{	2010	} &	Q. 56.00	&	Q. 56.00	&	Q51.75	\\
	\rowcolor{color1!5!white}\multicolumn{1}{c}{	2011	} &	Q.  63.70	&	Q.  63.70	&	Q59.45	\\
	\multicolumn{1}{c}{	2012	} &	Q.  68.00	&	Q.  68.00	&	Q62.50	\\
	\rowcolor{color1!5!white}\multicolumn{1}{c}{	2013	} &	Q.  71.40	&	Q.  71.40	&	Q65.63	\\
	\multicolumn{1}{c}{	2014	} &	Q.  74.97	&	Q.  74.97	&	Q68.91	\\
	\rowcolor{color1!5!white}\multicolumn{1}{c}{	2015	} &	Q.  78.72	&	Q.  78.72	&	Q72.36	\\
	\rowcolor{color1!40!white} \multicolumn{4}{c}{\Bold{Salario mensual	}} 	\\
		\multicolumn{1}{c}{	2008	} &	 1,433.5 	&	 1,479.3 	&	Q1,456.38	\\
		\rowcolor{color1!5!white}\multicolumn{1}{c}{	2009	} &	 1,581.7 	&	 1,581.7 	&	Q1,452.39	\\
		\multicolumn{1}{c}{	2010	} &	 1,703.3 	&	 1,703.3 	&	Q1,574.06	\\
		\rowcolor{color1!5!white}\multicolumn{1}{c}{	2011	} &	 1,937.5 	&	 1,937.5 	&	Q1,808.27	\\
		\multicolumn{1}{c}{	2012	} &	 2,074.0 	&	 2,074.0 	&	Q1,906.25	\\
		\rowcolor{color1!5!white}\multicolumn{1}{c}{	2013	} &	 2,171.8 	&	 2,171.8 	&	Q1,996.25	\\
		\multicolumn{1}{c}{	2014	} &	 2,280.3 	&	 2,280.3 	&	Q2,096.06	\\
		\rowcolor{color1!5!white}\multicolumn{1}{c}{	2015	} &	 2,394.4 	&	 2,394.4 	&	Q2,200.95	\\
		\rowcolor{color1!40!white} \multicolumn{4}{c}{\Bold{Salario total	}} \\	
\multicolumn{1}{c}{	2008	} &	 1,683.5 	&	 1,729.3 	&	Q1,706.38	\\
\rowcolor{color1!5!white}\multicolumn{1}{c}{	2009	} &	 1,831.7 	&	 1,831.7 	&	Q1,702.39	\\
\multicolumn{1}{c}{	2010	} &	 1,953.3 	&	 1,953.3 	&	Q1,824.06	\\
\rowcolor{color1!5!white}\multicolumn{1}{c}{	2011	} &	 2,187.5 	&	 2,187.5 	&	Q2,058.27	\\
\multicolumn{1}{c}{	2012	} &	 2,324.0 	&	 2,324.0 	&	Q2,156.25	\\
\rowcolor{color1!5!white}\multicolumn{1}{c}{	2013	} &	 2,421.8 	&	 2,421.8 	&	Q2,246.25	\\
\multicolumn{1}{c}{	2014	} &	 2,530.3 	&	 2,530.3 	&	Q2,346.06	\\
\rowcolor{color1!5!white}\multicolumn{1}{c}{	2015	} &	 2,644.4 	&	 2,644.4 	&	Q2,450.95	\\
\rowcolor{color1!40!white} \multicolumn{4}{c}{\Bold{Acuerdo Gubernativo	}} 	\\
	\multicolumn{1}{c}{	2008	} & \multicolumn{3}{c}{	No. 625-2007	} \\	
	\rowcolor{color1!5!white}\multicolumn{1}{c}{	2009	} & \multicolumn{3}{c}{	No.398-2008	}\\ 	
	\multicolumn{1}{c}{	2010	} & \multicolumn{3}{c}{	No. 347-2009	} \\	
	\rowcolor{color1!5!white}\multicolumn{1}{c}{	2011	} & \multicolumn{3}{c}{	No. 388-2010	} 		\\		
	\multicolumn{1}{c}{	2012	} & \multicolumn{3}{c}{	No. 520-2011	} 	\\
	\rowcolor{color1!5!white}\multicolumn{1}{c}{	2013	} & \multicolumn{3}{c}{	No. 359-2012	} 	\\
	\multicolumn{1}{c}{	2014	} & \multicolumn{3}{c}{	No. 537-2013	} 	\\
	\rowcolor{color1!5!white}\multicolumn{1}{c}{	2015	} & \multicolumn{3}{c}{	No. 470-2014	} 	\\	\hline
		&&&\\[-0.36cm]
		\multicolumn{4}{l}{\footnotesize Fuente: Mintrab, Observatorio Laboral.}\\
		\multicolumn{4}{l}{\footnotesize Nota: bonificación incentivo de Q. 250 según decreto No. 37-2001.}\\	
	\end{tabular}\addtocounter{Cuadro}{1}
		\end{spacing}
		
%		\begin{spacing}{1.2}
%			\begin{tabular}{m{2.5cm}rrr}
%				\hline &&&\\[-0.56cm] 
%				\rowcolor{color1!40!white}$\ $	&&&\\[-0.55cm]
%				\rowcolor{color1!40!white} \multicolumn{4}{c}{\Bold{Acuerdo Gubernativo	}} 	\\
%				\multicolumn{1}{c}{	2008	} & \multicolumn{3}{c}{	No. 625-2007	} \\	
%				\rowcolor{color1!5!white}\multicolumn{1}{c}{	2009	} & \multicolumn{3}{c}{	No.398-2008	}\\ 	
%				\multicolumn{1}{c}{	2010	} & \multicolumn{3}{c}{	No. 347-2009	} \\	
%				\rowcolor{color1!5!white}\multicolumn{1}{c}{	2011	} & \multicolumn{3}{c}{	No. 388-2010	} 		\\		
%				\multicolumn{1}{c}{	2012	} & \multicolumn{3}{c}{	No. 520-2011	} 	\\
%				\rowcolor{color1!5!white}\multicolumn{1}{c}{	2013	} & \multicolumn{3}{c}{	No. 359-2012	} 	\\
%				\multicolumn{1}{c}{	2014	} & \multicolumn{3}{c}{	No. 537-2013	} 	\\
%				\rowcolor{color1!5!white}\multicolumn{1}{c}{	2015	} & \multicolumn{3}{c}{	No. 470-2014	} 	\\	\hline
%				&&&\\[-0.36cm]
%				\multicolumn{4}{l}{\footnotesize Fuente: Mintrab, Observatorio Laboral.}\\
%			\end{tabular}\addtocounter{Cuadro}{1}
%		\end{spacing}
\end{center}
}





\hoja{
	{\Bold\color{color1!80!black}{Cuadro \theCuadro $\,-$  Salario medio mensual de trabajadores afiliados cotizantes al IGSS por año; según departamento. República de Guatemala, años 2010-2015.}}\\
	{\Bold\color{color1!80!black}{(Quetzales corrientes.)}}\\[-.5cm]
	\begin{center}
		\begin{spacing}{1.2}
			\begin{tabular}{p{2.5cm}ccccc}
				\hline &&&&&\\[-0.56cm]  
				\multicolumn{1}{r}{ } &	\multicolumn{5}{c}{\Bold{Año}}\\[0cm]\cline{2-6}
				&&&&&\\[-0.36cm] 
				\multicolumn{1}{l}{\raisebox{0.3cm}{\Bold{Departamento}}} & \multicolumn{1}{c}{\Bold{2010}} &	\multicolumn{1}{c}{\Bold{2011}} & \multicolumn{1}{c}{\Bold{2012}} & \multicolumn{1}{c}{\Bold{2013}} & \multicolumn{1}{c}{\Bold{2015}}\\[0.05cm]
				\hline
				\rowcolor{color1!40!white}$\ $	&&&&&\\[-0.55cm]
				%				\rowcolor{color1!40!white} \multicolumn{4}{c}{\Bold{Salario diario	}} 	\\
				\rowcolor{color1!40!white} \multicolumn{1}{l}{\Bold{	Total 	}}&	 3,048.0 	 & 	3,250.93	 & 	3,508.32	 & 	3,655.44	 & 	 3,874.55 	 \\ 
				\multicolumn{1}{l}{	Guatemala	}&	 3,328.0 	 & 	3,549.83	 & 	3,772.82	 & 	3,903.63	 & 	 4,180.22 	 \\ 
				\rowcolor{color1!5!white}\multicolumn{1}{l}{	El Progreso	}&	 3,108.2 	 & 	3,239.38	 & 	2,985.83	 & 	3,114.59	 & 	 3,539.03 	 \\ 
				\multicolumn{1}{l}{	Sacatepéquez	}&	 2,465.3 	 & 	2,641.15	 & 	2,753.79	 & 	2,873.10	 & 	 3,380.69 	 \\ 
				\rowcolor{color1!5!white}\multicolumn{1}{l}{	Chimaltenango	}&	 2,607.6 	 & 	2,920.47	 & 	3,127.39	 & 	3,329.25	 & 	 3,449.37 	 \\ 
				\multicolumn{1}{l}{	Escuintla	}&	 2,209.7 	 & 	2,375.15	 & 	2,624.13	 & 	2,708.05	 & 	 2,900.84 	 \\ 
				\rowcolor{color1!5!white}\multicolumn{1}{l}{	Santa Rosa	}&	 2,505.4 	 & 	2,567.92	 & 	3,073.14	 & 	3,463.90	 & 	 3,747.21 	 \\ 
				\multicolumn{1}{l}{	Sololá	}&	 2,892.0 	 & 	3,021.07	 & 	3,244.02	 & 	3,554.79	 & 	 3,851.51 	 \\ 
				\rowcolor{color1!5!white}\multicolumn{1}{l}{	Totonicapán	}&	 3,235.0 	 & 	3,296.25	 & 	3,599.22	 & 	3,879.86	 & 	 4,209.71 	 \\ 
				\multicolumn{1}{l}{	Quetzaltenango	}&	 2,670.8 	 & 	2,717.05	 & 	2,966.40	 & 	3,217.36	 & 	 3,367.37 	 \\ 
				\rowcolor{color1!5!white}\multicolumn{1}{l}{	Suchitepéquez	}&	 2,333.9 	 & 	2,513.45	 & 	2,745.54	 & 	2,861.96	 & 	 3,039.23 	 \\ 
				\multicolumn{1}{l}{	Retalhuleu	}&	 2,667.3 	 & 	2,708.15	 & 	2,885.37	 & 	3,060.26	 & 	 3,150.16 	 \\ 
				\rowcolor{color1!5!white}\multicolumn{1}{l}{	San Marcos	}&	 2,675.4 	 & 	3,083.79	 & 	3,239.86	 & 	3,452.58	 & 	 3,527.54 	 \\ 
				\multicolumn{1}{l}{	Huehuetenango	}&	 2,909.3 	 & 	3,012.42	 & 	3,268.39	 & 	3,476.98	 & 	 3,770.58 	 \\ 
				\rowcolor{color1!5!white}\multicolumn{1}{l}{	Quiché	}&	 3,077.0 	 & 	3,146.92	 & 	3,428.62	 & 	3,615.72	 & 	 3,994.57 	 \\ 
				\multicolumn{1}{l}{	Baja Verapaz	}&	 3,009.0 	 & 	3,139.09	 & 	3,488.57	 & 	3,766.08	 & 	 3,908.24 	 \\ 
				\rowcolor{color1!5!white}\multicolumn{1}{l}{	Alta Verapaz	}&	 2,652.2 	 & 	2,679.48	 & 	2,883.07	 & 	3,208.40	 & 	 3,587.20 	 \\ 
				\multicolumn{1}{l}{	Petén	}&	 2,918.3 	 & 	3,029.82	 & 	3,180.43	 & 	3,303.45	 & 	 3,535.21 	 \\ 
				\rowcolor{color1!5!white}\multicolumn{1}{l}{	Izabal	}&	 2,816.3 	 & 	3,208.02	 & 	3,444.45	 & 	3,611.02	 & 	 3,824.90 	 \\ 
				\multicolumn{1}{l}{	Zacapa	}&	 2,722.4 	 & 	2,980.87	 & 	3,131.15	 & 	3,301.09	 & 	 3,454.73 	 \\ 
				\rowcolor{color1!5!white}\multicolumn{1}{l}{	Chiquimula	}&	 2,996.7 	 & 	3,031.39	 & 	3,194.29	 & 	3,338.12	 & 	 3,600.57 	 \\ 
				\multicolumn{1}{l}{	Jalapa	}&	 2,815.3 	 & 	2,994.03	 & 	3,265.29	 & 	3,432.02	 & 	 3,703.20 	 \\ 
				\rowcolor{color1!5!white}\multicolumn{1}{l}{	Jutiapa	}&	 2,935.1 	 & 	3,072.86	 & 	3,444.38	 & 	3,687.63	 & 	 3,812.30 	 \\ 
				\hline
				&&&\\[-0.36cm]
				\multicolumn{4}{l}{\footnotesize Fuente: Instituto Guatemalteco de Seguridad Social.}\\
				%				\multicolumn{4}{l}{\footnotesize Nota: bonificación incentivo de Q. 250 según decreto No. 37-2001.}\\	
			\end{tabular}\addtocounter{Cuadro}{1}
		\end{spacing}
	\end{center}
}





\begin{landscape}
	{\Bold\color{color1!80!black}{Cuadro \theCuadro $\,-$  Ingreso laboral mensual promedio por características de la población. República de Guatemala, varios años.}}\\
	{\Bold\color{color1!80!black}{(Quetzales corrientes.)}}\\[-.5cm]
	\begin{center}
		\begin{spacing}{1.2}
			\begin{tabular}{lcccccccccc}
				\hline &&&&&&&&&&\\[-0.56cm]  
				\multicolumn{1}{c}{\small\raisebox{-.5cm}{\textbf{Año}}} &	\multicolumn{1}{c}{\small\raisebox{-.5cm}{\textbf{Total}}}&	\multicolumn{3}{c}{\textbf{Dominio de estudio}}&	\multicolumn{2}{c}{\textbf{Sexo}}&\multicolumn{2}{c}{\textbf{Étnia}}&\multicolumn{2}{c}{\textbf{Grupo etario}}\\[-.36cm]\cline{3-11}
				&&&&&&&&&&\\[-0.36cm] 
				\multicolumn{1}{c}{ } & \multicolumn{1}{c}{ }& \multicolumn{1}{c}{\parbox{2cm}{\Bold{$\ \ \ \ \ $Urbano\\ metropolitano} }}& \multicolumn{1}{c}{\Bold{Resto urbano}}& \multicolumn{1}{c}{\Bold{Rural}}& \multicolumn{1}{c}{\Bold{Hombre}} &	\multicolumn{1}{c}{\Bold{Mujer}} & \multicolumn{1}{c}{\Bold{Indígena}} & \multicolumn{1}{c}{\Bold{No indígena}} & \multicolumn{1}{c}{\Bold{15 a 24 años}} & \multicolumn{1}{c}{\Bold{25 años o más}}\\[0.05cm]
				\hline
				\rowcolor{color1!0!white}$\ $	&&&&&&&&&&\\[-0.55cm]
				%				\rowcolor{color1!40!white} \multicolumn{4}{c}{\Bold{Salario diario	}} 	\\
				\multicolumn{1}{l}{	ENEI 2002	}&	 1,216 	 & 	 1,989 	 & 	 1,124 	 & 	 853 	 & 	 1,412 	 & 	 879 	 & 	 923 	 & 	 1,434 	 & 	 967 	 & 	 1,303 	 \\ 
				\rowcolor{color1!5!white}\multicolumn{1}{l}{	ENEI 2003	}&	 1,424 	 & 	 2,193 	 & 	 1,521 	 & 	 1,019 	 & 	 1,634 	 & 	 1,050 	 & 	 881 	 & 	 1,688 	 & 	 1,068 	 & 	 1,570 	 \\ 
				\multicolumn{1}{l}{	ENEI 2004	}&	 1,205 	 & 	 1,728 	 & 	 1,301 	 & 	 820 	 & 	 1,339 	 & 	 951 	 & 	 817 	 & 	 1,397 	 & 	 1,008 	 & 	 1,269 	 \\ 
				\rowcolor{color1!5!white}\multicolumn{1}{l}{	ENEI 2010	}&	 1,680 	 & 	 2,563 	 & 	 1,768 	 & 	 1,133 	 & 	 1,831 	 & 	 1,393 	 & 	 1,119 	 & 	 2,001 	 & 	 1,214 	 & 	 1,817 	 \\ 
				\multicolumn{1}{l}{	ENEI 2011	}&	 1,685 	 & 	 2,657 	 & 	 1,719 	 & 	 1,223 	 & 	 1,801 	 & 	 1,443 	 & 	 1,198 	 & 	 1,988 	 & 	 1,229 	 & 	 1,828 	 \\ 
				\rowcolor{color1!5!white}\multicolumn{1}{l}{	ENEI 2012	}&	 1,734 	 & 	 2,628 	 & 	 1,837 	 & 	 1,258 	 & 	 1,880 	 & 	 1,465 	 & 	 1,214 	 & 	 2,056 	 & 	 1,241 	 & 	 1,877 	 \\ 
				\multicolumn{1}{l}{	ENEI 2013 / 1*	}&	 1,917 	 & 	 2,841 	 & 	 1,865 	 & 	 1,499 	 & 	 2,028 	 & 	 1,703 	 & 	 1,325 	 & 	 2,241 	 & 	 1,490 	 & 	 2,039 	 \\ 
				\rowcolor{color1!5!white}\multicolumn{1}{l}{	ENEI 2013 / 2	}&	 1,893 	 & 	 2,714 	 & 	 1,907 	 & 	 1,478 	 & 	 2,006 	 & 	 1,682 	 & 	 1,367 	 & 	 2,236 	 & 	 1,481 	 & 	 2,012 	 \\ 
				\multicolumn{1}{l}{	ENEI 2014 / 1	}&	 2,083 	 & 	 3,031 	 & 	 2,399 	 & 	 1,345 	 & 	 2,253 	 & 	 1,758 	 & 	 1,235 	 & 	 2,457 	 & 	 1,492 	 & 	 2,263 	 \\ 
				\rowcolor{color1!5!white}\multicolumn{1}{l}{	ENEI 2014 / 2	}&	 2,207 	 & 	 2,930 	 & 	 2,671 	 & 	 1,470 	 & 	 2,298 	 & 	 2,028 	 & 	 1,561 	 & 	 2,538 	 & 	 1,517 	 & 	 2,401 	 \\ 
				
				\hline
				&&&&&&&&&\\[-0.36cm]
				\multicolumn{10}{l}{\footnotesize Fuente: Instituto Guatemalteco de Seguridad Social.}\\
				%				\multicolumn{4}{l}{\footnotesize Nota: bonificación incentivo de Q. 250 según decreto No. 37-2001.}\\	
			\end{tabular}\addtocounter{Cuadro}{1}
		\end{spacing}
	\end{center}
\end{landscape}







\begin{landscape}
	{\Bold\color{color1!80!black}{Cuadro \theCuadro $\,-$  Salario medio mensual de trabajadores afiliados cotizantes al IGSS por año; según departamento. República de Guatemala, años 2010-2015.}}\\
	{\Bold\color{color1!80!black}{(Quetzales corrientes.)}}\\[-.5cm]
	\begin{center}
		\begin{spacing}{1.2}
			\begin{tabular}{p{5.5cm}ccccccccc}
				\hline &&&&&&&&&\\[-0.56cm]  
				\multicolumn{1}{p{5.5cm}}{\small\raisebox{-.5cm}{\textbf{Actividad económica}}} &	\multicolumn{9}{c}{\Bold{Año}}\\[0cm]\cline{2-10}
				&&&&&&&&&\\[-0.36cm] 
				\multicolumn{1}{p{5.5cm}}{\raisebox{0.3cm}{ }} & \multicolumn{1}{c}{\Bold{2006}}& \multicolumn{1}{c}{\Bold{2007}}& \multicolumn{1}{c}{\Bold{2008}}& \multicolumn{1}{c}{\Bold{2009}}& \multicolumn{1}{c}{\Bold{2010}} &	\multicolumn{1}{c}{\Bold{2011}} & \multicolumn{1}{c}{\Bold{2012}} & \multicolumn{1}{c}{\Bold{2013}} & \multicolumn{1}{c}{\Bold{2014}}\\[0.05cm]
				\hline
				\rowcolor{color1!40!white}$\ $	&&&&&&&&&\\[-0.55cm]
				%				\rowcolor{color1!40!white} \multicolumn{4}{c}{\Bold{Salario diario	}} 	\\
				\rowcolor{color1!40!white}\multicolumn{1}{p{5.5cm}}{\textbf{	Total}	}&	\textbf{2,454.24}	&	\textbf{2,580.30}	&	\textbf{2,798.84}	&	\textbf{2,854.14}	&	\textbf{3,048.20}	&	\textbf{3,250.93}	&	\textbf{3,508.32}	&	\textbf{3,655.44}	&	\textbf{3,874.55}	\\
				\multicolumn{1}{p{5.5cm}}{	Agricultura, Silv., Caza y Pezca	}&	 1,406.1 	 & 	 1,466.4 	 & 	 1,613.1 	 & 	 1,669.7 	 & 	 1,839.9 	 & 	 1,953.4 	 & 	 2,193.8 	 & 	 2,306.2 	 & 	 2,443.3 	 \\ 
				\rowcolor{color1!5!white}\multicolumn{1}{p{5.5cm}}{	Explotación de minas y canteras	}&	 4,556.2 	 & 	 5,025.9 	 & 	 5,766.9 	 & 	 6,194.6 	 & 	 7,585.7 	 & 	 7,844.1 	 & 	 7,440.1 	 & 	 7,698.8 	 & 	 8,150.6 	 \\ 
				\multicolumn{1}{p{5.5cm}}{	Industria manufacturera	}&	 2,482.1 	 & 	 2,653.0 	 & 	 2,883.3 	 & 	 2,971.6 	 & 	 3,129.8 	 & 	 3,399.2 	 & 	 3,620.1 	 & 	 3,789.8 	 & 	 4,000.5 	 \\ 
				\rowcolor{color1!5!white}\multicolumn{1}{p{5.5cm}}{	Construcción	}&	 1,864.6 	 & 	 1,908.0 	 & 	 2,042.2 	 & 	 2,165.1 	 & 	 2,326.5 	 & 	 2,519.5 	 & 	 2,700.6 	 & 	 2,825.1 	 & 	 3,040.5 	 \\ 
				\multicolumn{1}{p{5.5cm}}{	Electricidad, Gas, Agua y Servicios Sanitarios	}&	 4,659.3 	 & 	 5,038.4 	 & 	 5,209.7 	 & 	 5,605.6 	 & 	 5,432.0 	 & 	 5,929.9 	 & 	 6,199.8 	 & 	 6,716.4 	 & 	 7,019.2 	 \\ 
				\rowcolor{color1!5!white}\multicolumn{1}{p{5.5cm}}{	Comercio	}&	 2,747.6 	 & 	 2,850.0 	 & 	 2,988.0 	 & 	 3,077.2 	 & 	 3,224.1 	 & 	 3,500.9 	 & 	 3,648.6 	 & 	 3,770.6 	 & 	 4,073.9 	 \\ 
				\multicolumn{1}{p{5.5cm}}{	Transporte, almacenaje y comunicaciones	}&	 3,188.2 	 & 	 3,228.6 	 & 	 3,382.3 	 & 	 3,514.1 	 & 	 3,438.3 	 & 	 3,687.7 	 & 	 3,777.0 	 & 	 3,980.6 	 & 	 4,398.5 	 \\ 
				\rowcolor{color1!5!white}\multicolumn{1}{p{5.5cm}}{	Servicios	}&	 2,574.9 	 & 	 2,718.8 	 & 	 2,970.4 	 & 	 2,994.8 	 & 	 3,221.4 	 & 	 3,373.5 	 & 	 3,675.0 	 & 	 3,798.7 	 & 	 3,996.9 	 \\ 
				\hline
				&&&&&&&&&\\[-0.36cm]
				\multicolumn{10}{l}{\footnotesize Fuente: Instituto Guatemalteco de Seguridad Social.}\\
				%				\multicolumn{4}{l}{\footnotesize Nota: bonificación incentivo de Q. 250 según decreto No. 37-2001.}\\	
			\end{tabular}\addtocounter{Cuadro}{1}
		\end{spacing}
	\end{center}
\end{landscape}





\begin{landscape}
	{\Bold\color{color1!80!black}{Cuadro \theCuadro $\,-$  Ingreso laboral mensual promedio por características de la población. República de Guatemala, varios años.}}\\
	{\Bold\color{color1!80!black}{(Quetzales corrientes.)}}\\[-.5cm]
	\begin{center}
		\begin{spacing}{1.2}
			\begin{tabular}{lcccccccccc}
				\hline &&&&&&&&&&\\[-0.56cm]  
				\multicolumn{1}{c}{\small\raisebox{-.5cm}{\textbf{Año}}} &	\multicolumn{1}{c}{\small\raisebox{-.5cm}{\textbf{Total}}}&	\multicolumn{3}{c}{\textbf{Dominio de estudio}}&	\multicolumn{2}{c}{\textbf{Sexo}}&\multicolumn{2}{c}{\textbf{Étnia}}&\multicolumn{2}{c}{\textbf{Grupo etario}}\\[-.36cm]\cline{3-11}
				&&&&&&&&&&\\[-0.36cm] 
				\multicolumn{1}{c}{ } & \multicolumn{1}{c}{ }& \multicolumn{1}{c}{\parbox{2cm}{\Bold{$\ \ \ \ \ $Urbano\\ metropolitano} }}& \multicolumn{1}{c}{\Bold{Resto urbano}}& \multicolumn{1}{c}{\Bold{Rural}}& \multicolumn{1}{c}{\Bold{Hombre}} &	\multicolumn{1}{c}{\Bold{Mujer}} & \multicolumn{1}{c}{\Bold{Indígena}} & \multicolumn{1}{c}{\Bold{No indígena}} & \multicolumn{1}{c}{\Bold{15 a 24 años}} & \multicolumn{1}{c}{\Bold{25 años o más}}\\[0.05cm]
				\hline
				\rowcolor{color1!0!white}$\ $	&&&&&&&&&&\\[-0.55cm]
				%				\rowcolor{color1!40!white} \multicolumn{4}{c}{\Bold{Salario diario	}} 	\\
				\multicolumn{1}{l}{	ENEI 2002	}&	 1,216 	 & 	 1,989 	 & 	 1,124 	 & 	 853 	 & 	 1,412 	 & 	 879 	 & 	 923 	 & 	 1,434 	 & 	 967 	 & 	 1,303 	 \\ 
				\rowcolor{color1!5!white}\multicolumn{1}{l}{	ENEI 2003	}&	 1,424 	 & 	 2,193 	 & 	 1,521 	 & 	 1,019 	 & 	 1,634 	 & 	 1,050 	 & 	 881 	 & 	 1,688 	 & 	 1,068 	 & 	 1,570 	 \\ 
				\multicolumn{1}{l}{	ENEI 2004	}&	 1,205 	 & 	 1,728 	 & 	 1,301 	 & 	 820 	 & 	 1,339 	 & 	 951 	 & 	 817 	 & 	 1,397 	 & 	 1,008 	 & 	 1,269 	 \\ 
				\rowcolor{color1!5!white}\multicolumn{1}{l}{	ENEI 2010	}&	 1,680 	 & 	 2,563 	 & 	 1,768 	 & 	 1,133 	 & 	 1,831 	 & 	 1,393 	 & 	 1,119 	 & 	 2,001 	 & 	 1,214 	 & 	 1,817 	 \\ 
				\multicolumn{1}{l}{	ENEI 2011	}&	 1,685 	 & 	 2,657 	 & 	 1,719 	 & 	 1,223 	 & 	 1,801 	 & 	 1,443 	 & 	 1,198 	 & 	 1,988 	 & 	 1,229 	 & 	 1,828 	 \\ 
				\rowcolor{color1!5!white}\multicolumn{1}{l}{	ENEI 2012	}&	 1,734 	 & 	 2,628 	 & 	 1,837 	 & 	 1,258 	 & 	 1,880 	 & 	 1,465 	 & 	 1,214 	 & 	 2,056 	 & 	 1,241 	 & 	 1,877 	 \\ 
				\multicolumn{1}{l}{	ENEI 2013 / 1*	}&	 1,917 	 & 	 2,841 	 & 	 1,865 	 & 	 1,499 	 & 	 2,028 	 & 	 1,703 	 & 	 1,325 	 & 	 2,241 	 & 	 1,490 	 & 	 2,039 	 \\ 
				\rowcolor{color1!5!white}\multicolumn{1}{l}{	ENEI 2013 / 2	}&	 1,893 	 & 	 2,714 	 & 	 1,907 	 & 	 1,478 	 & 	 2,006 	 & 	 1,682 	 & 	 1,367 	 & 	 2,236 	 & 	 1,481 	 & 	 2,012 	 \\ 
				\multicolumn{1}{l}{	ENEI 2014 / 1	}&	 2,083 	 & 	 3,031 	 & 	 2,399 	 & 	 1,345 	 & 	 2,253 	 & 	 1,758 	 & 	 1,235 	 & 	 2,457 	 & 	 1,492 	 & 	 2,263 	 \\ 
				\rowcolor{color1!5!white}\multicolumn{1}{l}{	ENEI 2014 / 2	}&	 2,207 	 & 	 2,930 	 & 	 2,671 	 & 	 1,470 	 & 	 2,298 	 & 	 2,028 	 & 	 1,561 	 & 	 2,538 	 & 	 1,517 	 & 	 2,401 	 \\ 
				
				\hline
				&&&&&&&&&\\[-0.36cm]
				\multicolumn{10}{l}{\footnotesize Fuente: Instituto Guatemalteco de Seguridad Social.}\\
				%				\multicolumn{4}{l}{\footnotesize Nota: bonificación incentivo de Q. 250 según decreto No. 37-2001.}\\	
			\end{tabular}\addtocounter{Cuadro}{1}
		\end{spacing}
	\end{center}
\end{landscape}

%%%%%%%%% canasta básica






{\Bold\color{color1!80!black}{\normalsize Cuadro \theCuadro $\,-$  Mapa de pobreza rural, por departamento y municipio; según tipo de pobreza.  }}\\
{\Bold\color{color1!80!black}{\normalsize República de Guatemala, año 2011.}}\\
(Porcentaje de personas)\\
\begin{center}\fontsize{3.8mm}{1.6em}\selectfont \setlength{\arrayrulewidth}{0.7pt}
	$\ $\\[-2.5cm]
	$\!$\begin{longtable}{llrrrrrrrrr}
		\multicolumn{11}{l}{$\ $}\\[-.2cm]
		%			\multicolumn{9}{l}{\Bold\color{color1!80!black}{\normalsize Cuadro \theCuadro $\,-$ Número de habitantes total y por sexo; según grupos quinquenales de edad.}}\\
		%			\multicolumn{9}{l}{\normalsize (Personas)}
		%			\\[-0.1cm]
	\hline	\multicolumn{1}{l}{\multirow{3}[0]{*}{\Bold{Año}}} & \multicolumn{1}{c}{\multirow{3}[0]{*}{\Bold{Mes}}}& \multicolumn{4}{c}{\Bold{Canasta básica alimentaria}}&& \multicolumn{4}{c}{\Bold{Canasta básica ampliada}}\\\cline{3-6}\cline{8-11}
		&&&&&&&&&& \\[-0.6cm]
		\multicolumn{1}{l}{$\ $} &  \multicolumn{7}{c}{$\ $}&&& \\[-0.48cm]
		%			\multicolumn{1}{c}{$\ $} & \multicolumn{1}{c}{\Bold{IDH}} & \multicolumn{2}{c}{\Bold{IDH Salud}} & \multicolumn{2}{c}{\Bold{IDH Educación}} & \multicolumn{2}{c}{\Bold{IDH Ingresos}} \\
		\multicolumn{1}{c}{} &  \multicolumn{1}{c}{ } & \multicolumn{1}{c}{\small\parbox{1.2cm}{Costo\\ diario
				(Q.)}} &  \multicolumn{1}{c}{\small\parbox{1.4cm}{Costo\\ mensual (Q)}} &\multicolumn{1}{c}{\small\parbox{1.4cm}{Variación\\ mensual(\%)}} & \multicolumn{1}{c}{\small\parbox{1.4cm}{Variación\\ interanual(\%)}} & & \multicolumn{1}{c}{\small\parbox{1.2cm}{Costo\\ diario
				(Q.)}} &  \multicolumn{1}{c}{\small\parbox{1.4cm}{Costo\\ mensual (Q)}} &\multicolumn{1}{c}{\small\parbox{1.4cm}{Variación\\ mensual(\%)}} & \multicolumn{1}{c}{\small\parbox{1.4cm}{Variación\\ interanual(\%)}}\\						     
		\hline\endhead
		\hline \multicolumn{11}{r}{\textit{Continúa en la siguiente página}} \\
		\endfoot
		&&&&&&&&&& \\[-0.9cm]
		\multicolumn{11}{l}{\footnotesize Fuente: INE}\\
		\endlastfoot
		\rowcolor{color1!0!white}    &&&&&&&&&& \\[-0.55cm]
		%		\rowcolor{color1!0!white}\multicolumn{1}{l}{2006} &	Enero	&	 47.4 	&	 1,420.7 	&	 1.1 	&	 6.3 	&  &	 86.4 	&	 2,592.5 	&	 1.1 	&	 6.3 	\\
\rowcolor{color1!5!white}\multicolumn{1}{l}{	2006	}&	Febrero	&	 46.5 	&	 1,394.0 	&	 (1.9)	&	 3.2 	&  &	 84.8 	&	 2,543.9 	&	 (1.9)	&	 3.2 	\\
\multicolumn{1}{l}{	2006	}&	Marzo	&	 46.5 	&	 1,396.3 	&	 0.2 	&	 4.2 	&  &	 84.9 	&	 2,547.9 	&	 0.2 	&	 4.2 	\\
\rowcolor{color1!5!white}\multicolumn{1}{l}{	2006	}&	Abril	&	 47.1 	&	 1,414.3 	&	 1.3 	&	 5.2 	&  &	 86.0 	&	 2,580.8 	&	 1.3 	&	 5.2 	\\
\multicolumn{1}{l}{	2006	}&	Mayo	&	 47.2 	&	 1,416.3 	&	 0.1 	&	 5.9 	&  &	 86.1 	&	 2,584.5 	&	 0.1 	&	 5.9 	\\
\rowcolor{color1!5!white}\multicolumn{1}{l}{	2006	}&	Junio	&	 47.7 	&	 1,431.9 	&	 1.1 	&	 6.1 	&  &	 87.1 	&	 2,613.0 	&	 1.1 	&	 6.1 	\\
\multicolumn{1}{l}{	2006	}&	Julio	&	 47.9 	&	 1,438.1 	&	 0.4 	&	 5.8 	&  &	 87.5 	&	 2,624.3 	&	 0.4 	&	 5.8 	\\
\rowcolor{color1!5!white}\multicolumn{1}{l}{	2006	}&	Agosto	&	 47.6 	&	 1,428.6 	&	 (0.7)	&	 6.9 	&  &	 86.9 	&	 2,606.9 	&	 (0.7)	&	 6.9 	\\
\multicolumn{1}{l}{	2006	}&	Septiembre	&	 46.9 	&	 1,408.5 	&	 (1.4)	&	 2.5 	&  &	 85.7 	&	 2,570.2 	&	 (1.4)	&	 2.5 	\\
\rowcolor{color1!5!white}\multicolumn{1}{l}{	2006	}&	Octubre	&	 47.4 	&	 1,422.3 	&	 1.0 	&	 0.1 	&  &	 86.5 	&	 2,595.4 	&	 1.0 	&	 0.1 	\\
\multicolumn{1}{l}{	2006	}&	Noviembre	&	 48.2 	&	 1,447.5 	&	 1.8 	&	 2.0 	&  &	 88.0 	&	 2,641.4 	&	 1.8 	&	 2.0 	\\
\rowcolor{color1!5!white}\multicolumn{1}{l}{	2006	}&	Diciembre	&	 49.8 	&	 1,493.3 	&	 3.2 	&	 6.2 	&  &	 90.8 	&	 2,724.9 	&	 3.2 	&	 6.2 	\\
\multicolumn{1}{l}{	2007	}&	Enero	&	 50.6 	&	 1,518.6 	&	 1.7 	&	 6.9 	&  &	 92.4 	&	 2,771.2 	&	 1.7 	&	 6.9 	\\
\rowcolor{color1!5!white}\multicolumn{1}{l}{	2007	}&	Febrero	&	 50.4 	&	 1,512.2 	&	 (0.4)	&	 8.5 	&  &	 92.0 	&	 2,759.4 	&	 (0.4)	&	 8.5 	\\
\multicolumn{1}{l}{	2007	}&	Marzo	&	 51.1 	&	 1,532.7 	&	 1.4 	&	 9.8 	&  &	 93.2 	&	 2,796.8 	&	 1.4 	&	 9.8 	\\
\rowcolor{color1!5!white}\multicolumn{1}{l}{	2007	}&	Abril	&	 50.9 	&	 1,527.7 	&	 (0.3)	&	 8.0 	&  &	 92.9 	&	 2,787.8 	&	 (0.3)	&	 8.0 	\\
\multicolumn{1}{l}{	2007	}&	Mayo	&	 50.4 	&	 1,513.2 	&	 (1.0)	&	 6.8 	&  &	 92.0 	&	 2,761.2 	&	 (1.0)	&	 6.8 	\\
\rowcolor{color1!5!white}\multicolumn{1}{l}{	2007	}&	Junio	&	 50.8 	&	 1,525.4 	&	 0.8 	&	 6.5 	&  &	 92.8 	&	 2,783.7 	&	 0.8 	&	 6.5 	\\
\multicolumn{1}{l}{	2007	}&	Julio	&	 52.4 	&	 1,572.0 	&	 3.1 	&	 9.3 	&  &	 95.6 	&	 2,868.6 	&	 3.1 	&	 9.3 	\\
\rowcolor{color1!5!white}\multicolumn{1}{l}{	2007	}&	Agosto	&	 53.8 	&	 1,613.3 	&	 2.6 	&	 12.9 	&  &	 98.1 	&	 2,944.0 	&	 2.6 	&	 12.9 	\\
\multicolumn{1}{l}{	2007	}&	Septiembre	&	 54.8 	&	 1,644.2 	&	 1.9 	&	 16.7 	&  &	 100.0 	&	 3,000.3 	&	 1.9 	&	 16.7 	\\
\rowcolor{color1!5!white}\multicolumn{1}{l}{	2007	}&	Octubre	&	 54.4 	&	 1,631.6 	&	 (0.8)	&	 14.7 	&  &	 99.2 	&	 2,977.4 	&	 (0.8)	&	 14.7 	\\
\multicolumn{1}{l}{	2007	}&	Noviembre	&	 56.0 	&	 1,681.1 	&	 3.0 	&	 16.1 	&  &	 102.3 	&	 3,067.7 	&	 3.0 	&	 16.1 	\\
\rowcolor{color1!5!white}\multicolumn{1}{l}{	2007	}&	Diciembre	&	 55.4 	&	 1,662.6 	&	 (1.1)	&	 11.3 	&  &	 101.1 	&	 3,033.9 	&	 (1.1)	&	 11.3 	\\
\multicolumn{1}{l}{	2008	}&	Enero	&	 56.2 	&	 1,685.0 	&	 1.3 	&	 11.0 	&  &	 102.5 	&	 3,074.8 	&	 1.3 	&	 11.0 	\\
\rowcolor{color1!5!white}\multicolumn{1}{l}{	2008	}&	Febrero	&	 55.9 	&	 1,677.2 	&	 (0.5)	&	 10.9 	&  &	 102.0 	&	 3,060.6 	&	 (0.5)	&	 10.9 	\\
\multicolumn{1}{l}{	2008	}&	Marzo	&	 58.2 	&	 1,745.3 	&	 4.1 	&	 13.9 	&  &	 106.2 	&	 3,184.8 	&	 4.1 	&	 13.9 	\\
\rowcolor{color1!5!white}\multicolumn{1}{l}{	2008	}&	Abril	&	 59.2 	&	 1,776.0 	&	 1.8 	&	 16.3 	&  &	 108.0 	&	 3,240.9 	&	 1.8 	&	 16.3 	\\
\multicolumn{1}{l}{	2008	}&	Mayo	&	 60.6 	&	 1,817.0 	&	 2.3 	&	 20.1 	&  &	 110.5 	&	 3,315.7 	&	 2.3 	&	 20.1 	\\
\rowcolor{color1!5!white}\multicolumn{1}{l}{	2008	}&	Junio	&	 62.4 	&	 1,872.8 	&	 3.1 	&	 22.8 	&  &	 113.9 	&	 3,417.6 	&	 3.1 	&	 22.8 	\\
\multicolumn{1}{l}{	2008	}&	Julio	&	 64.6 	&	 1,939.1 	&	 3.5 	&	 23.3 	&  &	 117.9 	&	 3,538.4 	&	 3.5 	&	 23.3 	\\
\rowcolor{color1!5!white}\multicolumn{1}{l}{	2008	}&	Agosto	&	 65.2 	&	 1,954.7 	&	 0.8 	&	 21.2 	&  &	 118.9 	&	 3,566.9 	&	 0.8 	&	 21.2 	\\
\multicolumn{1}{l}{	2008	}&	Septiembre	&	 64.7 	&	 1,939.2 	&	 (0.8)	&	 17.9 	&  &	 118.0 	&	 3,538.7 	&	 (0.8)	&	 17.9 	\\
\rowcolor{color1!5!white}\multicolumn{1}{l}{	2008	}&	Octubre	&	 65.3 	&	 1,958.1 	&	 1.0 	&	 20.0 	&  &	 119.1 	&	 3,573.1 	&	 1.0 	&	 20.0 	\\
\multicolumn{1}{l}{	2008	}&	Noviembre	&	 65.8 	&	 1,974.7 	&	 0.8 	&	 17.5 	&  &	 120.1 	&	 3,603.5 	&	 0.8 	&	 17.5 	\\
\rowcolor{color1!5!white}\multicolumn{1}{l}{	2008	}&	Diciembre	&	 65.9 	&	 1,976.1 	&	 0.1 	&	 18.9 	&  &	 120.2 	&	 3,605.9 	&	 0.1 	&	 18.9 	\\
\multicolumn{1}{l}{	2009	}&	Enero	&	 66.3 	&	 1,989.1 	&	 0.7 	&	 18.1 	&  &	 121.0 	&	 3,629.8 	&	 0.7 	&	 18.0 	\\
\rowcolor{color1!5!white}\multicolumn{1}{l}{	2009	}&	Febrero	&	 65.9 	&	 1,978.1 	&	 (0.6)	&	 17.9 	&  &	 120.3 	&	 3,609.7 	&	 (0.6)	&	 17.9 	\\
\multicolumn{1}{l}{	2009	}&	Marzo	&	 65.9 	&	 1,976.8 	&	 (0.1)	&	 13.3 	&  &	 120.2 	&	 3,607.3 	&	 (0.1)	&	 13.3 	\\
\rowcolor{color1!5!white}\multicolumn{1}{l}{	2009	}&	Abril	&	 65.6 	&	 1,968.1 	&	 (0.4)	&	 10.8 	&  &	 119.7 	&	 3,591.4 	&	 (0.4)	&	 10.8 	\\
\multicolumn{1}{l}{	2009	}&	Mayo	&	 65.1 	&	 1,952.5 	&	 (0.8)	&	 7.5 	&  &	 118.8 	&	 3,563.0 	&	 (0.8)	&	 7.5 	\\
\rowcolor{color1!5!white}\multicolumn{1}{l}{	2009	}&	Junio	&	 65.2 	&	 1,955.2 	&	 0.1 	&	 4.4 	&  &	 118.9 	&	 3,568.0 	&	 0.1 	&	 4.4 	\\
\multicolumn{1}{l}{	2009	}&	Julio	&	 65.3 	&	 1,958.0 	&	 0.1 	&	 1.0 	&  &	 119.1 	&	 3,573.1 	&	 0.1 	&	 1.0 	\\
\rowcolor{color1!5!white}\multicolumn{1}{l}{	2009	}&	Agosto	&	 64.7 	&	 1,940.3 	&	 (0.9)	&	 (0.7)	&  &	 118.0 	&	 3,540.6 	&	 (0.9)	&	 (0.7)	\\
\multicolumn{1}{l}{	2009	}&	Septiembre	&	 65.1 	&	 1,952.3 	&	 0.6 	&	 0.7 	&  &	 118.8 	&	 3,562.5 	&	 0.6 	&	 0.7 	\\
\rowcolor{color1!5!white}\multicolumn{1}{l}{	2009	}&	Octubre	&	 65.1 	&	 1,951.8 	&	 (0.0)	&	 (0.3)	&  &	 118.7 	&	 3,561.7 	&	 (0.0)	&	 (0.3)	\\
\multicolumn{1}{l}{	2009	}&	Noviembre	&	 63.9 	&	 1,917.3 	&	 (1.8)	&	 (2.9)	&  &	 116.6 	&	 3,498.8 	&	 (1.8)	&	 (2.9)	\\
\rowcolor{color1!5!white}\multicolumn{1}{l}{	2009	}&	Diciembre	&	 63.2 	&	 1,897.3 	&	 (1.0)	&	 (4.0)	&  &	 115.4 	&	 3,462.3 	&	 (1.0)	&	 (4.0)	\\
\multicolumn{1}{l}{	2010	}&	Enero	&	 64.6 	&	 1,938.3 	&	 2.2 	&	 (2.6)	&  &	 117.9 	&	 3,537.0 	&	 2.2 	&	 (2.6)	\\
\rowcolor{color1!5!white}\multicolumn{1}{l}{	2010	}&	Febrero	&	 65.2 	&	 1,955.1 	&	 0.9 	&	 (1.2)	&  &	 118.9 	&	 3,567.6 	&	 0.9 	&	 (1.2)	\\
\multicolumn{1}{l}{	2010	}&	Marzo	&	 66.8 	&	 2,003.9 	&	 2.5 	&	 1.4 	&  &	 121.9 	&	 3,656.7 	&	 2.5 	&	 1.4 	\\
\rowcolor{color1!5!white}\multicolumn{1}{l}{	2010	}&	Abril	&	 66.5 	&	 1,996.1 	&	 (0.4)	&	 1.4 	&  &	 121.4 	&	 3,642.6 	&	 (0.4)	&	 1.4 	\\
\multicolumn{1}{l}{	2010	}&	Mayo	&	 66.4 	&	 1,992.6 	&	 (0.2)	&	 2.1 	&  &	 121.2 	&	 3,636.2 	&	 (0.2)	&	 2.1 	\\
\rowcolor{color1!5!white}\multicolumn{1}{l}{	2010	}&	Junio	&	 67.5 	&	 2,024.7 	&	 1.6 	&	 3.6 	&  &	 123.2 	&	 3,694.7 	&	 1.6 	&	 3.6 	\\
\multicolumn{1}{l}{	2010	}&	Julio	&	 67.8 	&	 2,034.6 	&	 0.5 	&	 3.9 	&  &	 123.8 	&	 3,712.8 	&	 0.5 	&	 3.9 	\\
\rowcolor{color1!5!white}\multicolumn{1}{l}{	2010	}&	Agosto	&	 67.2 	&	 2,017.2 	&	 (0.9)	&	 4.0 	&  &	 122.7 	&	 3,681.0 	&	 (0.9)	&	 4.0 	\\
\multicolumn{1}{l}{	2010	}&	Septiembre	&	 67.7 	&	 2,030.1 	&	 0.6 	&	 4.0 	&  &	 123.5 	&	 3,704.6 	&	 0.6 	&	 4.0 	\\
\rowcolor{color1!5!white}\multicolumn{1}{l}{	2010	}&	Octubre	&	 69.6 	&	 2,089.2 	&	 2.9 	&	 7.0 	&  &	 127.1 	&	 3,812.4 	&	 2.9 	&	 7.0 	\\
\multicolumn{1}{l}{	2010	}&	Noviembre	&	 71.3 	&	 2,138.4 	&	 2.4 	&	 11.5 	&  &	 130.1 	&	 3,902.2 	&	 2.4 	&	 11.5 	\\
\rowcolor{color1!5!white}\multicolumn{1}{l}{	2010	}&	Diciembre	&	 71.6 	&	 2,149.2 	&	 0.5 	&	 13.3 	&  &	 130.7 	&	 3,921.9 	&	 0.5 	&	 13.3 	\\
\multicolumn{1}{l}{	2011	}&	Enero	&	 72.4 	&	 2,172.6 	&	 1.1 	&	 12.1 	&  &	 132.2 	&	 3,964.6 	&	 1.1 	&	 12.1 	\\
\rowcolor{color1!5!white}\multicolumn{1}{l}{	2011	}&	Febrero	&	 72.9 	&	 2,187.3 	&	 0.7 	&	 11.9 	&  &	 133.0 	&	 3,991.4 	&	 0.7 	&	 11.9 	\\
\multicolumn{1}{l}{	2011	}&	Marzo	&	 74.2 	&	 2,224.8 	&	 1.7 	&	 11.0 	&  &	 135.3 	&	 4,059.9 	&	 1.7 	&	 11.0 	\\
\rowcolor{color1!5!white}\multicolumn{1}{l}{	2011	}&	Abril	&	 75.1 	&	 2,252.7 	&	 1.3 	&	 12.9 	&  &	 137.0 	&	 4,110.8 	&	 1.3 	&	 12.9 	\\
\multicolumn{1}{l}{	2011	}&	Mayo	&	 75.4 	&	 2,262.0 	&	 0.4 	&	 13.5 	&  &	 137.6 	&	 4,127.7 	&	 0.4 	&	 13.5 	\\
\rowcolor{color1!5!white}\multicolumn{1}{l}{	2011	}&	Junio	&	 76.9 	&	 2,307.6 	&	 2.0 	&	 14.0 	&  &	 140.4 	&	 4,211.0 	&	 2.0 	&	 14.0 	\\
\multicolumn{1}{l}{	2011	}&	Julio	&	 79.2 	&	 2,376.0 	&	 3.0 	&	 16.8 	&  &	 144.5 	&	 4,335.8 	&	 3.0 	&	 16.8 	\\
\rowcolor{color1!5!white}\multicolumn{1}{l}{	2011	}&	Agosto	&	 80.1 	&	 2,403.3 	&	 1.1 	&	 19.1 	&  &	 146.2 	&	 4,385.6 	&	 1.1 	&	 19.1 	\\
\multicolumn{1}{l}{	2011	}&	Septiembre	&	 79.5 	&	 2,383.5 	&	 (0.8)	&	 17.4 	&  &	 145.0 	&	 4,349.5 	&	 (0.8)	&	 17.4 	\\
\rowcolor{color1!5!white}\multicolumn{1}{l}{	2011	}&	Octubre	&	 79.9 	&	 2,397.3 	&	 0.6 	&	 14.7 	&  &	 145.8 	&	 4,374.6 	&	 0.6 	&	 14.7 	\\
\multicolumn{1}{l}{	2011	}&	Noviembre	&	 80.5 	&	 2,415.6 	&	 0.8 	&	 13.0 	&  &	 146.9 	&	 4,408.0 	&	 0.8 	&	 13.0 	\\
\rowcolor{color1!5!white}\multicolumn{1}{l}{	2011	}&	Diciembre	&	 81.3 	&	 2,440.2 	&	 1.0 	&	 13.5 	&  &	 148.4 	&	 4,452.9 	&	 1.0 	&	 13.5 	\\
\multicolumn{1}{l}{	2012	}&	Enero	&	 81.7 	&	 2,449.8 	&	 0.4 	&	 12.8 	&  &	 149.0 	&	 4,470.4 	&	 0.4 	&	 12.8 	\\
\rowcolor{color1!5!white}\multicolumn{1}{l}{	2012	}&	Febrero	&	 83.1 	&	 2,494.2 	&	 1.8 	&	 14.0 	&  &	 151.7 	&	 4,551.5 	&	 1.8 	&	 14.0 	\\
\multicolumn{1}{l}{	2012	}&	Marzo	&	 83.4 	&	 2,501.1 	&	 0.3 	&	 12.4 	&  &	 152.1 	&	 4,564.1 	&	 0.3 	&	 12.4 	\\
\rowcolor{color1!5!white}\multicolumn{1}{l}{	2012	}&	Abril	&	 83.8 	&	 2,513.1 	&	 0.5 	&	 11.6 	&  &	 152.9 	&	 4,586.0 	&	 0.5 	&	 11.6 	\\
\multicolumn{1}{l}{	2012	}&	Mayo	&	 84.1 	&	 2,522.1 	&	 0.4 	&	 11.5 	&  &	 153.4 	&	 4,602.4 	&	 0.4 	&	 11.5 	\\
\rowcolor{color1!5!white}\multicolumn{1}{l}{	2012	}&	Junio	&	 84.5 	&	 2,534.1 	&	 0.5 	&	 9.8 	&  &	 154.1 	&	 4,624.3 	&	 0.5 	&	 9.8 	\\
\multicolumn{1}{l}{	2012	}&	Julio	&	 85.3 	&	 2,558.4 	&	 1.0 	&	 7.7 	&  &	 155.6 	&	 4,668.6 	&	 1.0 	&	 7.7 	\\
\rowcolor{color1!5!white}\multicolumn{1}{l}{	2012	}&	Agosto	&	 85.5 	&	 2,565.9 	&	 0.3 	&	 6.8 	&  &	 156.1 	&	 4,682.3 	&	 0.3 	&	 6.8 	\\
\multicolumn{1}{l}{	2012	}&	Septiembre	&	 86.2 	&	 2,585.1 	&	 0.8 	&	 8.5 	&  &	 157.2 	&	 4,717.3 	&	 0.8 	&	 8.5 	\\
\rowcolor{color1!5!white}\multicolumn{1}{l}{	2012	}&	Octubre	&	 86.6 	&	 2,596.8 	&	 0.5 	&	 8.3 	&  &	 158.0 	&	 4,738.7 	&	 0.5 	&	 8.3 	\\
\multicolumn{1}{l}{	2012	}&	Noviembre	&	 87.0 	&	 2,609.1 	&	 0.5 	&	 8.0 	&  &	 158.7 	&	 4,761.1 	&	 0.5 	&	 8.0 	\\
\rowcolor{color1!5!white}\multicolumn{1}{l}{	2012	}&	Diciembre	&	 87.3 	&	 2,617.8 	&	 0.3 	&	 7.3 	&  &	 159.2 	&	 4,777.0 	&	 0.3 	&	 7.3 	\\
\multicolumn{1}{l}{	2013	}&	Enero	&	 88.0 	&	 2,639.4 	&	 0.8 	&	 7.7 	&  &	 160.5 	&	 4,816.4 	&	 0.8 	&	 7.7 	\\
\rowcolor{color1!5!white}\multicolumn{1}{l}{	2013	}&	Febrero	&	 89.8 	&	 2,692.5 	&	 2.0 	&	 8.0 	&  &	 163.8 	&	 4,913.3 	&	 2.0 	&	 8.0 	\\
\multicolumn{1}{l}{	2013	}&	Marzo	&	 90.8 	&	 2,723.7 	&	 1.2 	&	 8.9 	&  &	 165.7 	&	 4,970.3 	&	 1.2 	&	 8.9 	\\
\rowcolor{color1!5!white}\multicolumn{1}{l}{	2013	}&	Abril	&	 91.0 	&	 2,730.9 	&	 0.3 	&	 8.7 	&  &	 166.1 	&	 4,983.4 	&	 0.3 	&	 8.7 	\\
\multicolumn{1}{l}{	2013	}&	Mayo	&	 92.4 	&	 2,772.0 	&	 1.5 	&	 9.9 	&  &	 168.6 	&	 5,058.4 	&	 1.5 	&	 9.9 	\\
\rowcolor{color1!5!white}\multicolumn{1}{l}{	2013	}&	Junio	&	 94.3 	&	 2,829.0 	&	 2.1 	&	 11.6 	&  &	 172.1 	&	 5,162.4 	&	 2.1 	&	 11.6 	\\
\multicolumn{1}{l}{	2013	}&	Julio	&	 94.3 	&	 2,829.3 	&	 0.0 	&	 10.6 	&  &	 172.1 	&	 5,163.0 	&	 0.0 	&	 10.6 	\\
\rowcolor{color1!5!white}\multicolumn{1}{l}{	2013	}&	Agosto	&	 94.0 	&	 2,821.2 	&	 (0.3)	&	 10.0 	&  &	 171.6 	&	 5,148.2 	&	 (0.3)	&	 10.0 	\\
\multicolumn{1}{l}{	2013	}&	Septiembre	&	 94.6 	&	 2,838.0 	&	 0.6 	&	 9.8 	&  &	 172.6 	&	 5,178.8 	&	 0.6 	&	 9.8 	\\
\rowcolor{color1!5!white}\multicolumn{1}{l}{	2013	}&	Octubre	&	 94.7 	&	 2,841.3 	&	 0.1 	&	 9.4 	&  &	 172.8 	&	 5,184.9 	&	 0.1 	&	 9.4 	\\
\multicolumn{1}{l}{	2013	}&	Noviembre	&	 96.6 	&	 2,896.5 	&	 1.9 	&	 11.0 	&  &	 176.2 	&	 5,285.6 	&	 1.9 	&	 11.0 	\\
\rowcolor{color1!5!white}\multicolumn{1}{l}{	2013	}&	Diciembre	&	 96.7 	&	 2,900.1 	&	 0.1 	&	 10.8 	&  &	 176.4 	&	 5,292.2 	&	 0.1 	&	 10.8 	\\
\multicolumn{1}{l}{	2014	}&	Enero	&	 97.4 	&	 2,922.3 	&	 0.8 	&	 10.7 	&  &	 177.8 	&	 5,332.7 	&	 0.8 	&	 10.7 	\\
\rowcolor{color1!5!white}\multicolumn{1}{l}{	2014	}&	Febrero	&	 97.7 	&	 2,929.5 	&	 0.3 	&	 8.8 	&  &	 178.2 	&	 5,345.8 	&	 0.3 	&	 8.8 	\\
\multicolumn{1}{l}{	2014	}&	Marzo	&	 98.2 	&	 2,945.1 	&	 0.5 	&	 8.1 	&  &	 179.1 	&	 5,374.3 	&	 0.5 	&	 8.1 	\\
\rowcolor{color1!5!white}\multicolumn{1}{l}{	2014	}&	Abril	&	 98.9 	&	 2,966.4 	&	 0.7 	&	 8.6 	&  &	 180.4 	&	 5,413.1 	&	 0.7 	&	 8.6 	\\
\multicolumn{1}{l}{	2014	}&	Mayo	&	 99.4 	&	 2,982.0 	&	 0.5 	&	 7.6 	&  &	 181.4 	&	 5,441.6 	&	 0.5 	&	 7.6 	\\
\rowcolor{color1!5!white}\multicolumn{1}{l}{	2014	}&	Junio	&	 100.4 	&	 3,012.3 	&	 1.0 	&	 6.5 	&  &	 183.2 	&	 5,496.9 	&	 1.0 	&	 6.5 	\\
\multicolumn{1}{l}{	2014	}&	Julio	&	 101.3 	&	 3,039.0 	&	 0.9 	&	 7.4 	&  &	 184.9 	&	 5,545.6 	&	 0.9 	&	 7.4 	\\
\rowcolor{color1!5!white}\multicolumn{1}{l}{	2014	}&	Agosto	&	 102.8 	&	 3,084.6 	&	 1.5 	&	 9.3 	&  &	 187.6 	&	 5,628.8 	&	 1.5 	&	 9.3 	\\
\multicolumn{1}{l}{	2014	}&	Septiembre	&	 104.1 	&	 3,123.6 	&	 1.3 	&	 10.1 	&  &	 190.0 	&	 5,700.0 	&	 1.3 	&	 10.1 	\\
\rowcolor{color1!5!white}\multicolumn{1}{l}{	2014	}&	Octubre	&	 106.5 	&	 3,193.5 	&	 2.2 	&	 12.4 	&  &	 194.3 	&	 5,827.6 	&	 2.2 	&	 12.4 	\\
\multicolumn{1}{l}{	2014	}&	Noviembre	&	 107.3 	&	 3,218.1 	&	 0.8 	&	 11.1 	&  &	 195.8 	&	 5,872.5 	&	 0.8 	&	 11.1 	\\
\rowcolor{color1!5!white}\multicolumn{1}{l}{	2014	}&	Diciembre	&	 107.9 	&	 3,236.7 	&	 0.6 	&	 11.6 	&  &	 196.9 	&	 5,906.4 	&	 0.6 	&	 11.6 	\\
\multicolumn{1}{l}{	2015	}&	Enero	&	 108.2 	&	 3,247.2 	&	 0.3 	&	 11.1 	&  &	 197.5 	&	 5,925.6 	&	 0.3 	&	 11.1 	\\
\rowcolor{color1!5!white}\multicolumn{1}{l}{	2015	}&	Febrero	&	 109.4 	&	 3,281.7 	&	 1.1 	&	 12.0 	&  &	 199.6 	&	 5,988.5 	&	 1.1 	&	 12.0 	\\
\multicolumn{1}{l}{	2015	}&	Marzo	&	 109.4 	&	 3,282.3 	&	 0.0 	&	 11.5 	&  &	 199.7 	&	 5,989.6 	&	 0.0 	&	 11.5 	\\
\rowcolor{color1!5!white}\multicolumn{1}{l}{	2015	}&	Abril	&	 110.5 	&	 3,315.0 	&	 1.0 	&	 11.8 	&  &	 201.6 	&	 6,049.3 	&	 1.0 	&	 11.8 	\\
\multicolumn{1}{l}{	2015	}&	Mayo	&	 112.0 	&	 3,358.5 	&	 1.3 	&	 12.6 	&  &	 204.3 	&	 6,128.7 	&	 1.3 	&	 12.6 	\\
\rowcolor{color1!5!white}\multicolumn{1}{l}{	2015	}&	Junio	&	 113.5 	&	 3,405.6 	&	 1.4 	&	 13.1 	&  &	 207.2 	&	 6,214.6 	&	 1.4 	&	 13.1 	\\
\multicolumn{1}{l}{	2015	}&	Julio	&	 114.1 	&	 3,422.7 	&	 0.5 	&	 12.6 	&  &	 208.2 	&	 6,245.8 	&	 0.5 	&	 12.6 	\\
\rowcolor{color1!5!white}\multicolumn{1}{l}{	2015	}&	Agosto	&	 114.0 	&	 3,420.9 	&	 (0.1)	&	 10.9 	&  &	 208.1 	&	 6,242.5 	&	 (0.1)	&	 10.9 	\\
\multicolumn{1}{l}{	2015	}&	Septiembre	&	 114.6 	&	 3,436.8 	&	 0.5 	&	 10.0 	&  &	 209.1 	&	 6,271.5 	&	 0.5 	&	 10.0 	\\
\rowcolor{color1!5!white}\multicolumn{1}{l}{	2015	}&	Octubre	&	 116.9 	&	 3,507.6 	&	 2.1 	&	 9.8 	&  &	 213.4 	&	 6,400.7 	&	 2.1 	&	 9.8 	\\
\multicolumn{1}{l}{	2015	}&	Noviembre	&	 118.0 	&	 3,540.6 	&	 0.9 	&	 10.0 	&  &	 215.4 	&	 6,461.0 	&	 0.9 	&	 10.0 	\\
\rowcolor{color1!5!white}\multicolumn{1}{l}{	2015	}&	Diciembre	&	 119.7 	&	 3,589.8 	&	 1.4 	&	 10.9 	&  &	 218.4 	&	 6,550.7 	&	 1.4 	&	 10.9 	\\

		\rowcolor{color1!0!white}\multicolumn{1}{l}{2006} &	Enero	&	 47.4 	&	 1,420.7 	&	 1.1 	&	 6.3 	&  &	 86.4 	&	 2,592.5 	&	 1.1 	&	 6.3 	\\
		\rowcolor{color1!5!white}\multicolumn{1}{l}{	2006	}&	Febrero	&	 46.5 	&	 1,394.0 	&	 (1.9)	&	 3.2 	&  &	 84.8 	&	 2,543.9 	&	 (1.9)	&	 3.2 	\\
		\multicolumn{1}{l}{	2006	}&	Marzo	&	 46.5 	&	 1,396.3 	&	 0.2 	&	 4.2 	&  &	 84.9 	&	 2,547.9 	&	 0.2 	&	 4.2 	\\
		\rowcolor{color1!5!white}\multicolumn{1}{l}{	2006	}&	Abril	&	 47.1 	&	 1,414.3 	&	 1.3 	&	 5.2 	&  &	 86.0 	&	 2,580.8 	&	 1.3 	&	 5.2 	\\
		\multicolumn{1}{l}{	2006	}&	Mayo	&	 47.2 	&	 1,416.3 	&	 0.1 	&	 5.9 	&  &	 86.1 	&	 2,584.5 	&	 0.1 	&	 5.9 	\\
		\rowcolor{color1!5!white}\multicolumn{1}{l}{	2006	}&	Junio	&	 47.7 	&	 1,431.9 	&	 1.1 	&	 6.1 	&  &	 87.1 	&	 2,613.0 	&	 1.1 	&	 6.1 	\\
		\multicolumn{1}{l}{	2006	}&	Julio	&	 47.9 	&	 1,438.1 	&	 0.4 	&	 5.8 	&  &	 87.5 	&	 2,624.3 	&	 0.4 	&	 5.8 	\\
		\rowcolor{color1!5!white}\multicolumn{1}{l}{	2006	}&	Agosto	&	 47.6 	&	 1,428.6 	&	 (0.7)	&	 6.9 	&  &	 86.9 	&	 2,606.9 	&	 (0.7)	&	 6.9 	\\
		\multicolumn{1}{l}{	2006	}&	Septiembre	&	 46.9 	&	 1,408.5 	&	 (1.4)	&	 2.5 	&  &	 85.7 	&	 2,570.2 	&	 (1.4)	&	 2.5 	\\
		\rowcolor{color1!5!white}\multicolumn{1}{l}{	2006	}&	Octubre	&	 47.4 	&	 1,422.3 	&	 1.0 	&	 0.1 	&  &	 86.5 	&	 2,595.4 	&	 1.0 	&	 0.1 	\\
		\multicolumn{1}{l}{	2006	}&	Noviembre	&	 48.2 	&	 1,447.5 	&	 1.8 	&	 2.0 	&  &	 88.0 	&	 2,641.4 	&	 1.8 	&	 2.0 	\\
		\rowcolor{color1!5!white}\multicolumn{1}{l}{	2006	}&	Diciembre	&	 49.8 	&	 1,493.3 	&	 3.2 	&	 6.2 	&  &	 90.8 	&	 2,724.9 	&	 3.2 	&	 6.2 	\\
		\multicolumn{1}{l}{	2007	}&	Enero	&	 50.6 	&	 1,518.6 	&	 1.7 	&	 6.9 	&  &	 92.4 	&	 2,771.2 	&	 1.7 	&	 6.9 	\\
		\rowcolor{color1!5!white}\multicolumn{1}{l}{	2007	}&	Febrero	&	 50.4 	&	 1,512.2 	&	 (0.4)	&	 8.5 	&  &	 92.0 	&	 2,759.4 	&	 (0.4)	&	 8.5 	\\
		\multicolumn{1}{l}{	2007	}&	Marzo	&	 51.1 	&	 1,532.7 	&	 1.4 	&	 9.8 	&  &	 93.2 	&	 2,796.8 	&	 1.4 	&	 9.8 	\\
		\rowcolor{color1!5!white}\multicolumn{1}{l}{	2007	}&	Abril	&	 50.9 	&	 1,527.7 	&	 (0.3)	&	 8.0 	&  &	 92.9 	&	 2,787.8 	&	 (0.3)	&	 8.0 	\\
		\multicolumn{1}{l}{	2007	}&	Mayo	&	 50.4 	&	 1,513.2 	&	 (1.0)	&	 6.8 	&  &	 92.0 	&	 2,761.2 	&	 (1.0)	&	 6.8 	\\
		\rowcolor{color1!5!white}\multicolumn{1}{l}{	2007	}&	Junio	&	 50.8 	&	 1,525.4 	&	 0.8 	&	 6.5 	&  &	 92.8 	&	 2,783.7 	&	 0.8 	&	 6.5 	\\
		\multicolumn{1}{l}{	2007	}&	Julio	&	 52.4 	&	 1,572.0 	&	 3.1 	&	 9.3 	&  &	 95.6 	&	 2,868.6 	&	 3.1 	&	 9.3 	\\
		\rowcolor{color1!5!white}\multicolumn{1}{l}{	2007	}&	Agosto	&	 53.8 	&	 1,613.3 	&	 2.6 	&	 12.9 	&  &	 98.1 	&	 2,944.0 	&	 2.6 	&	 12.9 	\\
		\multicolumn{1}{l}{	2007	}&	Septiembre	&	 54.8 	&	 1,644.2 	&	 1.9 	&	 16.7 	&  &	 100.0 	&	 3,000.3 	&	 1.9 	&	 16.7 	\\
		\rowcolor{color1!5!white}\multicolumn{1}{l}{	2007	}&	Octubre	&	 54.4 	&	 1,631.6 	&	 (0.8)	&	 14.7 	&  &	 99.2 	&	 2,977.4 	&	 (0.8)	&	 14.7 	\\
		\multicolumn{1}{l}{	2007	}&	Noviembre	&	 56.0 	&	 1,681.1 	&	 3.0 	&	 16.1 	&  &	 102.3 	&	 3,067.7 	&	 3.0 	&	 16.1 	\\
		\rowcolor{color1!5!white}\multicolumn{1}{l}{	2007	}&	Diciembre	&	 55.4 	&	 1,662.6 	&	 (1.1)	&	 11.3 	&  &	 101.1 	&	 3,033.9 	&	 (1.1)	&	 11.3 	\\
		\multicolumn{1}{l}{	2008	}&	Enero	&	 56.2 	&	 1,685.0 	&	 1.3 	&	 11.0 	&  &	 102.5 	&	 3,074.8 	&	 1.3 	&	 11.0 	\\
		\rowcolor{color1!5!white}\multicolumn{1}{l}{	2008	}&	Febrero	&	 55.9 	&	 1,677.2 	&	 (0.5)	&	 10.9 	&  &	 102.0 	&	 3,060.6 	&	 (0.5)	&	 10.9 	\\
		\multicolumn{1}{l}{	2008	}&	Marzo	&	 58.2 	&	 1,745.3 	&	 4.1 	&	 13.9 	&  &	 106.2 	&	 3,184.8 	&	 4.1 	&	 13.9 	\\
		\rowcolor{color1!5!white}\multicolumn{1}{l}{	2008	}&	Abril	&	 59.2 	&	 1,776.0 	&	 1.8 	&	 16.3 	&  &	 108.0 	&	 3,240.9 	&	 1.8 	&	 16.3 	\\
		\multicolumn{1}{l}{	2008	}&	Mayo	&	 60.6 	&	 1,817.0 	&	 2.3 	&	 20.1 	&  &	 110.5 	&	 3,315.7 	&	 2.3 	&	 20.1 	\\
		\rowcolor{color1!5!white}\multicolumn{1}{l}{	2008	}&	Junio	&	 62.4 	&	 1,872.8 	&	 3.1 	&	 22.8 	&  &	 113.9 	&	 3,417.6 	&	 3.1 	&	 22.8 	\\
		\multicolumn{1}{l}{	2008	}&	Julio	&	 64.6 	&	 1,939.1 	&	 3.5 	&	 23.3 	&  &	 117.9 	&	 3,538.4 	&	 3.5 	&	 23.3 	\\
		\rowcolor{color1!5!white}\multicolumn{1}{l}{	2008	}&	Agosto	&	 65.2 	&	 1,954.7 	&	 0.8 	&	 21.2 	&  &	 118.9 	&	 3,566.9 	&	 0.8 	&	 21.2 	\\
		\multicolumn{1}{l}{	2008	}&	Septiembre	&	 64.7 	&	 1,939.2 	&	 (0.8)	&	 17.9 	&  &	 118.0 	&	 3,538.7 	&	 (0.8)	&	 17.9 	\\
		\rowcolor{color1!5!white}\multicolumn{1}{l}{	2008	}&	Octubre	&	 65.3 	&	 1,958.1 	&	 1.0 	&	 20.0 	&  &	 119.1 	&	 3,573.1 	&	 1.0 	&	 20.0 	\\
		\multicolumn{1}{l}{	2008	}&	Noviembre	&	 65.8 	&	 1,974.7 	&	 0.8 	&	 17.5 	&  &	 120.1 	&	 3,603.5 	&	 0.8 	&	 17.5 	\\
		\rowcolor{color1!5!white}\multicolumn{1}{l}{	2008	}&	Diciembre	&	 65.9 	&	 1,976.1 	&	 0.1 	&	 18.9 	&  &	 120.2 	&	 3,605.9 	&	 0.1 	&	 18.9 	\\
		\multicolumn{1}{l}{	2009	}&	Enero	&	 66.3 	&	 1,989.1 	&	 0.7 	&	 18.1 	&  &	 121.0 	&	 3,629.8 	&	 0.7 	&	 18.0 	\\
		\rowcolor{color1!5!white}\multicolumn{1}{l}{	2009	}&	Febrero	&	 65.9 	&	 1,978.1 	&	 (0.6)	&	 17.9 	&  &	 120.3 	&	 3,609.7 	&	 (0.6)	&	 17.9 	\\
		\multicolumn{1}{l}{	2009	}&	Marzo	&	 65.9 	&	 1,976.8 	&	 (0.1)	&	 13.3 	&  &	 120.2 	&	 3,607.3 	&	 (0.1)	&	 13.3 	\\
		\rowcolor{color1!5!white}\multicolumn{1}{l}{	2009	}&	Abril	&	 65.6 	&	 1,968.1 	&	 (0.4)	&	 10.8 	&  &	 119.7 	&	 3,591.4 	&	 (0.4)	&	 10.8 	\\
		\multicolumn{1}{l}{	2009	}&	Mayo	&	 65.1 	&	 1,952.5 	&	 (0.8)	&	 7.5 	&  &	 118.8 	&	 3,563.0 	&	 (0.8)	&	 7.5 	\\
		\rowcolor{color1!5!white}\multicolumn{1}{l}{	2009	}&	Junio	&	 65.2 	&	 1,955.2 	&	 0.1 	&	 4.4 	&  &	 118.9 	&	 3,568.0 	&	 0.1 	&	 4.4 	\\
		\multicolumn{1}{l}{	2009	}&	Julio	&	 65.3 	&	 1,958.0 	&	 0.1 	&	 1.0 	&  &	 119.1 	&	 3,573.1 	&	 0.1 	&	 1.0 	\\
		\rowcolor{color1!5!white}\multicolumn{1}{l}{	2009	}&	Agosto	&	 64.7 	&	 1,940.3 	&	 (0.9)	&	 (0.7)	&  &	 118.0 	&	 3,540.6 	&	 (0.9)	&	 (0.7)	\\
		\multicolumn{1}{l}{	2009	}&	Septiembre	&	 65.1 	&	 1,952.3 	&	 0.6 	&	 0.7 	&  &	 118.8 	&	 3,562.5 	&	 0.6 	&	 0.7 	\\
		\rowcolor{color1!5!white}\multicolumn{1}{l}{	2009	}&	Octubre	&	 65.1 	&	 1,951.8 	&	 (0.0)	&	 (0.3)	&  &	 118.7 	&	 3,561.7 	&	 (0.0)	&	 (0.3)	\\
		\multicolumn{1}{l}{	2009	}&	Noviembre	&	 63.9 	&	 1,917.3 	&	 (1.8)	&	 (2.9)	&  &	 116.6 	&	 3,498.8 	&	 (1.8)	&	 (2.9)	\\
		\rowcolor{color1!5!white}\multicolumn{1}{l}{	2009	}&	Diciembre	&	 63.2 	&	 1,897.3 	&	 (1.0)	&	 (4.0)	&  &	 115.4 	&	 3,462.3 	&	 (1.0)	&	 (4.0)	\\
		\multicolumn{1}{l}{	2010	}&	Enero	&	 64.6 	&	 1,938.3 	&	 2.2 	&	 (2.6)	&  &	 117.9 	&	 3,537.0 	&	 2.2 	&	 (2.6)	\\
		\rowcolor{color1!5!white}\multicolumn{1}{l}{	2010	}&	Febrero	&	 65.2 	&	 1,955.1 	&	 0.9 	&	 (1.2)	&  &	 118.9 	&	 3,567.6 	&	 0.9 	&	 (1.2)	\\
		\multicolumn{1}{l}{	2010	}&	Marzo	&	 66.8 	&	 2,003.9 	&	 2.5 	&	 1.4 	&  &	 121.9 	&	 3,656.7 	&	 2.5 	&	 1.4 	\\
		\rowcolor{color1!5!white}\multicolumn{1}{l}{	2010	}&	Abril	&	 66.5 	&	 1,996.1 	&	 (0.4)	&	 1.4 	&  &	 121.4 	&	 3,642.6 	&	 (0.4)	&	 1.4 	\\
		\multicolumn{1}{l}{	2010	}&	Mayo	&	 66.4 	&	 1,992.6 	&	 (0.2)	&	 2.1 	&  &	 121.2 	&	 3,636.2 	&	 (0.2)	&	 2.1 	\\
		\rowcolor{color1!5!white}\multicolumn{1}{l}{	2010	}&	Junio	&	 67.5 	&	 2,024.7 	&	 1.6 	&	 3.6 	&  &	 123.2 	&	 3,694.7 	&	 1.6 	&	 3.6 	\\
		\multicolumn{1}{l}{	2010	}&	Julio	&	 67.8 	&	 2,034.6 	&	 0.5 	&	 3.9 	&  &	 123.8 	&	 3,712.8 	&	 0.5 	&	 3.9 	\\
		\rowcolor{color1!5!white}\multicolumn{1}{l}{	2010	}&	Agosto	&	 67.2 	&	 2,017.2 	&	 (0.9)	&	 4.0 	&  &	 122.7 	&	 3,681.0 	&	 (0.9)	&	 4.0 	\\
		\multicolumn{1}{l}{	2010	}&	Septiembre	&	 67.7 	&	 2,030.1 	&	 0.6 	&	 4.0 	&  &	 123.5 	&	 3,704.6 	&	 0.6 	&	 4.0 	\\
		\rowcolor{color1!5!white}\multicolumn{1}{l}{	2010	}&	Octubre	&	 69.6 	&	 2,089.2 	&	 2.9 	&	 7.0 	&  &	 127.1 	&	 3,812.4 	&	 2.9 	&	 7.0 	\\
		\multicolumn{1}{l}{	2010	}&	Noviembre	&	 71.3 	&	 2,138.4 	&	 2.4 	&	 11.5 	&  &	 130.1 	&	 3,902.2 	&	 2.4 	&	 11.5 	\\
		\rowcolor{color1!5!white}\multicolumn{1}{l}{	2010	}&	Diciembre	&	 71.6 	&	 2,149.2 	&	 0.5 	&	 13.3 	&  &	 130.7 	&	 3,921.9 	&	 0.5 	&	 13.3 	\\
		\multicolumn{1}{l}{	2011	}&	Enero	&	 72.4 	&	 2,172.6 	&	 1.1 	&	 12.1 	&  &	 132.2 	&	 3,964.6 	&	 1.1 	&	 12.1 	\\
		\rowcolor{color1!5!white}\multicolumn{1}{l}{	2011	}&	Febrero	&	 72.9 	&	 2,187.3 	&	 0.7 	&	 11.9 	&  &	 133.0 	&	 3,991.4 	&	 0.7 	&	 11.9 	\\
		\multicolumn{1}{l}{	2011	}&	Marzo	&	 74.2 	&	 2,224.8 	&	 1.7 	&	 11.0 	&  &	 135.3 	&	 4,059.9 	&	 1.7 	&	 11.0 	\\
		\rowcolor{color1!5!white}\multicolumn{1}{l}{	2011	}&	Abril	&	 75.1 	&	 2,252.7 	&	 1.3 	&	 12.9 	&  &	 137.0 	&	 4,110.8 	&	 1.3 	&	 12.9 	\\
		\multicolumn{1}{l}{	2011	}&	Mayo	&	 75.4 	&	 2,262.0 	&	 0.4 	&	 13.5 	&  &	 137.6 	&	 4,127.7 	&	 0.4 	&	 13.5 	\\
		\rowcolor{color1!5!white}\multicolumn{1}{l}{	2011	}&	Junio	&	 76.9 	&	 2,307.6 	&	 2.0 	&	 14.0 	&  &	 140.4 	&	 4,211.0 	&	 2.0 	&	 14.0 	\\
		\multicolumn{1}{l}{	2011	}&	Julio	&	 79.2 	&	 2,376.0 	&	 3.0 	&	 16.8 	&  &	 144.5 	&	 4,335.8 	&	 3.0 	&	 16.8 	\\
		\rowcolor{color1!5!white}\multicolumn{1}{l}{	2011	}&	Agosto	&	 80.1 	&	 2,403.3 	&	 1.1 	&	 19.1 	&  &	 146.2 	&	 4,385.6 	&	 1.1 	&	 19.1 	\\
		\multicolumn{1}{l}{	2011	}&	Septiembre	&	 79.5 	&	 2,383.5 	&	 (0.8)	&	 17.4 	&  &	 145.0 	&	 4,349.5 	&	 (0.8)	&	 17.4 	\\
		\rowcolor{color1!5!white}\multicolumn{1}{l}{	2011	}&	Octubre	&	 79.9 	&	 2,397.3 	&	 0.6 	&	 14.7 	&  &	 145.8 	&	 4,374.6 	&	 0.6 	&	 14.7 	\\
		\multicolumn{1}{l}{	2011	}&	Noviembre	&	 80.5 	&	 2,415.6 	&	 0.8 	&	 13.0 	&  &	 146.9 	&	 4,408.0 	&	 0.8 	&	 13.0 	\\
		\rowcolor{color1!5!white}\multicolumn{1}{l}{	2011	}&	Diciembre	&	 81.3 	&	 2,440.2 	&	 1.0 	&	 13.5 	&  &	 148.4 	&	 4,452.9 	&	 1.0 	&	 13.5 	\\
		\multicolumn{1}{l}{	2012	}&	Enero	&	 81.7 	&	 2,449.8 	&	 0.4 	&	 12.8 	&  &	 149.0 	&	 4,470.4 	&	 0.4 	&	 12.8 	\\
		\rowcolor{color1!5!white}\multicolumn{1}{l}{	2012	}&	Febrero	&	 83.1 	&	 2,494.2 	&	 1.8 	&	 14.0 	&  &	 151.7 	&	 4,551.5 	&	 1.8 	&	 14.0 	\\
		\multicolumn{1}{l}{	2012	}&	Marzo	&	 83.4 	&	 2,501.1 	&	 0.3 	&	 12.4 	&  &	 152.1 	&	 4,564.1 	&	 0.3 	&	 12.4 	\\
		\rowcolor{color1!5!white}\multicolumn{1}{l}{	2012	}&	Abril	&	 83.8 	&	 2,513.1 	&	 0.5 	&	 11.6 	&  &	 152.9 	&	 4,586.0 	&	 0.5 	&	 11.6 	\\
		\multicolumn{1}{l}{	2012	}&	Mayo	&	 84.1 	&	 2,522.1 	&	 0.4 	&	 11.5 	&  &	 153.4 	&	 4,602.4 	&	 0.4 	&	 11.5 	\\
		\rowcolor{color1!5!white}\multicolumn{1}{l}{	2012	}&	Junio	&	 84.5 	&	 2,534.1 	&	 0.5 	&	 9.8 	&  &	 154.1 	&	 4,624.3 	&	 0.5 	&	 9.8 	\\
		\multicolumn{1}{l}{	2012	}&	Julio	&	 85.3 	&	 2,558.4 	&	 1.0 	&	 7.7 	&  &	 155.6 	&	 4,668.6 	&	 1.0 	&	 7.7 	\\
		\rowcolor{color1!5!white}\multicolumn{1}{l}{	2012	}&	Agosto	&	 85.5 	&	 2,565.9 	&	 0.3 	&	 6.8 	&  &	 156.1 	&	 4,682.3 	&	 0.3 	&	 6.8 	\\
		\multicolumn{1}{l}{	2012	}&	Septiembre	&	 86.2 	&	 2,585.1 	&	 0.8 	&	 8.5 	&  &	 157.2 	&	 4,717.3 	&	 0.8 	&	 8.5 	\\
		\rowcolor{color1!5!white}\multicolumn{1}{l}{	2012	}&	Octubre	&	 86.6 	&	 2,596.8 	&	 0.5 	&	 8.3 	&  &	 158.0 	&	 4,738.7 	&	 0.5 	&	 8.3 	\\
		\multicolumn{1}{l}{	2012	}&	Noviembre	&	 87.0 	&	 2,609.1 	&	 0.5 	&	 8.0 	&  &	 158.7 	&	 4,761.1 	&	 0.5 	&	 8.0 	\\
		\rowcolor{color1!5!white}\multicolumn{1}{l}{	2012	}&	Diciembre	&	 87.3 	&	 2,617.8 	&	 0.3 	&	 7.3 	&  &	 159.2 	&	 4,777.0 	&	 0.3 	&	 7.3 	\\
		\multicolumn{1}{l}{	2013	}&	Enero	&	 88.0 	&	 2,639.4 	&	 0.8 	&	 7.7 	&  &	 160.5 	&	 4,816.4 	&	 0.8 	&	 7.7 	\\
		\rowcolor{color1!5!white}\multicolumn{1}{l}{	2013	}&	Febrero	&	 89.8 	&	 2,692.5 	&	 2.0 	&	 8.0 	&  &	 163.8 	&	 4,913.3 	&	 2.0 	&	 8.0 	\\
		\multicolumn{1}{l}{	2013	}&	Marzo	&	 90.8 	&	 2,723.7 	&	 1.2 	&	 8.9 	&  &	 165.7 	&	 4,970.3 	&	 1.2 	&	 8.9 	\\
		\rowcolor{color1!5!white}\multicolumn{1}{l}{	2013	}&	Abril	&	 91.0 	&	 2,730.9 	&	 0.3 	&	 8.7 	&  &	 166.1 	&	 4,983.4 	&	 0.3 	&	 8.7 	\\
		\multicolumn{1}{l}{	2013	}&	Mayo	&	 92.4 	&	 2,772.0 	&	 1.5 	&	 9.9 	&  &	 168.6 	&	 5,058.4 	&	 1.5 	&	 9.9 	\\
		\rowcolor{color1!5!white}\multicolumn{1}{l}{	2013	}&	Junio	&	 94.3 	&	 2,829.0 	&	 2.1 	&	 11.6 	&  &	 172.1 	&	 5,162.4 	&	 2.1 	&	 11.6 	\\
		\multicolumn{1}{l}{	2013	}&	Julio	&	 94.3 	&	 2,829.3 	&	 0.0 	&	 10.6 	&  &	 172.1 	&	 5,163.0 	&	 0.0 	&	 10.6 	\\
		\rowcolor{color1!5!white}\multicolumn{1}{l}{	2013	}&	Agosto	&	 94.0 	&	 2,821.2 	&	 (0.3)	&	 10.0 	&  &	 171.6 	&	 5,148.2 	&	 (0.3)	&	 10.0 	\\
		\multicolumn{1}{l}{	2013	}&	Septiembre	&	 94.6 	&	 2,838.0 	&	 0.6 	&	 9.8 	&  &	 172.6 	&	 5,178.8 	&	 0.6 	&	 9.8 	\\
		\rowcolor{color1!5!white}\multicolumn{1}{l}{	2013	}&	Octubre	&	 94.7 	&	 2,841.3 	&	 0.1 	&	 9.4 	&  &	 172.8 	&	 5,184.9 	&	 0.1 	&	 9.4 	\\
		\multicolumn{1}{l}{	2013	}&	Noviembre	&	 96.6 	&	 2,896.5 	&	 1.9 	&	 11.0 	&  &	 176.2 	&	 5,285.6 	&	 1.9 	&	 11.0 	\\
		\rowcolor{color1!5!white}\multicolumn{1}{l}{	2013	}&	Diciembre	&	 96.7 	&	 2,900.1 	&	 0.1 	&	 10.8 	&  &	 176.4 	&	 5,292.2 	&	 0.1 	&	 10.8 	\\
		\multicolumn{1}{l}{	2014	}&	Enero	&	 97.4 	&	 2,922.3 	&	 0.8 	&	 10.7 	&  &	 177.8 	&	 5,332.7 	&	 0.8 	&	 10.7 	\\
		\rowcolor{color1!5!white}\multicolumn{1}{l}{	2014	}&	Febrero	&	 97.7 	&	 2,929.5 	&	 0.3 	&	 8.8 	&  &	 178.2 	&	 5,345.8 	&	 0.3 	&	 8.8 	\\
		\multicolumn{1}{l}{	2014	}&	Marzo	&	 98.2 	&	 2,945.1 	&	 0.5 	&	 8.1 	&  &	 179.1 	&	 5,374.3 	&	 0.5 	&	 8.1 	\\
		\rowcolor{color1!5!white}\multicolumn{1}{l}{	2014	}&	Abril	&	 98.9 	&	 2,966.4 	&	 0.7 	&	 8.6 	&  &	 180.4 	&	 5,413.1 	&	 0.7 	&	 8.6 	\\
		\multicolumn{1}{l}{	2014	}&	Mayo	&	 99.4 	&	 2,982.0 	&	 0.5 	&	 7.6 	&  &	 181.4 	&	 5,441.6 	&	 0.5 	&	 7.6 	\\
		\rowcolor{color1!5!white}\multicolumn{1}{l}{	2014	}&	Junio	&	 100.4 	&	 3,012.3 	&	 1.0 	&	 6.5 	&  &	 183.2 	&	 5,496.9 	&	 1.0 	&	 6.5 	\\
		\multicolumn{1}{l}{	2014	}&	Julio	&	 101.3 	&	 3,039.0 	&	 0.9 	&	 7.4 	&  &	 184.9 	&	 5,545.6 	&	 0.9 	&	 7.4 	\\
		\rowcolor{color1!5!white}\multicolumn{1}{l}{	2014	}&	Agosto	&	 102.8 	&	 3,084.6 	&	 1.5 	&	 9.3 	&  &	 187.6 	&	 5,628.8 	&	 1.5 	&	 9.3 	\\
		\multicolumn{1}{l}{	2014	}&	Septiembre	&	 104.1 	&	 3,123.6 	&	 1.3 	&	 10.1 	&  &	 190.0 	&	 5,700.0 	&	 1.3 	&	 10.1 	\\
		\rowcolor{color1!5!white}\multicolumn{1}{l}{	2014	}&	Octubre	&	 106.5 	&	 3,193.5 	&	 2.2 	&	 12.4 	&  &	 194.3 	&	 5,827.6 	&	 2.2 	&	 12.4 	\\
		\multicolumn{1}{l}{	2014	}&	Noviembre	&	 107.3 	&	 3,218.1 	&	 0.8 	&	 11.1 	&  &	 195.8 	&	 5,872.5 	&	 0.8 	&	 11.1 	\\
		\rowcolor{color1!5!white}\multicolumn{1}{l}{	2014	}&	Diciembre	&	 107.9 	&	 3,236.7 	&	 0.6 	&	 11.6 	&  &	 196.9 	&	 5,906.4 	&	 0.6 	&	 11.6 	\\
		\multicolumn{1}{l}{	2015	}&	Enero	&	 108.2 	&	 3,247.2 	&	 0.3 	&	 11.1 	&  &	 197.5 	&	 5,925.6 	&	 0.3 	&	 11.1 	\\
		\rowcolor{color1!5!white}\multicolumn{1}{l}{	2015	}&	Febrero	&	 109.4 	&	 3,281.7 	&	 1.1 	&	 12.0 	&  &	 199.6 	&	 5,988.5 	&	 1.1 	&	 12.0 	\\
		\multicolumn{1}{l}{	2015	}&	Marzo	&	 109.4 	&	 3,282.3 	&	 0.0 	&	 11.5 	&  &	 199.7 	&	 5,989.6 	&	 0.0 	&	 11.5 	\\
		\rowcolor{color1!5!white}\multicolumn{1}{l}{	2015	}&	Abril	&	 110.5 	&	 3,315.0 	&	 1.0 	&	 11.8 	&  &	 201.6 	&	 6,049.3 	&	 1.0 	&	 11.8 	\\
		\multicolumn{1}{l}{	2015	}&	Mayo	&	 112.0 	&	 3,358.5 	&	 1.3 	&	 12.6 	&  &	 204.3 	&	 6,128.7 	&	 1.3 	&	 12.6 	\\
		\rowcolor{color1!5!white}\multicolumn{1}{l}{	2015	}&	Junio	&	 113.5 	&	 3,405.6 	&	 1.4 	&	 13.1 	&  &	 207.2 	&	 6,214.6 	&	 1.4 	&	 13.1 	\\
		\multicolumn{1}{l}{	2015	}&	Julio	&	 114.1 	&	 3,422.7 	&	 0.5 	&	 12.6 	&  &	 208.2 	&	 6,245.8 	&	 0.5 	&	 12.6 	\\
		\rowcolor{color1!5!white}\multicolumn{1}{l}{	2015	}&	Agosto	&	 114.0 	&	 3,420.9 	&	 (0.1)	&	 10.9 	&  &	 208.1 	&	 6,242.5 	&	 (0.1)	&	 10.9 	\\
		\multicolumn{1}{l}{	2015	}&	Septiembre	&	 114.6 	&	 3,436.8 	&	 0.5 	&	 10.0 	&  &	 209.1 	&	 6,271.5 	&	 0.5 	&	 10.0 	\\
		\rowcolor{color1!5!white}\multicolumn{1}{l}{	2015	}&	Octubre	&	 116.9 	&	 3,507.6 	&	 2.1 	&	 9.8 	&  &	 213.4 	&	 6,400.7 	&	 2.1 	&	 9.8 	\\
		\multicolumn{1}{l}{	2015	}&	Noviembre	&	 118.0 	&	 3,540.6 	&	 0.9 	&	 10.0 	&  &	 215.4 	&	 6,461.0 	&	 0.9 	&	 10.0 	\\
		\rowcolor{color1!5!white}\multicolumn{1}{l}{	2015	}&	Diciembre	&	 119.7 	&	 3,589.8 	&	 1.4 	&	 10.9 	&  &	 218.4 	&	 6,550.7 	&	 1.4 	&	 10.9 	\\
		\hline
		&&&&&&&&&&\\[-0.28cm]
		%			\multicolumn{9}{l}{\footnotesize Fuente: Informe Nacional de Desarrollo Humano (PNUD), con base en las Encuestas Nacionales de Condiciones de Vida (Encovi).}
	\end{longtable}
\end{center}


		
		\INEchaptercarta[Consumo de los alimentos. (Dimensión 4)]{Consumo de los alimentos.\\
	(Dimensión 4)}	
		
		
		\addtocounter{Cuadro}{1}
		
			{\Bold\Large 4.1 Hábitos y costumbres}\\[-.5cm]
				
		{\Bold\color{color1!80!black}{\normalsize Cuadro \theCuadro $\,-$  Niños menores de 5 años que recibieron lactancia materna; según características varias. }}\\
		{\Bold\color{color1!80!black}{\normalsize República de Guatemala, año 2008/2009.}}\\
		(Porcentajes)\\
		\begin{center}\fontsize{3.8mm}{1.6em}\selectfont \setlength{\arrayrulewidth}{0.7pt}
			$\ $\\[-2.5cm]
			$\!$\begin{longtable}{x{4.3cm}ccc}
				\multicolumn{4}{l}{$\ $}\\[-.2cm]
			\hline\multicolumn{1}{c}{\raisebox{-.3cm}{\small \Bold{Características}}}& \multicolumn{1}{c}{\small \Bold{Alguna vez}} &\multicolumn{1}{c}{\small \Bold{Empezó a lactar dentro}}&\multicolumn{1}{c}{\small \Bold{Empezó a lactar}}\\[-0.3cm]
				\multicolumn{1}{l}{ } & \multicolumn{1}{c}{\textbf{lactó}}& \multicolumn{1}{c}{\small \Bold{de la primera hora}} &\multicolumn{1}{c}{\small \Bold{durante el primer día}}\\
				&&& \\[-0.6cm]
				\multicolumn{1}{l}{$\ $} &  \multicolumn{3}{c}{$\ $} \\[-0.48cm]					     
				\hline\endhead
%				\hline \multicolumn{4}{r}{\textit{Continúa en la siguiente página}} \\
%				\endfoot
				&&& \\[-0.9cm]
				\multicolumn{4}{l}{\footnotesize INE. Encuesta Nacional de Salud Materno Infantil (Ensmi) 2008/2009.}\\\endlastfoot
\rowcolor{color1!60!white} \multicolumn{1}{l}{\Bold{	Total	}}&	 96.0 	 & 	 55.5 	 & 	 79.2 	 \\ 
\rowcolor{color1!40!white} \multicolumn{1}{l}{\Bold{	Área geográfica	}}&		&		&		\\
\multicolumn{1}{l}{	Urbana	}&	94.4	&	49	&	76.5	\\
\rowcolor{color1!5!white}\multicolumn{1}{l}{	Rural	}&	97.1	&	59.8	&	81	\\
\rowcolor{color1!40!white} \multicolumn{1}{l}{\Bold{	Región	}}&		&		&		\\
\multicolumn{1}{l}{	Metropolitana	}&	94.2	&	47.3	&	78.6	\\
\rowcolor{color1!5!white}\multicolumn{1}{l}{	Norte	}&	95.9	&	62.5	&	80.3	\\
\multicolumn{1}{l}{	Nororiente	}&	95.4	&	58.4	&	80.9	\\
\rowcolor{color1!5!white}\multicolumn{1}{l}{	Suroriente	}&	93.7	&	57.2	&	80.2	\\
\multicolumn{1}{l}{	Central	}&	96.5	&	56.1	&	81.1	\\
\rowcolor{color1!5!white}\multicolumn{1}{l}{	Suroccidente	}&	97.1	&	54.2	&	78.9	\\
\multicolumn{1}{l}{	Noroccidente	}&	97.3	&	59.8	&	79	\\
\rowcolor{color1!5!white}\multicolumn{1}{l}{	Petén	}&	97.6	&	52.9	&	71.8	\\
\rowcolor{color1!40!white} \multicolumn{1}{l}{\Bold{	Categoría étnica de la madre	}}&		&		&		\\
\multicolumn{1}{l}{	Indígena	}&	97.1	&	59.6	&	79.4	\\
\rowcolor{color1!5!white}\multicolumn{1}{l}{	Ladino	}&	95.2	&	52.4	&	79.1	\\
\rowcolor{color1!40!white} \multicolumn{1}{l}{\Bold{	Nivel de educación de la madre	}}&		&		&		\\
\multicolumn{1}{l}{	Sin educación	}&	96.9	&	60.7	&	80.5	\\
\rowcolor{color1!5!white}\multicolumn{1}{l}{	Primaria	}&	96.7	&	56.9	&	80.5	\\
\multicolumn{1}{l}{	Secundaria	}&	93.3	&	47.7	&	75.6	\\
\rowcolor{color1!5!white}\multicolumn{1}{l}{	Superior	}&	94.1	&	41.3	&	72.3	\\
\rowcolor{color1!40!white} \multicolumn{1}{l}{\Bold{	Asistencia en el parto	}}&		&		&		\\
\multicolumn{1}{l}{	Personal médico	}&	94.9	&	49.2	&	78	\\
\rowcolor{color1!5!white}\multicolumn{1}{l}{	Comadrona	}&	97.6	&	63.8	&	81	\\
\multicolumn{1}{l}{	Otro o ninguno	}&	96.3	&	63.6	&	79.7	\\
\rowcolor{color1!40!white} \multicolumn{1}{l}{\Bold{	Lugar del parto	}}&		&		&		\\
\multicolumn{1}{l}{	Establecimiento de salud	}&	94.9	&	49.2	&	78.1	\\
\rowcolor{color1!5!white}\multicolumn{1}{l}{	En casa	}&	97.6	&	63.9	&	80.9	\\
\multicolumn{1}{l}{	Otro	}&	88.2	&	35.1	&	57.8	\\
				\hline
				&&&\\[-0.28cm]
			\end{longtable}\addtocounter{Cuadro}{1}
		\end{center}
		


	\newpage
	
	
	
	
	{\Bold\color{color1!80!black}{\normalsize Cuadro \theCuadro $\,-$   Duración mediana de la lactancia para menores de 2 años e intensidad de lactancia en menores de 6 meses por características varias. }}\\
	{\Bold\color{color1!80!black}{\normalsize República de Guatemala, año 2008/2009.}}\\
	(Meses y porcentajes)\\[-.9mm]
	\begin{center}\fontsize{3.8mm}{1.6em}\selectfont \setlength{\arrayrulewidth}{0.7pt}
		$\ $\\[-2.5cm]
		$\!$\begin{longtable}{x{4.3cm}cccx{0.1mm}c}
			\multicolumn{4}{l}{$\ $}\\[-.2cm]
			%			\multicolumn{9}{l}{\Bold\color{color1!80!black}{\normalsize Cuadro \theCuadro $\,-$ Número de habitantes total y por sexo; según grupos quinquenales de edad.}}\\
			%			\multicolumn{9}{l}{\normalsize (Personas)}
			%			\\[-0.1cm]
		\hline	
			\multicolumn{1}{c}{\multirow{3}[0]{*}{ \Bold{Característica}}}	& \multicolumn{3}{c}{\multirow{2}{*}[2mm]{\small \Bold{Duración mediana (meses)}}}&&\multicolumn{1}{c}{\small \Bold{Intensidad de}}\\[-0.2cm]
			\multicolumn{1}{c}{ }& \multicolumn{1}{c}{} &\multicolumn{3}{c}{ }&\multicolumn{1}{c}{\small \Bold{lactancia (porcentaje)}}\\[-0.1cm]\cline{2-4}\cline{6-6}
			& \multicolumn{1}{c}{\small \Bold{Cualquier}} &\multicolumn{1}{c}{\small \Bold{Lactancia}}&\multicolumn{1}{c}{\small \Bold{Lactancia}}&\multicolumn{1}{c}{ }&\multicolumn{1}{c}{\small \Bold{Pecho 6 o más veces}}\\[-0.2cm]
			\multicolumn{1}{c}{ }& \multicolumn{1}{c}{\small \Bold{lactancia}} &\multicolumn{1}{c}{\small \Bold{exclusiva}}&\multicolumn{1}{c}{\small \Bold{completa}}& &\multicolumn{1}{c}{\small \Bold{en las últimas 24 horas}}\\[-0.2cm]
			%											&&&&&\multicolumn{1}{c}{\small \Bold{24 horas}}\\[-0.3cm]
			&&& \\[-0.6cm]
			\multicolumn{1}{l}{$\ $} &  \multicolumn{3}{c}{$\ $} \\[-0.48cm]
			%			\multicolumn{1}{c}{$\ $} & \multicolumn{1}{c}{\Bold{IDH}} & \multicolumn{2}{c}{\Bold{IDH Salud}} & \multicolumn{2}{c}{\Bold{IDH Educación}} & \multicolumn{2}{c}{\Bold{IDH Ingresos}} \\
			%		\multicolumn{1}{c}{} &  \multicolumn{1}{c}{\Bold{Incidencia}} & \multicolumn{1}{c}{\Bold{Error estándar}} &  \multicolumn{1}{c}{ } &\multicolumn{1}{c}{\Bold{Incidencia}} & \multicolumn{1}{c}{\Bold{Error estándar}} &  \multicolumn{1}{c}{ }\\						     
			\hline\endhead
			\hline \multicolumn{4}{r}{\textit{Continúa en la siguiente página}} \\
			\endfoot
			&&& \\[-0.9cm]
			\multicolumn{6}{l}{\footnotesize INE. Encuesta Nacional de Salud Materno Infantil (Ensmi) 2008/2009.}\\[-0.1cm]
			\multicolumn{6}{l}{\parbox{15cm}{\footnotesize \textbf{Definiciones:} - Lactancia exclusiva: solo pecho.}}\\[-0.1cm]			\multicolumn{6}{l}{\parbox{15cm}{\footnotesize Lactancia completa: pecho y agua.}}\\[-0.1cm]
			\endlastfoot
			\rowcolor{color1!60!white} \multicolumn{1}{l}{\Bold{Total}}&	 21.0 	&	 2.9 	&	 4.2 	&&	93.7	\\
			\rowcolor{color1!40!white} \multicolumn{1}{l}{\Bold{	Área geográfica	}}&		&		&		&&		\\
			\multicolumn{1}{l}{	Urbana	}&	19.6	&	0.6	&	1.3	&&	90.8	\\
			\rowcolor{color1!5!white}\multicolumn{1}{l}{	Rural	}&	21.8	&	4.6	&	5	&&	95.4	\\
			\rowcolor{color1!40!white} \multicolumn{1}{l}{\Bold{	Región	}}&		&		&		&&		\\
			\multicolumn{1}{l}{	Metropolitana	}&	13.8	&	0.5	&	0.6	&&	91.2	\\
			\rowcolor{color1!5!white}\multicolumn{1}{l}{	Norte	}&	21.3	&	5.9	&	6	&&	95.5	\\
			\multicolumn{1}{l}{	Nororiente	}&	22.8	&	0.5	&	1.5	&&	91.5	\\
			\rowcolor{color1!5!white}\multicolumn{1}{l}{	Suroriente	}&	18	&	0.4	&	1.8	&&	94.2	\\
			\multicolumn{1}{l}{	Central	}&	19.2	&	0.5	&	3	&&	93.7	\\
			\rowcolor{color1!5!white}\multicolumn{1}{l}{	Suroccidente	}&	21.5	&	4.4	&	4.9	&&	94.5	\\
			\multicolumn{1}{l}{	Noroccidente	}&	22.8	&	5.1	&	5.4	&&	93.2	\\
			\rowcolor{color1!5!white}\multicolumn{1}{l}{	Petén	}&	21.7	&	3.8	&	3.9	&&	96.3	\\
			\rowcolor{color1!40!white} \multicolumn{1}{l}{\Bold{	Categoría étnica de la madre	}}&		&		&		&&		\\
			\multicolumn{1}{l}{	Indígena	}&	22.6	&	5.2	&	5.6	&&	95.3	\\
			\rowcolor{color1!5!white}\multicolumn{1}{l}{	Ladino	}&	17.9	&	0.9	&	1.5	&&	92.2	\\
			\rowcolor{color1!40!white} \multicolumn{1}{l}{\Bold{	Nivel de educación de la madre	}}&		&		&		&&		\\
			\multicolumn{1}{l}{	Sin educación	}&	21.9	&	5.2	&	5.3	&&	93.6	\\
			\rowcolor{color1!5!white}\multicolumn{1}{l}{	Primaria	}&	21.3	&	4	&	4.5	&&	96.8	\\
			\multicolumn{1}{l}{	Secundaria	}&	14.1	&	0.5	&	0.6	&&	86.8	\\
			\rowcolor{color1!5!white}\multicolumn{1}{l}{	Superior	}&	8.1	&	0.3	&	0.3	&&	84.7	\\
			\rowcolor{color1!40!white} \multicolumn{1}{l}{\Bold{	Asistencia en el parto	}}&		&		&		&&		\\
			\multicolumn{1}{l}{	Personal médico	}&	18.3	&	1	&	1.5	&&	92.8	\\
			\rowcolor{color1!5!white}\multicolumn{1}{l}{	Comadrona	}&	22.6	&	5.1	&	5.4	&&	84.9	\\
			\multicolumn{1}{l}{	Otro o ninguno	}&	21.4	&	5	&	5.2	&&	94.7	\\
			\rowcolor{color1!40!white} \multicolumn{1}{l}{\Bold{	Lugar del parto	}}&		&		&		&&		\\
			\multicolumn{1}{l}{	Establecimiento de salud	}&	18.4	&	1	&	1.6	&&	92.7	\\
			\rowcolor{color1!5!white}\multicolumn{1}{l}{	En casa	}&	22.4	&	5.1	&	5.4	&&	94.9	\\
			
			%				\rowcolor{color1!40!white}				     &&& \\[-0.5cm]
			%						\multicolumn{1}{l}{\multirow{3}[0]{*}{\Bold{\raisebox{-0.6cm}{ }}}} & \multicolumn{7}{c}{\Bold{Año}} \\\cline{2-8}
			%						\multicolumn{1}{l}{$\ $} &  \multicolumn{7}{c}{$\ $} \\[-0.28cm]
			%						\rowcolor{color1!0!white} &&&&&& \\[-0.28cm]     
			%						\rowcolor{color1!0!white} { } & \multicolumn{1}{c}{2008} & \multicolumn{1}{c}{2009} & \multicolumn{1}{c}{2010} & \multicolumn{1}{c}{2011} & \multicolumn{1}{c}{2012} & \multicolumn{1}{c}{2013} & \multicolumn{1}{c}{2014}  \\ \hline
			%						\rowcolor{color1!40!white} &&&&&&& \\[-0.28cm]
			%							&&&&&& \\[-0.58cm]
			
			
			
			\hline
			&&&\\[-0.28cm]
			%			\multicolumn{9}{l}{\footnotesize Fuente: Informe Nacional de Desarrollo Humano (PNUD), con base en las Encuestas Nacionales de Condiciones de Vida (Encovi).}
		\end{longtable}\addtocounter{Cuadro}{1}
	\end{center}
	

%%%%%%%%%%%%%%%%%%





\begin{landscape}
%	$\ $\\[-1.8cm]
	{\Bold\color{color1!80!black}{\normalsize Cuadro \theCuadro $\,-$   Niños de 0 a 23 meses por tipo de lactancia; según características varias.}}\\
	{\Bold\color{color1!80!black}{\normalsize República de Guatemala, año 2008/2009.}}\\
	(Porcentajes)\\[-5mm]
	\begin{center}\fontsize{3.8mm}{1.6em}\selectfont \setlength{\arrayrulewidth}{0.7pt}
		$\ $\\[-2.5cm]
		$\!$\begin{longtable}{x{3.0cm} D{.}{.}{-1} D{.}{.}{-1}D{.}{.}{-1}x{0.05mm}D{.}{.}{-1}D{.}{.}{-1}D{.}{.}{-1}x{0.05mm}D{.}{.}{-1}x{0.05mm}D{.}{.}{-1}x{0.05mm}D{.}{.}{-1}}
			\multicolumn{14}{l}{$\ $}\\[-.2cm]\hline
			\multicolumn{1}{c}{\multirow{3}[0]{*}{ \Bold{Característica}}}	& \multicolumn{3}{c}{\Bold{0 - 3 meses}}&\multicolumn{1}{c}{} &\multicolumn{3}{c}{\Bold{0 - 5 meses}}&\multicolumn{1}{c}{ }&\multicolumn{1}{c}{ \Bold{6 - 9 meses}}&\multicolumn{1}{c}{ }&\multicolumn{1}{c}{ \Bold{12 - 15 meses}}&\multicolumn{1}{c}{}&\multicolumn{1}{c}{ \Bold{20 - 23 meses}}\\[-0.1cm]\cline{2-4}\cline{6-8}\cline{10-10}\cline{12-12}\cline{14-14}
			\multicolumn{1}{c}{ }& \multicolumn{1}{c}{\scriptsize\textbf{ No lactando}} &\multicolumn{1}{c}{\scriptsize \textbf{Lactancia}}&\multicolumn{1}{c}{\scriptsize \Bold{Lactancia}}&& \multicolumn{1}{c}{\scriptsize \textbf{No lactando}} &\multicolumn{1}{c}{\scriptsize \textbf{Lactancia}}&\multicolumn{1}{c}{\scriptsize \Bold{Lactancia}}&&\multicolumn{1}{c}{\scriptsize \textbf{Lactancia}}&&\multicolumn{1}{c}{\scriptsize \Bold{Lactancia}}&&\multicolumn{1}{c}{\scriptsize \Bold{Lactancia}}\\[-0.25cm]
			\multicolumn{1}{c}{ }& &\multicolumn{1}{c}{\scriptsize \textbf{exclusiva}}&\multicolumn{1}{c}{\scriptsize \Bold{predominante}}&& &\multicolumn{1}{c}{\scriptsize \textbf{exclusiva}}&\multicolumn{1}{c}{\scriptsize \Bold{predominante}}&&\multicolumn{1}{c}{\scriptsize \textbf{complementaria}}&&\multicolumn{1}{c}{\scriptsize \Bold{continuada}}&&\multicolumn{1}{c}{\scriptsize \Bold{continuada}}\\[-0.1cm]
			&&&&&&&&&&&&& \\[-0.6cm]
			\multicolumn{1}{l}{$\ $} &  \multicolumn{13}{c}{$\ $} \\[-0.48cm]				     
			\hline\endhead
			\hline \multicolumn{14}{r}{\textit{Continúa en la siguiente página}} \\[2cm]
			\endfoot
			&&&&&&&&&&&&& \\[-0.9cm]
			\multicolumn{14}{l}{\footnotesize INE. Encuesta Nacional de Salud Materno Infantil (Ensmi) 2008/2009.}\\[-0.1cm]
			\multicolumn{14}{l}{\parbox{15cm}{\footnotesize \textbf{Definiciones:} - Lactancia exclusiva: solo pecho.}}\\[-0.1cm]			\multicolumn{14}{l}{\parbox{15cm}{\footnotesize Lactancia completa: pecho y agua.}}\\[-0.1cm]
			\endlastfoot
			\rowcolor{color1!60!white} \multicolumn{1}{l}{\Bold{	Total	}}&	 4.2 	 & 	 55.6 	 & 	 20.9 	 & 	 & 	5.9	&	49.6	&	 19.4 	 & 	 & 	 71.3 	 & 	 & 	 78.6 	 & 	 & 	46.2	 \\ 
			\rowcolor{color1!40!white} \multicolumn{1}{l}{\Bold{	Área Geográfica	}}&		&		&		&	&		&		&		&	&		&	&		&	&		 \\ 
			\multicolumn{1}{l}{	Urbana	}&	5.5	&	37.9	&	24.4	&	&	9.9	&	32.5	&	22.6	&	&	63.7	&	&	64	&	&	39.9	 \\ 
			\rowcolor{color1!5!white}\multicolumn{1}{l}{	Rural	}&	3.3	&	67.4	&	18.6	&	&	3.4	&	60.4	&	17.4	&	&	75.8	&	&	87.2	&	&	50	 \\ 
			\rowcolor{color1!40!white} \multicolumn{1}{l}{\Bold{	Región	}}&		&		&		&	&		&		&		&	&		&	&		&	&		 \\ 
			\multicolumn{1}{l}{	Metropolitana	}&	2.3	&	35.6	&	22.2	&	&	11.9	&	25.8	&	22.8	&	&	61.4	&	&	51.1	&	&	30.6	 \\ 
			\rowcolor{color1!5!white}\multicolumn{1}{l}{	Norte	}&	7.3	&	78.2	&	6.4	&	&	5.7	&	74.4	&	8.4	&	&	60.4	&	&	79.5	&	&	40.1	 \\ 
			\multicolumn{1}{l}{	Nororiente	}&	9.3	&	46.2	&	18.4	&	&	10.2	&	36.1	&	14.2	&	&	71.2	&	&	68	&	&	55.8	 \\ 
			\rowcolor{color1!5!white}\multicolumn{1}{l}{	Suroriente	}&	4.8	&	40.3	&	34.9	&	&	3.4	&	35.4	&	29.5	&	&	81	&	&	71.6	&	&	36.3	 \\ 
			\multicolumn{1}{l}{	Central	}&	4.4	&	40.6	&	30.8	&	&	6.7	&	35.8	&	28.3	&	&	67.9	&	&	82.3	&	&	47.8	 \\ 
			\rowcolor{color1!5!white}\multicolumn{1}{l}{	Suroccidente	}&	3.5	&	63.4	&	22.7	&	&	4	&	58.3	&	20	&	&	74.3	&	&	91	&	&	48.1	 \\ 
			\multicolumn{1}{l}{	Noroccidente	}&	3.1	&	62.5	&	17.9	&	&	2.7	&	60.4	&	15.8	&	&	80.4	&	&	92	&	&	59.6	 \\ 
			\rowcolor{color1!5!white}\multicolumn{1}{l}{	Petén	}&	0	&	61.7	&	22.4	&	&	2.4	&	52.1	&	20.6	&	&	78.1	&	&	89	&	&	49.3	 \\ 
			\rowcolor{color1!40!white} \multicolumn{1}{l}{\Bold{	Categoría étnica	}}&		&		&		&	&		&		&		&	&		&	&		&	&		 \\ 
			\multicolumn{1}{l}{	Indígena	}&	3.8	&	71.7	&	14.4	&	&	3.2	&	66.4	&	15.5	&	&	69.4	&	&	88.9	&	&	57.9	 \\ 
			\rowcolor{color1!5!white}\multicolumn{1}{l}{	Ladino	}&	4.5	&	41.3	&	26.7	&	&	8.4	&	34.4	&	23	&	&	72.9	&	&	69.6	&	&	35.6	 \\ 
			\rowcolor{color1!40!white} \multicolumn{1}{l}{\Bold{	Nivel de educación	}}&		&		&		&	&		&		&		&	&		&	&		&	&		 \\ 
			\multicolumn{1}{l}{	Sin educación	}&	3.6	&	67.3	&	22.5	&	&	3	&	64	&	19.9	&	&	75.2	&	&	90	&	&	52.2	 \\ 
			\rowcolor{color1!5!white}\multicolumn{1}{l}{	Primaria	}&	4	&	62.6	&	19.4	&	&	5	&	55.1	&	18.9	&	&	72.4	&	&	86.9	&	&	47.2	 \\ 
			\multicolumn{1}{l}{	Secundaria	}&	5.2	&	32.6	&	24.3	&	&	10.9	&	26.4	&	21.3	&	&	67.5	&	&	51.6	&	&	40.2	 \\ 
			\rowcolor{color1!5!white}\multicolumn{1}{l}{	Superior	}&	3.9	&	5.7	&	12	&	&	8.9	&	4.7	&	9.9	&	&	37.2	&	&	46.8	&	&	11.1	 \\ 
			\rowcolor{color1!40!white} \multicolumn{1}{l}{\Bold{	Asistencia en el parto	}}&		&		&		&	&		&		&		&	&		&	&		&	&		 \\ 
			\multicolumn{1}{l}{	Personal médico	}&	5.8	&	42.7	&	25	&	&	8.8	&	36.5	&	23.2	&	&	72.5	&	&	67.7	&	&	38.5	 \\ 
			\rowcolor{color1!5!white}\multicolumn{1}{l}{	Comadrona	}&	2	&	75	&	15.2	&	&	2.4	&	67.7	&	14.6	&	&	69.6	&	&	90.2	&	&	56.5	 \\ 
			\rowcolor{color1!40!white} \multicolumn{1}{l}{\Bold{	Lugar del parto	}}&		&		&		&	&		&		&		&	&		&	&		&	&		 \\ 
			\multicolumn{1}{l}{	Establecimiento de salud	}&	5.6	&	43	&	25	&	&	8.8	&	36.6	&	23.2	&	&	72.4	&	&	68.1	&	&	38.6	 \\ 
			\rowcolor{color1!5!white}\multicolumn{1}{l}{	En casa	}&	2	&	74.2	&	14.8	&	&	2.1	&	67	&	14.5	&	&	70.1	&	&	90.6	&	&	55	 \\ 
			\hline
			&&&&&&&&&&&&&\\[-0.28cm]
		\end{longtable}\addtocounter{Cuadro}{1}
	\end{center}
\end{landscape}	


%%%%%%%%%%%%
\begin{landscape}
	{\Bold\Large 4.2 Educación}\\[-.5cm]

	$\ $\\%[-1.8cm]
	{\Bold\color{color1!80!black}{\normalsize Cuadro \theCuadro $\,-$   Distribución porcentual de las mujeres en edad fértil (15 a 49 años de edad), por nivel de educación más alto alcanzado; según características seleccionadas. }}\\
	{\Bold\color{color1!80!black}{\normalsize República de Guatemala, año 2008/2009.}}\\
	(Porcentajes)\\[-5mm]
	\begin{center}\fontsize{3.8mm}{1.6em}\selectfont \setlength{\arrayrulewidth}{0.7pt}
		$\ $\\[-2.5cm]
		$\!$\begin{longtable}{x{3.0cm}D{.}{.}{-1} x{0.05mm}D{.}{.}{-1}x{0.05mm} D{.}{.}{-1}D{.}{.}{-1}x{0.05mm}D{.}{.}{-1}D{.}{.}{-1}x{0.05mm}D{.}{.}{-1}x{0.05mm}D{.}{.}{-1}}
			\multicolumn{14}{l}{$\ $}\\[-.2cm]\hline
			%			\multicolumn{9}{l}{\Bold\color{color1!80!black}{\normalsize Cuadro \theCuadro $\,-$ Número de habitantes total y por sexo; según grupos quinquenales de edad.}}\\
			%			\multicolumn{9}{l}{\normalsize (Personas)}
			%			\\[-0.1cm]		
			\multicolumn{1}{c}{\multirow{2}{*}[2mm]{ \Bold{Característica}}}	& \multicolumn{1}{c}{\scriptsize\Bold{Sin}}&&\multicolumn{1}{c}{\multirow{2}{*}[2mm]{\textbf{Alfabetización}}}& &\multicolumn{2}{c}{\Bold{Primaria}}&\multicolumn{1}{c}{ }&\multicolumn{2}{c}{ \Bold{Secundaria}}&\multicolumn{1}{c}{ }&\multicolumn{1}{c}{\multirow{2}{*}[2mm]{ \Bold{Superior}}}&\multicolumn{1}{c}{}&\multicolumn{1}{c}{\multirow{2}{*}[2mm]{ \Bold{Total}}}\\[-0.1cm]\cline{6-7}\cline{9-10}
			\multicolumn{1}{c}{ }& \multicolumn{1}{c}{\scriptsize\textbf{educación}} &\multicolumn{1}{c}{ }&\multicolumn{1}{c}{ }&\multicolumn{1}{c}{  }&\multicolumn{1}{c}{\scriptsize \Bold{Incompleta}}&\multicolumn{1}{c}{\scriptsize \Bold{Completa}}& \multicolumn{1}{c}{ }&\multicolumn{1}{c}{\scriptsize \Bold{Incompleta}}&\multicolumn{1}{c}{\scriptsize \Bold{Completa}}& \multicolumn{1}{c}{ }&\multicolumn{1}{c}{} &\multicolumn{1}{c}{}&\multicolumn{1}{c}{}\\[-0.1cm]
			%			& \multicolumn{1}{c}{\scriptsize \Bold{Cualquier}} &\multicolumn{1}{c}{\scriptsize \Bold{Lactancia}}&\multicolumn{1}{c}{\scriptsize \Bold{Lactancia}}&\multicolumn{1}{c}{ }&\multicolumn{1}{c}{\scriptsize \Bold{Pecho 6 o más veces}}\\[-0.2cm]
			%			\multicolumn{1}{c}{ }& \multicolumn{1}{c}{\scriptsize \Bold{lactancia}} &\multicolumn{1}{c}{\scriptsize \Bold{exclusiva}}&\multicolumn{1}{c}{\scriptsize \Bold{completa}}& &\multicolumn{1}{c}{\scriptsize \Bold{en las últimas 24 horas}}\\[-0.2cm]
			%											&&&&&\multicolumn{1}{c}{\scriptsize \Bold{24 horas}}\\[-0.3cm]
			&&&&&&&&&&&&& \\[-0.6cm]
			\multicolumn{1}{l}{$\ $} &  \multicolumn{13}{c}{$\ $} \\[-0.48cm]
			%			\multicolumn{1}{c}{$\ $} & \multicolumn{1}{c}{\Bold{IDH}} & \multicolumn{2}{c}{\Bold{IDH Salud}} & \multicolumn{2}{c}{\Bold{IDH Educación}} & \multicolumn{2}{c}{\Bold{IDH Ingresos}} \\
			%		\multicolumn{1}{c}{} &  \multicolumn{1}{c}{\Bold{Incidencia}} & \multicolumn{1}{c}{\Bold{Error estándar}} &  \multicolumn{1}{c}{ } &\multicolumn{1}{c}{\Bold{Incidencia}} & \multicolumn{1}{c}{\Bold{Error estándar}} &  \multicolumn{1}{c}{ }\\						     
			\hline\endhead
			\hline \multicolumn{14}{r}{\textit{Continúa en la siguiente página}} \\[2cm]
			\endfoot
			&&&&&&&&&&&&& \\[-0.9cm]
			\multicolumn{14}{l}{\footnotesize INE. Encuesta Nacional de Salud Materno Infantil (Ensmi) 2008/2009.}\\[-0.1cm]
			%			\multicolumn{14}{l}{\parbox{15cm}{\footnotesize \textbf{Definiciones:} - Lactancia exclusiva: solo pecho.}}\\[-0.1cm]			\multicolumn{14}{l}{\parbox{15cm}{\footnotesize Lactancia completa: pecho y agua.}}\\[-0.1cm]
			\endlastfoot
			\rowcolor{color1!60!white} \multicolumn{1}{l}{\Bold{	Total	}}&	 20.2 	&&	 1.0 	&&	 30.8 	&	15.1	&&	 20.2 	&	 7.1 	&&	5.6	&&	 100 	\\
			\rowcolor{color1!40!white} \multicolumn{1}{l}{\Bold{	Área geográfica	}}&		&&		&&		&		&&		&		&&		&&		\\
			\multicolumn{1}{l}{	Urbana	}&	10	&&	0.8	&&	21.3	&	14.6	&&	29.7	&	12.5	&&	11.1	&&	 100 	\\
			\rowcolor{color1!5!white}\multicolumn{1}{l}{	Rural	}&	28.8	&&	1.1	&&	38.8	&	15.4	&&	12.2	&	2.7	&&	1	&&	 100 	\\
			\rowcolor{color1!40!white} \multicolumn{1}{l}{\Bold{	Departamento	}}&		&&		&&		&		&&		&		&&		&&		\\
			\multicolumn{1}{l}{	Guatemala	 }& 	 7.2 	 && 	 0.7 	 && 	 17.9 	 & 	 16.6 	 && 	 32.7 	 & 	 11.7 	 && 	 13.2 	 && 	 100 	 \\ 
			\rowcolor{color1!5!white}\multicolumn{1}{l}{	El Progreso	 }& 	 10.9 	 && 	 0.1 	 && 	 26.3 	 & 	 23.8 	 && 	 25.7 	 & 	 9.9 	 && 	 3.3 	 && 	 100 	 \\ 
			\multicolumn{1}{l}{	Sacatepéquez	 }& 	 12.4 	 && 	 1.9 	 && 	 29.6 	 & 	 18.1 	 && 	 24.2 	 & 	 8.2 	 && 	 5.7 	 && 	 100 	 \\ 
			\rowcolor{color1!5!white}\multicolumn{1}{l}{	Chimaltenango	 }& 	 14.7 	 && 	 0.8 	 && 	 34.4 	 & 	 17.3 	 && 	 21.4 	 & 	 8.3 	 && 	 3.0 	 && 	 100 	 \\ 
			\multicolumn{1}{l}{	Escuintla	 }& 	 13.3 	 && 	 1.3 	 && 	 33.6 	 & 	 16.9 	 && 	 24.6 	 & 	 6.0 	 && 	 4.3 	 && 	 100 	 \\ 
			\rowcolor{color1!5!white}\multicolumn{1}{l}{	Santa Rosa	 }& 	 9.9 	 && 	 0.9 	 && 	 33.5 	 & 	 22.7 	 && 	 22.1 	 & 	 7.2 	 && 	 3.6 	 && 	 100 	 \\ 
			\multicolumn{1}{l}{	Sololá	 }& 	 31.6 	 && 	 0.6 	 && 	 29.0 	 & 	 13.2 	 && 	 14.5 	 & 	 9.1 	 && 	 1.9 	 && 	 100 	 \\ 
			\rowcolor{color1!5!white}\multicolumn{1}{l}{	Totonicapán	 }& 	 28.1 	 && 	 1.8 	 && 	 30.9 	 & 	 14.4 	 && 	 16.3 	 & 	 5.7 	 && 	 2.7 	 && 	 100 	 \\ 
			\multicolumn{1}{l}{	Quetzaltenango	 }& 	 14.4 	 && 	 0.7 	 && 	 30.8 	 & 	 11.0 	 && 	 25.3 	 & 	 8.5 	 && 	 9.4 	 && 	 100 	 \\ 
			\rowcolor{color1!5!white}\multicolumn{1}{l}{	Suchitepéquez	 }& 	 15.2 	 && 	 1.8 	 && 	 36.9 	 & 	 13.5 	 && 	 21.3 	 & 	 8.1 	 && 	 3.2 	 && 	 100 	 \\ 
			\multicolumn{1}{l}{	Retalhuleu	 }& 	 15.4 	 && 	 0.8 	 && 	 35.1 	 & 	 16.2 	 && 	 20.5 	 & 	 6.9 	 && 	 5.1 	 && 	 100 	 \\ 
			\rowcolor{color1!5!white}\multicolumn{1}{l}{	San Marcos	 }& 	 19.3 	 && 	 0.6 	 && 	 39.6 	 & 	 16.1 	 && 	 15.3 	 & 	 5.9 	 && 	 3.1 	 && 	 100 	 \\ 
			\multicolumn{1}{l}{	Huehuetenango	 }& 	 34.4 	 && 	 0.8 	 && 	 39.6 	 & 	 12.3 	 && 	 8.3 	 & 	 2.7 	 && 	 1.9 	 && 	 100 	 \\ 
			\rowcolor{color1!5!white}\multicolumn{1}{l}{	Quiché	 }& 	 41.9 	 && 	 0.4 	 && 	 34.3 	 & 	 9.6 	 && 	 8.7 	 & 	 2.9 	 && 	 2.2 	 && 	 100 	 \\ 
			\multicolumn{1}{l}{	Baja Verapaz	 }& 	 32.5 	 && 	 1.1 	 && 	 32.1 	 & 	 12.1 	 && 	 13.6 	 & 	 5.2 	 && 	 3.5 	 && 	 100 	 \\ 
			\rowcolor{color1!5!white}\multicolumn{1}{l}{	Alta Verapaz	 }& 	 37.8 	 && 	 2.2 	 && 	 31.0 	 & 	 11.9 	 && 	 11.0 	 & 	 3.6 	 && 	 2.6 	 && 	 100 	 \\ 
			\multicolumn{1}{l}{	Petén	 }& 	 25.2 	 && 	 0.3 	 && 	 34.8 	 & 	 13.7 	 && 	 18.0 	 & 	 6.3 	 && 	 1.7 	 && 	 100 	 \\ 
			\rowcolor{color1!5!white}\multicolumn{1}{l}{	Izabal	 }& 	 17.4 	 && 	 1.9 	 && 	 27.8 	 & 	 15.0 	 && 	 25.0 	 & 	 9.5 	 && 	 3.4 	 && 	 100 	 \\ 
			\multicolumn{1}{l}{	Zacapa	 }& 	 18.4 	 && 	 0.3 	 && 	 31.0 	 & 	 14.3 	 && 	 22.7 	 & 	 8.2 	 && 	 5.2 	 && 	 100 	 \\ 
			\rowcolor{color1!5!white}\multicolumn{1}{l}{	Chiquimula	 }& 	 26.5 	 && 	 1.2 	 && 	 27.0 	 & 	 18.8 	 && 	 13.3 	 & 	 6.4 	 && 	 6.9 	 && 	 100 	 \\ 
			\multicolumn{1}{l}{	Jalapa	 }& 	 16.4 	 && 	 0.7 	 && 	 42.1 	 & 	 11.5 	 && 	 17.3 	 & 	 6.8 	 && 	 5.2 	 && 	 100 	 \\ 
			\rowcolor{color1!5!white}\multicolumn{1}{l}{	Jutiapa	 }& 	 14.4 	 && 	 0.9 	 && 	 37.5 	 & 	 22.0 	 && 	 17.2 	 & 	 4.6 	 && 	 3.5 	 && 	 100 	 \\ 
			\rowcolor{color1!40!white} \multicolumn{1}{l}{\Bold{	Grupo de edad	}}&		 && 		 && 		 & 		 && 		 & 		 && 		 && 		 \\ 
			\multicolumn{1}{l}{	15-19	 }& 	 7.0 	 && 	 0.6 	 && 	 26.2 	 & 	 20.9 	 && 	 40.7 	 & 	 3.6 	 && 	 1.0 	 && 	 100 	 \\ 
			\rowcolor{color1!5!white}\multicolumn{1}{l}{	20-24	 }& 	 13.8 	 && 	 0.6 	 && 	 30.4 	 & 	 15.1 	 && 	 19.6 	 & 	 11.4 	 && 	 9.2 	 && 	 100 	 \\ 
			\multicolumn{1}{l}{	25-29	 }& 	 19.9 	 && 	 0.9 	 && 	 32.0 	 & 	 15.2 	 && 	 15.5 	 & 	 8.6 	 && 	 7.9 	 && 	 100 	 \\ 
			\rowcolor{color1!5!white}\multicolumn{1}{l}{	30-34	 }& 	 23.9 	 && 	 1.3 	 && 	 33.3 	 & 	 12.4 	 && 	 14.5 	 & 	 7.8 	 && 	 6.8 	 && 	 100 	 \\ 
			\multicolumn{1}{l}{	35-39	 }& 	 28.5 	 && 	 1.0 	 && 	 37.0 	 & 	 11.7 	 && 	 10.9 	 & 	 6.0 	 && 	 4.8 	 && 	 100 	 \\ 
			\rowcolor{color1!5!white}\multicolumn{1}{l}{	40-44	 }& 	 33.9 	 && 	 1.4 	 && 	 31.8 	 & 	 11.8 	 && 	 8.9 	 & 	 6.4 	 && 	 5.8 	 && 	 100 	 \\ 
			\multicolumn{1}{l}{	45-49	 }& 	 38.5 	 && 	 1.8 	 && 	 28.9 	 & 	 11.0 	 && 	 7.6 	 & 	 6.8 	 && 	 5.4 	 && 	 100 	 \\ 
			\rowcolor{color1!40!white} \multicolumn{1}{l}{\Bold{	Grupo étnico	}}&		 && 		 && 		 & 		 && 		 & 		 && 		 && 		 \\ 
			\multicolumn{1}{l}{	Indígena	 }& 	 34.7 	 && 	 1.3 	 && 	 35.6 	 & 	 13.1 	 && 	 10.6 	 & 	 3.0 	 && 	 1.7 	 && 	 100 	 \\ 
			\rowcolor{color1!5!white}\multicolumn{1}{l}{	No indígena	 }& 	 11.1 	 && 	 0.8 	 && 	 27.9 	 & 	 16.3 	 && 	 26.2 	 & 	 9.7 	 && 	 8.1 	 && 	 100 	 \\ 
			
			\hline
			&&&&&&&&&&&&&\\[-0.28cm]
			%			\multicolumn{9}{l}{\footnotesize Fuente: Informe Nacional de Desarrollo Humano (PNUD), con base en las Encuestas Nacionales de Condiciones de Vida (Encovi).}
		\end{longtable}\addtocounter{Cuadro}{1}
	\end{center}
\end{landscape}	


%%%%%%%%%%%%%%%%

	{\Bold\Large 4.3 Orientación al consumidor}\\[-.5cm]

{\Bold\color{color1!80!black}{\normalsize Cuadro \theCuadro $\,-$   Mujeres de 15 a 49 años de edad que leen con facilidad, leen un periódico o ven televisión por lo menos una vez a la semana o escuchan la radio todos los días; según características seleccionadas. }}\\
{\Bold\color{color1!80!black}{\normalsize República de Guatemala, año 2008/2009.}}\\
(Porcentajes)\\[-5mm]
\begin{center}\fontsize{3.8mm}{1.6em}\selectfont \setlength{\arrayrulewidth}{0.7pt}
	$\ $\\[-2.5cm]
	$\!$\begin{longtable}{x{3.0cm}D{.}{.}{-1}D{.}{.}{-1}D{.}{.}{-1}D{.}{.}{-1}D{.}{.}{-1}D{.}{.}{-1}}
		\multicolumn{7}{l}{$\ $}\\[-.2cm]
		%			\multicolumn{9}{l}{\Bold\color{color1!80!black}{\normalsize Cuadro \theCuadro $\,-$ Número de habitantes total y por sexo; según grupos quinquenales de edad.}}\\
		%			\multicolumn{9}{l}{\normalsize (Personas)}
		%			\\[-0.1cm]	
		\hline
		\multicolumn{1}{c}{\multirow{2}{*}[2mm]{ \Bold{Característica}}}	& \multicolumn{1}{c}{\scriptsize\Bold{Capacidad de}}&\multicolumn{1}{c}{\multirow{2}{*}[2mm]{\textbf{Nigún medio}}}&\multicolumn{2}{c}{\Bold{Una vez a la semana}}&\multicolumn{1}{c}{\textbf{Radio todos}}&\multicolumn{1}{c}{ \Bold{Los 3 medios}}\\[-0.1cm]\cline{4-5}
		\multicolumn{1}{c}{ }& \multicolumn{1}{c}{\scriptsize\textbf{comprensión}} &\multicolumn{1}{c}{ }&\multicolumn{1}{c}{\textbf{Periódico} }&\multicolumn{1}{c}{\textbf{Televisión}}&\multicolumn{1}{c}{ \Bold{los días}}&\multicolumn{1}{c}{}\\[-0.1cm]
		%			& \multicolumn{1}{c}{\scriptsize \Bold{Cualquier}} &\multicolumn{1}{c}{\scriptsize \Bold{Lactancia}}&\multicolumn{1}{c}{\scriptsize \Bold{Lactancia}}&\multicolumn{1}{c}{ }&\multicolumn{1}{c}{\scriptsize \Bold{Pecho 6 o más veces}}\\[-0.2cm]
		%			\multicolumn{1}{c}{ }& \multicolumn{1}{c}{\scriptsize \Bold{lactancia}} &\multicolumn{1}{c}{\scriptsize \Bold{exclusiva}}&\multicolumn{1}{c}{\scriptsize \Bold{completa}}& &\multicolumn{1}{c}{\scriptsize \Bold{en las últimas 24 horas}}\\[-0.2cm]
		%											&&&&&\multicolumn{1}{c}{\scriptsize \Bold{24 horas}}\\[-0.3cm]
		&&&&&& \\[-0.6cm]
		\multicolumn{1}{l}{$\ $} &  \multicolumn{6}{c}{$\ $} \\[-0.48cm]					     
		\hline\endhead
		\hline \multicolumn{7}{r}{\textit{Continúa en la siguiente página}} \\[2cm]
		\endfoot
		&&&&&&\\[-0.9cm]
		\multicolumn{7}{l}{\footnotesize INE. Encuesta Nacional de Salud Materno Infantil (Ensmi) 2008/2009.}\\[-0.1cm]
		%			\multicolumn{14}{l}{\parbox{15cm}{\footnotesize \textbf{Definiciones:} - Lactancia exclusiva: solo pecho.}}\\[-0.1cm]			\multicolumn{14}{l}{\parbox{15cm}{\footnotesize Lactancia completa: pecho y agua.}}\\[-0.1cm]
		\endlastfoot
		\rowcolor{color1!60!white} \multicolumn{1}{l}{\Bold{	Total	}}&	 58.3 	&	 2.8 	&	 60.3 	&	 71.4 	&	 66.8 	&	 38.1 	\\
		\rowcolor{color1!40!white} \multicolumn{1}{l}{\Bold{	Área geográfica	}}&		&		&		&		&		&		\\
		\multicolumn{1}{l}{	Urbana	}&	75.1	&	1	&	74.8	&	89	&	69.7	&	51.8	\\
		\rowcolor{color1!5!white}\multicolumn{1}{l}{	Rural	}&	44	&	4.2	&	48.1	&	56.5	&	64.4	&	26.6	\\
		\rowcolor{color1!40!white} \multicolumn{1}{l}{\Bold{	Grupo de edad	}}&		&		&		&		&		&		\\
		\multicolumn{1}{l}{	15-19	 }& 	74.5	&	2.1	&	74	&	74.6	&	76.1	&	49.8	\\
		\rowcolor{color1!5!white}\multicolumn{1}{l}{	20-24	 }& 	63.8	&	3.7	&	63.7	&	72.2	&	67.1	&	39.3	\\
		\multicolumn{1}{l}{	25-29	 }& 	58.2	&	3.4	&	58.5	&	70.3	&	63.4	&	36.3	\\
		\rowcolor{color1!5!white}\multicolumn{1}{l}{	30-34	 }& 	53.8	&	3.2	&	55.9	&	70.8	&	62.1	&	33.9	\\
		\multicolumn{1}{l}{	35-39	 }& 	46.7	&	2.5	&	52.5	&	68.5	&	65.4	&	34	\\
		\rowcolor{color1!5!white}\multicolumn{1}{l}{	40-44	 }& 	44.8	&	2.6	&	51.3	&	67.4	&	60.3	&	29.8	\\
		\multicolumn{1}{l}{	45-49	 }& 	39.2	&	1.1	&	46.1	&	72.2	&	62.9	&	28.3	\\
		\rowcolor{color1!40!white} \multicolumn{1}{l}{\Bold{	Grupo étnico	}}&		&		&		&		&		&		\\
		\multicolumn{1}{l}{	Indígena	 }& 	35.9	&	4.3	&	41.9	&	51.4	&	64.5	&	23.4	\\
		\rowcolor{color1!5!white}\multicolumn{1}{l}{	No indígena	 }& 	72.2	&	1.8	&	71.8	&	83.8	&	68.2	&	74.3	\\
		\rowcolor{color1!40!white} \multicolumn{1}{l}{\Bold{	Nivel de educación	}}&		&		&		&		&		&		\\
		\multicolumn{1}{l}{	Sin educación	 }& 	0.8	&	0.6	&	3.9	&	39	&	53.6	&	1.7	\\
		\rowcolor{color1!5!white}\multicolumn{1}{l}{	Primaria incompleta	 }& 	41.7	&	6.6	&	58.1	&	65.6	&	65.3	&	30.7	\\
		\multicolumn{1}{l}{	Primaria completa	 }& 	79.9	&	2.8	&	77.2	&	79.9	&	72.4	&	48.5	\\
		\rowcolor{color1!5!white}\multicolumn{1}{l}{	Secundaria incompleta	 }& 	100.0	&	0.6	&	88.2	&	91.2	&	75.5	&	62.2	\\
		\multicolumn{1}{l}{	Secundaria completa	 }& 	100.0	&	0.5	&	90.5	&	95.3	&	69.1	&	62.1	\\
		\rowcolor{color1!5!white}\multicolumn{1}{l}{	Superior	 }& 	100.0	&	0	&	92.3	&	96.1	&	73.5	&	67.2	\\
		\multicolumn{1}{l}{	Alfabetización	 }& 	26.3	&	4.1	&	51.4	&	62.6	&	64.8	&	25.2	\\
		\hline
		&&&&&&\\[-0.28cm]
		%			\multicolumn{9}{l}{\footnotesize Fuente: Informe Nacional de Desarrollo Humano (PNUD), con base en las Encuestas Nacionales de Condiciones de Vida (Encovi).}
	\end{longtable}\addtocounter{Cuadro}{1}
\end{center}


%%%%%%%%%%%%%%%%%%%%%%5




		\INEchaptercarta[Utilización biológica de los alimentos (Dimensión 5)]{Utilización biológica\\ de los alimentos \\ (Dimensión 5)}{}
\addtocounter{Cuadro}{1}

\begin{landscape}
		{\Bold\Large 5.1 Morbilidad relacionada}\\%[-.5cm]
%	$\ $\\[-1.8cm]
	{\Bold\color{color1!80!black}{\normalsize Cuadro \theCuadro $\,-$   Consultas por diarrea aguda en niños y niñas menores de cinco años; según departamento. República de Guatemala, año 2011/2015.}}\\
%	{\Bold\color{color1!80!black}{\normalsize }}\\
	(Número de consultas)\\[-10mm]
	\begin{center}\fontsize{3mm}{1.5em}\selectfont \setlength{\arrayrulewidth}{0.7pt}
		$\ $\\[-2.0cm]
		$\!$\begin{longtable}{lrrrx{0.05mm}rrrx{0.05mm}rrrx{0.05mm}rrrx{0.05mm}rrr}
			\multicolumn{19}{l}{$\ $}\\[-.2cm]\hline
			\multicolumn{1}{c}{\multirow{2}{*}[0.5mm]{ \Bold{Departamento}}}	& \multicolumn{3}{c}{\textbf{2011}}&\tiny&\multicolumn{3}{c}{\textbf{2012}}&& \multicolumn{3}{c}{\textbf{2013}}&\tiny& \multicolumn{3}{c}{\textbf{2014}}&& \multicolumn{3}{c}{\textbf{2015}}\\[-0.1cm]\cline{2-4}\cline{6-8}\cline{10-12}\cline{14-16}\cline{18-20}
			\multicolumn{1}{c}{}& \multicolumn{1}{c}{\scriptsize\textbf{Mujer}} &\multicolumn{1}{c}{\Bold{Hombre}}&\multicolumn{1}{c}{\Bold{Total} } &\tiny&  \multicolumn{1}{c}{\scriptsize\textbf{Mujer}} &\multicolumn{1}{c}{\textbf{\Bold{Hombre}}}&\multicolumn{1}{c}{\Bold{Total} } &\tiny&  \multicolumn{1}{c}{\scriptsize\textbf{Mujer}} &\multicolumn{1}{c}{\Bold{Hombre}}&\multicolumn{1}{c}{\Bold{Total} } &\tiny&  \multicolumn{1}{c}{\scriptsize\textbf{Mujer}} &\multicolumn{1}{c}{\Bold{Hombre}}&\multicolumn{1}{c}{\Bold{Total} } &\tiny&  \multicolumn{1}{c}{\scriptsize\textbf{Mujer}} &\multicolumn{1}{c}{\Bold{Hombre}}&\multicolumn{1}{c}{\Bold{Total} }\\[-0.1cm]
			&&&&\tiny&&&&\tiny&&&&\tiny&&&& \\[-0.6cm]
			\multicolumn{1}{l}{$\ $} &  \multicolumn{18}{c}{$\ $} \\[-0.48cm]				     
			\hline\endhead
			\hline \multicolumn{19}{r}{\textit{Continúa en la siguiente página}} \\[2cm]
			\endfoot
			&&&&\tiny&&&&\tiny&&&&\tiny&&&& \\[-0.9cm]
			\multicolumn{19}{l}{\footnotesize INE. Encuesta Nacional de Salud Materno Infantil (Ensmi) 2008/2009.}\\[-0.1cm]
			\endlastfoot
	\rowcolor{color1!60!white} \multicolumn{1}{l}{\Bold{\textcolor{white}{Total}} }& \Bold{\textcolor{white}{145,156}}& \Bold{\textcolor{white}{146,248}}	 & \Bold{\textcolor{white}{291,404}}	 &\tiny&\Bold{\textcolor{white}{189,029}}	 &\Bold{\textcolor{white}{202,946}}	 &  \Bold{\textcolor{white}{391,975}}	 &\tiny&  \Bold{\textcolor{white}{208,999}}	 &\Bold{\textcolor{white}{223,790}}	 &\Bold{\textcolor{white}{432,789}}	 &\tiny&  \Bold{\textcolor{white}{207,233}}	 &\Bold{\textcolor{white}{220,973}}& \Bold{\textcolor{white}{428,206}}	 &\tiny&\Bold{\textcolor{white}{159,126}}	 &\Bold{\textcolor{white}{172,748}}	 &\Bold{\textcolor{white}{331,874}}	 \\ 
\multicolumn{1}{l}{	 Alta Verapaz }&10,387&9,835&20,222&&15,584&15,969&31,553&&16,744&17,499&34,243&&15,404&16,258&31,662&&10,878&12,057&22,935\\ 
\rowcolor{color1!5!white}\multicolumn{1}{l}{	  Baja Verapaz }&2,921&2,906&5,827&&4,763&4,956&9,719&&4,116&4,326&8,442&&4,856&5,127&9,983&&3,851&4,024&7,875\\ 
\multicolumn{1}{l}{	 Chimaltenango }&4,784&4,973&9,757&&5,590&6,098&11,688&&7,183&8,027&15,210&&6,886&7,571&14,457&&5,113&5,658&10,771\\ 
\rowcolor{color1!5!white}\multicolumn{1}{l}{	 Chiquimula }&7,185&7,185&14,370&&7,993&8,433&16,426&&9,279&9,159&18,438&&7,804&7,652&15,456&&6,149&6,223&12,372\\ 
\multicolumn{1}{l}{	  El Peten }&6,728&6,894&13,622&&8,480&9,133&17,613&&9,454&10,385&19,839&&12,457&13,636&26,093&&9,978&11,018&20,996\\ 
\rowcolor{color1!5!white}\multicolumn{1}{l}{	  El Progreso }&1,536&1,553&3,089&&2,370&2,854&5,224&&2,494&2,883&5,377&&2,473&2,872&5,345&&1,923&2,132&4,055\\ 
\multicolumn{1}{l}{	 Escuintla }&8,515&8,946&17,461&&10,469&12,173&22,642&&9,920&11,987&21,907&&11,150&12,655&23,805&&9,261&10,228&19,489\\ 
\rowcolor{color1!5!white}\multicolumn{1}{l}{	 Guatemala }&11,799&12,582&24,381&&19,492&21,679&41,171&&19,521&21,667&41,188&&19,714&21,332&41,046&&14,169&15,488&29,657\\ 
\multicolumn{1}{l}{	  Huehuetenango }&15,067&14,924&29,991&&23,453&24,320&47,773&&27,031&28,038&55,069&&21,327&22,324&43,651&&17,072&18,082&35,154\\ 
\rowcolor{color1!5!white}\multicolumn{1}{l}{	  Izabal }&4,008&3,954&7,962&&5,097&5,155&10,252&&4,198&4,322&8,520&&4,264&4,396&8,660&&2,968&3,278&6,246\\ 
\multicolumn{1}{l}{	 Jalapa }&5,721&6,153&11,874&&4,963&5,311&10,274&&4,918&5,139&10,057&&4,719&5,245&9,964&&3,860&4,117&7,977\\ 
\rowcolor{color1!5!white}\multicolumn{1}{l}{	  Jutiapa }&4,172&4,380&8,552&&6,197&7,082&13,279&&7,654&8,210&15,864&&7,024&7,647&14,671&&5,863&6,439&12,302\\ 
\multicolumn{1}{l}{	  Quetzaltenango }&6,950&6,775&13,725&&9,305&9,734&19,039&&9,703&10,351&20,054&&11,005&11,571&22,576&&7,591&8,430&16,021\\ 
\rowcolor{color1!5!white}\multicolumn{1}{l}{	  Quiché }&17,316&17,425&34,741&&21,064&23,224&44,288&&22,585&24,774&47,359&&19,796&21,481&41,277&&16,419&17,773&34,192\\ 
\multicolumn{1}{l}{	  Retalhuleu }&1,734&1,557&3,291&&2,078&1,993&4,071&&4,553&4,806&9,359&&5,352&5,637&10,989&&4,231&4,751&8,982\\ 
\rowcolor{color1!5!white}\multicolumn{1}{l}{	  Sacatepéquez }&2,031&2,330&4,361&&3,366&4,016&7,382&&3,407&3,688&7,095&&3,227&3,750&6,977&&2,575&2,953&5,528\\ 
\multicolumn{1}{l}{	 San Marcos }&16,544&15,326&31,870&&16,029&15,959&31,988&&19,407&19,613&39,020&&20,444&20,856&41,300&&16,849&17,739&34,588\\ 
\rowcolor{color1!5!white}\multicolumn{1}{l}{	 Santa Rosa }&4,085&4,678&8,763&&5,386&6,313&11,699&&5,878&6,371&12,249&&6,638&7,134&13,772&&5,054&5,205&10,259\\ 
\multicolumn{1}{l}{	 Solola }&4,093&4,095&8,188&&3,928&3,976&7,904&&5,948&6,283&12,231&&5,301&5,583&10,884&&3,358&3,910&7,268\\ 
\rowcolor{color1!5!white}\multicolumn{1}{l}{	 Suchitepequez }&2,912&3,151&6,063&&5,333&5,711&11,044&&5,985&6,802&12,787&&6,766&7,172&13,938&&4,426&5,064&9,490\\ 
\multicolumn{1}{l}{	 Totonicapan }&4,034&3,619&7,653&&4,444&4,741&9,185&&5,592&5,776&11,368&&7,236&7,241&14,477&&4,962&5,287&10,249\\ 
\rowcolor{color1!5!white}\multicolumn{1}{l}{	 Zacapa }&2,634&3,007&5,641&&3,645&4,116&7,761&&3,429&3,684&7,113&&3,390&3,833&7,223&&2,576&2,892&5,468\\ 
			\hline
		&&&&\tiny&&&&\tiny&&&&\tiny&&&&\\[-0.28cm]
		\end{longtable}\addtocounter{Cuadro}{1}
	\end{center}
\end{landscape}	




\begin{landscape}\fontsize{3.3mm}{1.9em}\selectfont \setlength{\arrayrulewidth}{01pt}
%	$\ $\\[-1.8cm]
	{\Bold\color{color1!80!black}{\normalsize Cuadro \theCuadro $\,-$   Consultas por Infecciones Respiratorias Agudas (IRA) en niños y niñas menores de cinco años; según departamento. }}\\
	{\Bold\color{color1!80!black}{\normalsize República de Guatemala, año 2011/2015.}}\\
	(Número de consultas)\\[-10mm]
	\begin{center}\fontsize{3mm}{1.5em}\selectfont \setlength{\arrayrulewidth}{0.7pt}
		$\ $\\[-2.0cm]
		$\!$\begin{longtable}{lrrrx{0.05mm}rrrx{0.05mm}rrrx{0.05mm}rrrx{0.05mm}rrr}
			\multicolumn{19}{l}{$\ $}\\[-.3cm]\hline
			$\ $\\[-.3cm]
			\multicolumn{1}{c}{\multirow{2}{*}[0.5mm]{\begin{sideways}\Bold{Departamento}\end{sideways}}}	& \multicolumn{3}{c}{\normalsize\textbf{2011}}&&\multicolumn{3}{c}{\normalsize\textbf{2012}}&& \multicolumn{3}{c}{\normalsize\textbf{2013}}&& \multicolumn{3}{c}{\normalsize\textbf{2014}}&& \multicolumn{3}{c}{\normalsize\textbf{2015}}\\[0.1cm]\cline{2-4}\cline{6-8}\cline{10-12}\cline{14-16}\cline{18-20}
			$\ $\\[0.05cm]
			\multicolumn{1}{c}{}& \multicolumn{1}{c}{\raisebox{.3cm}{\scriptsize\textbf{Mujer}}} &\multicolumn{1}{c}{\raisebox{.3cm}{\Bold{Hombre}}}&\multicolumn{1}{c}{\raisebox{.3cm}{\Bold{Total} }} &\tiny&  \multicolumn{1}{c}{\raisebox{.3cm}{\scriptsize\textbf{Mujer}}} &\multicolumn{1}{c}{\raisebox{.3cm}{\Bold{Hombre}}}&\multicolumn{1}{c}{\raisebox{.3cm}{\Bold{Total} }} &\tiny&  \multicolumn{1}{c}{\raisebox{.3cm}{\scriptsize\textbf{Mujer}}} &\multicolumn{1}{c}{\raisebox{.3cm}{\Bold{Hombre}}}&\multicolumn{1}{c}{\raisebox{.3cm}{\Bold{Total} }} &\tiny&  \multicolumn{1}{c}{\raisebox{.3cm}{\scriptsize\textbf{Mujer}}} &\multicolumn{1}{c}{\raisebox{.3cm}{\Bold{Hombre}}}&\multicolumn{1}{c}{\raisebox{.3cm}{\Bold{Total} }} &\tiny&  \multicolumn{1}{c}{\raisebox{.3cm}{\scriptsize\textbf{Mujer}}} &\multicolumn{1}{c}{\raisebox{.3cm}{\Bold{Hombre}}}&\multicolumn{1}{c}{\raisebox{.3cm}{\Bold{Total} }}\\[0.05cm]
%			&&&&&&&&&&&&&&&& \\[-0.1cm]
			\multicolumn{1}{l}{$\ $} &  \multicolumn{18}{c}{$\ $} \\[-0.48cm]				     
			\hline\endhead
			\hline \multicolumn{19}{r}{\textit{Continúa en la siguiente página}} \\[2cm]
			\endfoot
%			&&&&\tiny&&&&\tiny&&&&\tiny&&&& \\[-0.3cm]
			\multicolumn{19}{l}{\footnotesize \textbf{Fuente: }INE. Encuesta Nacional de Salud Materno Infantil (Ensmi) 2008/2009.  Datos preliminares a noviembre, para el año 2015. }\\[-0.15cm]
			\endlastfoot
			\rowcolor{color1!60!white} \multicolumn{1}{l}{\Bold{\textcolor{white}{Total }} }& \Bold{\scriptsize\textcolor{white}{617,843}}   &    \Bold{\scriptsize\textcolor{white}{587,268}}&\Bold{\scriptsize\textcolor{white}{1,205,111}}&\tiny&\Bold{\scriptsize\textcolor{white}{674,529}}&\Bold{\scriptsize\textcolor{white}{677,308}}&\Bold{\scriptsize\textcolor{white}{1,351,837}}&\tiny&\Bold{\scriptsize\textcolor{white}{809,593}}&\Bold{\scriptsize\textcolor{white}{815,765}}&\Bold{\scriptsize\textcolor{white}{1,625,358}}&\tiny&\Bold{\scriptsize\textcolor{white}{745,808}}&\Bold{\scriptsize\textcolor{white}{749,793}}&\Bold{\scriptsize\textcolor{white}{1,495,601}}&\tiny&\Bold{\scriptsize\textcolor{white}{586,666}}&\Bold{\scriptsize\textcolor{white}{598,090}}&\Bold{\scriptsize\textcolor{white}{1,184,756}}\\
			\multicolumn{1}{l}{Alta Verapaz}&35,181&32,061&67,242&&45,419&43,908&89,327&&60,356&59,433&119,789&&48,715&46,957&95,672&&37,227&37,623&74,850\\
			\rowcolor{color1!5!white}\multicolumn{1}{l}{Baja Verapaz}&21,318&20,258&41,576&&24,873&23,859&48,732&&21,098&20,093&41,191&&23,040&22,460&45,500&&15,712&16,144&31,856\\
			\multicolumn{1}{l}{Chimaltenango}&29,272&27,990&57,262&&29,563&31,050&60,613&&39,570&41,333&80,903&&36,023&36,451&72,474&&26,924&27,617&54,541\\
			\rowcolor{color1!5!white}\multicolumn{1}{l}{Chiquimula}&28,815&27,446&56,261&&27,957&26,683&54,640&&35,531&33,402&68,933&&27,313&26,176&53,489&&20,898&19,964&40,862\\
			\multicolumn{1}{l}{Peten}&36,902&34,128&71,030&&39,894&37,359&77,253&&50,355&50,383&100,738&&54,042&53,531&107,573&&49,174&49,754&98,928\\
			\rowcolor{color1!5!white}\multicolumn{1}{l}{El Progreso}&7,048&7,019&14,067&&8,755&9,784&18,539&&9,851&10,853&20,704&&9,851&10,500&20,351&&7,484&7,866&15,350\\
			\multicolumn{1}{l}{Escuintla}&57,161&53,015&110,176&&43,005&46,114&89,119&&42,428&45,492&87,920&&43,251&45,023&88,274&&35,610&36,983&72,593\\
			\rowcolor{color1!5!white}\multicolumn{1}{l}{Guatemala}&52,798&51,851&104,649&&75,305&79,130&154,435&&88,562&90,093&178,655&&82,014&83,492&165,506&&61,081&63,105&124,186\\
			\multicolumn{1}{l}{Huehuetenango}&45,457&42,544&88,001&&62,305&60,278&122,583&&75,661&75,829&151,490&&57,479&57,322&114,801&&44,782&45,227&90,009\\
			\rowcolor{color1!5!white}\multicolumn{1}{l}{Izabal}&20,389&19,229&39,618&&19,396&18,573&37,969&&19,280&19,525&38,805&&18,600&18,435&37,035&&12,171&12,723&24,894\\
			\multicolumn{1}{l}{Jalapa}&19,685&19,767&39,452&&16,755&16,558&33,313&&19,289&19,692&38,981&&16,339&16,517&32,856&&12,560&12,390&24,950\\
			\rowcolor{color1!5!white}\multicolumn{1}{l}{Jutiapa}&25,456&25,176&50,632&&31,731&32,303&64,034&&40,031&40,008&80,039&&36,017&36,722&72,739&&29,589&30,125&59,714\\
			\multicolumn{1}{l}{Quetzaltenango}&30,386&28,456&58,842&&33,344&33,419&66,763&&36,381&37,202&73,583&&34,261&34,643&68,904&&25,955&27,162&53,117\\
			\rowcolor{color1!5!white}\multicolumn{1}{l}{Quiché}&47,930&46,056&93,986&&55,228&55,364&110,592&&67,545&68,487&136,032&&55,785&56,943&112,728&&47,127&48,130&95,257\\
			\multicolumn{1}{l}{Retalhuleu}&12,538&11,417&23,955&&12,962&12,134&25,096&&15,216&14,964&30,180&&16,432&16,082&32,514&&13,586&13,687&27,273\\
			\rowcolor{color1!5!white}\multicolumn{1}{l}{Sacatepéquez}&7,722&7,449&15,171&&12,045&12,931&24,976&&15,144&15,558&30,702&&11,983&12,911&24,894&&10,055&10,559&20,614\\
			\multicolumn{1}{l}{San Marcos}&57,622&51,751&109,373&&47,553&46,903&94,456&&62,182&61,683&123,865&&64,197&63,767&127,964&&54,416&54,365&108,781\\
			\rowcolor{color1!5!white}\multicolumn{1}{l}{Santa Rosa}&15,967&17,143&33,110&&17,423&19,233&36,656&&22,055&23,279&45,334&&22,751&24,055&46,806&&18,974&19,799&38,773\\
			\multicolumn{1}{l}{Sololá}&24,438&24,208&48,646&&20,983&20,854&41,837&&29,204&28,546&57,750&&26,209&25,703&51,912&&19,416&20,087&39,503\\
			\rowcolor{color1!5!white}\multicolumn{1}{l}{Suchitepéquez}&12,990&12,425&25,415&&17,667&18,277&35,944&&20,464&20,963&41,427&&21,885&22,762&44,647&&16,815&17,199&34,014\\
			\multicolumn{1}{l}{Totonicapán}&14,854&13,701&28,555&&17,192&16,864&34,056&&21,092&20,750&41,842&&25,019&24,467&49,486&&16,153&16,541&32,694\\
			\rowcolor{color1!5!white}\multicolumn{1}{l}{Zacapa}&13,914&14,178&28,092&&15,174&15,730&30,904&&18,298&18,197&36,495&&14,602&14,874&29,476&&10,957&11,040&21,997\\
			\hline
			&&&&\tiny&&&&\tiny&&&&\tiny&&&&\\[-0.28cm]
		\end{longtable}\addtocounter{Cuadro}{1}
	\end{center}
\end{landscape}






%%%%%%%%%%%%%%%%%



\begin{landscape}\fontsize{4mm}{1.9em}\selectfont \setlength{\arrayrulewidth}{01pt}
	$\ $\\[-1.8cm]
	{\Bold\Large 5.2 Acceso a servicios de salud}\\[.5cm]
	{\Bold\color{color1!80!black}{Cuadro \theCuadro $\,-$  Mujeres embarazadas al momento de la encuesta, que recibieron atención pre natal, por establecimiento o lugar a donde asistieron; según características varias. }}\\
	{\Bold\color{color1!80!black}{República de Guatemala, año 2008/2009. }}\\
	\normalsize (Porcentajes)\\[0.4cm]
	\begin{center}\fontsize{4mm}{1.7em}
		\selectfont \setlength{\arrayrulewidth}{1pt}
		$\ $\\[-2.0cm]
		$\!$\begin{longtable}{lrrrrrrrrrrrr}
			\multicolumn{12}{l}{$\ $}\\[-1.5cm]\hline
			$\ $\\[-.5cm]
			\multicolumn{1}{c}{\multirow{2}{*}[1mm]{\Bold{Característica}}}	& \multicolumn{11}{c}{\normalsize\textbf{Lugar de atención prenatal}}\\[0.1cm]\cline{2-12}
			%$\ $\\[0.05cm]
			\multicolumn{1}{c}{}& \multicolumn{1}{c}{\scriptsize\textbf{Hospital}} &\multicolumn{1}{c}{\scriptsize\Bold{Centro }}&  \multicolumn{1}{c}{\scriptsize\textbf{Puesto }} &\multicolumn{1}{c}{\scriptsize\Bold{Centro }}&\multicolumn{1}{c}{\scriptsize\Bold{IGSS}} & \multicolumn{1}{c}{\scriptsize\textbf{Hospital}} &\multicolumn{1}{c}{\scriptsize\Bold{Consultorio}}&\multicolumn{1}{c}{\scriptsize\Bold{Aprofam }} &  \multicolumn{1}{c}{\scriptsize\textbf{Casa }} &\multicolumn{1}{c}{\scriptsize\Bold{Casa de}}&\multicolumn{1}{c}{\scriptsize\Bold{Otro }}\\%[0.1cm]
			\multicolumn{1}{c}{}& \multicolumn{1}{c}{\scriptsize\textbf{público}} &\multicolumn{1}{c}{\scriptsize\Bold{de salud }}&  \multicolumn{1}{c}{\scriptsize\textbf{de salud}} &\multicolumn{1}{c}{\scriptsize\Bold{comunitario }}&\multicolumn{1}{c}{ } & \multicolumn{1}{c}{\scriptsize\textbf{privado*}} &\multicolumn{1}{c}{ }&\multicolumn{1}{c}{ } &  \multicolumn{1}{c}{\scriptsize\textbf{Comadrona}} &\multicolumn{1}{c}{\scriptsize\Bold{entrevistada}}&\multicolumn{1}{c}{ }\\[-0.1cm]
			%			&&&&&&&&&&&&&&&& \\[-0.1cm]
			\multicolumn{1}{l}{$\ $} &  \multicolumn{11}{c}{$\ $} \\[-0.48cm]				     
			\hline\endhead
			\hline \multicolumn{12}{r}{\textit{Continúa en la siguiente página}} \\[2cm]
			\endfoot
			%			&&&&\tiny&&&&\tiny&&&&\tiny&&&& \\[-0.3cm]
			\multicolumn{12}{l}{\footnotesize INE. Encuesta Nacional de Salud Materno Infantil (Ensmi) 2008/2009.}\\[-0.1cm]
			\multicolumn{12}{l}{\footnotesize * Menos de 25 casos. }\\[-0.1cm]
			\endlastfoot
			%					\rowcolor{color1!60!white} \multicolumn{1}{l}{\Bold{\textcolor{white}{Total }} }& \Bold{\scriptsize\textcolor{white}{617,843}}   &    \Bold{\scriptsize\textcolor{white}{587,268}}&\Bold{\scriptsize\textcolor{white}{1,205,111}}&\tiny&\Bold{\scriptsize\textcolor{white}{674,529}}&\Bold{\scriptsize\textcolor{white}{677,308}}&\Bold{\scriptsize\textcolor{white}{1,351,837}}&\tiny&\Bold{\scriptsize\textcolor{white}{809,593}}&\Bold{\scriptsize\textcolor{white}{815,765}}&\Bold{\scriptsize\textcolor{white}{1,625,358}}&\tiny&\Bold{\scriptsize\textcolor{white}{745,808}}&\Bold{\scriptsize\textcolor{white}{749,793}}&\Bold{\scriptsize\textcolor{white}{1,495,601}}&\tiny&\Bold{\scriptsize\textcolor{white}{586,666}}&\Bold{\scriptsize\textcolor{white}{598,090}}&\Bold{\scriptsize\textcolor{white}{1,184,756}}\\
			\rowcolor{color1!60!white} \multicolumn{1}{l}{\Bold{	Total	}}&	8.1	&	26.8	&	16.4	&	17	&	6.6	&	15.3	&	7.6	&	2.2	&	33.2	&	10.7	&	2.8	\\
			\rowcolor{color1!40!white}\multicolumn{1}{l}{\Bold{	Área geográfica	}}&		&		&		&		&		&		&		&		&		&		&	\\
			\multicolumn{1}{l}{	Urbana	}&	 13.3&27.2&9.0&1.4&11.3&27.2&13.4&2.8&27.4&7.1&1.7\\ 
			\rowcolor{color1!5!white}\multicolumn{1}{l}{	Rural	}&	 5.0&26.5&20.8&26.4&3.7&8.1&4.1&1.8&36.7&12.9&3.4\\ 
			\rowcolor{color1!40!white} \multicolumn{1}{l}{\Bold{	Departamentos	}}&		 & 		 & 		 & 		 & 		 & 		 & 		 & 		 & 		 & 		 & 		 \\ 
			\multicolumn{1}{l}{	Guatemala	}&	 10.0&28.3&7.2&-  &14.0&25.2&17.5&2.9&30.3&3.3&1.8\\ 
			\rowcolor{color1!5!white}\multicolumn{1}{l}{	El Progreso	}&	 20.1&26.3&8.0&-  &25.5&22.7&4.1&3.2&23.7&-  &-  \\ 
			\multicolumn{1}{l}{	Sacatepéquez	}&	*	&	*	&	*	&	*	&	*	&	*	&	*	&	*	&	*	&	*	&	*	\\
			\rowcolor{color1!5!white}\multicolumn{1}{l}{	Chimaltenango	}&	 7.0&16.3&23.4&14.2&3.0&21.2&6.1&-  &34.3&2.9&-  \\ 
			\multicolumn{1}{l}{	Escuintla	}&	 10.1&13.7&4.7&-  &29.6&21.4&5.1&11.8&54.4&1.4&2.2\\ 
			\rowcolor{color1!5!white}\multicolumn{1}{l}{	Santa Rosa	}&	 5.3&47.2&4.0&14.3&-  &11.5&7.4&10.2&34.9&8.9&-  \\ 
			\multicolumn{1}{l}{	Sololá	}&	 -  &51.1&21.4&7.9&-  &14.6&-  &-  &12.0&24.8&10.2\\ 
			\rowcolor{color1!5!white}\multicolumn{1}{l}{	Totonicapán	}&	 1.4&43.6&33.8&14.5&-  &13.3&2.0&-  &6.9&38.1&-  \\ 
			\multicolumn{1}{l}{	Quetzaltenango	}&	 18.8&25.5&16.5&9.8&-  &13.6&11.0&-  &37.0&3.3&-  \\ 
			\rowcolor{color1!5!white}\multicolumn{1}{l}{	Suchitepéquez	}&	 10.0&29.7&10.3&10.3&12.2&13.2&12.3&-  &44.8&11.0&2.8\\ 
			\multicolumn{1}{l}{	Retalhuleu	}&	 13.2&25.7&15.7&11.6&7.5&20.5&6.1&-  &43.9&16.1&1.9\\ 
			\rowcolor{color1!5!white}\multicolumn{1}{l}{	San Marcos	}&	 3.2&28.1&19.4&15.7&5.3&21.8&1.8&3.8&43.4&2.8&1.5\\ 
			\multicolumn{1}{l}{	Huehuetenango	}&	 8.3&12.8&45.1&7.3&-  &9.2&4.6&-  &44.1&19.0&3.4\\ 
			\rowcolor{color1!5!white}\multicolumn{1}{l}{	Quiché	}&	 9.0&28.2&16.2&29.6&-  &7.7&-  &-  &35.0&13.5&3.1\\ 
			\multicolumn{1}{l}{	Baja Verapaz	}&	 2.5&11.9&9.1&47.2&7.2&1.9&14.1&-  &10.9&25.9&-  \\ 
			\rowcolor{color1!5!white}\multicolumn{1}{l}{	Alta Verapaz	}&	 4.9&26.2&4.9&49.3&3.0&5.5&4.9&2.3&11.8&21.1&8.0\\ 
			\multicolumn{1}{l}{	Petén	}&	 37.7&18.9&39.6&9.1&-  &14.3&2.8&1.8&39.5&6.4&1.8\\ 
			\rowcolor{color1!5!white}\multicolumn{1}{l}{	Izabal	}&	 5.0&33.0&21.4&14.3&8.5&13.0&9.1&-  &24.6&18.3&-  \\ 
			\multicolumn{1}{l}{	Zacapa	}&	 2.7&38.0&17.9&-  &12.0&15.2&19.1&-  &22.4&-  &-  \\ 
			\rowcolor{color1!5!white}\multicolumn{1}{l}{	Chiquimula	}&	 -  &21.7&6.1&59.6&-  &17.7&3.2&-  &27.2&11.1&3.6\\ 
			\multicolumn{1}{l}{	Jalapa	}&	 1.4&31.8&21.3&18.7&-  &7.5&4.7&-  &40.9&2.0&12.5\\ 
			\rowcolor{color1!5!white}\multicolumn{1}{l}{	Jutiapa	}&	 4.0&28.4&14.6&36.2&4.5&13.0&9.8&-  &46.4&7.7&-  \\ 
			\rowcolor{color1!40!white} \multicolumn{1}{l}{\Bold{	Categoría étnica	}}&		&		&		&		&		&		&		&		&		&		&		\\
			\multicolumn{1}{l}{	Indígena	}&	 3.1&27.4&19.3&26.7&2.1&8.7&4.3&1.2&30.8&20.0&3.3\\ 
			\rowcolor{color1!5!white}\multicolumn{1}{l}{	Ladino	}&	 11.7&26.3&14.3&10.1&9.8&20.0&10.0&2.9&35.0&4.1&2.4\\ 
			\rowcolor{color1!40!white} \multicolumn{1}{l}{\Bold{	Nivel de educación	}}&		&		&		&		&		&		&		&		&		&		&		\\
			\multicolumn{1}{l}{	Sin educación	}&	 4.2&21.4&21.4&29.0&1.3&4.1&0.5&0.6&38.9&17.1&3.5\\ 
			\rowcolor{color1!5!white}\multicolumn{1}{l}{	Primaria	}&	 9.0&30.5&19.0&18.2&3.3&10.4&6.9&2.0&37.7&11.1&3.1\\ 
			\multicolumn{1}{l}{	Secundaria	}&	 10.6&27.5&8.3&5.1&17.3&28.6&13.2&4.5&22.8&5.4&1.8\\ 
			\rowcolor{color1!5!white}\multicolumn{1}{l}{	Superior	}&	 5.4&8.0&1.0&-  &15.4&65.0&25.6&1.4&4.6&-  &-  \\ \hline
			&&&&&&&&&&&\\[-0.28cm]
		\end{longtable}\addtocounter{Cuadro}{1}
	\end{center}
\end{landscape}



%%%%%%%%%%%5



\hoja{
	\begin{center}\fontsize{3.mm}{1.45em}\selectfont \setlength{\arrayrulewidth}{0.7pt}
		$\!$\begin{tabular}{lrrrrrr}
			\multicolumn{7}{l}{$\ $}\\[0.15cm]
			\multicolumn{7}{l}{\Bold\color{color1!80!black}{\parbox{15cm}{\normalsize Cuadro \theCuadro $\,-$   Distribución porcentual de niñas y niños nacidos vivos en los cinco años anteriores a la encuesta, por trimestre en que las madres recibieron su primera atención prenatal; según características varias. }}}\\
			\multicolumn{7}{l}{\Bold\color{color1!80!black}{República de Guatemala, años 2008/2009.}}\\
			\multicolumn{7}{l}{\normalsize (Porcentajes)}
		\end{tabular}
	\end{center}			
	\begin{center}\fontsize{3.mm}{1.2em}\selectfont \setlength{\arrayrulewidth}{0.7pt}
		$\!$\begin{tabular}{lrrrrrr}				
			\multicolumn{1}{l}{$\ $} &  \multicolumn{6}{c}{$\ $} \\[-0.4cm]\hline
			\multicolumn{1}{l}{\multirow{3}[0]{*}{\Bold{Característica}}} & \multicolumn{1}{c}{\raisebox{-2mm}{\Bold{Tuvo}}} &\multicolumn{3}{c}{\Bold{Trimestre}}&\multicolumn{1}{c}{\multirow{3}[0]{*}{\Bold{No sabe}}}&\multicolumn{1}{c}{\multirow{3}[0]{*}{\Bold{No tuvo}}} \\[1mm]\cline{3-5}
			\multicolumn{7}{l}{$\ $}  \\[-2mm]
			\multicolumn{1}{l}{ } & \multicolumn{1}{c}{\Bold{atención}} &\multicolumn{1}{c}{\Bold{Primero}}&\multicolumn{1}{c}{\Bold{Segundo}}&\multicolumn{1}{c}{\Bold{Tercero}}&\multicolumn{1}{c}{}&\multicolumn{1}{c}{} \\
			\multicolumn{1}{l}{$\ $} &  \multicolumn{6}{c}{$\ $} \\[-0.4cm]
			\hline
			\rowcolor{color1!60!white} \multicolumn{1}{l}{\Bold{	Total	}}&	93.2	&	60.4	&	28.3	&	4.4	&	0.1	&	6.8	\\
			\rowcolor{color1!40!white} \multicolumn{1}{l}{\Bold{	Área geográfica	}}&		&		&		&		&		&		\\
			\multicolumn{1}{l}{	Urbana	}&	95.9	 & 	70.6	 & 	21.9	 & 	3.3	 & 	0.1	 & 	4.1	 \\ 
			\rowcolor{color1!5!white}\multicolumn{1}{l}{	Rural	}&	91.7	 & 	54.5	 & 	32	 & 	5	 & 	0.1	 & 	8.3	 \\ 
			\rowcolor{color1!40!white} \multicolumn{1}{l}{\Bold{	Departamentos	}}&		 & 		 & 		 & 		 & 		 & 		 \\ 
			\multicolumn{1}{l}{	Guatemala	}&	96.6	 & 	71.9	 & 	21.6	 & 	3.2	 & 	0	 & 	3.4	 \\ 
			\rowcolor{color1!5!white}\multicolumn{1}{l}{	El Progreso	}&	90.8	 & 	74.4	 & 	14.9	 & 	1.4	 & 	0.1	 & 	9.2	 \\ 
			\multicolumn{1}{l}{	Sacatepéquez	}&	94.2	&	66.9	&	24.4	&	3	&	0	&	5.8	\\
			\rowcolor{color1!5!white}\multicolumn{1}{l}{	Chimaltenango	}&	95.9	 & 	57	 & 	32.6	 & 	5.8	 & 	0.5	 & 	4.1	 \\ 
			\multicolumn{1}{l}{	Escuintla	}&	91.4	 & 	66.2	 & 	20.5	 & 	4.4	 & 	0.4	 & 	8.6	 \\ 
			\rowcolor{color1!5!white}\multicolumn{1}{l}{	Santa Rosa	}&	93.4	 & 	69.5	 & 	20.6	 & 	2.8	 & 	0.5	 & 	6.6	 \\ 
			\multicolumn{1}{l}{	Sololá	}&	92.6	 & 	44.2	 & 	35.2	 & 	12.8	 & 	0.5	 & 	7.4	 \\ 
			\rowcolor{color1!5!white}\multicolumn{1}{l}{	Totonicapán	}&	96.2	 & 	46	 & 	42.5	 & 	7.6	 & 	0.2	 & 	3.8	 \\ 
			\multicolumn{1}{l}{	Quetzaltenango	}&	97.2	 & 	67	 & 	26.9	 & 	3.3	 & 	0	 & 	2.8	 \\ 
			\rowcolor{color1!5!white}\multicolumn{1}{l}{	Suchitepéquez	}&	94.5	 & 	61.3	 & 	27.7	 & 	5.6	 & 	0	 & 	5.5	 \\ 
			\multicolumn{1}{l}{	Retalhuleu	}&	90.8	 & 	59.1	 & 	26.1	 & 	4.9	 & 	0.7	 & 	9.2	 \\ 
			\rowcolor{color1!5!white}\multicolumn{1}{l}{	San Marcos	}&	91.8	 & 	54.7	 & 	33.5	 & 	3.4	 & 	0.1	 & 	8.2	 \\ 
			\multicolumn{1}{l}{	Huehuetenango	}&	91.8	 & 	52	 & 	32.8	 & 	7	 & 	0	 & 	8.2	 \\ 
			\rowcolor{color1!5!white}\multicolumn{1}{l}{	Quiché	}&	94.9	 & 	50.7	 & 	37.9	 & 	6.2	 & 	0.1	 & 	5.1	 \\ 
			\multicolumn{1}{l}{	Baja Verapaz	}&	96.2	 & 	59.7	 & 	34.6	 & 	1.9	 & 	0	 & 	3.8	 \\ 
			\rowcolor{color1!5!white}\multicolumn{1}{l}{	Alta Verapaz	}&	92.4	 & 	61.3	 & 	28	 & 	2.9	 & 	0.2	 & 	7.6	 \\ 
			\multicolumn{1}{l}{	Petén	}&	88.5	 & 	56	 & 	28.5	 & 	3.9	 & 	0.1	 & 	11.5	 \\ 
			\rowcolor{color1!5!white}\multicolumn{1}{l}{	Izabal	}&	91.2	 & 	65.5	 & 	24.2	 & 	1.5	 & 	0	 & 	8.8	 \\ 
			\multicolumn{1}{l}{	Zacapa	}&	78.2	 & 	57.8	 & 	17.9	 & 	2.4	 & 	0	 & 	21.8	 \\ 
			\rowcolor{color1!5!white}\multicolumn{1}{l}{	Chiquimula	}&	84	 & 	48.8	 & 	31.4	 & 	3.8	 & 	0	 & 	16	 \\ 
			\multicolumn{1}{l}{	Jalapa	}&	96.9	 & 	65.8	 & 	25.4	 & 	5.7	 & 	0	 & 	3.1	 \\ 
			\rowcolor{color1!5!white}\multicolumn{1}{l}{	Jutiapa	}&	94.2	 & 	71.6	 & 	21.9	 & 	0.7	 & 	0	 & 	5.8	 \\ 
			\rowcolor{color1!40!white} \multicolumn{1}{l}{\Bold{	Categoría étnica	}}&		&		&		&		&		&		\\
			\multicolumn{1}{l}{	Indígena	}&	92.7	 & 	51.8	 & 	35.2	 & 	5.6	 & 	0.1	 & 	7.3	 \\ 
			\rowcolor{color1!5!white}\multicolumn{1}{l}{	Ladino	}&	93.7	 & 	67.8	 & 	22.4	 & 	3.4	 & 	0.1	 & 	6.3	 \\ 
			\rowcolor{color1!40!white} \multicolumn{1}{l}{\Bold{	Nivel de educación	}}&		&		&		&		&		&		\\
			\multicolumn{1}{l}{	Sin educación	}&	89.8	 & 	49.2	 & 	34.3	 & 	6.2	 & 	0.1	 & 	10.2	 \\ 
			\rowcolor{color1!5!white}\multicolumn{1}{l}{	Primaria	}&	93.1	 & 	59.2	 & 	29.6	 & 	4.2	 & 	0.2	 & 	6.9	 \\ 
			\multicolumn{1}{l}{	Secundaria	}&	98.3	 & 	77.9	 & 	17.8	 & 	2.6	 & 	0	 & 	1.7	 \\ 
			\rowcolor{color1!5!white}\multicolumn{1}{l}{	Superior	}&	99.7	 & 	90	 & 	9	 & 	0.7	 & 	0	 & 	0.3	 \\ \hline
			&&&&&&\\[-0.28cm]
			\multicolumn{7}{l}{\footnotesize Fuente:  INE. Encuesta Nacional de Salud Materno Infantil (Ensmi), 2008/2009.}
		\end{tabular}\addtocounter{Cuadro}{1}
	\end{center}}




%%%%%%%%%%%%%%%%%%%%%%






\begin{center}\fontsize{3.mm}{1.45em}\selectfont \setlength{\arrayrulewidth}{0.7pt}
	$\!$\begin{tabular}{lrrrrrr}
		\multicolumn{7}{l}{$\ $}\\[0.15cm]
		\multicolumn{7}{l}{\Bold\color{color1!80!black}{\parbox{15cm}{\normalsize Cuadro \theCuadro $\,-$ Esquema de vacunación por mes y departamento; según tipo de vacunas y grupos de edad. }}}\\
		\multicolumn{7}{l}{\Bold\color{color1!80!black}{República de Guatemala, años 2010-2015.}}\\
		\multicolumn{7}{l}{\normalsize (Dosis aplicadas)}
	\end{tabular}
\end{center}			
\begin{center}\fontsize{3.mm}{1.2em}\selectfont \setlength{\arrayrulewidth}{0.7pt}
	$\!$\begin{longtable}{lrrrrrr}				
		%			\multicolumn{7}{l}{$\ $}  \\[-2mm]
		&&&&&& \\[-2cm]\hline
		\multicolumn{1}{l}{\Bold{Departamento}} & \multicolumn{1}{c}{\Bold{2010}} &\multicolumn{1}{c}{\Bold{2011}}&\multicolumn{1}{c}{\Bold{2012}}&\multicolumn{1}{c}{\Bold{2013}}&\multicolumn{1}{c}{\textbf{2014}}&\multicolumn{1}{c}{\textbf{2015}} \\
		\multicolumn{1}{l}{$\ $} &  \multicolumn{6}{c}{$\ $} \\[-0.4cm]
		\hline\endhead
		\hline \multicolumn{7}{r}{\textit{Continúa en la siguiente página}} \\[2cm]
		\endfoot
		&&&&&& \\[-0.1cm]
		\multicolumn{7}{l}{\footnotesize INE. Encuesta Nacional de Salud Materno Infantil (Ensmi) 2008/2009.}\\[-0.1cm]
		\endlastfoot
		\rowcolor{color1!60!white} \multicolumn{1}{l}{\Bold{	Total general	}}&	5,818,857	 & 	2,897,786	 & 	4,012,987	 & 	3,277,046	 & 	3,253,982	 & 	3,253,951	 \\ 
		\rowcolor{color1!40!white} \multicolumn{1}{l}{\Bold{	Alta Verapaz	}}&	676,557	 & 	298,083	 & 	299,852	 & 	270,848	 & 	147,273	 & 	249,026	 \\ 
		\multicolumn{1}{l}{	 Ácido fólico 	}&	308,789	 & 	89,156	 & 	55,170	 & 	33,856	 & 	8,566	 & 	1,047	 \\ 
		\rowcolor{color1!5!white}\multicolumn{1}{l}{	 Sulfato ferroso 	}&	310,396	&	91,694	&	57,593	&	34,739	&	9,988	&	1,505	\\
		\multicolumn{1}{l}{	 Vitamina A 	}&	54,488	 & 	110,767	 & 	123,385	 & 	124,278	 & 	60,396	 & 	93,415	 \\ 
		\rowcolor{color1!5!white}\multicolumn{1}{l}{	 Vitaminas y minerales espolvoreados 	}&	2,884	 & 	6,466	 & 	18,102	 & 	21,063	 & 	39,510	 & 	99,052	 \\ 
		\multicolumn{1}{l}{	 Desparasitante 	}&	0	&	0	&	45,602	&	56,912	&	28,813	&	54,007	\\
		\rowcolor{color1!40!white} \multicolumn{1}{l}{\Bold{	Baja Verapaz	}}&	168,654	&	74,437	&	95,368	&	72,480	&	77,389	&	84,859	\\
		\multicolumn{1}{l}{	 Ácido fólico 	}&	63,005	&	20,738	&	24,107	&	3,451	&	298	&	25	\\
		\rowcolor{color1!5!white}\multicolumn{1}{l}{	 Sulfato ferroso 	}&	63,864	&	21,806	&	25,525	&	3,836	&	404	&	12	\\
		\multicolumn{1}{l}{	 Vitamina A 	}&	38,893	&	28,101	&	39,065	&	29,521	&	32,182	&	26,100	\\
		\rowcolor{color1!5!white}\multicolumn{1}{l}{	 Vitaminas y minerales espolvoreados 	}&	2,892	&	3,792	&	6,671	&	18,846	&	26,594	&	38,713	\\
		\multicolumn{1}{l}{	 Desparasitante 	}&	0	&	0	&	18,681	&	16,826	&	17,911	&	20,009	\\
		\rowcolor{color1!40!white} \multicolumn{1}{l}{\Bold{	Chimaltenango	}}&	334,669	&	169,275	&	300,169	&	217,131	&	201,093	&	194,297	\\
		\multicolumn{1}{l}{	 Ácido fólico 	}&	126,483	&	64,766	&	107,944	&	19,755	&	988	&	7	\\
		\rowcolor{color1!5!white}\multicolumn{1}{l}{	 Sulfato ferroso 	}&	128,074	&	65,023	&	109,558	&	20,565	&	2,918	&	6	\\
		\multicolumn{1}{l}{	 Vitamina A 	}&	80,084	&	39,428	&	74,255	&	81,901	&	85,552	&	69,914	\\
		\rowcolor{color1!5!white}\multicolumn{1}{l}{	 Vitaminas y minerales espolvoreados 	}&	28	&	58	&	8,412	&	51,365	&	60,465	&	79,293	\\
		\multicolumn{1}{l}{	 Desparasitante 	}&	0	&	0	&	38,650	&	43,545	&	51,170	&	45,077	\\
		\rowcolor{color1!40!white} \multicolumn{1}{l}{\Bold{	Chiquimula	}}&	257,733	&	140,742	&	171,295	&	141,162	&	122,570	&	111,942	\\
		\multicolumn{1}{l}{	 Ácido fólico 	}&	82,758	&	44,626	&	54,264	&	20,913	&	14,732	&	294	\\
		\rowcolor{color1!5!white}\multicolumn{1}{l}{	 Sulfato ferroso 	}&	81,816	&	46,233	&	56,528	&	21,308	&	15,440	&	255	\\
		\multicolumn{1}{l}{	 Vitamina A 	}&	93,126	&	49,005	&	57,715	&	53,704	&	47,409	&	40,021	\\
		\rowcolor{color1!5!white}\multicolumn{1}{l}{	 Vitaminas y minerales espolvoreados 	}&	33	&	878	&	2,788	&	17,262	&	21,095	&	45,951	\\
		\multicolumn{1}{l}{	 Desparasitante 	}&	0	&	0	&	28,323	&	27,975	&	23,894	&	25,421	\\
		\rowcolor{color1!40!white} \multicolumn{1}{l}{\Bold{	Petén	}}&	310,174	&	159,506	&	290,736	&	239,607	&	253,237	&	241,389	\\
		\multicolumn{1}{l}{	 Ácido fólico 	}&	118,175	&	54,942	&	83,942	&	13,671	&	1,163	&	483	\\
		\rowcolor{color1!5!white}\multicolumn{1}{l}{	 Sulfato ferroso 	}&	120,577	&	56,180	&	87,994	&	17,429	&	2,142	&	458	\\
		\multicolumn{1}{l}{	 Vitamina A 	}&	67,013	&	40,957	&	104,757	&	85,413	&	97,615	&	86,501	\\
		\rowcolor{color1!5!white}\multicolumn{1}{l}{	 Vitaminas y minerales espolvoreados 	}&	4,409	&	7,427	&	14,043	&	75,274	&	96,449	&	96,362	\\
		\multicolumn{1}{l}{	 Desparasitante 	}&	0	&	0	&	56,801	&	47,820	&	55,868	&	57,585	\\
		\rowcolor{color1!40!white} \multicolumn{1}{l}{\Bold{	El Progreso	}}&	100,822	&	16,254	&	26,427	&	23,025	&	32,811	&	35,839	\\
		\multicolumn{1}{l}{	 Ácido fólico 	}&	40,770	&	5,721	&	10,141	&	1,512	&	1,413	&	169	\\
		\rowcolor{color1!5!white}\multicolumn{1}{l}{	 Sulfato ferroso 	}&	36,908	&	5,663	&	8,632	&	1,773	&	1,670	&	193	\\
		\multicolumn{1}{l}{	 Vitamina A 	}&	23,135	&	4,854	&	7,578	&	8,287	&	13,635	&	14,191	\\
		\rowcolor{color1!5!white}\multicolumn{1}{l}{	 Vitaminas y minerales espolvoreados 	}&	9	&	16	&	76	&	7,992	&	13,343	&	15,996	\\
		\multicolumn{1}{l}{	 Desparasitante 	}&	0	&	0	&	2,461	&	3,461	&	2,750	&	5,290	\\
		\rowcolor{color1!40!white} \multicolumn{1}{l}{\Bold{	Escuintla	}}&	291,259	&	92,160	&	320,399	&	182,696	&	217,270	&	181,492	\\
		\multicolumn{1}{l}{	 Ácido fólico 	}&	113,689	&	26,656	&	120,237	&	1,097	&	87	&	58	\\
		\rowcolor{color1!5!white}\multicolumn{1}{l}{	 Sulfato ferroso 	}&	109,985	&	38,211	&	112,802	&	1,044	&	121	&	46	\\
		\multicolumn{1}{l}{	 Vitamina A 	}&	67,555	&	27,215	&	82,914	&	72,094	&	90,704	&	66,892	\\
		\rowcolor{color1!5!white}\multicolumn{1}{l}{	 Vitaminas y minerales espolvoreados 	}&	30	&	78	&	4,446	&	80,068	&	89,989	&	80,725	\\
		\multicolumn{1}{l}{	 Desparasitante 	}&	0	&	0	&	35,134	&	28,393	&	36,369	&	33,771	\\
		\rowcolor{color1!40!white} \multicolumn{1}{l}{\Bold{	Guatemala	}}&	522,835	&	527,912	&	633,965	&	441,264	&	435,114	&	373,748	\\
		\multicolumn{1}{l}{	 Ácido fólico 	}&	194,234	&	189,958	&	225,923	&	56,380	&	8,973	&	1,737	\\
		\rowcolor{color1!5!white}\multicolumn{1}{l}{	 Sulfato ferroso 	}&	203,597	&	194,853	&	232,826	&	57,482	&	9,340	&	1,989	\\
		\multicolumn{1}{l}{	 Vitamina A 	}&	124,928	&	142,930	&	173,813	&	176,127	&	181,336	&	147,636	\\
		\rowcolor{color1!5!white}\multicolumn{1}{l}{	 Vitaminas y minerales espolvoreados 	}&	76	&	171	&	1,403	&	122,882	&	159,786	&	162,240	\\
		\multicolumn{1}{l}{	 Desparasitante 	}&	0	&	0	&	61,838	&	28,393	&	75,679	&	60,146	\\
		\rowcolor{color1!40!white} \multicolumn{1}{l}{\Bold{	Huehuetenango	}}&	565,152	&	145,201	&	417,399	&	277,705	&	215,078	&	241,351	\\
		\multicolumn{1}{l}{	 Ácido fólico 	}&	250,839	&	43,379	&	138,016	&	34,051	&	20,194	&	4,337	\\
		\rowcolor{color1!5!white}\multicolumn{1}{l}{	 Sulfato ferroso 	}&	252,256	&	46,475	&	138,050	&	36,514	&	21,335	&	6,105	\\
		\multicolumn{1}{l}{	 Vitamina A 	}&	60,569	&	49,713	&	121,369	&	156,458	&	93,937	&	101,450	\\
		\rowcolor{color1!5!white}\multicolumn{1}{l}{	 Vitaminas y minerales espolvoreados 	}&	1,488	&	5,634	&	19,964	&	50,682	&	44,066	&	85,079	\\
		\multicolumn{1}{l}{	 Desparasitante 	}&	0	&	0	&	46,850	&	76,552	&	35,546	&	44,380	\\
		\rowcolor{color1!40!white} \multicolumn{1}{l}{\Bold{	Izabal	}}&	156,302	&	100,449	&	120,110	&	95,112	&	71,318	&	51,989	\\
		\multicolumn{1}{l}{	 Ácido fólico 	}&	58,385	&	30,867	&	40,410	&	15,675	&	6,948	&	1,147	\\
		\rowcolor{color1!5!white}\multicolumn{1}{l}{	 Sulfato ferroso 	}&	59,439	&	33,133	&	42,470	&	16,044	&	8,643	&	1,443	\\
		\multicolumn{1}{l}{	 Vitamina A 	}&	38,447	&	36,363	&	36,098	&	43,777	&	32,007	&	18,065	\\
		\rowcolor{color1!5!white}\multicolumn{1}{l}{	 Vitaminas y minerales espolvoreados 	}&	31	&	86	&	1,132	&	20	&	9,165	&	20,846	\\
		\multicolumn{1}{l}{	 Desparasitante 	}&	0	&	0	&	17,156	&	19,596	&	14,555	&	10,488	\\
		\rowcolor{color1!40!white} \multicolumn{1}{l}{\Bold{	Jalapa	}}&	141,523	&	41,147	&	73,987	&	104,116	&	99,288	&	97,043	\\
		\multicolumn{1}{l}{	 Ácido fólico 	}&	44,523	&	6,260	&	18,179	&	10,480	&	374	&	11	\\
		\rowcolor{color1!5!white}\multicolumn{1}{l}{	 Sulfato ferroso 	}&	45,590	&	5,967	&	17,309	&	11,705	&	1,137	&	3	\\
		\multicolumn{1}{l}{	 Vitamina A 	}&	48,028	&	23,911	&	27,434	&	44,271	&	48,629	&	35,271	\\
		\rowcolor{color1!5!white}\multicolumn{1}{l}{	 Vitaminas y minerales espolvoreados 	}&	3,382	&	5,009	&	11,065	&	16,236	&	23,199	&	40,541	\\
		\multicolumn{1}{l}{	 Desparasitante 	}&	0	&	0	&	13,113	&	21,424	&	25,949	&	21,217	\\
		\rowcolor{color1!40!white} \multicolumn{1}{l}{\Bold{	Jutiapa	}}&	98,343	&	61,250	&	96,721	&	76,586	&	120,300	&	145,241	\\
		\multicolumn{1}{l}{	 Ácido fólico 	}&	34,979	&	18,571	&	27,441	&	3,772	&	2,900	&	225	\\
		\rowcolor{color1!5!white}\multicolumn{1}{l}{	 Sulfato ferroso 	}&	37,596	&	18,751	&	29,164	&	3,869	&	4,289	&	224	\\
		\multicolumn{1}{l}{	 Vitamina A 	}&	23,534	&	18,001	&	28,201	&	31,293	&	48,876	&	53,711	\\
		\rowcolor{color1!5!white}\multicolumn{1}{l}{	 Vitaminas y minerales espolvoreados 	}&	2,234	&	5,927	&	11,915	&	24,319	&	40,331	&	59,871	\\
		\multicolumn{1}{l}{	 Desparasitante 	}&	0	&	0	&	10,854	&	13,333	&	23,904	&	31,210	\\
		\rowcolor{color1!40!white} \multicolumn{1}{l}{\Bold{	Quetzaltenango	}}&	263,348	&	79,281	&	153,991	&	168,727	&	216,336	&	184,840	\\
		\multicolumn{1}{l}{	 Ácido fólico 	}&	93,958	&	26,894	&	53,649	&	32,845	&	21,205	&	86	\\
		\rowcolor{color1!5!white}\multicolumn{1}{l}{	 Sulfato ferroso 	}&	101,928	&	27,966	&	53,235	&	33,036	&	21,456	&	117	\\
		\multicolumn{1}{l}{	 Vitamina A 	}&	67,322	&	23,834	&	42,279	&	51,916	&	75,727	&	70,291	\\
		\rowcolor{color1!5!white}\multicolumn{1}{l}{	 Vitaminas y minerales espolvoreados 	}&	140	&	587	&	4,828	&	30,421	&	62,472	&	77,105	\\
		\multicolumn{1}{l}{	 Desparasitante 	}&	0	&	0	&	14,458	&	20,509	&	35,476	&	37,241	\\
		\rowcolor{color1!40!white} \multicolumn{1}{l}{\Bold{	Quiché	}}&	504,455	&	314,251	&	343,531	&	365,826	&	261,067	&	280,623	\\
		\multicolumn{1}{l}{	 Ácido fólico 	}&	197,546	&	110,266	&	100,495	&	79,958	&	24,185	&	1,977	\\
		\rowcolor{color1!5!white}\multicolumn{1}{l}{	 Sulfato ferroso 	}&	199,444	&	111,370	&	107,337	&	82,003	&	24,470	&	2,183	\\
		\multicolumn{1}{l}{	 Vitamina A 	}&	105,575	&	86,037	&	109,816	&	115,713	&	95,321	&	101,880	\\
		\rowcolor{color1!5!white}\multicolumn{1}{l}{	 Vitaminas y minerales espolvoreados 	}&	1,890	&	6,578	&	25,883	&	29,400	&	66,847	&	111,507	\\
		\multicolumn{1}{l}{	 Desparasitante 	}&	0	&	0	&	44,847	&	58,752	&	50,244	&	63,076	\\
		\rowcolor{color1!40!white} \multicolumn{1}{l}{\Bold{	Retalhuleu	}}&	95,041	&	29,413	&	91,789	&	64,996	&	83,680	&	82,806	\\
		\multicolumn{1}{l}{	 Ácido fólico 	}&	30,230	&	9,661	&	30,468	&	6,667	&	4,285	&	320	\\
		\rowcolor{color1!5!white}\multicolumn{1}{l}{	 Sulfato ferroso 	}&	37,319	&	10,884	&	30,979	&	7,288	&	5,561	&	497	\\
		\multicolumn{1}{l}{	 Vitamina A 	}&	27,417	&	8,651	&	29,064	&	21,721	&	31,387	&	30,948	\\
		\rowcolor{color1!5!white}\multicolumn{1}{l}{	 Vitaminas y minerales espolvoreados 	}&	75	&	217	&	1,278	&	19,785	&	27,571	&	33,672	\\
		\multicolumn{1}{l}{	 Desparasitante 	}&	0	&	0	&	12,159	&	9,535	&	14,876	&	17,369	\\
		\rowcolor{color1!40!white} \multicolumn{1}{l}{\Bold{	Sacatepéquez	}}&	115,886	&	77,089	&	60,128	&	30,914	&	49,414	&	67,304	\\
		\multicolumn{1}{l}{	 Ácido fólico 	}&	36,934	&	23,905	&	23,085	&	5,583	&	108	&	10	\\
		\rowcolor{color1!5!white}\multicolumn{1}{l}{	 Sulfato ferroso 	}&	47,425	&	31,045	&	23,004	&	5,596	&	190	&	15	\\
		\multicolumn{1}{l}{	 Vitamina A 	}&	31,518	&	22,097	&	13,920	&	9,309	&	19,608	&	24,161	\\
		\rowcolor{color1!5!white}\multicolumn{1}{l}{	 Vitaminas y minerales espolvoreados 	}&	9	&	42	&	119	&	6,850	&	20,741	&	29,606	\\
		\multicolumn{1}{l}{	 Desparasitante 	}&	0	&	0	&	7,777	&	3,576	&	8,767	&	13,512	\\
		\rowcolor{color1!40!white} \multicolumn{1}{l}{\Bold{	San Marcos	}}&	347,730	&	148,955	&	97,456	&	67,073	&	191,824	&	154,420	\\
		\multicolumn{1}{l}{	 Ácido fólico 	}&	95,188	&	42,327	&	29,923	&	12,001	&	38,775	&	12,410	\\
		\rowcolor{color1!5!white}\multicolumn{1}{l}{	 Sulfato ferroso 	}&	118,553	&	48,322	&	30,668	&	13,436	&	41,612	&	13,449	\\
		\multicolumn{1}{l}{	 Vitamina A 	}&	132,795	&	55,237	&	25,173	&	22,100	&	73,885	&	64,487	\\
		\rowcolor{color1!5!white}\multicolumn{1}{l}{	 Vitaminas y minerales espolvoreados 	}&	1,194	&	3,069	&	11,692	&	13,050	&	28,785	&	42,820	\\
		\multicolumn{1}{l}{	 Desparasitante 	}&	0	&	0	&	10,194	&	6,486	&	8,767	&	21,254	\\
		\rowcolor{color1!40!white} \multicolumn{1}{l}{\Bold{	Santa Rosa	}}&	231,323	&	127,352	&	127,635	&	100,421	&	69,969	&	85,457	\\
		\multicolumn{1}{l}{	 Ácido fólico 	}&	86,861	&	46,956	&	44,118	&	13,190	&	2,350	&	14	\\
		\rowcolor{color1!5!white}\multicolumn{1}{l}{	 Sulfato ferroso 	}&	87,775	&	46,709	&	44,028	&	15,602	&	2,889	&	47	\\
		\multicolumn{1}{l}{	 Vitamina A 	}&	56,175	&	32,456	&	36,248	&	32,916	&	27,196	&	29,820	\\
		\rowcolor{color1!5!white}\multicolumn{1}{l}{	 Vitaminas y minerales espolvoreados 	}&	512	&	1,231	&	3,241	&	24,421	&	24,626	&	36,735	\\
		\multicolumn{1}{l}{	 Desparasitante 	}&	0	&	0	&	14,951	&	14,292	&	12,908	&	18,841	\\
		\rowcolor{color1!40!white} \multicolumn{1}{l}{\Bold{	Sololá	}}&	260,816	&	59,698	&	63,489	&	100,417	&	122,299	&	114,384	\\
		\multicolumn{1}{l}{	 Ácido fólico 	}&	99,642	&	17,430	&	14,143	&	19,239	&	11,526	&	44	\\
		\rowcolor{color1!5!white}\multicolumn{1}{l}{	 Sulfato ferroso 	}&	104,550	&	17,962	&	14,344	&	19,722	&	12,005	&	63	\\
		\multicolumn{1}{l}{	 Vitamina A 	}&	53,951	&	19,571	&	24,099	&	31,361	&	44,041	&	39,642	\\
		\rowcolor{color1!5!white}\multicolumn{1}{l}{	 Vitaminas y minerales espolvoreados 	}&	2,673	&	4,735	&	10,903	&	13,223	&	30,559	&	47,108	\\
		\multicolumn{1}{l}{	 Desparasitante 	}&	0	&	0	&	10,458	&	16,872	&	24,168	&	27,527	\\
		\rowcolor{color1!40!white} \multicolumn{1}{l}{\Bold{	Suchitepéquez	}}&	124,230	&	75,017	&	80,974	&	140,845	&	131,261	&	131,543	\\
		\multicolumn{1}{l}{	 Ácido fólico 	}&	31,422	&	12,231	&	18,467	&	1,557	&	724	&	170	\\
		\rowcolor{color1!5!white}\multicolumn{1}{l}{	 Sulfato ferroso 	}&	39,805	&	27,262	&	28,387	&	4,562	&	2,812	&	631	\\
		\multicolumn{1}{l}{	 Vitamina A 	}&	52,987	&	35,426	&	33,707	&	56,294	&	56,663	&	54,366	\\
		\rowcolor{color1!5!white}\multicolumn{1}{l}{	 Vitaminas y minerales espolvoreados 	}&	16	&	98	&	413	&	55,469	&	47,922	&	53,792	\\
		\multicolumn{1}{l}{	 Desparasitante 	}&	0	&	0	&	9,365	&	22,963	&	23,140	&	22,584	\\
		\rowcolor{color1!40!white} \multicolumn{1}{l}{\Bold{	Totonicapán	}}&	114,182	&	123,607	&	64,202	&	67,422	&	75,234	&	88,291	\\
		\multicolumn{1}{l}{	 Ácido fólico 	}&	36,541	&	28,081	&	12,426	&	10,908	&	3,805	&	19	\\
		\rowcolor{color1!5!white}\multicolumn{1}{l}{	 Sulfato ferroso 	}&	36,709	&	35,162	&	12,925	&	10,679	&	4,311	&	33	\\
		\multicolumn{1}{l}{	 Vitamina A 	}&	37,190	&	53,712	&	29,051	&	24,501	&	28,788	&	34,179	\\
		\rowcolor{color1!5!white}\multicolumn{1}{l}{	 Vitaminas y minerales espolvoreados 	}&	3,742	&	6,652	&	9,800	&	10,985	&	23,588	&	37,841	\\
		\multicolumn{1}{l}{	 Desparasitante 	}&	0	&	0	&	9,971	&	10,349	&	14,742	&	16,219	\\
		\rowcolor{color1!40!white} \multicolumn{1}{l}{\Bold{	Zacapa	}}&	137,822	&	36,707	&	83,364	&	28,673	&	60,157	&	56,067	\\
		\multicolumn{1}{l}{	 Ácido fólico 	}&	54,147	&	13,030	&	25,411	&	1,429	&	2,841	&	265	\\
		\rowcolor{color1!5!white}\multicolumn{1}{l}{	 Sulfato ferroso 	}&	53,522	&	13,514	&	29,371	&	2,330	&	3,362	&	273	\\
		\multicolumn{1}{l}{	 Vitamina A 	}&	30,035	&	9,738	&	26,628	&	12,787	&	23,847	&	22,089	\\
		\rowcolor{color1!5!white}\multicolumn{1}{l}{	 Vitaminas y minerales espolvoreados 	}&	118	&	425	&	1,954	&	6,559	&	20,028	&	21,880	\\
		\multicolumn{1}{l}{	 Desparasitante 	}&	0	&	0	&	10,410	&	5,568	&	10,079	&	11,560	\\
		\hline
	\end{longtable}\addtocounter{Cuadro}{1}
\end{center}





%%%%%%%%%%%%%%%%%%%%%%%%%%%%%%5






%%%%%%%%%%%%%%%%%



\begin{landscape}%\fontsize{4mm}{1.9em}\selectfont \setlength{\arrayrulewidth}{01pt}
	%	$\ $\\[-1.8cm]
	%	{\Bold\color{color1!80!black}{Cuadro \theCuadro $\,-$  Mujeres embarazadas al momento de la encuesta, que recibieron atención pre natal, por establecimiento o lugar a donde asistieron; según características varias. }}\\
	%	{\Bold\color{color1!80!black}{República de Guatemala, año 2008/2009. }}\\
	%	\normalsize (Porcentajes)\\[0.4cm]
	\begin{center}\fontsize{4mm}{1.8em}
		\selectfont \setlength{\arrayrulewidth}{1pt}
		$\ $\\[-2.0cm]
		$\!$\begin{longtable}{llrrrrrrrrrrrrrrr}
			\multicolumn{17}{l}{\Bold\color{color1!80!black}{\parbox{20cm}{Cuadro \theCuadro $\,-$ Esquema de vacunación por mes y departamento; según tipo de vacunas y grupos de edad. }}}\\
			\multicolumn{17}{l}{\Bold\color{color1!80!black}{República de Guatemala, año 2015. }}\\
			\multicolumn{17}{l}{	\normalsize (Dosis aplicadas)}\\
			\multicolumn{17}{l}{$\ $}\\[-.5cm]\hline
			$\ $\\[-.3cm]
			\multicolumn{1}{c}{\multirow{3}{*}[1mm]{\begin{sideways}\Bold{Mes} \end{sideways}}}	&	\multicolumn{1}{c}{\multirow{3}{*}[1mm]{\begin{sideways}\Bold{Departamento}\end{sideways}}}&	\multicolumn{10}{c}{\Bold{Menores a 1 año}} &\multicolumn{3}{c}{\Bold{Entre 1 y 2 años}}&\multicolumn{2}{c}{\Bold{4 años}}\\[0.1cm]\cline{3-12}\cline{16-17}
			%$\ $\\[0.05cm]
			\multicolumn{1}{c}{ }	&	\multicolumn{1}{c}{ }&	\multicolumn{1}{c}{\Bold{BCG}} &\multicolumn{1}{c}{\Bold{Hep B}}&\multicolumn{3}{c}{\Bold{OPV}}&\multicolumn{2}{c}{\Bold{Rotavirus 2*}}&\multicolumn{3}{c}{\Bold{Rotavirus 3*}}&	\multicolumn{1}{c}{\Bold{DPT}}&	\multicolumn{1}{c}{\Bold{OPV}}&	\multicolumn{1}{c}{\Bold{SPR}}&	\multicolumn{2}{c}{\Bold{DPT}}\\[0.1cm]
			\multicolumn{1}{c}{ }&\multicolumn{1}{c}{ }&	\multicolumn{1}{c}{1°}&		\multicolumn{1}{c}{1°}&		\multicolumn{1}{c}{1°}&		\multicolumn{1}{c}{2°}&		\multicolumn{1}{c}{3°}&		\multicolumn{1}{c}{1°}&		\multicolumn{1}{c}{2°}&		\multicolumn{1}{c}{1°}&		\multicolumn{1}{c}{2°}&		\multicolumn{1}{c}{3°}&		\multicolumn{1}{c}{R1}&		\multicolumn{1}{c}{R1}&		\multicolumn{1}{c}{1°}&	\multicolumn{1}{c}{R2}&		\multicolumn{1}{c}{R2}	\\\hline\endhead
						\hline \multicolumn{17}{r}{\textit{Continúa en la siguiente página}} \\
						\endfoot
						\hline \multicolumn{17}{l}{\textbf{Fuente:}\textit{SIGSA}} \\\endlastfoot
			\rowcolor{color1!40!white} \multicolumn{1}{l}{\Bold{	\footnotesize	 Total 	}}&		&	 286,123 	&	 96,437 	&	 354,206 	&	 334,688 	&	 282,830 	&	 334,945 	&	 265,266 	&	 172 	&	 54 	&	 27 	&	 295,202 	&	 271,830 	&	 330,677 	&	 254,214 	&	 236,152 	\\
			\multicolumn{1}{l}{	\footnotesize	 Enero 	}&	 Alta Verapaz 	&	 2,305 	&	 477 	&	 2,823 	&	 1,707 	&	 891 	&	 2,110 	&	 901 	&	 -   	&	 -   	&	 -   	&	 1,486 	&	 1,500 	&	 252 	&	 1,450 	&	 1,396 	\\
			\rowcolor{color1!5!white}\multicolumn{1}{l}{	\footnotesize	 Enero 	}&	 Baja Verapaz 	&	 597 	&	 239 	&	 492 	&	 647 	&	 589 	&	 533 	&	 539 	&	 -   	&	 -   	&	 -   	&	 920 	&	 531 	&	 214 	&	 807 	&	 496 	\\
			\multicolumn{1}{l}{	\footnotesize	 Enero 	}&	 Chimaltenango 	&	 1,138 	&	 394 	&	 1,473 	&	 1,209 	&	 799 	&	 1,729 	&	 848 	&	 -   	&	 -   	&	 -   	&	 1,222 	&	 1,041 	&	 9 	&	 1,027 	&	 955 	\\
			\rowcolor{color1!5!white}\multicolumn{1}{l}{	\footnotesize	 Enero 	}&	 Chiquimula 	&	 837 	&	 444 	&	 885 	&	 1,041 	&	 593 	&	 708 	&	 546 	&	 -   	&	 -   	&	 -   	&	 761 	&	 648 	&	 46 	&	 670 	&	 571 	\\
			\multicolumn{1}{l}{	\footnotesize	 Enero 	}&	 Petén 	&	 1,191 	&	 478 	&	 1,445 	&	 1,449 	&	 1,005 	&	 1,349 	&	 1,185 	&	 1 	&	 -   	&	 -   	&	 726 	&	 655 	&	 457 	&	 655 	&	 593 	\\
			\rowcolor{color1!5!white}\multicolumn{1}{l}{	\footnotesize	 Enero 	}&	 El Progreso 	&	 352 	&	 130 	&	 454 	&	 492 	&	 269 	&	 322 	&	 286 	&	 -   	&	 -   	&	 -   	&	 433 	&	 274 	&	 50 	&	 395 	&	 235 	\\
			\multicolumn{1}{l}{	\footnotesize	 Enero 	}&	 Escuintla 	&	 1,132 	&	 133 	&	 777 	&	 578 	&	 363 	&	 980 	&	 700 	&	 -   	&	 -   	&	 -   	&	 444 	&	 465 	&	 166 	&	 424 	&	 429 	\\
			\rowcolor{color1!5!white}\multicolumn{1}{l}{	\footnotesize	 Enero 	}&	 Guatemala 	&	 4,095 	&	 721 	&	 5,163 	&	 4,975 	&	 3,452 	&	 7,049 	&	 4,129 	&	 -   	&	 -   	&	 -   	&	 5,695 	&	 4,046 	&	 75 	&	 3,764 	&	 2,903 	\\
			\multicolumn{1}{l}{	\footnotesize	 Enero 	}&	 Huehuetenango 	&	 1,922 	&	 334 	&	 1,341 	&	 696 	&	 503 	&	 974 	&	 444 	&	 -   	&	 2 	&	 -   	&	 772 	&	 807 	&	 36 	&	 616 	&	 635 	\\
			\rowcolor{color1!5!white}\multicolumn{1}{l}{	\footnotesize	 Enero 	}&	 Izabal 	&	 786 	&	 32 	&	 826 	&	 621 	&	 307 	&	 568 	&	 360 	&	 1 	&	 -   	&	 -   	&	 245 	&	 377 	&	 27 	&	 174 	&	 287 	\\
			\multicolumn{1}{l}{	\footnotesize	 Enero 	}&	 Jalapa 	&	 680 	&	 402 	&	 460 	&	 766 	&	 352 	&	 750 	&	 710 	&	 -   	&	 -   	&	 -   	&	 1,116 	&	 331 	&	 1 	&	 784 	&	 285 	\\
			\rowcolor{color1!5!white}\multicolumn{1}{l}{	\footnotesize	 Enero 	}&	 Jutiapa 	&	 1,269 	&	 222 	&	 1,287 	&	 1,283 	&	 714 	&	 1,051 	&	 933 	&	 -   	&	 -   	&	 -   	&	 1,056 	&	 954 	&	 86 	&	 873 	&	 779 	\\
			\multicolumn{1}{l}{	\footnotesize	 Enero 	}&	 Quetzaltenango 	&	 1,645 	&	 79 	&	 1,979 	&	 1,965 	&	 1,449 	&	 758 	&	 506 	&	 -   	&	 -   	&	 -   	&	 588 	&	 1,030 	&	 116 	&	 482 	&	 683 	\\
			\rowcolor{color1!5!white}\multicolumn{1}{l}{	\footnotesize	 Enero 	}&	 Quiché 	&	 2,391 	&	 843 	&	 2,235 	&	 2,261 	&	 1,359 	&	 2,017 	&	 1,229 	&	 5 	&	 -   	&	 3 	&	 2,219 	&	 1,812 	&	 43 	&	 1,743 	&	 1,481 	\\
			\multicolumn{1}{l}{	\footnotesize	 Enero 	}&	 Retalhuleu 	&	 585 	&	 55 	&	 839 	&	 944 	&	 507 	&	 719 	&	 535 	&	 -   	&	 -   	&	 -   	&	 572 	&	 541 	&	 29 	&	 454 	&	 376 	\\
			\rowcolor{color1!5!white}\multicolumn{1}{l}{	\footnotesize	 Enero 	}&	 Sacatepéquez 	&	 294 	&	 16 	&	 843 	&	 1,042 	&	 594 	&	 742 	&	 542 	&	 -   	&	 -   	&	 -   	&	 574 	&	 541 	&	 -   	&	 498 	&	 510 	\\
			\multicolumn{1}{l}{	\footnotesize	 Enero 	}&	 San Marcos 	&	 3,015 	&	 617 	&	 2,082 	&	 1,623 	&	 951 	&	 2,874 	&	 1,380 	&	 1 	&	 2 	&	 2 	&	 2,332 	&	 1,122 	&	 63 	&	 1,545 	&	 875 	\\
			\rowcolor{color1!5!white}\multicolumn{1}{l}{	\footnotesize	 Enero 	}&	 Santa Rosa 	&	 622 	&	 373 	&	 1,101 	&	 883 	&	 489 	&	 814 	&	 617 	&	 -   	&	 -   	&	 -   	&	 812 	&	 839 	&	 105 	&	 806 	&	 812 	\\
			\multicolumn{1}{l}{	\footnotesize	 Enero 	}&	 Solola 	&	 416 	&	 71 	&	 548 	&	 509 	&	 301 	&	 524 	&	 430 	&	 6 	&	 -   	&	 -   	&	 273 	&	 328 	&	 20 	&	 260 	&	 264 	\\
			\rowcolor{color1!5!white}\multicolumn{1}{l}{	\footnotesize	 Enero 	}&	 Suchitepéquez 	&	 947 	&	 178 	&	 1,292 	&	 1,289 	&	 883 	&	 909 	&	 711 	&	 1 	&	 1 	&	 -   	&	 924 	&	 847 	&	 93 	&	 791 	&	 675 	\\
			\multicolumn{1}{l}{	\footnotesize	 Enero 	}&	 Totonicapán 	&	 799 	&	 126 	&	 749 	&	 581 	&	 436 	&	 100 	&	 102 	&	 -   	&	 -   	&	 1 	&	 69 	&	 336 	&	 15 	&	 52 	&	 176 	\\
			\rowcolor{color1!5!white}\multicolumn{1}{l}{	\footnotesize	 Enero 	}&	 Zacapa 	&	 460 	&	 193 	&	 721 	&	 722 	&	 478 	&	 634 	&	 457 	&	 -   	&	 -   	&	 -   	&	 473 	&	 474 	&	 67 	&	 308 	&	 305 	\\
			\multicolumn{1}{l}{	\footnotesize	 Febrero 	}&	 Alta Verapaz 	&	 3,036 	&	 707 	&	 3,361 	&	 2,286 	&	 1,125 	&	 2,659 	&	 978 	&	 -   	&	 -   	&	 -   	&	 2,078 	&	 1,803 	&	 1,871 	&	 1,609 	&	 1,488 	\\
			\rowcolor{color1!5!white}\multicolumn{1}{l}{	\footnotesize	 Febrero 	}&	 Baja Verapaz 	&	 652 	&	 251 	&	 716 	&	 804 	&	 690 	&	 694 	&	 584 	&	 -   	&	 -   	&	 -   	&	 806 	&	 730 	&	 1,003 	&	 638 	&	 579 	\\
			\multicolumn{1}{l}{	\footnotesize	 Febrero 	}&	 Chimaltenango 	&	 1,343 	&	 714 	&	 1,499 	&	 1,250 	&	 1,031 	&	 1,450 	&	 831 	&	 -   	&	 -   	&	 -   	&	 1,562 	&	 1,334 	&	 1,918 	&	 1,214 	&	 1,117 	\\
			\rowcolor{color1!5!white}\multicolumn{1}{l}{	\footnotesize	 Febrero 	}&	 Chiquimula 	&	 1,037 	&	 460 	&	 1,054 	&	 994 	&	 749 	&	 1,186 	&	 648 	&	 -   	&	 -   	&	 -   	&	 998 	&	 728 	&	 1,413 	&	 1,012 	&	 794 	\\
			\multicolumn{1}{l}{	\footnotesize	 Febrero 	}&	 Petén 	&	 1,561 	&	 852 	&	 2,020 	&	 2,058 	&	 1,559 	&	 1,837 	&	 1,602 	&	 -   	&	 -   	&	 -   	&	 3,001 	&	 2,807 	&	 1,335 	&	 2,661 	&	 2,509 	\\
			\rowcolor{color1!5!white}\multicolumn{1}{l}{	\footnotesize	 Febrero 	}&	 El Progreso 	&	 377 	&	 152 	&	 491 	&	 568 	&	 387 	&	 637 	&	 411 	&	 -   	&	 -   	&	 -   	&	 545 	&	 443 	&	 745 	&	 413 	&	 327 	\\
			\multicolumn{1}{l}{	\footnotesize	 Febrero 	}&	 Escuintla 	&	 2,242 	&	 476 	&	 2,754 	&	 2,157 	&	 1,824 	&	 2,588 	&	 1,874 	&	 -   	&	 -   	&	 -   	&	 2,337 	&	 2,260 	&	 2,173 	&	 2,627 	&	 2,527 	\\
			\rowcolor{color1!5!white}\multicolumn{1}{l}{	\footnotesize	 Febrero 	}&	 Guatemala 	&	 3,875 	&	 1,152 	&	 5,813 	&	 4,964 	&	 3,882 	&	 7,355 	&	 4,387 	&	 -   	&	 -   	&	 -   	&	 5,893 	&	 4,530 	&	 7,714 	&	 3,794 	&	 2,987 	\\
			\multicolumn{1}{l}{	\footnotesize	 Febrero 	}&	 Huehuetenango 	&	 4,540 	&	 539 	&	 5,416 	&	 3,467 	&	 2,200 	&	 5,321 	&	 2,172 	&	 19 	&	 5 	&	 -   	&	 4,837 	&	 4,535 	&	 5,158 	&	 4,015 	&	 3,807 	\\
			\rowcolor{color1!5!white}\multicolumn{1}{l}{	\footnotesize	 Febrero 	}&	 Izabal 	&	 765 	&	 262 	&	 1,176 	&	 985 	&	 471 	&	 953 	&	 465 	&	 -   	&	 -   	&	 -   	&	 425 	&	 599 	&	 799 	&	 372 	&	 485 	\\
			\multicolumn{1}{l}{	\footnotesize	 Febrero 	}&	 Jalapa 	&	 924 	&	 482 	&	 1,158 	&	 1,528 	&	 684 	&	 1,268 	&	 572 	&	 1 	&	 -   	&	 -   	&	 1,509 	&	 1,007 	&	 2,144 	&	 1,210 	&	 801 	\\
			\rowcolor{color1!5!white}\multicolumn{1}{l}{	\footnotesize	 Febrero 	}&	 Jutiapa 	&	 1,282 	&	 341 	&	 913 	&	 1,010 	&	 625 	&	 1,301 	&	 1,132 	&	 -   	&	 1 	&	 -   	&	 1,508 	&	 933 	&	 2,440 	&	 1,248 	&	 757 	\\
			\multicolumn{1}{l}{	\footnotesize	 Febrero 	}&	 Quetzaltenango 	&	 1,867 	&	 337 	&	 1,740 	&	 1,571 	&	 1,314 	&	 3,139 	&	 1,776 	&	 1 	&	 -   	&	 -   	&	 2,807 	&	 1,459 	&	 2,207 	&	 2,267 	&	 1,240 	\\
			\rowcolor{color1!5!white}\multicolumn{1}{l}{	\footnotesize	 Febrero 	}&	 Quiché 	&	 2,959 	&	 967 	&	 2,572 	&	 2,274 	&	 1,678 	&	 3,129 	&	 1,738 	&	 1 	&	 1 	&	 1 	&	 2,857 	&	 2,034 	&	 3,477 	&	 2,285 	&	 1,814 	\\
			\multicolumn{1}{l}{	\footnotesize	 Febrero 	}&	 Retalhuleu 	&	 675 	&	 102 	&	 813 	&	 744 	&	 590 	&	 1,020 	&	 678 	&	 -   	&	 -   	&	 -   	&	 841 	&	 642 	&	 971 	&	 625 	&	 438 	\\
			\rowcolor{color1!5!white}\multicolumn{1}{l}{	\footnotesize	 Febrero 	}&	 Sacatepéquez 	&	 279 	&	 39 	&	 991 	&	 835 	&	 711 	&	 1,350 	&	 815 	&	 -   	&	 -   	&	 -   	&	 1,096 	&	 720 	&	 1,107 	&	 884 	&	 678 	\\
			\multicolumn{1}{l}{	\footnotesize	 Febrero 	}&	 San Marcos 	&	 4,436 	&	 1,176 	&	 4,396 	&	 3,473 	&	 2,216 	&	 5,430 	&	 2,206 	&	 1 	&	 -   	&	 1 	&	 4,233 	&	 2,718 	&	 5,469 	&	 3,255 	&	 2,300 	\\
			\rowcolor{color1!5!white}\multicolumn{1}{l}{	\footnotesize	 Febrero 	}&	 Santa Rosa 	&	 652 	&	 180 	&	 687 	&	 608 	&	 425 	&	 1,117 	&	 803 	&	 -   	&	 -   	&	 -   	&	 1,119 	&	 528 	&	 909 	&	 972 	&	 520 	\\
			\multicolumn{1}{l}{	\footnotesize	 Febrero 	}&	 Sololá 	&	 499 	&	 195 	&	 779 	&	 698 	&	 602 	&	 787 	&	 567 	&	 1 	&	 -   	&	 -   	&	 972 	&	 766 	&	 742 	&	 760 	&	 646 	\\
			\rowcolor{color1!5!white}\multicolumn{1}{l}{	\footnotesize	 Febrero 	}&	 Suchitepéquez 	&	 1,117 	&	 344 	&	 1,387 	&	 1,310 	&	 1,059 	&	 1,666 	&	 1,081 	&	 -   	&	 1 	&	 -   	&	 2,192 	&	 1,467 	&	 2,390 	&	 1,644 	&	 1,191 	\\
			\multicolumn{1}{l}{	\footnotesize	 Febrero 	}&	 Totonicapán 	&	 940 	&	 228 	&	 1,009 	&	 762 	&	 529 	&	 1,001 	&	 651 	&	 -   	&	 -   	&	 -   	&	 813 	&	 659 	&	 626 	&	 544 	&	 455 	\\
			\rowcolor{color1!5!white}\multicolumn{1}{l}{	\footnotesize	 Febrero 	}&	 Zacapa 	&	 563 	&	 182 	&	 673 	&	 726 	&	 532 	&	 734 	&	 590 	&	 -   	&	 -   	&	 1 	&	 704 	&	 603 	&	 620 	&	 493 	&	 396 	\\
			\multicolumn{1}{l}{	\footnotesize	 Marzo 	}&	 Alta Verapaz 	&	 2,256 	&	 627 	&	 2,903 	&	 2,161 	&	 1,128 	&	 2,695 	&	 1,329 	&	 -   	&	 -   	&	 -   	&	 1,830 	&	 1,489 	&	 2,859 	&	 1,675 	&	 1,366 	\\
			\rowcolor{color1!5!white}\multicolumn{1}{l}{	\footnotesize	 Marzo 	}&	 Baja Verapaz 	&	 553 	&	 258 	&	 602 	&	 594 	&	 588 	&	 479 	&	 459 	&	 1 	&	 -   	&	 -   	&	 588 	&	 599 	&	 753 	&	 481 	&	 496 	\\
			\multicolumn{1}{l}{	\footnotesize	 Marzo 	}&	 Chimaltenango 	&	 1,113 	&	 615 	&	 1,015 	&	 908 	&	 755 	&	 1,534 	&	 988 	&	 -   	&	 -   	&	 -   	&	 1,000 	&	 737 	&	 1,573 	&	 869 	&	 603 	\\
			\rowcolor{color1!5!white}\multicolumn{1}{l}{	\footnotesize	 Marzo 	}&	 Chiquimula 	&	 816 	&	 274 	&	 1,148 	&	 1,054 	&	 912 	&	 1,227 	&	 775 	&	 -   	&	 -   	&	 -   	&	 1,059 	&	 888 	&	 1,791 	&	 1,021 	&	 889 	\\
			\multicolumn{1}{l}{	\footnotesize	 Marzo 	}&	 Petén 	&	 1,372 	&	 792 	&	 1,379 	&	 1,685 	&	 1,583 	&	 1,296 	&	 1,462 	&	 -   	&	 -   	&	 -   	&	 1,556 	&	 1,499 	&	 1,847 	&	 1,263 	&	 1,240 	\\
			\rowcolor{color1!5!white}\multicolumn{1}{l}{	\footnotesize	 Marzo 	}&	 El Progreso 	&	 274 	&	 195 	&	 549 	&	 555 	&	 508 	&	 452 	&	 378 	&	 -   	&	 -   	&	 -   	&	 365 	&	 440 	&	 571 	&	 335 	&	 357 	\\
			\multicolumn{1}{l}{	\footnotesize	 Marzo 	}&	 Escuintla 	&	 1,627 	&	 310 	&	 1,168 	&	 848 	&	 715 	&	 1,220 	&	 774 	&	 -   	&	 -   	&	 -   	&	 1,118 	&	 1,035 	&	 1,558 	&	 1,006 	&	 909 	\\
			\rowcolor{color1!5!white}\multicolumn{1}{l}{	\footnotesize	 Marzo 	}&	 Guatemala 	&	 3,191 	&	 1,272 	&	 6,184 	&	 5,818 	&	 4,756 	&	 4,722 	&	 4,374 	&	 1 	&	 -   	&	 -   	&	 5,124 	&	 4,950 	&	 8,160 	&	 3,565 	&	 3,366 	\\
			\multicolumn{1}{l}{	\footnotesize	 Marzo 	}&	 Huehuetenango 	&	 3,230 	&	 810 	&	 4,446 	&	 3,001 	&	 2,033 	&	 4,360 	&	 1,998 	&	 20 	&	 2 	&	 1 	&	 4,279 	&	 3,936 	&	 4,613 	&	 3,301 	&	 3,047 	\\
			\rowcolor{color1!5!white}\multicolumn{1}{l}{	\footnotesize	 Marzo 	}&	 Izabal 	&	 446 	&	 238 	&	 981 	&	 800 	&	 567 	&	 905 	&	 555 	&	 1 	&	 1 	&	 1 	&	 707 	&	 558 	&	 1,122 	&	 608 	&	 486 	\\
			\multicolumn{1}{l}{	\footnotesize	 Marzo 	}&	 Jalapa 	&	 638 	&	 220 	&	 1,016 	&	 847 	&	 729 	&	 636 	&	 504 	&	 -   	&	 -   	&	 -   	&	 930 	&	 793 	&	 1,083 	&	 666 	&	 630 	\\
			\rowcolor{color1!5!white}\multicolumn{1}{l}{	\footnotesize	 Marzo 	}&	 Jutiapa 	&	 1,040 	&	 294 	&	 1,906 	&	 1,772 	&	 1,523 	&	 1,396 	&	 1,092 	&	 -   	&	 -   	&	 -   	&	 1,736 	&	 1,542 	&	 1,835 	&	 1,430 	&	 1,329 	\\
			\multicolumn{1}{l}{	\footnotesize	 Marzo 	}&	 Quetzaltenango 	&	 1,555 	&	 409 	&	 2,182 	&	 2,166 	&	 1,688 	&	 2,111 	&	 1,808 	&	 -   	&	 -   	&	 -   	&	 2,033 	&	 1,455 	&	 3,127 	&	 1,518 	&	 1,130 	\\
			\rowcolor{color1!5!white}\multicolumn{1}{l}{	\footnotesize	 Marzo 	}&	 Quiché 	&	 2,814 	&	 723 	&	 3,533 	&	 2,990 	&	 2,317 	&	 3,372 	&	 2,189 	&	 -   	&	 1 	&	 1 	&	 4,046 	&	 2,950 	&	 4,224 	&	 3,330 	&	 2,636 	\\
			\multicolumn{1}{l}{	\footnotesize	 Marzo 	}&	 Retalhuleu 	&	 545 	&	 135 	&	 921 	&	 935 	&	 754 	&	 748 	&	 628 	&	 -   	&	 -   	&	 -   	&	 715 	&	 723 	&	 907 	&	 538 	&	 514 	\\
			\rowcolor{color1!5!white}\multicolumn{1}{l}{	\footnotesize	 Marzo 	}&	 Sacatepéquez 	&	 300 	&	 54 	&	 630 	&	 638 	&	 668 	&	 691 	&	 711 	&	 -   	&	 -   	&	 -   	&	 591 	&	 496 	&	 1,059 	&	 494 	&	 411 	\\
			\multicolumn{1}{l}{	\footnotesize	 Marzo 	}&	 San Marcos 	&	 2,735 	&	 828 	&	 3,895 	&	 3,163 	&	 2,341 	&	 3,163 	&	 2,260 	&	 1 	&	 -   	&	 2 	&	 1,936 	&	 2,153 	&	 2,708 	&	 1,454 	&	 1,409 	\\
			\rowcolor{color1!5!white}\multicolumn{1}{l}{	\footnotesize	 Marzo 	}&	 Santa Rosa 	&	 663 	&	 237 	&	 979 	&	 980 	&	 697 	&	 674 	&	 614 	&	 -   	&	 -   	&	 -   	&	 766 	&	 822 	&	 1,415 	&	 668 	&	 643 	\\
			\multicolumn{1}{l}{	\footnotesize	 Marzo 	}&	 Sololá 	&	 525 	&	 178 	&	 911 	&	 730 	&	 645 	&	 852 	&	 586 	&	 -   	&	 -   	&	 -   	&	 649 	&	 683 	&	 1,465 	&	 647 	&	 668 	\\
			\rowcolor{color1!5!white}\multicolumn{1}{l}{	\footnotesize	 Marzo 	}&	 Suchitepéquez 	&	 792 	&	 298 	&	 1,317 	&	 1,341 	&	 1,227 	&	 1,160 	&	 994 	&	 1 	&	 -   	&	 -   	&	 1,379 	&	 1,308 	&	 2,102 	&	 1,035 	&	 972 	\\
			\multicolumn{1}{l}{	\footnotesize	 Marzo 	}&	 Totonicapán 	&	 847 	&	 196 	&	 809 	&	 774 	&	 604 	&	 859 	&	 568 	&	 -   	&	 -   	&	 -   	&	 526 	&	 472 	&	 1,089 	&	 417 	&	 385 	\\
			\rowcolor{color1!5!white}\multicolumn{1}{l}{	\footnotesize	 Marzo 	}&	 Zacapa 	&	 465 	&	 173 	&	 569 	&	 614 	&	 586 	&	 573 	&	 590 	&	 -   	&	 -   	&	 -   	&	 676 	&	 564 	&	 784 	&	 386 	&	 340 	\\
			\multicolumn{1}{l}{	\footnotesize	 Mayo 	}&	 Alta Verapaz 	&	 1,931 	&	 607 	&	 2,453 	&	 2,237 	&	 1,458 	&	 2,163 	&	 1,757 	&	 1 	&	 -   	&	 1 	&	 2,000 	&	 1,603 	&	 2,238 	&	 1,663 	&	 1,447 	\\
			\rowcolor{color1!5!white}\multicolumn{1}{l}{	\footnotesize	 Mayo 	}&	 Baja Verapaz 	&	 702 	&	 240 	&	 916 	&	 810 	&	 711 	&	 753 	&	 660 	&	 -   	&	 -   	&	 -   	&	 831 	&	 847 	&	 660 	&	 812 	&	 748 	\\
			\multicolumn{1}{l}{	\footnotesize	 Mayo 	}&	 Chimaltenango 	&	 1,229 	&	 540 	&	 1,593 	&	 1,516 	&	 1,296 	&	 1,364 	&	 1,151 	&	 -   	&	 -   	&	 -   	&	 1,489 	&	 1,426 	&	 1,480 	&	 1,309 	&	 1,231 	\\
			\rowcolor{color1!5!white}\multicolumn{1}{l}{	\footnotesize	 Mayo 	}&	 Chiquimula 	&	 1,045 	&	 320 	&	 1,544 	&	 1,451 	&	 1,119 	&	 1,015 	&	 895 	&	 -   	&	 -   	&	 -   	&	 1,093 	&	 1,065 	&	 614 	&	 1,007 	&	 979 	\\
			\multicolumn{1}{l}{	\footnotesize	 Mayo 	}&	 Petén 	&	 1,484 	&	 889 	&	 1,632 	&	 1,672 	&	 1,899 	&	 1,420 	&	 1,322 	&	 -   	&	 -   	&	 -   	&	 1,622 	&	 1,600 	&	 1,679 	&	 1,277 	&	 1,265 	\\
			\rowcolor{color1!5!white}\multicolumn{1}{l}{	\footnotesize	 Mayo 	}&	 El Progreso 	&	 265 	&	 173 	&	 446 	&	 601 	&	 567 	&	 397 	&	 433 	&	 1 	&	 -   	&	 -   	&	 403 	&	 457 	&	 478 	&	 303 	&	 316 	\\
			\multicolumn{1}{l}{	\footnotesize	 Mayo 	}&	 Escuintla 	&	 1,156 	&	 525 	&	 1,777 	&	 1,940 	&	 1,484 	&	 1,440 	&	 1,396 	&	 -   	&	 -   	&	 -   	&	 2,104 	&	 1,976 	&	 2,385 	&	 2,025 	&	 1,887 	\\
			\rowcolor{color1!5!white}\multicolumn{1}{l}{	\footnotesize	 Mayo 	}&	 Guatemala 	&	 2,977 	&	 1,128 	&	 5,873 	&	 5,893 	&	 5,182 	&	 5,123 	&	 4,482 	&	 -   	&	 -   	&	 -   	&	 4,943 	&	 4,930 	&	 6,357 	&	 4,366 	&	 4,290 	\\
			\multicolumn{1}{l}{	\footnotesize	 Mayo 	}&	 Huehuetenango 	&	 2,857 	&	 934 	&	 3,711 	&	 4,070 	&	 2,857 	&	 2,719 	&	 2,569 	&	 7 	&	 3 	&	 -   	&	 3,473 	&	 3,420 	&	 2,325 	&	 2,530 	&	 2,439 	\\
			\rowcolor{color1!5!white}\multicolumn{1}{l}{	\footnotesize	 Mayo 	}&	 Izabal 	&	 836 	&	 156 	&	 667 	&	 786 	&	 622 	&	 146 	&	 152 	&	 2 	&	 1 	&	 1 	&	 553 	&	 489 	&	 644 	&	 487 	&	 422 	\\
			\multicolumn{1}{l}{	\footnotesize	 Mayo 	}&	 Jalapa 	&	 764 	&	 404 	&	 924 	&	 1,118 	&	 893 	&	 1,109 	&	 843 	&	 -   	&	 -   	&	 -   	&	 695 	&	 935 	&	 890 	&	 512 	&	 669 	\\
			\rowcolor{color1!5!white}\multicolumn{1}{l}{	\footnotesize	 Mayo 	}&	 Jutiapa 	&	 888 	&	 438 	&	 1,659 	&	 1,864 	&	 1,767 	&	 1,048 	&	 1,005 	&	 1 	&	 -   	&	 -   	&	 1,808 	&	 1,607 	&	 496 	&	 1,535 	&	 1,443 	\\
			\multicolumn{1}{l}{	\footnotesize	 Mayo 	}&	 Quetzaltenango 	&	 1,832 	&	 209 	&	 2,090 	&	 2,208 	&	 2,102 	&	 1,500 	&	 1,257 	&	 -   	&	 -   	&	 -   	&	 1,813 	&	 1,764 	&	 2,332 	&	 1,778 	&	 1,630 	\\
			\rowcolor{color1!5!white}\multicolumn{1}{l}{	\footnotesize	 Mayo 	}&	 Quiché 	&	 2,728 	&	 1,103 	&	 3,205 	&	 3,181 	&	 2,759 	&	 2,700 	&	 2,395 	&	 1 	&	 -   	&	 -   	&	 3,087 	&	 2,688 	&	 2,527 	&	 2,824 	&	 2,513 	\\
			\multicolumn{1}{l}{	\footnotesize	 Mayo 	}&	 Retalhuleu 	&	 649 	&	 204 	&	 836 	&	 935 	&	 907 	&	 742 	&	 677 	&	 1 	&	 1 	&	 -   	&	 851 	&	 963 	&	 678 	&	 693 	&	 740 	\\
			\rowcolor{color1!5!white}\multicolumn{1}{l}{	\footnotesize	 Mayo 	}&	 Sacatepéquez 	&	 413 	&	 54 	&	 791 	&	 781 	&	 754 	&	 666 	&	 635 	&	 -   	&	 -   	&	 -   	&	 598 	&	 596 	&	 709 	&	 564 	&	 551 	\\
			\multicolumn{1}{l}{	\footnotesize	 Mayo 	}&	 San Marcos 	&	 3,274 	&	 732 	&	 3,877 	&	 4,213 	&	 3,296 	&	 2,938 	&	 2,529 	&	 3 	&	 1 	&	 1 	&	 2,928 	&	 2,739 	&	 2,159 	&	 2,104 	&	 1,978 	\\
			\rowcolor{color1!5!white}\multicolumn{1}{l}{	\footnotesize	 Mayo 	}&	 Santa Rosa 	&	 660 	&	 464 	&	 899 	&	 985 	&	 877 	&	 687 	&	 652 	&	 1 	&	 -   	&	 -   	&	 839 	&	 846 	&	 387 	&	 767 	&	 768 	\\
			\multicolumn{1}{l}{	\footnotesize	 Mayo 	}&	 Solola 	&	 529 	&	 238 	&	 894 	&	 937 	&	 742 	&	 751 	&	 753 	&	 -   	&	 -   	&	 -   	&	 842 	&	 817 	&	 670 	&	 791 	&	 744 	\\
			\rowcolor{color1!5!white}\multicolumn{1}{l}{	\footnotesize	 Mayo 	}&	 Suchitepéquez 	&	 876 	&	 327 	&	 1,294 	&	 1,370 	&	 1,236 	&	 1,075 	&	 1,101 	&	 4 	&	 -   	&	 -   	&	 1,401 	&	 1,459 	&	 939 	&	 1,298 	&	 1,310 	\\
			\multicolumn{1}{l}{	\footnotesize	 Mayo 	}&	 Totonicapán 	&	 1,040 	&	 318 	&	 945 	&	 930 	&	 964 	&	 896 	&	 711 	&	 -   	&	 -   	&	 -   	&	 638 	&	 629 	&	 135 	&	 569 	&	 566 	\\
			\rowcolor{color1!5!white}\multicolumn{1}{l}{	\footnotesize	 Mayo 	}&	 Zacapa 	&	 358 	&	 138 	&	 544 	&	 575 	&	 584 	&	 387 	&	 409 	&	 2 	&	 1 	&	 1 	&	 491 	&	 534 	&	 348 	&	 323 	&	 335 	\\
			\multicolumn{1}{l}{	\footnotesize	 Abril 	}&	 Alta Verapaz 	&	 1,068 	&	 698 	&	 2,072 	&	 2,140 	&	 1,270 	&	 1,915 	&	 1,558 	&	 -   	&	 -   	&	 -   	&	 1,584 	&	 1,279 	&	 2,012 	&	 1,030 	&	 887 	\\
			\rowcolor{color1!5!white}\multicolumn{1}{l}{	\footnotesize	 Abril 	}&	 Baja Verapaz 	&	 497 	&	 267 	&	 709 	&	 620 	&	 590 	&	 825 	&	 696 	&	 -   	&	 -   	&	 -   	&	 722 	&	 550 	&	 849 	&	 649 	&	 543 	\\
			\multicolumn{1}{l}{	\footnotesize	 Abril 	}&	 Chimaltenango 	&	 909 	&	 636 	&	 1,520 	&	 1,451 	&	 1,064 	&	 1,295 	&	 1,087 	&	 -   	&	 -   	&	 -   	&	 1,382 	&	 1,290 	&	 1,615 	&	 1,286 	&	 1,191 	\\
			\rowcolor{color1!5!white}\multicolumn{1}{l}{	\footnotesize	 Abril 	}&	 Chiquimula 	&	 704 	&	 24 	&	 425 	&	 446 	&	 355 	&	 490 	&	 387 	&	 -   	&	 -   	&	 -   	&	 474 	&	 328 	&	 1,013 	&	 441 	&	 303 	\\
			\multicolumn{1}{l}{	\footnotesize	 Abril 	}&	 Petén 	&	 1,248 	&	 636 	&	 1,165 	&	 1,334 	&	 1,320 	&	 1,172 	&	 1,229 	&	 -   	&	 -   	&	 -   	&	 1,444 	&	 1,454 	&	 1,947 	&	 1,146 	&	 1,134 	\\
			\rowcolor{color1!5!white}\multicolumn{1}{l}{	\footnotesize	 Abril 	}&	 El Progreso 	&	 299 	&	 175 	&	 455 	&	 545 	&	 524 	&	 398 	&	 428 	&	 -   	&	 -   	&	 -   	&	 298 	&	 348 	&	 492 	&	 300 	&	 327 	\\
			\multicolumn{1}{l}{	\footnotesize	 Abril 	}&	 Escuintla 	&	 1,435 	&	 431 	&	 1,228 	&	 1,338 	&	 1,011 	&	 1,348 	&	 1,414 	&	 -   	&	 -   	&	 -   	&	 985 	&	 813 	&	 1,389 	&	 827 	&	 669 	\\
			\rowcolor{color1!5!white}\multicolumn{1}{l}{	\footnotesize	 Abril 	}&	 Guatemala 	&	 3,270 	&	 1,121 	&	 5,832 	&	 5,540 	&	 4,653 	&	 4,936 	&	 4,850 	&	 2 	&	 1 	&	 -   	&	 4,367 	&	 4,099 	&	 5,875 	&	 3,236 	&	 3,104 	\\
			\multicolumn{1}{l}{	\footnotesize	 Abril 	}&	 Huehuetenango 	&	 2,637 	&	 806 	&	 3,232 	&	 3,245 	&	 2,214 	&	 3,264 	&	 2,620 	&	 4 	&	 3 	&	 1 	&	 3,321 	&	 3,249 	&	 1,701 	&	 2,690 	&	 2,606 	\\
			\rowcolor{color1!5!white}\multicolumn{1}{l}{	\footnotesize	 Abril 	}&	 Izabal 	&	 253 	&	 115 	&	 663 	&	 763 	&	 598 	&	 545 	&	 451 	&	 1 	&	 -   	&	 -   	&	 395 	&	 414 	&	 701 	&	 312 	&	 323 	\\
			\multicolumn{1}{l}{	\footnotesize	 Abril 	}&	 Jalapa 	&	 528 	&	 535 	&	 999 	&	 1,018 	&	 957 	&	 896 	&	 944 	&	 -   	&	 -   	&	 -   	&	 858 	&	 893 	&	 1,291 	&	 686 	&	 714 	\\
			\rowcolor{color1!5!white}\multicolumn{1}{l}{	\footnotesize	 Abril 	}&	 Jutiapa 	&	 853 	&	 561 	&	 1,375 	&	 1,412 	&	 1,297 	&	 890 	&	 723 	&	 1 	&	 -   	&	 -   	&	 1,516 	&	 1,368 	&	 591 	&	 1,162 	&	 1,089 	\\
			\multicolumn{1}{l}{	\footnotesize	 Abril 	}&	 Quetzaltenango 	&	 1,119 	&	 150 	&	 2,035 	&	 2,070 	&	 1,685 	&	 1,250 	&	 1,285 	&	 -   	&	 -   	&	 -   	&	 1,865 	&	 1,554 	&	 2,171 	&	 1,676 	&	 1,314 	\\
			\rowcolor{color1!5!white}\multicolumn{1}{l}{	\footnotesize	 Abril 	}&	 Quiché 	&	 2,402 	&	 631 	&	 2,987 	&	 2,920 	&	 2,364 	&	 2,326 	&	 1,907 	&	 -   	&	 -   	&	 -   	&	 2,447 	&	 2,275 	&	 2,659 	&	 2,237 	&	 2,057 	\\
			\multicolumn{1}{l}{	\footnotesize	 Abril 	}&	 Retalhuleu 	&	 556 	&	 156 	&	 871 	&	 938 	&	 727 	&	 752 	&	 792 	&	 -   	&	 -   	&	 -   	&	 804 	&	 853 	&	 850 	&	 591 	&	 594 	\\
			\rowcolor{color1!5!white}\multicolumn{1}{l}{	\footnotesize	 Abril 	}&	 Sacatepéquez 	&	 427 	&	 46 	&	 766 	&	 850 	&	 644 	&	 749 	&	 831 	&	 -   	&	 -   	&	 -   	&	 630 	&	 526 	&	 845 	&	 540 	&	 457 	\\
			\multicolumn{1}{l}{	\footnotesize	 Abril 	}&	 San Marcos 	&	 1,638 	&	 918 	&	 3,642 	&	 3,108 	&	 2,446 	&	 2,466 	&	 2,142 	&	 1 	&	 -   	&	 -   	&	 2,133 	&	 2,136 	&	 3,623 	&	 1,512 	&	 1,432 	\\
			\rowcolor{color1!5!white}\multicolumn{1}{l}{	\footnotesize	 Abril 	}&	 Santa Rosa 	&	 673 	&	 258 	&	 928 	&	 939 	&	 752 	&	 863 	&	 891 	&	 -   	&	 -   	&	 -   	&	 765 	&	 783 	&	 770 	&	 740 	&	 731 	\\
			\multicolumn{1}{l}{	\footnotesize	 Abril 	}&	 Sololá 	&	 576 	&	 215 	&	 827 	&	 739 	&	 655 	&	 814 	&	 655 	&	 -   	&	 -   	&	 -   	&	 596 	&	 570 	&	 951 	&	 472 	&	 483 	\\
			\rowcolor{color1!5!white}\multicolumn{1}{l}{	\footnotesize	 Abril 	}&	 Suchitepéquez 	&	 838 	&	 359 	&	 1,221 	&	 1,228 	&	 1,037 	&	 1,017 	&	 1,055 	&	 1 	&	 -   	&	 -   	&	 1,139 	&	 1,125 	&	 1,299 	&	 997 	&	 948 	\\
			\multicolumn{1}{l}{	\footnotesize	 Abril 	}&	 Totonicapán 	&	 683 	&	 208 	&	 831 	&	 792 	&	 602 	&	 823 	&	 652 	&	 -   	&	 -   	&	 -   	&	 522 	&	 500 	&	 245 	&	 433 	&	 428 	\\
			\rowcolor{color1!5!white}\multicolumn{1}{l}{	\footnotesize	 Abril 	}&	 Zacapa 	&	 452 	&	 140 	&	 591 	&	 717 	&	 672 	&	 453 	&	 549 	&	 -   	&	 -   	&	 -   	&	 530 	&	 576 	&	 545 	&	 335 	&	 365 	\\
			\multicolumn{1}{l}{	\footnotesize	 Agosto 	}&	 Alta Verapaz 	&	 2,804 	&	 579 	&	 2,150 	&	 2,076 	&	 1,630 	&	 2,048 	&	 1,591 	&	 2 	&	 -   	&	 -   	&	 1,554 	&	 1,267 	&	 1,933 	&	 1,514 	&	 1,260 	\\
			\rowcolor{color1!5!white}\multicolumn{1}{l}{	\footnotesize	 Agosto 	}&	 Baja Verapaz 	&	 758 	&	 281 	&	 572 	&	 648 	&	 634 	&	 604 	&	 617 	&	 -   	&	 1 	&	 -   	&	 631 	&	 668 	&	 733 	&	 581 	&	 598 	\\
			\multicolumn{1}{l}{	\footnotesize	 Agosto 	}&	 Chimaltenango 	&	 1,230 	&	 716 	&	 960 	&	 844 	&	 723 	&	 1,240 	&	 1,154 	&	 -   	&	 -   	&	 -   	&	 1,149 	&	 978 	&	 1,140 	&	 1,015 	&	 892 	\\
			\rowcolor{color1!5!white}\multicolumn{1}{l}{	\footnotesize	 Agosto 	}&	 Chiquimula 	&	 846 	&	 138 	&	 983 	&	 1,098 	&	 969 	&	 1,032 	&	 879 	&	 -   	&	 -   	&	 -   	&	 737 	&	 742 	&	 1,045 	&	 683 	&	 674 	\\
			\multicolumn{1}{l}{	\footnotesize	 Agosto 	}&	 Petén 	&	 1,427 	&	 1,007 	&	 1,318 	&	 1,283 	&	 1,303 	&	 1,297 	&	 1,209 	&	 2 	&	 -   	&	 1 	&	 1,257 	&	 1,252 	&	 1,370 	&	 1,244 	&	 1,242 	\\
			\rowcolor{color1!5!white}\multicolumn{1}{l}{	\footnotesize	 Agosto 	}&	 El Progreso 	&	 332 	&	 178 	&	 420 	&	 468 	&	 518 	&	 392 	&	 408 	&	 -   	&	 -   	&	 -   	&	 288 	&	 286 	&	 416 	&	 305 	&	 318 	\\
			\multicolumn{1}{l}{	\footnotesize	 Agosto 	}&	 Escuintla 	&	 1,367 	&	 565 	&	 1,244 	&	 1,206 	&	 1,012 	&	 1,358 	&	 1,194 	&	 -   	&	 -   	&	 -   	&	 1,276 	&	 1,237 	&	 1,460 	&	 1,125 	&	 1,101 	\\
			\rowcolor{color1!5!white}\multicolumn{1}{l}{	\footnotesize	 Agosto 	}&	 Guatemala 	&	 3,447 	&	 1,328 	&	 5,006 	&	 4,892 	&	 4,224 	&	 4,670 	&	 4,472 	&	 -   	&	 -   	&	 -   	&	 3,155 	&	 3,322 	&	 3,930 	&	 2,599 	&	 2,700 	\\
			\multicolumn{1}{l}{	\footnotesize	 Agosto 	}&	 Huehuetenango 	&	 2,682 	&	 865 	&	 2,496 	&	 2,433 	&	 2,305 	&	 2,697 	&	 2,342 	&	 12 	&	 3 	&	 -   	&	 2,593 	&	 2,513 	&	 3,072 	&	 1,877 	&	 1,848 	\\
			\rowcolor{color1!5!white}\multicolumn{1}{l}{	\footnotesize	 Agosto 	}&	 Izabal 	&	 901 	&	 64 	&	 649 	&	 590 	&	 540 	&	 860 	&	 477 	&	 1 	&	 -   	&	 -   	&	 528 	&	 478 	&	 720 	&	 588 	&	 507 	\\
			\multicolumn{1}{l}{	\footnotesize	 Agosto 	}&	 Jalapa 	&	 706 	&	 216 	&	 685 	&	 643 	&	 669 	&	 737 	&	 811 	&	 -   	&	 -   	&	 -   	&	 756 	&	 737 	&	 924 	&	 738 	&	 662 	\\
			\rowcolor{color1!5!white}\multicolumn{1}{l}{	\footnotesize	 Agosto 	}&	 Jutiapa 	&	 1,138 	&	 283 	&	 1,606 	&	 1,347 	&	 1,346 	&	 1,669 	&	 1,305 	&	 -   	&	 -   	&	 -   	&	 1,063 	&	 1,057 	&	 1,505 	&	 940 	&	 900 	\\
			\multicolumn{1}{l}{	\footnotesize	 Agosto 	}&	 Quetzaltenango 	&	 1,585 	&	 198 	&	 1,983 	&	 1,960 	&	 1,682 	&	 2,002 	&	 1,773 	&	 2 	&	 -   	&	 -   	&	 1,392 	&	 1,275 	&	 1,666 	&	 1,245 	&	 1,225 	\\
			\rowcolor{color1!5!white}\multicolumn{1}{l}{	\footnotesize	 Agosto 	}&	 Quiché 	&	 1,965 	&	 1,021 	&	 2,236 	&	 2,238 	&	 2,079 	&	 2,227 	&	 2,254 	&	 2 	&	 -   	&	 -   	&	 1,893 	&	 1,817 	&	 2,027 	&	 1,808 	&	 1,776 	\\
			\multicolumn{1}{l}{	\footnotesize	 Agosto 	}&	 Retalhuleu 	&	 667 	&	 65 	&	 772 	&	 736 	&	 789 	&	 761 	&	 634 	&	 -   	&	 -   	&	 -   	&	 575 	&	 563 	&	 669 	&	 484 	&	 488 	\\
			\rowcolor{color1!5!white}\multicolumn{1}{l}{	\footnotesize	 Agosto 	}&	 Sacatepéquez 	&	 251 	&	 46 	&	 628 	&	 651 	&	 562 	&	 620 	&	 605 	&	 -   	&	 -   	&	 -   	&	 345 	&	 342 	&	 514 	&	 297 	&	 290 	\\
			\multicolumn{1}{l}{	\footnotesize	 Agosto 	}&	 San Marcos 	&	 2,610 	&	 1,094 	&	 2,588 	&	 2,745 	&	 2,551 	&	 2,218 	&	 1,864 	&	 2 	&	 -   	&	 -   	&	 2,267 	&	 2,265 	&	 2,629 	&	 1,760 	&	 1,745 	\\
			\rowcolor{color1!5!white}\multicolumn{1}{l}{	\footnotesize	 Agosto 	}&	 Santa Rosa 	&	 752 	&	 201 	&	 731 	&	 843 	&	 818 	&	 746 	&	 814 	&	 -   	&	 -   	&	 -   	&	 757 	&	 791 	&	 1,004 	&	 739 	&	 748 	\\
			\multicolumn{1}{l}{	\footnotesize	 Agosto 	}&	 Sololá 	&	 554 	&	 265 	&	 814 	&	 687 	&	 642 	&	 759 	&	 586 	&	 -   	&	 -   	&	 -   	&	 650 	&	 616 	&	 905 	&	 511 	&	 500 	\\
			\rowcolor{color1!5!white}\multicolumn{1}{l}{	\footnotesize	 Agosto 	}&	 Suchitepéquez 	&	 628 	&	 247 	&	 1,072 	&	 1,008 	&	 986 	&	 1,025 	&	 921 	&	 -   	&	 -   	&	 -   	&	 917 	&	 908 	&	 1,010 	&	 840 	&	 830 	\\
			\multicolumn{1}{l}{	\footnotesize	 Agosto 	}&	 Totonicapán 	&	 978 	&	 259 	&	 920 	&	 828 	&	 779 	&	 944 	&	 808 	&	 -   	&	 -   	&	 -   	&	 661 	&	 658 	&	 1,172 	&	 516 	&	 520 	\\
			\rowcolor{color1!5!white}\multicolumn{1}{l}{	\footnotesize	 Agosto 	}&	 Zacapa 	&	 363 	&	 123 	&	 454 	&	 447 	&	 449 	&	 444 	&	 444 	&	 -   	&	 -   	&	 -   	&	 303 	&	 303 	&	 420 	&	 286 	&	 283 	\\
			\multicolumn{1}{l}{	\footnotesize	 Julio 	}&	 Alta Verapaz 	&	 2,240 	&	 521 	&	 2,387 	&	 2,122 	&	 1,673 	&	 1,730 	&	 1,423 	&	 -   	&	 -   	&	 -   	&	 1,661 	&	 1,559 	&	 2,014 	&	 1,401 	&	 1,301 	\\
			\rowcolor{color1!5!white}\multicolumn{1}{l}{	\footnotesize	 Julio 	}&	 Baja Verapaz 	&	 573 	&	 234 	&	 581 	&	 626 	&	 664 	&	 575 	&	 577 	&	 -   	&	 -   	&	 -   	&	 576 	&	 577 	&	 694 	&	 546 	&	 584 	\\
			\multicolumn{1}{l}{	\footnotesize	 Julio 	}&	 Chimaltenango 	&	 1,364 	&	 683 	&	 1,412 	&	 1,449 	&	 1,436 	&	 1,557 	&	 1,413 	&	 -   	&	 -   	&	 -   	&	 1,209 	&	 1,213 	&	 1,371 	&	 1,268 	&	 1,283 	\\
			\rowcolor{color1!5!white}\multicolumn{1}{l}{	\footnotesize	 Julio 	}&	 Chiquimula 	&	 873 	&	 405 	&	 713 	&	 884 	&	 855 	&	 717 	&	 686 	&	 -   	&	 -   	&	 -   	&	 682 	&	 667 	&	 893 	&	 613 	&	 613 	\\
			\multicolumn{1}{l}{	\footnotesize	 Julio 	}&	 Petén 	&	 1,128 	&	 955 	&	 1,032 	&	 1,020 	&	 1,050 	&	 1,037 	&	 953 	&	 -   	&	 -   	&	 -   	&	 913 	&	 914 	&	 1,090 	&	 877 	&	 881 	\\
			\rowcolor{color1!5!white}\multicolumn{1}{l}{	\footnotesize	 Julio 	}&	 El Progreso 	&	 202 	&	 202 	&	 208 	&	 255 	&	 290 	&	 187 	&	 196 	&	 -   	&	 -   	&	 -   	&	 152 	&	 151 	&	 219 	&	 144 	&	 145 	\\
			\multicolumn{1}{l}{	\footnotesize	 Julio 	}&	 Escuintla 	&	 1,177 	&	 544 	&	 925 	&	 1,033 	&	 1,036 	&	 885 	&	 823 	&	 -   	&	 -   	&	 -   	&	 790 	&	 794 	&	 1,004 	&	 780 	&	 757 	\\
			\rowcolor{color1!5!white}\multicolumn{1}{l}{	\footnotesize	 Julio 	}&	 Guatemala 	&	 3,041 	&	 1,214 	&	 4,992 	&	 5,056 	&	 4,608 	&	 4,535 	&	 4,344 	&	 -   	&	 -   	&	 -   	&	 3,188 	&	 3,338 	&	 4,104 	&	 2,449 	&	 2,505 	\\
			\multicolumn{1}{l}{	\footnotesize	 Julio 	}&	 Huehuetenango 	&	 3,028 	&	 898 	&	 3,716 	&	 3,557 	&	 3,539 	&	 3,369 	&	 2,487 	&	 3 	&	 1 	&	 1 	&	 3,108 	&	 3,069 	&	 3,631 	&	 2,301 	&	 2,228 	\\
			\rowcolor{color1!5!white}\multicolumn{1}{l}{	\footnotesize	 Julio 	}&	 Izabal 	&	 287 	&	 62 	&	 318 	&	 316 	&	 277 	&	 276 	&	 150 	&	 -   	&	 -   	&	 -   	&	 324 	&	 280 	&	 356 	&	 247 	&	 189 	\\
			\multicolumn{1}{l}{	\footnotesize	 Julio 	}&	 Jalapa 	&	 924 	&	 402 	&	 1,007 	&	 1,143 	&	 1,065 	&	 827 	&	 856 	&	 -   	&	 -   	&	 -   	&	 1,358 	&	 988 	&	 885 	&	 1,090 	&	 810 	\\
			\rowcolor{color1!5!white}\multicolumn{1}{l}{	\footnotesize	 Julio 	}&	 Jutiapa 	&	 767 	&	 215 	&	 792 	&	 901 	&	 865 	&	 778 	&	 628 	&	 -   	&	 -   	&	 -   	&	 721 	&	 566 	&	 1,000 	&	 606 	&	 500 	\\
			\multicolumn{1}{l}{	\footnotesize	 Julio 	}&	 Quetzaltenango 	&	 1,370 	&	 199 	&	 1,854 	&	 1,772 	&	 1,610 	&	 1,882 	&	 1,608 	&	 3 	&	 1 	&	 -   	&	 1,210 	&	 1,238 	&	 1,519 	&	 1,022 	&	 1,045 	\\
			\rowcolor{color1!5!white}\multicolumn{1}{l}{	\footnotesize	 Julio 	}&	 Quiché 	&	 2,032 	&	 1,189 	&	 2,340 	&	 2,409 	&	 2,324 	&	 2,337 	&	 2,086 	&	 2 	&	 -   	&	 -   	&	 2,093 	&	 2,002 	&	 2,446 	&	 1,847 	&	 1,762 	\\
			\multicolumn{1}{l}{	\footnotesize	 Julio 	}&	 Retalhuleu 	&	 528 	&	 88 	&	 631 	&	 628 	&	 715 	&	 601 	&	 518 	&	 1 	&	 -   	&	 -   	&	 523 	&	 532 	&	 615 	&	 430 	&	 432 	\\
			\rowcolor{color1!5!white}\multicolumn{1}{l}{	\footnotesize	 Julio 	}&	 Sacatepéquez 	&	 291 	&	 59 	&	 700 	&	 717 	&	 612 	&	 681 	&	 653 	&	 -   	&	 -   	&	 -   	&	 422 	&	 417 	&	 550 	&	 428 	&	 430 	\\
			\multicolumn{1}{l}{	\footnotesize	 Julio 	}&	 San marcos 	&	 2,537 	&	 1,074 	&	 2,528 	&	 2,871 	&	 2,700 	&	 2,144 	&	 1,806 	&	 2 	&	 1 	&	 -   	&	 2,289 	&	 2,228 	&	 2,927 	&	 1,707 	&	 1,656 	\\
			\rowcolor{color1!5!white}\multicolumn{1}{l}{	\footnotesize	 Julio 	}&	 Santa rosa 	&	 463 	&	 157 	&	 515 	&	 573 	&	 597 	&	 537 	&	 519 	&	 -   	&	 -   	&	 -   	&	 490 	&	 459 	&	 614 	&	 459 	&	 406 	\\
			\multicolumn{1}{l}{	\footnotesize	 Julio 	}&	 Sololá 	&	 537 	&	 241 	&	 812 	&	 716 	&	 765 	&	 785 	&	 595 	&	 -   	&	 -   	&	 -   	&	 705 	&	 691 	&	 822 	&	 605 	&	 594 	\\
			\rowcolor{color1!5!white}\multicolumn{1}{l}{	\footnotesize	 Julio 	}&	 Suchitepéquez 	&	 749 	&	 256 	&	 1,184 	&	 1,222 	&	 1,165 	&	 1,002 	&	 871 	&	 -   	&	 -   	&	 -   	&	 1,101 	&	 1,110 	&	 1,530 	&	 982 	&	 970 	\\
			\multicolumn{1}{l}{	\footnotesize	 Julio 	}&	 Totonicapán 	&	 831 	&	 267 	&	 902 	&	 904 	&	 756 	&	 960 	&	 832 	&	 -   	&	 -   	&	 -   	&	 674 	&	 677 	&	 1,336 	&	 516 	&	 516 	\\
			\rowcolor{color1!5!white}\multicolumn{1}{l}{	\footnotesize	 Julio 	}&	 Zacapa 	&	 204 	&	 104 	&	 340 	&	 368 	&	 368 	&	 354 	&	 336 	&	 -   	&	 -   	&	 -   	&	 248 	&	 262 	&	 351 	&	 200 	&	 207 	\\
			\multicolumn{1}{l}{	\footnotesize	 Junio 	}&	 Alta Verapaz 	&	 1,518 	&	 581 	&	 1,924 	&	 1,810 	&	 1,426 	&	 2,192 	&	 1,493 	&	 1 	&	 1 	&	 -   	&	 1,697 	&	 1,302 	&	 1,867 	&	 1,416 	&	 1,041 	\\
			\rowcolor{color1!5!white}\multicolumn{1}{l}{	\footnotesize	 Junio 	}&	 Baja Verapaz 	&	 572 	&	 229 	&	 649 	&	 752 	&	 610 	&	 720 	&	 685 	&	 -   	&	 -   	&	 -   	&	 645 	&	 597 	&	 658 	&	 658 	&	 595 	\\
			\multicolumn{1}{l}{	\footnotesize	 Junio 	}&	 Chimaltenango 	&	 1,239 	&	 728 	&	 1,357 	&	 1,440 	&	 1,227 	&	 1,670 	&	 1,314 	&	 -   	&	 -   	&	 -   	&	 1,309 	&	 1,325 	&	 1,293 	&	 1,227 	&	 1,205 	\\
			\rowcolor{color1!5!white}\multicolumn{1}{l}{	\footnotesize	 Junio 	}&	 Chiquimula 	&	 904 	&	 367 	&	 1,181 	&	 1,096 	&	 924 	&	 1,055 	&	 760 	&	 -   	&	 -   	&	 -   	&	 933 	&	 982 	&	 1,503 	&	 807 	&	 851 	\\
			\multicolumn{1}{l}{	\footnotesize	 Junio 	}&	 Petén 	&	 1,560 	&	 1,051 	&	 1,306 	&	 1,424 	&	 1,479 	&	 1,321 	&	 1,345 	&	 -   	&	 -   	&	 -   	&	 1,363 	&	 1,350 	&	 1,488 	&	 1,065 	&	 1,054 	\\
			\rowcolor{color1!5!white}\multicolumn{1}{l}{	\footnotesize	 Junio 	}&	 El Progreso 	&	 286 	&	 207 	&	 403 	&	 445 	&	 445 	&	 381 	&	 349 	&	 -   	&	 -   	&	 -   	&	 274 	&	 312 	&	 354 	&	 278 	&	 309 	\\
			\multicolumn{1}{l}{	\footnotesize	 Junio 	}&	 Escuintla 	&	 1,393 	&	 509 	&	 1,723 	&	 1,555 	&	 1,495 	&	 1,626 	&	 1,266 	&	 -   	&	 -   	&	 -   	&	 1,442 	&	 1,456 	&	 1,662 	&	 1,467 	&	 1,458 	\\
			\rowcolor{color1!5!white}\multicolumn{1}{l}{	\footnotesize	 Junio 	}&	 Guatemala 	&	 3,159 	&	 1,202 	&	 5,780 	&	 5,517 	&	 5,038 	&	 5,571 	&	 4,857 	&	 1 	&	 -   	&	 -   	&	 3,901 	&	 4,134 	&	 5,082 	&	 3,221 	&	 3,202 	\\
			\multicolumn{1}{l}{	\footnotesize	 Junio 	}&	 Huehuetenango 	&	 2,382 	&	 984 	&	 3,899 	&	 3,954 	&	 3,390 	&	 3,558 	&	 2,970 	&	 4 	&	 5 	&	 3 	&	 3,417 	&	 3,303 	&	 5,042 	&	 2,572 	&	 2,435 	\\
			\rowcolor{color1!5!white}\multicolumn{1}{l}{	\footnotesize	 Junio 	}&	 Izabal 	&	 632 	&	 74 	&	 612 	&	 652 	&	 564 	&	 692 	&	 462 	&	 -   	&	 -   	&	 -   	&	 567 	&	 471 	&	 592 	&	 452 	&	 387 	\\
			\multicolumn{1}{l}{	\footnotesize	 Junio 	}&	 Jalapa 	&	 739 	&	 23 	&	 450 	&	 525 	&	 416 	&	 903 	&	 764 	&	 -   	&	 -   	&	 -   	&	 116 	&	 324 	&	 634 	&	 88 	&	 263 	\\
			\rowcolor{color1!5!white}\multicolumn{1}{l}{	\footnotesize	 Junio 	}&	 Jutiapa 	&	 1,001 	&	 292 	&	 926 	&	 997 	&	 943 	&	 1,221 	&	 933 	&	 -   	&	 -   	&	 -   	&	 730 	&	 648 	&	 1,209 	&	 624 	&	 570 	\\
			\multicolumn{1}{l}{	\footnotesize	 Junio 	}&	 Quetzaltenango 	&	 1,760 	&	 172 	&	 1,984 	&	 2,100 	&	 2,001 	&	 2,455 	&	 1,924 	&	 -   	&	 -   	&	 -   	&	 1,553 	&	 1,437 	&	 2,013 	&	 1,526 	&	 1,430 	\\
			\rowcolor{color1!5!white}\multicolumn{1}{l}{	\footnotesize	 Junio 	}&	 Quiché 	&	 2,667 	&	 1,379 	&	 2,501 	&	 2,689 	&	 2,465 	&	 2,939 	&	 2,417 	&	 3 	&	 -   	&	 -   	&	 2,284 	&	 2,124 	&	 3,405 	&	 2,059 	&	 1,924 	\\
			\multicolumn{1}{l}{	\footnotesize	 Junio 	}&	 Retalhuleu 	&	 604 	&	 194 	&	 709 	&	 811 	&	 793 	&	 661 	&	 638 	&	 -   	&	 -   	&	 -   	&	 556 	&	 607 	&	 782 	&	 454 	&	 457 	\\
			\rowcolor{color1!5!white}\multicolumn{1}{l}{	\footnotesize	 Junio 	}&	 Sacatepéquez 	&	 423 	&	 68 	&	 806 	&	 854 	&	 805 	&	 778 	&	 727 	&	 -   	&	 -   	&	 -   	&	 547 	&	 567 	&	 721 	&	 545 	&	 564 	\\
			\multicolumn{1}{l}{	\footnotesize	 Junio 	}&	 San Marcos 	&	 3,086 	&	 1,335 	&	 3,566 	&	 3,862 	&	 3,272 	&	 2,643 	&	 2,154 	&	 3 	&	 -   	&	 -   	&	 2,951 	&	 2,801 	&	 4,255 	&	 2,264 	&	 2,152 	\\
			\rowcolor{color1!5!white}\multicolumn{1}{l}{	\footnotesize	 Junio 	}&	 Santa Rosa 	&	 547 	&	 272 	&	 789 	&	 853 	&	 854 	&	 872 	&	 810 	&	 -   	&	 1 	&	 -   	&	 744 	&	 709 	&	 803 	&	 667 	&	 610 	\\
			\multicolumn{1}{l}{	\footnotesize	 Junio 	}&	 Sololá 	&	 676 	&	 298 	&	 807 	&	 848 	&	 668 	&	 775 	&	 715 	&	 -   	&	 -   	&	 -   	&	 701 	&	 686 	&	 657 	&	 607 	&	 592 	\\
			\rowcolor{color1!5!white}\multicolumn{1}{l}{	\footnotesize	 Junio 	}&	 Suchitepéquez 	&	 712 	&	 333 	&	 1,280 	&	 1,305 	&	 1,213 	&	 1,079 	&	 967 	&	 4 	&	 -   	&	 -   	&	 1,125 	&	 1,147 	&	 1,586 	&	 943 	&	 940 	\\
			\multicolumn{1}{l}{	\footnotesize	 Junio 	}&	 Totonicapán 	&	 1,048 	&	 324 	&	 983 	&	 897 	&	 822 	&	 1,129 	&	 896 	&	 -   	&	 -   	&	 -   	&	 652 	&	 646 	&	 1,779 	&	 529 	&	 522 	\\
			\rowcolor{color1!5!white}\multicolumn{1}{l}{	\footnotesize	 Junio 	}&	 Zacapa 	&	 373 	&	 169 	&	 490 	&	 519 	&	 559 	&	 553 	&	 512 	&	 2 	&	 1 	&	 1 	&	 358 	&	 361 	&	 474 	&	 254 	&	 264 	\\
			\multicolumn{1}{l}{	\footnotesize	 Septiembre 	}&	 Alta Verapaz 	&	 1,653 	&	 576 	&	 1,296 	&	 1,123 	&	 1,012 	&	 1,277 	&	 1,112 	&	 -   	&	 -   	&	 -   	&	 747 	&	 681 	&	 1,189 	&	 894 	&	 776 	\\
			\rowcolor{color1!5!white}\multicolumn{1}{l}{	\footnotesize	 Septiembre 	}&	 Baja Verapaz 	&	 329 	&	 241 	&	 471 	&	 453 	&	 488 	&	 347 	&	 300 	&	 -   	&	 -   	&	 -   	&	 510 	&	 497 	&	 565 	&	 484 	&	 462 	\\
			\multicolumn{1}{l}{	\footnotesize	 Septiembre 	}&	 Chimaltenango 	&	 1,194 	&	 679 	&	 1,143 	&	 1,178 	&	 1,091 	&	 1,065 	&	 1,074 	&	 -   	&	 -   	&	 -   	&	 854 	&	 808 	&	 982 	&	 899 	&	 846 	\\
			\rowcolor{color1!5!white}\multicolumn{1}{l}{	\footnotesize	 Septiembre 	}&	 Chiquimula 	&	 685 	&	 233 	&	 870 	&	 763 	&	 804 	&	 853 	&	 652 	&	 -   	&	 -   	&	 -   	&	 548 	&	 536 	&	 773 	&	 641 	&	 676 	\\
			\multicolumn{1}{l}{	\footnotesize	 Septiembre 	}&	 Petén 	&	 1,602 	&	 1,137 	&	 1,296 	&	 1,192 	&	 1,100 	&	 1,273 	&	 1,110 	&	 -   	&	 -   	&	 -   	&	 1,085 	&	 1,062 	&	 1,495 	&	 1,171 	&	 1,169 	\\
			\rowcolor{color1!5!white}\multicolumn{1}{l}{	\footnotesize	 Septiembre 	}&	 El Progreso 	&	 307 	&	 198 	&	 303 	&	 278 	&	 321 	&	 303 	&	 236 	&	 -   	&	 -   	&	 -   	&	 248 	&	 222 	&	 296 	&	 291 	&	 304 	\\
			\multicolumn{1}{l}{	\footnotesize	 Septiembre 	}&	 Escuintla 	&	 1,267 	&	 444 	&	 1,548 	&	 1,278 	&	 1,236 	&	 1,338 	&	 1,015 	&	 -   	&	 -   	&	 1 	&	 1,057 	&	 1,081 	&	 1,283 	&	 1,094 	&	 1,096 	\\
			\rowcolor{color1!5!white}\multicolumn{1}{l}{	\footnotesize	 Septiembre 	}&	 Guatemala 	&	 3,539 	&	 1,143 	&	 3,763 	&	 3,628 	&	 3,197 	&	 4,148 	&	 3,878 	&	 -   	&	 -   	&	 1 	&	 2,779 	&	 2,421 	&	 3,753 	&	 2,331 	&	 2,060 	\\
			\multicolumn{1}{l}{	\footnotesize	 Septiembre 	}&	 Huehuetenango 	&	 2,681 	&	 560 	&	 2,521 	&	 2,488 	&	 2,203 	&	 2,490 	&	 2,176 	&	 3 	&	 -   	&	 -   	&	 1,888 	&	 1,865 	&	 2,384 	&	 1,595 	&	 1,573 	\\
			\rowcolor{color1!5!white}\multicolumn{1}{l}{	\footnotesize	 Septiembre 	}&	 Izabal 	&	 330 	&	 50 	&	 588 	&	 349 	&	 315 	&	 495 	&	 269 	&	 -   	&	 -   	&	 -   	&	 301 	&	 299 	&	 478 	&	 430 	&	 398 	\\
			\multicolumn{1}{l}{	\footnotesize	 Septiembre 	}&	 Jalapa 	&	 611 	&	 275 	&	 159 	&	 173 	&	 162 	&	 580 	&	 557 	&	 -   	&	 -   	&	 -   	&	 156 	&	 117 	&	 482 	&	 153 	&	 123 	\\
			\rowcolor{color1!5!white}\multicolumn{1}{l}{	\footnotesize	 Septiembre 	}&	 Jutiapa 	&	 967 	&	 439 	&	 1,110 	&	 1,023 	&	 1,042 	&	 1,135 	&	 839 	&	 -   	&	 -   	&	 -   	&	 781 	&	 726 	&	 596 	&	 847 	&	 825 	\\
			\multicolumn{1}{l}{	\footnotesize	 Septiembre 	}&	 Quetzaltenango 	&	 1,481 	&	 216 	&	 1,519 	&	 1,451 	&	 1,400 	&	 1,263 	&	 1,158 	&	 -   	&	 -   	&	 -   	&	 1,114 	&	 1,102 	&	 1,448 	&	 1,131 	&	 1,093 	\\
			\rowcolor{color1!5!white}\multicolumn{1}{l}{	\footnotesize	 Septiembre 	}&	 Quiché 	&	 1,766 	&	 990 	&	 2,024 	&	 2,049 	&	 1,931 	&	 1,937 	&	 1,893 	&	 1 	&	 -   	&	 -   	&	 1,576 	&	 1,463 	&	 1,830 	&	 1,461 	&	 1,440 	\\
			\multicolumn{1}{l}{	\footnotesize	 Septiembre 	}&	 Retalhuleu 	&	 542 	&	 128 	&	 661 	&	 640 	&	 605 	&	 641 	&	 589 	&	 -   	&	 1 	&	 -   	&	 425 	&	 432 	&	 543 	&	 442 	&	 448 	\\
			\rowcolor{color1!5!white}\multicolumn{1}{l}{	\footnotesize	 Septiembre 	}&	 Sacatepéquez 	&	 252 	&	 62 	&	 622 	&	 557 	&	 518 	&	 607 	&	 547 	&	 -   	&	 -   	&	 -   	&	 424 	&	 398 	&	 581 	&	 454 	&	 452 	\\
			\multicolumn{1}{l}{	\footnotesize	 Septiembre 	}&	 San Marcos 	&	 1,939 	&	 886 	&	 2,307 	&	 2,270 	&	 2,236 	&	 2,066 	&	 1,672 	&	 -   	&	 -   	&	 -   	&	 1,560 	&	 1,607 	&	 2,417 	&	 1,444 	&	 1,416 	\\
			\rowcolor{color1!5!white}\multicolumn{1}{l}{	\footnotesize	 Septiembre 	}&	 Santa Rosa 	&	 679 	&	 228 	&	 655 	&	 629 	&	 610 	&	 675 	&	 574 	&	 -   	&	 -   	&	 -   	&	 580 	&	 562 	&	 744 	&	 664 	&	 622 	\\
			\multicolumn{1}{l}{	\footnotesize	 Septiembre 	}&	 Sololá 	&	 624 	&	 281 	&	 716 	&	 649 	&	 496 	&	 662 	&	 554 	&	 -   	&	 -   	&	 -   	&	 543 	&	 508 	&	 665 	&	 503 	&	 478 	\\
			\rowcolor{color1!5!white}\multicolumn{1}{l}{	\footnotesize	 Septiembre 	}&	 Suchitepéquez 	&	 806 	&	 298 	&	 1,298 	&	 1,168 	&	 1,080 	&	 1,119 	&	 954 	&	 1 	&	 -   	&	 -   	&	 1,000 	&	 953 	&	 1,220 	&	 899 	&	 899 	\\
			\multicolumn{1}{l}{	\footnotesize	 Septiembre 	}&	 Totonicapán 	&	 924 	&	 267 	&	 888 	&	 779 	&	 729 	&	 870 	&	 701 	&	 -   	&	 -   	&	 -   	&	 546 	&	 545 	&	 934 	&	 509 	&	 502 	\\
			\rowcolor{color1!5!white}\multicolumn{1}{l}{	\footnotesize	 Septiembre 	}&	 Zacapa 	&	 290 	&	 88 	&	 353 	&	 313 	&	 297 	&	 350 	&	 292 	&	 -   	&	 -   	&	 -   	&	 275 	&	 269 	&	 376 	&	 249 	&	 253 	\\
			\multicolumn{1}{l}{	\footnotesize	 Octubre 	}&	 Alta Verapaz 	&	 1,941 	&	 368 	&	 2,243 	&	 2,035 	&	 1,654 	&	 1,677 	&	 1,350 	&	 -   	&	 1 	&	 -   	&	 1,183 	&	 1,209 	&	 1,786 	&	 1,590 	&	 1,523 	\\
			\rowcolor{color1!5!white}\multicolumn{1}{l}{	\footnotesize	 Octubre 	}&	 Baja Verapaz 	&	 472 	&	 189 	&	 507 	&	 463 	&	 503 	&	 622 	&	 565 	&	 -   	&	 -   	&	 -   	&	 452 	&	 456 	&	 460 	&	 419 	&	 414 	\\
			\multicolumn{1}{l}{	\footnotesize	 Octubre 	}&	 Chimaltenango 	&	 1,214 	&	 668 	&	 1,579 	&	 1,280 	&	 1,206 	&	 1,399 	&	 1,167 	&	 -   	&	 -   	&	 -   	&	 1,098 	&	 1,139 	&	 1,341 	&	 1,176 	&	 1,199 	\\
			\rowcolor{color1!5!white}\multicolumn{1}{l}{	\footnotesize	 Octubre 	}&	 Chiquimula 	&	 626 	&	 16 	&	 871 	&	 764 	&	 693 	&	 854 	&	 690 	&	 -   	&	 -   	&	 -   	&	 562 	&	 513 	&	 889 	&	 686 	&	 704 	\\
			\multicolumn{1}{l}{	\footnotesize	 Octubre 	}&	 Petén 	&	 1,447 	&	 958 	&	 1,368 	&	 1,298 	&	 1,269 	&	 1,306 	&	 1,208 	&	 -   	&	 -   	&	 -   	&	 1,256 	&	 1,251 	&	 1,568 	&	 1,332 	&	 1,328 	\\
			\rowcolor{color1!5!white}\multicolumn{1}{l}{	\footnotesize	 Octubre 	}&	 El Progreso 	&	 280 	&	 169 	&	 315 	&	 303 	&	 338 	&	 310 	&	 256 	&	 -   	&	 -   	&	 -   	&	 221 	&	 189 	&	 316 	&	 278 	&	 281 	\\
			\multicolumn{1}{l}{	\footnotesize	 Octubre 	}&	 Escuintla 	&	 1,310 	&	 358 	&	 1,502 	&	 1,250 	&	 1,101 	&	 1,315 	&	 962 	&	 -   	&	 -   	&	 -   	&	 1,060 	&	 1,084 	&	 1,406 	&	 1,160 	&	 1,177 	\\
			\rowcolor{color1!5!white}\multicolumn{1}{l}{	\footnotesize	 Octubre 	}&	 Guatemala 	&	 2,571 	&	 778 	&	 3,828 	&	 3,472 	&	 3,233 	&	 3,601 	&	 3,129 	&	 -   	&	 -   	&	 -   	&	 2,673 	&	 2,640 	&	 3,146 	&	 2,012 	&	 2,012 	\\
			\multicolumn{1}{l}{	\footnotesize	 Octubre 	}&	 Huehuetenango 	&	 2,756 	&	 583 	&	 3,253 	&	 2,782 	&	 2,560 	&	 2,852 	&	 2,170 	&	 2 	&	 3 	&	 -   	&	 2,063 	&	 2,040 	&	 3,014 	&	 1,938 	&	 1,932 	\\
			\rowcolor{color1!5!white}\multicolumn{1}{l}{	\footnotesize	 Octubre 	}&	 Izabal 	&	 444 	&	 27 	&	 606 	&	 398 	&	 312 	&	 642 	&	 384 	&	 -   	&	 -   	&	 -   	&	 280 	&	 287 	&	 491 	&	 360 	&	 375 	\\
			\multicolumn{1}{l}{	\footnotesize	 Octubre 	}&	 Jalapa 	&	 749 	&	 82 	&	 1,398 	&	 1,410 	&	 1,361 	&	 875 	&	 803 	&	 -   	&	 -   	&	 -   	&	 1,087 	&	 1,137 	&	 880 	&	 1,042 	&	 1,092 	\\
			\rowcolor{color1!5!white}\multicolumn{1}{l}{	\footnotesize	 Octubre 	}&	 Jutiapa 	&	 616 	&	 152 	&	 1,144 	&	 1,154 	&	 1,026 	&	 1,068 	&	 979 	&	 1 	&	 1 	&	 -   	&	 896 	&	 841 	&	 1,385 	&	 925 	&	 914 	\\
			\multicolumn{1}{l}{	\footnotesize	 Octubre 	}&	 Quetzaltenango 	&	 1,327 	&	 176 	&	 1,855 	&	 1,698 	&	 1,561 	&	 1,921 	&	 1,644 	&	 1 	&	 3 	&	 -   	&	 1,183 	&	 1,187 	&	 1,482 	&	 1,150 	&	 1,180 	\\
			\rowcolor{color1!5!white}\multicolumn{1}{l}{	\footnotesize	 Octubre 	}&	 Quiché 	&	 1,429 	&	 688 	&	 2,063 	&	 2,008 	&	 1,878 	&	 1,933 	&	 1,828 	&	 -   	&	 -   	&	 -   	&	 1,695 	&	 1,692 	&	 1,892 	&	 1,605 	&	 1,626 	\\
			\multicolumn{1}{l}{	\footnotesize	 Octubre 	}&	 Retalhuleu 	&	 295 	&	 66 	&	 687 	&	 730 	&	 673 	&	 681 	&	 639 	&	 2 	&	 -   	&	 -   	&	 561 	&	 543 	&	 668 	&	 605 	&	 608 	\\
			\rowcolor{color1!5!white}\multicolumn{1}{l}{	\footnotesize	 Octubre 	}&	 Sacatepéquez 	&	 240 	&	 43 	&	 625 	&	 527 	&	 548 	&	 595 	&	 469 	&	 -   	&	 -   	&	 -   	&	 436 	&	 418 	&	 540 	&	 412 	&	 415 	\\
			\multicolumn{1}{l}{	\footnotesize	 Octubre 	}&	 San Marcos 	&	 1,751 	&	 905 	&	 1,846 	&	 1,566 	&	 1,563 	&	 1,877 	&	 1,373 	&	 3 	&	 -   	&	 -   	&	 1,218 	&	 1,161 	&	 1,781 	&	 1,248 	&	 1,147 	\\
			\rowcolor{color1!5!white}\multicolumn{1}{l}{	\footnotesize	 Octubre 	}&	 Santa Rosa 	&	 723 	&	 253 	&	 772 	&	 736 	&	 769 	&	 722 	&	 635 	&	 -   	&	 -   	&	 -   	&	 648 	&	 661 	&	 877 	&	 709 	&	 708 	\\
			\multicolumn{1}{l}{	\footnotesize	 Octubre 	}&	 Sololá 	&	 639 	&	 241 	&	 786 	&	 740 	&	 624 	&	 810 	&	 720 	&	 -   	&	 -   	&	 -   	&	 560 	&	 549 	&	 708 	&	 543 	&	 527 	\\
			\rowcolor{color1!5!white}\multicolumn{1}{l}{	\footnotesize	 Octubre 	}&	 Suchitepéquez 	&	 592 	&	 258 	&	 1,027 	&	 983 	&	 953 	&	 984 	&	 912 	&	 -   	&	 -   	&	 -   	&	 860 	&	 851 	&	 1,082 	&	 945 	&	 939 	\\
			\multicolumn{1}{l}{	\footnotesize	 Octubre 	}&	 Totonicapán 	&	 896 	&	 253 	&	 979 	&	 848 	&	 795 	&	 969 	&	 792 	&	 -   	&	 -   	&	 -   	&	 638 	&	 647 	&	 1,018 	&	 606 	&	 608 	\\
			\rowcolor{color1!5!white}\multicolumn{1}{l}{	\footnotesize	 Octubre 	}&	 Zacapa 	&	 238 	&	 50 	&	 367 	&	 327 	&	 366 	&	 380 	&	 312 	&	 -   	&	 -   	&	 -   	&	 252 	&	 259 	&	 356 	&	 251 	&	 256 	\\
			\multicolumn{1}{l}{	\footnotesize	 Noviembre 	}&	 Alta Verapaz 	&	 1,006 	&	 89 	&	 1,023 	&	 798 	&	 752 	&	 810 	&	 510 	&	 -   	&	 -   	&	 -   	&	 514 	&	 520 	&	 937 	&	 725 	&	 745 	\\
			\rowcolor{color1!5!white}\multicolumn{1}{l}{	\footnotesize	 Noviembre 	}&	 Baja Verapaz 	&	 406 	&	 120 	&	 522 	&	 430 	&	 378 	&	 474 	&	 355 	&	 1 	&	 -   	&	 -   	&	 385 	&	 375 	&	 448 	&	 389 	&	 395 	\\
			\multicolumn{1}{l}{	\footnotesize	 Noviembre 	}&	 Chimaltenango 	&	 929 	&	 465 	&	 1,087 	&	 1,017 	&	 1,060 	&	 645 	&	 506 	&	 -   	&	 -   	&	 -   	&	 853 	&	 865 	&	 1,060 	&	 1,004 	&	 998 	\\
			\rowcolor{color1!5!white}\multicolumn{1}{l}{	\footnotesize	 Noviembre 	}&	 Chiquimula 	&	 529 	&	 20 	&	 769 	&	 691 	&	 553 	&	 638 	&	 524 	&	 -   	&	 -   	&	 -   	&	 370 	&	 380 	&	 599 	&	 534 	&	 556 	\\
			\multicolumn{1}{l}{	\footnotesize	 Noviembre 	}&	 Petén 	&	 940 	&	 677 	&	 1,107 	&	 848 	&	 925 	&	 914 	&	 659 	&	 -   	&	 -   	&	 -   	&	 861 	&	 856 	&	 1,083 	&	 935 	&	 929 	\\
			\rowcolor{color1!5!white}\multicolumn{1}{l}{	\footnotesize	 Noviembre 	}&	 El Progreso 	&	 264 	&	 65 	&	 274 	&	 221 	&	 231 	&	 267 	&	 210 	&	 -   	&	 -   	&	 -   	&	 202 	&	 194 	&	 223 	&	 234 	&	 241 	\\
			\multicolumn{1}{l}{	\footnotesize	 Noviembre 	}&	 Escuintla 	&	 1,051 	&	 157 	&	 1,321 	&	 1,067 	&	 1,011 	&	 1,240 	&	 961 	&	 -   	&	 -   	&	 -   	&	 1,031 	&	 1,080 	&	 1,467 	&	 1,130 	&	 1,165 	\\
			\rowcolor{color1!5!white}\multicolumn{1}{l}{	\footnotesize	 Noviembre 	}&	 Guatemala 	&	 1,958 	&	 490 	&	 3,251 	&	 2,556 	&	 2,463 	&	 3,019 	&	 2,265 	&	 -   	&	 -   	&	 -   	&	 2,275 	&	 2,321 	&	 2,825 	&	 2,140 	&	 2,156 	\\
			\multicolumn{1}{l}{	\footnotesize	 Noviembre 	}&	 Huehuetenango 	&	 1,554 	&	 307 	&	 1,718 	&	 1,225 	&	 1,167 	&	 1,488 	&	 940 	&	 1 	&	 -   	&	 -   	&	 1,029 	&	 1,006 	&	 1,524 	&	 1,083 	&	 1,067 	\\
			\rowcolor{color1!5!white}\multicolumn{1}{l}{	\footnotesize	 Noviembre 	}&	 Izabal 	&	 168 	&	 3 	&	 203 	&	 201 	&	 114 	&	 258 	&	 159 	&	 -   	&	 -   	&	 -   	&	 86 	&	 77 	&	 185 	&	 178 	&	 168 	\\
			\multicolumn{1}{l}{	\footnotesize	 Noviembre 	}&	 Jalapa 	&	 695 	&	 83 	&	 720 	&	 310 	&	 320 	&	 594 	&	 469 	&	 -   	&	 -   	&	 -   	&	 768 	&	 680 	&	 764 	&	 806 	&	 766 	\\
			\rowcolor{color1!5!white}\multicolumn{1}{l}{	\footnotesize	 Noviembre 	}&	 Jutiapa 	&	 447 	&	 104 	&	 979 	&	 1,025 	&	 923 	&	 930 	&	 837 	&	 -   	&	 -   	&	 -   	&	 705 	&	 756 	&	 1,316 	&	 955 	&	 1,025 	\\
			\multicolumn{1}{l}{	\footnotesize	 Noviembre 	}&	 Quetzaltenango 	&	 1,141 	&	 87 	&	 1,388 	&	 1,189 	&	 1,086 	&	 1,364 	&	 1,071 	&	 1 	&	 -   	&	 -   	&	 1,016 	&	 1,010 	&	 1,313 	&	 1,030 	&	 1,063 	\\
			\rowcolor{color1!5!white}\multicolumn{1}{l}{	\footnotesize	 Noviembre 	}&	 Quiché 	&	 1,082 	&	 425 	&	 1,240 	&	 1,195 	&	 1,125 	&	 1,111 	&	 949 	&	 -   	&	 -   	&	 -   	&	 923 	&	 926 	&	 1,140 	&	 981 	&	 1,013 	\\
			\multicolumn{1}{l}{	\footnotesize	 Noviembre 	}&	 Retalhuleu 	&	 320 	&	 61 	&	 676 	&	 581 	&	 555 	&	 661 	&	 551 	&	 -   	&	 -   	&	 -   	&	 497 	&	 492 	&	 615 	&	 520 	&	 517 	\\
			\rowcolor{color1!5!white}\multicolumn{1}{l}{	\footnotesize	 Noviembre 	}&	 Sacatepéquez 	&	 217 	&	 43 	&	 553 	&	 481 	&	 413 	&	 546 	&	 463 	&	 -   	&	 -   	&	 -   	&	 349 	&	 346 	&	 454 	&	 357 	&	 355 	\\
			\multicolumn{1}{l}{	\footnotesize	 Noviembre 	}&	 San Marcos 	&	 1,062 	&	 474 	&	 1,007 	&	 792 	&	 767 	&	 672 	&	 488 	&	 1 	&	 1 	&	 -   	&	 544 	&	 576 	&	 807 	&	 562 	&	 532 	\\
			\rowcolor{color1!5!white}\multicolumn{1}{l}{	\footnotesize	 Noviembre 	}&	 Santa Rosa 	&	 566 	&	 53 	&	 620 	&	 505 	&	 513 	&	 615 	&	 464 	&	 -   	&	 -   	&	 -   	&	 457 	&	 468 	&	 690 	&	 552 	&	 558 	\\
			\multicolumn{1}{l}{	\footnotesize	 Noviembre 	}&	 Sololá 	&	 304 	&	 89 	&	 519 	&	 493 	&	 432 	&	 500 	&	 426 	&	 -   	&	 1 	&	 -   	&	 375 	&	 360 	&	 610 	&	 432 	&	 425 	\\
			\rowcolor{color1!5!white}\multicolumn{1}{l}{	\footnotesize	 Noviembre 	}&	 Suchitepéquez 	&	 555 	&	 161 	&	 947 	&	 890 	&	 767 	&	 917 	&	 804 	&	 4 	&	 -   	&	 -   	&	 739 	&	 711 	&	 1,001 	&	 847 	&	 844 	\\
			\multicolumn{1}{l}{	\footnotesize	 Noviembre 	}&	 Totonicapán 	&	 829 	&	 181 	&	 786 	&	 676 	&	 626 	&	 765 	&	 623 	&	 -   	&	 -   	&	 -   	&	 409 	&	 406 	&	 755 	&	 501 	&	 497 	\\
			\rowcolor{color1!5!white}\multicolumn{1}{l}{	\footnotesize	 Noviembre 	}&	 Zacapa 	&	 204 	&	 26 	&	 243 	&	 221 	&	 197 	&	 243 	&	 196 	&	 -   	&	 -   	&	 -   	&	 153 	&	 152 	&	 237 	&	 143 	&	 142 	\\
%			[-0.28cm]
		\end{longtable}\addtocounter{Cuadro}{1}
	\end{center}
\end{landscape}




%%%%%%%%%%%%%%%%%%%%%%





%%%%%%%%%%%%%%%%%



%\fontsize{4mm}{1.9em}\selectfont \setlength{\arrayrulewidth}{01pt}
	%	$\ $\\[-1.8cm]
	%	{\Bold\color{color1!80!black}{Cuadro \theCuadro $\,-$  Mujeres embarazadas al momento de la encuesta, que recibieron atención pre natal, por establecimiento o lugar a donde asistieron; según características varias. }}\\
	%	{\Bold\color{color1!80!black}{República de Guatemala, año 2008/2009. }}\\
	%	\normalsize (Porcentajes)\\[0.4cm]
	\begin{center}\fontsize{4mm}{1.8em}
		\selectfont \setlength{\arrayrulewidth}{1pt}
		$\ $\\[-2.0cm]
		$\!$\begin{longtable}{llrrrrrrrrrrr}
			\multicolumn{13}{l}{\Bold\color{color1!80!black}{Cuadro \theCuadro $\,-$  Esquema de vacunación por mes y departamento; según tipo de vacunas y grupos de edad. }}\\[-0.2cm]
			\multicolumn{13}{l}{\Bold\color{color1!80!black}{República de Guatemala, año 2015. }}\\
			\multicolumn{13}{l}{	\normalsize (Dosis aplicadas)}\\
			\multicolumn{13}{l}{$\ $}\\[-.5cm]\hline
			$\ $\\[-.3cm]
			\multicolumn{1}{c}{\multirow{3}{*}[1mm]{\begin{sideways}\Bold{Mes} \end{sideways}}}	&	\multicolumn{1}{c}{\multirow{3}{*}[1mm]{\begin{sideways}\Bold{Departamento}\end{sideways}}}&	\multicolumn{8}{c}{\Bold{Entre 1 y 6 años}} &\multicolumn{3}{c}{\Bold{Otras}}\\[0.1cm]\cline{3-10}
			%$\ $\\[0.05cm]
			\multicolumn{1}{c}{ }	&	\multicolumn{1}{c}{ }&	\multicolumn{2}{c}{\Bold{DTP}} &\multicolumn{5}{c}{\Bold{OPV}}&\multicolumn{1}{c}{\Bold{SPR}}&\multicolumn{3}{c}{\Bold{Rotavirus 3 Dosis}}\\[0.1cm]
			\multicolumn{1}{c}{ }&\multicolumn{1}{c}{ }&	\multicolumn{1}{c}{R1}&		\multicolumn{1}{c}{R2}&		\multicolumn{1}{c}{1°}&		\multicolumn{1}{c}{2°}&		\multicolumn{1}{c}{3°}&		\multicolumn{1}{c}{R1}&		\multicolumn{1}{c}{R2}&		\multicolumn{1}{c}{1°}&		\multicolumn{1}{c}{1°}&		\multicolumn{1}{c}{2°}&		\multicolumn{1}{c}{3°}\\\hline\endhead
			\hline \multicolumn{13}{r}{\textit{Continúa en la siguiente página}} \\
			\endfoot
									\hline \multicolumn{13}{l}{\textbf{Fuente:}\textit{SIGSA}} \\\endlastfoot
			\rowcolor{color1!40!white} \multicolumn{1}{l}{\Bold{	\footnotesize	 Total 	}}&		&	 43,118 	&	 20,232 	&	 10,109 	&	 21,818 	&	 45,535 	&	 38,686 	&	 17,673 	&	 23,757 	&	 6 	&	 -   	&	 -   	\\
			\multicolumn{1}{l}{	\footnotesize	 Enero 	}&	 Alta Verapaz 	&	 180 	&	 144 	&	 47 	&	 153 	&	 355 	&	 240 	&	 123 	&	 12 	&	 -   	&	 -   	&	 -   	\\
			\rowcolor{color1!5!white}\multicolumn{1}{l}{	\footnotesize	 Enero 	}&	 Baja Verapaz 	&	 25 	&	 63 	&	 6 	&	 6 	&	 24 	&	 13 	&	 26 	&	 1 	&	 -   	&	 -   	&	 -   	\\
			\multicolumn{1}{l}{	\footnotesize	 Enero 	}&	 Chimaltenango 	&	 35 	&	 347 	&	 5 	&	 8 	&	 33 	&	 37 	&	 244 	&	 -   	&	 -   	&	 -   	&	 -   	\\
			\rowcolor{color1!5!white}\multicolumn{1}{l}{	\footnotesize	 Enero 	}&	 Chiquimula 	&	 27 	&	 6 	&	 10 	&	 32 	&	 76 	&	 42 	&	 7 	&	 -   	&	 -   	&	 -   	&	 -   	\\
			\multicolumn{1}{l}{	\footnotesize	 Enero 	}&	 Petén 	&	 25 	&	 7 	&	 25 	&	 26 	&	 74 	&	 41 	&	 5 	&	 9 	&	 -   	&	 -   	&	 -   	\\
			\rowcolor{color1!5!white}\multicolumn{1}{l}{	\footnotesize	 Enero 	}&	 El Progreso 	&	 20 	&	 1 	&	 21 	&	 -   	&	 -   	&	 4 	&	 1 	&	 -   	&	 -   	&	 -   	&	 -   	\\
			\multicolumn{1}{l}{	\footnotesize	 Enero 	}&	 Escuintla 	&	 31 	&	 43 	&	 7 	&	 3 	&	 13 	&	 33 	&	 46 	&	 4 	&	 -   	&	 -   	&	 -   	\\
			\rowcolor{color1!5!white}\multicolumn{1}{l}{	\footnotesize	 Enero 	}&	 Guatemala 	&	 387 	&	 354 	&	 66 	&	 73 	&	 283 	&	 342 	&	 295 	&	 -   	&	 -   	&	 -   	&	 -   	\\
			\multicolumn{1}{l}{	\footnotesize	 Enero 	}&	 Huehuetenango 	&	 65 	&	 36 	&	 14 	&	 13 	&	 18 	&	 42 	&	 23 	&	 2 	&	 -   	&	 -   	&	 -   	\\
			\rowcolor{color1!5!white}\multicolumn{1}{l}{	\footnotesize	 Enero 	}&	 Izabal 	&	 46 	&	 5 	&	 19 	&	 21 	&	 43 	&	 52 	&	 17 	&	 3 	&	 -   	&	 -   	&	 -   	\\
			\multicolumn{1}{l}{	\footnotesize	 Enero 	}&	 Jalapa 	&	 30 	&	 98 	&	 -   	&	 4 	&	 14 	&	 16 	&	 63 	&	 -   	&	 -   	&	 -   	&	 -   	\\
			\rowcolor{color1!5!white}\multicolumn{1}{l}{	\footnotesize	 Enero 	}&	 Jutiapa 	&	 37 	&	 48 	&	 12 	&	 3 	&	 3 	&	 24 	&	 42 	&	 3 	&	 -   	&	 -   	&	 -   	\\
			\multicolumn{1}{l}{	\footnotesize	 Enero 	}&	 Quetzaltenango 	&	 20 	&	 21 	&	 16 	&	 38 	&	 118 	&	 76 	&	 30 	&	 15 	&	 -   	&	 -   	&	 -   	\\
			\rowcolor{color1!5!white}\multicolumn{1}{l}{	\footnotesize	 Enero 	}&	 Quiché 	&	 118 	&	 124 	&	 35 	&	 41 	&	 127 	&	 95 	&	 94 	&	 3 	&	 -   	&	 -   	&	 -   	\\
			\multicolumn{1}{l}{	\footnotesize	 Enero 	}&	 Retalhuleu 	&	 19 	&	 4 	&	 14 	&	 8 	&	 28 	&	 13 	&	 5 	&	 1 	&	 -   	&	 -   	&	 -   	\\
			\rowcolor{color1!5!white}\multicolumn{1}{l}{	\footnotesize	 Enero 	}&	 Sacatepéquez 	&	 44 	&	 50 	&	 10 	&	 28 	&	 109 	&	 83 	&	 51 	&	 -   	&	 -   	&	 -   	&	 -   	\\
			\multicolumn{1}{l}{	\footnotesize	 Enero 	}&	 San Marcos 	&	 230 	&	 226 	&	 34 	&	 14 	&	 28 	&	 45 	&	 51 	&	 7 	&	 -   	&	 -   	&	 -   	\\
			\rowcolor{color1!5!white}\multicolumn{1}{l}{	\footnotesize	 Enero 	}&	 Santa Rosa 	&	 43 	&	 7 	&	 3 	&	 12 	&	 63 	&	 44 	&	 7 	&	 1 	&	 -   	&	 -   	&	 -   	\\
			\multicolumn{1}{l}{	\footnotesize	 Enero 	}&	 Solola 	&	 14 	&	 5 	&	 4 	&	 5 	&	 14 	&	 28 	&	 12 	&	 9 	&	 -   	&	 -   	&	 -   	\\
			\rowcolor{color1!5!white}\multicolumn{1}{l}{	\footnotesize	 Enero 	}&	 Suchitepéquez 	&	 30 	&	 35 	&	 8 	&	 13 	&	 14 	&	 32 	&	 40 	&	 3 	&	 -   	&	 -   	&	 -   	\\
			\multicolumn{1}{l}{	\footnotesize	 Enero 	}&	 Totonicapán 	&	 3 	&	 2 	&	 5 	&	 14 	&	 16 	&	 10 	&	 3 	&	 -   	&	 -   	&	 -   	&	 -   	\\
			\rowcolor{color1!5!white}\multicolumn{1}{l}{	\footnotesize	 Enero 	}&	 Zacapa 	&	 20 	&	 4 	&	 21 	&	 5 	&	 11 	&	 5 	&	 1 	&	 -   	&	 -   	&	 -   	&	 -   	\\
			\multicolumn{1}{l}{	\footnotesize	 Febrero 	}&	 Alta Verapaz 	&	 466 	&	 416 	&	 72 	&	 304 	&	 722 	&	 501 	&	 338 	&	 798 	&	 -   	&	 -   	&	 -   	\\
			\rowcolor{color1!5!white}\multicolumn{1}{l}{	\footnotesize	 Febrero 	}&	 Baja Verapaz 	&	 43 	&	 89 	&	 1 	&	 9 	&	 31 	&	 42 	&	 96 	&	 4 	&	 -   	&	 -   	&	 -   	\\
			\multicolumn{1}{l}{	\footnotesize	 Febrero 	}&	 Chimaltenango 	&	 159 	&	 381 	&	 11 	&	 33 	&	 65 	&	 158 	&	 307 	&	 1,558 	&	 -   	&	 -   	&	 -   	\\
			\rowcolor{color1!5!white}\multicolumn{1}{l}{	\footnotesize	 Febrero 	}&	 Chiquimula 	&	 71 	&	 12 	&	 14 	&	 41 	&	 101 	&	 78 	&	 10 	&	 8 	&	 -   	&	 -   	&	 -   	\\
			\multicolumn{1}{l}{	\footnotesize	 Febrero 	}&	 Petén 	&	 258 	&	 17 	&	 49 	&	 83 	&	 184 	&	 250 	&	 16 	&	 34 	&	 -   	&	 -   	&	 -   	\\
			\rowcolor{color1!5!white}\multicolumn{1}{l}{	\footnotesize	 Febrero 	}&	 El Progreso 	&	 13 	&	 3 	&	 2 	&	 4 	&	 4 	&	 4 	&	 1 	&	 8 	&	 -   	&	 -   	&	 -   	\\
			\multicolumn{1}{l}{	\footnotesize	 Febrero 	}&	 Escuintla 	&	 231 	&	 111 	&	 28 	&	 56 	&	 168 	&	 223 	&	 109 	&	 69 	&	 -   	&	 -   	&	 -   	\\
			\rowcolor{color1!5!white}\multicolumn{1}{l}{	\footnotesize	 Febrero 	}&	 Guatemala 	&	 577 	&	 304 	&	 69 	&	 176 	&	 492 	&	 447 	&	 212 	&	 137 	&	 -   	&	 -   	&	 -   	\\
			\multicolumn{1}{l}{	\footnotesize	 Febrero 	}&	 Huehuetenango 	&	 504 	&	 203 	&	 114 	&	 88 	&	 221 	&	 411 	&	 131 	&	 80 	&	 -   	&	 -   	&	 -   	\\
			\rowcolor{color1!5!white}\multicolumn{1}{l}{	\footnotesize	 Febrero 	}&	 Izabal 	&	 70 	&	 7 	&	 52 	&	 52 	&	 168 	&	 76 	&	 8 	&	 216 	&	 -   	&	 -   	&	 -   	\\
			\multicolumn{1}{l}{	\footnotesize	 Febrero 	}&	 Jalapa 	&	 192 	&	 214 	&	 6 	&	 57 	&	 210 	&	 122 	&	 117 	&	 532 	&	 -   	&	 -   	&	 -   	\\
			\rowcolor{color1!5!white}\multicolumn{1}{l}{	\footnotesize	 Febrero 	}&	 Jutiapa 	&	 23 	&	 18 	&	 7 	&	 4 	&	 10 	&	 11 	&	 6 	&	 12 	&	 -   	&	 -   	&	 -   	\\
			\multicolumn{1}{l}{	\footnotesize	 Febrero 	}&	 Quetzaltenango 	&	 219 	&	 145 	&	 30 	&	 88 	&	 222 	&	 155 	&	 93 	&	 70 	&	 -   	&	 -   	&	 -   	\\
			\rowcolor{color1!5!white}\multicolumn{1}{l}{	\footnotesize	 Febrero 	}&	 Quiché 	&	 269 	&	 159 	&	 36 	&	 116 	&	 239 	&	 164 	&	 90 	&	 291 	&	 -   	&	 -   	&	 -   	\\
			\multicolumn{1}{l}{	\footnotesize	 Febrero 	}&	 Retalhuleu 	&	 48 	&	 22 	&	 28 	&	 25 	&	 52 	&	 36 	&	 19 	&	 122 	&	 -   	&	 -   	&	 -   	\\
			\rowcolor{color1!5!white}\multicolumn{1}{l}{	\footnotesize	 Febrero 	}&	 Sacatepéquez 	&	 86 	&	 81 	&	 15 	&	 43 	&	 141 	&	 107 	&	 68 	&	 19 	&	 -   	&	 -   	&	 -   	\\
			\multicolumn{1}{l}{	\footnotesize	 Febrero 	}&	 San Marcos 	&	 351 	&	 297 	&	 91 	&	 32 	&	 126 	&	 153 	&	 185 	&	 503 	&	 -   	&	 -   	&	 -   	\\
			\rowcolor{color1!5!white}\multicolumn{1}{l}{	\footnotesize	 Febrero 	}&	 Santa Rosa 	&	 63 	&	 22 	&	 5 	&	 16 	&	 70 	&	 46 	&	 12 	&	 6 	&	 -   	&	 -   	&	 -   	\\
			\multicolumn{1}{l}{	\footnotesize	 Febrero 	}&	 Sololá 	&	 133 	&	 62 	&	 9 	&	 22 	&	 37 	&	 135 	&	 66 	&	 20 	&	 -   	&	 -   	&	 -   	\\
			\rowcolor{color1!5!white}\multicolumn{1}{l}{	\footnotesize	 Febrero 	}&	 Suchitepéquez 	&	 82 	&	 31 	&	 18 	&	 38 	&	 126 	&	 88 	&	 40 	&	 19 	&	 -   	&	 -   	&	 -   	\\
			\multicolumn{1}{l}{	\footnotesize	 Febrero 	}&	 Totonicapán 	&	 34 	&	 3 	&	 25 	&	 35 	&	 38 	&	 38 	&	 3 	&	 4 	&	 -   	&	 -   	&	 -   	\\
			\rowcolor{color1!5!white}\multicolumn{1}{l}{	\footnotesize	 Febrero 	}&	 Zacapa 	&	 50 	&	 18 	&	 30 	&	 17 	&	 39 	&	 26 	&	 18 	&	 12 	&	 5 	&	 -   	&	 -   	\\
			\multicolumn{1}{l}{	\footnotesize	 Marzo 	}&	 Alta Verapaz 	&	 743 	&	 315 	&	 128 	&	 410 	&	 641 	&	 588 	&	 255 	&	 1,251 	&	 -   	&	 -   	&	 -   	\\
			\rowcolor{color1!5!white}\multicolumn{1}{l}{	\footnotesize	 Marzo 	}&	 Baja Verapaz 	&	 21 	&	 17 	&	 1 	&	 26 	&	 64 	&	 24 	&	 24 	&	 22 	&	 -   	&	 -   	&	 -   	\\
			\multicolumn{1}{l}{	\footnotesize	 Marzo 	}&	 Chimaltenango 	&	 170 	&	 218 	&	 19 	&	 17 	&	 53 	&	 73 	&	 87 	&	 799 	&	 -   	&	 -   	&	 -   	\\
			\rowcolor{color1!5!white}\multicolumn{1}{l}{	\footnotesize	 Marzo 	}&	 Chiquimula 	&	 126 	&	 16 	&	 21 	&	 73 	&	 172 	&	 128 	&	 16 	&	 26 	&	 -   	&	 -   	&	 -   	\\
			\multicolumn{1}{l}{	\footnotesize	 Marzo 	}&	 Petén 	&	 224 	&	 33 	&	 51 	&	 120 	&	 139 	&	 220 	&	 27 	&	 89 	&	 -   	&	 -   	&	 -   	\\
			\rowcolor{color1!5!white}\multicolumn{1}{l}{	\footnotesize	 Marzo 	}&	 El Progreso 	&	 16 	&	 2 	&	 7 	&	 22 	&	 35 	&	 13 	&	 3 	&	 14 	&	 -   	&	 -   	&	 -   	\\
			\multicolumn{1}{l}{	\footnotesize	 Marzo 	}&	 Escuintla 	&	 114 	&	 45 	&	 15 	&	 51 	&	 74 	&	 92 	&	 37 	&	 69 	&	 -   	&	 -   	&	 -   	\\
			\rowcolor{color1!5!white}\multicolumn{1}{l}{	\footnotesize	 Marzo 	}&	 Guatemala 	&	 658 	&	 396 	&	 105 	&	 304 	&	 734 	&	 623 	&	 382 	&	 350 	&	 -   	&	 -   	&	 -   	\\
			\multicolumn{1}{l}{	\footnotesize	 Marzo 	}&	 Huehuetenango 	&	 525 	&	 217 	&	 110 	&	 115 	&	 233 	&	 399 	&	 176 	&	 93 	&	 -   	&	 -   	&	 -   	\\
			\rowcolor{color1!5!white}\multicolumn{1}{l}{	\footnotesize	 Marzo 	}&	 Izabal 	&	 132 	&	 28 	&	 31 	&	 69 	&	 137 	&	 93 	&	 35 	&	 130 	&	 -   	&	 -   	&	 -   	\\
			\multicolumn{1}{l}{	\footnotesize	 Marzo 	}&	 Jalapa 	&	 148 	&	 119 	&	 10 	&	 51 	&	 167 	&	 120 	&	 81 	&	 210 	&	 -   	&	 -   	&	 -   	\\
			\rowcolor{color1!5!white}\multicolumn{1}{l}{	\footnotesize	 Marzo 	}&	 Jutiapa 	&	 48 	&	 15 	&	 10 	&	 27 	&	 43 	&	 52 	&	 7 	&	 16 	&	 -   	&	 -   	&	 -   	\\
			\multicolumn{1}{l}{	\footnotesize	 Marzo 	}&	 Quetzaltenango 	&	 232 	&	 89 	&	 42 	&	 110 	&	 246 	&	 157 	&	 83 	&	 58 	&	 -   	&	 -   	&	 -   	\\
			\rowcolor{color1!5!white}\multicolumn{1}{l}{	\footnotesize	 Marzo 	}&	 Quiché 	&	 469 	&	 136 	&	 66 	&	 209 	&	 420 	&	 366 	&	 96 	&	 716 	&	 -   	&	 -   	&	 -   	\\
			\multicolumn{1}{l}{	\footnotesize	 Marzo 	}&	 Retalhuleu 	&	 46 	&	 16 	&	 15 	&	 30 	&	 103 	&	 41 	&	 10 	&	 113 	&	 -   	&	 -   	&	 -   	\\
			\rowcolor{color1!5!white}\multicolumn{1}{l}{	\footnotesize	 Marzo 	}&	 Sacatepéquez 	&	 108 	&	 46 	&	 7 	&	 71 	&	 141 	&	 135 	&	 20 	&	 27 	&	 -   	&	 -   	&	 -   	\\
			\multicolumn{1}{l}{	\footnotesize	 Marzo 	}&	 San Marcos 	&	 177 	&	 131 	&	 64 	&	 70 	&	 127 	&	 140 	&	 124 	&	 235 	&	 -   	&	 -   	&	 -   	\\
			\rowcolor{color1!5!white}\multicolumn{1}{l}{	\footnotesize	 Marzo 	}&	 Santa Rosa 	&	 90 	&	 10 	&	 10 	&	 54 	&	 133 	&	 99 	&	 12 	&	 24 	&	 -   	&	 -   	&	 -   	\\
			\multicolumn{1}{l}{	\footnotesize	 Marzo 	}&	 Sololá 	&	 140 	&	 53 	&	 15 	&	 30 	&	 62 	&	 157 	&	 49 	&	 104 	&	 -   	&	 -   	&	 -   	\\
			\rowcolor{color1!5!white}\multicolumn{1}{l}{	\footnotesize	 Marzo 	}&	 Suchitepéquez 	&	 91 	&	 48 	&	 13 	&	 46 	&	 126 	&	 93 	&	 46 	&	 11 	&	 -   	&	 -   	&	 -   	\\
			\multicolumn{1}{l}{	\footnotesize	 Marzo 	}&	 Totonicapán 	&	 28 	&	 6 	&	 17 	&	 28 	&	 66 	&	 25 	&	 5 	&	 13 	&	 -   	&	 -   	&	 -   	\\
			\rowcolor{color1!5!white}\multicolumn{1}{l}{	\footnotesize	 Marzo 	}&	 Zacapa 	&	 45 	&	 5 	&	 26 	&	 9 	&	 30 	&	 21 	&	 4 	&	 13 	&	 -   	&	 -   	&	 -   	\\
			\multicolumn{1}{l}{	\footnotesize	 Mayo 	}&	 Alta Verapaz 	&	 791 	&	 425 	&	 252 	&	 506 	&	 734 	&	 628 	&	 390 	&	 621 	&	 -   	&	 -   	&	 -   	\\
			\rowcolor{color1!5!white}\multicolumn{1}{l}{	\footnotesize	 Mayo 	}&	 Baja Verapaz 	&	 141 	&	 61 	&	 18 	&	 47 	&	 133 	&	 113 	&	 48 	&	 6 	&	 -   	&	 -   	&	 -   	\\
			\multicolumn{1}{l}{	\footnotesize	 Mayo 	}&	 Chimaltenango 	&	 294 	&	 131 	&	 24 	&	 67 	&	 166 	&	 238 	&	 108 	&	 354 	&	 -   	&	 -   	&	 -   	\\
			\rowcolor{color1!5!white}\multicolumn{1}{l}{	\footnotesize	 Mayo 	}&	 Chiquimula 	&	 120 	&	 18 	&	 71 	&	 161 	&	 289 	&	 159 	&	 17 	&	 14 	&	 -   	&	 -   	&	 -   	\\
			\multicolumn{1}{l}{	\footnotesize	 Mayo 	}&	 Petén 	&	 165 	&	 22 	&	 69 	&	 119 	&	 223 	&	 184 	&	 16 	&	 54 	&	 -   	&	 -   	&	 -   	\\
			\rowcolor{color1!5!white}\multicolumn{1}{l}{	\footnotesize	 Mayo 	}&	 El Progreso 	&	 18 	&	 2 	&	 7 	&	 31 	&	 70 	&	 15 	&	 1 	&	 8 	&	 -   	&	 -   	&	 -   	\\
			\multicolumn{1}{l}{	\footnotesize	 Mayo 	}&	 Escuintla 	&	 399 	&	 153 	&	 38 	&	 97 	&	 262 	&	 361 	&	 128 	&	 128 	&	 -   	&	 -   	&	 -   	\\
			\rowcolor{color1!5!white}\multicolumn{1}{l}{	\footnotesize	 Mayo 	}&	 Guatemala 	&	 1,546 	&	 876 	&	 261 	&	 517 	&	 1,016 	&	 1,375 	&	 844 	&	 566 	&	 -   	&	 -   	&	 -   	\\
			\multicolumn{1}{l}{	\footnotesize	 Mayo 	}&	 Huehuetenango 	&	 728 	&	 348 	&	 140 	&	 222 	&	 432 	&	 568 	&	 307 	&	 55 	&	 -   	&	 -   	&	 -   	\\
			\rowcolor{color1!5!white}\multicolumn{1}{l}{	\footnotesize	 Mayo 	}&	 Izabal 	&	 145 	&	 21 	&	 33 	&	 66 	&	 113 	&	 101 	&	 9 	&	 50 	&	 -   	&	 -   	&	 -   	\\
			\multicolumn{1}{l}{	\footnotesize	 Mayo 	}&	 Jalapa 	&	 116 	&	 72 	&	 39 	&	 131 	&	 347 	&	 177 	&	 83 	&	 79 	&	 -   	&	 -   	&	 -   	\\
			\rowcolor{color1!5!white}\multicolumn{1}{l}{	\footnotesize	 Mayo 	}&	 Jutiapa 	&	 101 	&	 23 	&	 42 	&	 36 	&	 88 	&	 60 	&	 16 	&	 5 	&	 -   	&	 -   	&	 -   	\\
			\multicolumn{1}{l}{	\footnotesize	 Mayo 	}&	 Quetzaltenango 	&	 533 	&	 98 	&	 99 	&	 242 	&	 454 	&	 478 	&	 91 	&	 113 	&	 -   	&	 -   	&	 -   	\\
			\rowcolor{color1!5!white}\multicolumn{1}{l}{	\footnotesize	 Mayo 	}&	 Quiché 	&	 793 	&	 310 	&	 170 	&	 395 	&	 707 	&	 601 	&	 247 	&	 485 	&	 -   	&	 -   	&	 -   	\\
			\multicolumn{1}{l}{	\footnotesize	 Mayo 	}&	 Retalhuleu 	&	 146 	&	 46 	&	 31 	&	 57 	&	 103 	&	 129 	&	 51 	&	 54 	&	 -   	&	 -   	&	 -   	\\
			\rowcolor{color1!5!white}\multicolumn{1}{l}{	\footnotesize	 Mayo 	}&	 Sacatepéquez 	&	 195 	&	 34 	&	 41 	&	 95 	&	 173 	&	 199 	&	 34 	&	 63 	&	 -   	&	 -   	&	 -   	\\
			\multicolumn{1}{l}{	\footnotesize	 Mayo 	}&	 San Marcos 	&	 474 	&	 413 	&	 146 	&	 141 	&	 256 	&	 266 	&	 275 	&	 189 	&	 -   	&	 -   	&	 -   	\\
			\rowcolor{color1!5!white}\multicolumn{1}{l}{	\footnotesize	 Mayo 	}&	 Santa Rosa 	&	 149 	&	 27 	&	 24 	&	 75 	&	 156 	&	 142 	&	 25 	&	 11 	&	 -   	&	 -   	&	 -   	\\
			\multicolumn{1}{l}{	\footnotesize	 Mayo 	}&	 Solola 	&	 263 	&	 106 	&	 35 	&	 69 	&	 130 	&	 238 	&	 106 	&	 27 	&	 -   	&	 -   	&	 -   	\\
			\rowcolor{color1!5!white}\multicolumn{1}{l}{	\footnotesize	 Mayo 	}&	 Suchitepéquez 	&	 159 	&	 69 	&	 47 	&	 131 	&	 288 	&	 150 	&	 68 	&	 35 	&	 -   	&	 -   	&	 -   	\\
			\multicolumn{1}{l}{	\footnotesize	 Mayo 	}&	 Totonicapán 	&	 85 	&	 22 	&	 43 	&	 51 	&	 130 	&	 74 	&	 20 	&	 2 	&	 -   	&	 -   	&	 -   	\\
			\rowcolor{color1!5!white}\multicolumn{1}{l}{	\footnotesize	 Mayo 	}&	 Zacapa 	&	 86 	&	 31 	&	 32 	&	 30 	&	 34 	&	 33 	&	 14 	&	 3 	&	 -   	&	 -   	&	 -   	\\
			\multicolumn{1}{l}{	\footnotesize	 Abril 	}&	 Alta Verapaz 	&	 424 	&	 189 	&	 68 	&	 282 	&	 507 	&	 360 	&	 151 	&	 333 	&	 -   	&	 -   	&	 -   	\\
			\rowcolor{color1!5!white}\multicolumn{1}{l}{	\footnotesize	 Abril 	}&	 Baja Verapaz 	&	 83 	&	 47 	&	 6 	&	 26 	&	 67 	&	 56 	&	 25 	&	 14 	&	 -   	&	 -   	&	 -   	\\
			\multicolumn{1}{l}{	\footnotesize	 Abril 	}&	 Chimaltenango 	&	 226 	&	 178 	&	 19 	&	 38 	&	 111 	&	 146 	&	 142 	&	 417 	&	 -   	&	 -   	&	 -   	\\
			\rowcolor{color1!5!white}\multicolumn{1}{l}{	\footnotesize	 Abril 	}&	 Chiquimula 	&	 61 	&	 7 	&	 6 	&	 13 	&	 66 	&	 39 	&	 3 	&	 19 	&	 -   	&	 -   	&	 -   	\\
			\multicolumn{1}{l}{	\footnotesize	 Abril 	}&	 Petén 	&	 144 	&	 17 	&	 44 	&	 77 	&	 148 	&	 130 	&	 12 	&	 82 	&	 -   	&	 -   	&	 -   	\\
			\rowcolor{color1!5!white}\multicolumn{1}{l}{	\footnotesize	 Abril 	}&	 El Progreso 	&	 14 	&	 5 	&	 2 	&	 13 	&	 25 	&	 12 	&	 4 	&	 9 	&	 -   	&	 -   	&	 -   	\\
			\multicolumn{1}{l}{	\footnotesize	 Abril 	}&	 Escuintla 	&	 110 	&	 62 	&	 5 	&	 19 	&	 83 	&	 85 	&	 55 	&	 36 	&	 -   	&	 -   	&	 -   	\\
			\rowcolor{color1!5!white}\multicolumn{1}{l}{	\footnotesize	 Abril 	}&	 Guatemala 	&	 649 	&	 453 	&	 106 	&	 291 	&	 676 	&	 594 	&	 406 	&	 262 	&	 -   	&	 -   	&	 -   	\\
			\multicolumn{1}{l}{	\footnotesize	 Abril 	}&	 Huehuetenango 	&	 526 	&	 211 	&	 109 	&	 72 	&	 159 	&	 426 	&	 204 	&	 30 	&	 -   	&	 -   	&	 -   	\\
			\rowcolor{color1!5!white}\multicolumn{1}{l}{	\footnotesize	 Abril 	}&	 Izabal 	&	 85 	&	 17 	&	 22 	&	 45 	&	 89 	&	 86 	&	 22 	&	 28 	&	 -   	&	 -   	&	 -   	\\
			\multicolumn{1}{l}{	\footnotesize	 Abril 	}&	 Jalapa 	&	 180 	&	 72 	&	 43 	&	 91 	&	 283 	&	 187 	&	 73 	&	 199 	&	 -   	&	 -   	&	 -   	\\
			\rowcolor{color1!5!white}\multicolumn{1}{l}{	\footnotesize	 Abril 	}&	 Jutiapa 	&	 41 	&	 6 	&	 7 	&	 19 	&	 42 	&	 37 	&	 6 	&	 4 	&	 -   	&	 -   	&	 -   	\\
			\multicolumn{1}{l}{	\footnotesize	 Abril 	}&	 Quetzaltenango 	&	 323 	&	 92 	&	 58 	&	 113 	&	 295 	&	 284 	&	 84 	&	 126 	&	 -   	&	 -   	&	 -   	\\
			\rowcolor{color1!5!white}\multicolumn{1}{l}{	\footnotesize	 Abril 	}&	 Quiché 	&	 338 	&	 135 	&	 54 	&	 139 	&	 309 	&	 250 	&	 106 	&	 281 	&	 -   	&	 -   	&	 -   	\\
			\multicolumn{1}{l}{	\footnotesize	 Abril 	}&	 Retalhuleu 	&	 75 	&	 47 	&	 33 	&	 53 	&	 108 	&	 90 	&	 36 	&	 81 	&	 -   	&	 -   	&	 -   	\\
			\rowcolor{color1!5!white}\multicolumn{1}{l}{	\footnotesize	 Abril 	}&	 Sacatepéquez 	&	 110 	&	 25 	&	 27 	&	 63 	&	 114 	&	 110 	&	 21 	&	 40 	&	 -   	&	 -   	&	 -   	\\
			\multicolumn{1}{l}{	\footnotesize	 Abril 	}&	 San Marcos 	&	 117 	&	 135 	&	 63 	&	 94 	&	 161 	&	 81 	&	 105 	&	 100 	&	 -   	&	 -   	&	 -   	\\
			\rowcolor{color1!5!white}\multicolumn{1}{l}{	\footnotesize	 Abril 	}&	 Santa Rosa 	&	 107 	&	 10 	&	 25 	&	 44 	&	 155 	&	 103 	&	 10 	&	 12 	&	 -   	&	 -   	&	 -   	\\
			\multicolumn{1}{l}{	\footnotesize	 Abril 	}&	 Sololá 	&	 115 	&	 66 	&	 15 	&	 39 	&	 72 	&	 119 	&	 59 	&	 79 	&	 -   	&	 -   	&	 -   	\\
			\rowcolor{color1!5!white}\multicolumn{1}{l}{	\footnotesize	 Abril 	}&	 Suchitepéquez 	&	 63 	&	 28 	&	 5 	&	 46 	&	 123 	&	 67 	&	 28 	&	 19 	&	 -   	&	 -   	&	 -   	\\
			\multicolumn{1}{l}{	\footnotesize	 Abril 	}&	 Totonicapán 	&	 38 	&	 14 	&	 21 	&	 29 	&	 56 	&	 34 	&	 13 	&	 2 	&	 -   	&	 -   	&	 -   	\\
			\rowcolor{color1!5!white}\multicolumn{1}{l}{	\footnotesize	 Abril 	}&	 Zacapa 	&	 35 	&	 10 	&	 34 	&	 8 	&	 28 	&	 17 	&	 7 	&	 8 	&	 -   	&	 -   	&	 -   	\\
			\multicolumn{1}{l}{	\footnotesize	 Agosto 	}&	 Alta Verapaz 	&	 623 	&	 239 	&	 189 	&	 519 	&	 781 	&	 537 	&	 184 	&	 298 	&	 -   	&	 -   	&	 -   	\\
			\rowcolor{color1!5!white}\multicolumn{1}{l}{	\footnotesize	 Agosto 	}&	 Baja Verapaz 	&	 38 	&	 8 	&	 11 	&	 29 	&	 71 	&	 43 	&	 10 	&	 18 	&	 -   	&	 -   	&	 -   	\\
			\multicolumn{1}{l}{	\footnotesize	 Agosto 	}&	 Chimaltenango 	&	 142 	&	 53 	&	 12 	&	 35 	&	 95 	&	 113 	&	 43 	&	 68 	&	 -   	&	 -   	&	 -   	\\
			\rowcolor{color1!5!white}\multicolumn{1}{l}{	\footnotesize	 Agosto 	}&	 Chiquimula 	&	 122 	&	 21 	&	 38 	&	 82 	&	 282 	&	 128 	&	 18 	&	 26 	&	 -   	&	 -   	&	 -   	\\
			\multicolumn{1}{l}{	\footnotesize	 Agosto 	}&	 Petén 	&	 82 	&	 24 	&	 37 	&	 51 	&	 118 	&	 68 	&	 24 	&	 47 	&	 -   	&	 -   	&	 -   	\\
			\rowcolor{color1!5!white}\multicolumn{1}{l}{	\footnotesize	 Agosto 	}&	 El Progreso 	&	 5 	&	 3 	&	 3 	&	 10 	&	 16 	&	 6 	&	 3 	&	 8 	&	 -   	&	 -   	&	 -   	\\
			\multicolumn{1}{l}{	\footnotesize	 Agosto 	}&	 Escuintla 	&	 169 	&	 35 	&	 18 	&	 41 	&	 116 	&	 158 	&	 36 	&	 35 	&	 -   	&	 -   	&	 -   	\\
			\rowcolor{color1!5!white}\multicolumn{1}{l}{	\footnotesize	 Agosto 	}&	 Guatemala 	&	 481 	&	 322 	&	 109 	&	 249 	&	 575 	&	 525 	&	 414 	&	 255 	&	 -   	&	 -   	&	 -   	\\
			\multicolumn{1}{l}{	\footnotesize	 Agosto 	}&	 Huehuetenango 	&	 582 	&	 250 	&	 109 	&	 152 	&	 308 	&	 513 	&	 227 	&	 213 	&	 -   	&	 -   	&	 -   	\\
			\rowcolor{color1!5!white}\multicolumn{1}{l}{	\footnotesize	 Agosto 	}&	 Izabal 	&	 114 	&	 19 	&	 20 	&	 46 	&	 103 	&	 86 	&	 12 	&	 28 	&	 -   	&	 -   	&	 -   	\\
			\multicolumn{1}{l}{	\footnotesize	 Agosto 	}&	 Jalapa 	&	 123 	&	 62 	&	 16 	&	 53 	&	 167 	&	 101 	&	 47 	&	 39 	&	 -   	&	 -   	&	 -   	\\
			\rowcolor{color1!5!white}\multicolumn{1}{l}{	\footnotesize	 Agosto 	}&	 Jutiapa 	&	 22 	&	 7 	&	 12 	&	 24 	&	 58 	&	 19 	&	 5 	&	 20 	&	 -   	&	 -   	&	 -   	\\
			\multicolumn{1}{l}{	\footnotesize	 Agosto 	}&	 Quetzaltenango 	&	 282 	&	 93 	&	 54 	&	 179 	&	 336 	&	 318 	&	 91 	&	 73 	&	 -   	&	 -   	&	 -   	\\
			\rowcolor{color1!5!white}\multicolumn{1}{l}{	\footnotesize	 Agosto 	}&	 Quiché 	&	 268 	&	 81 	&	 66 	&	 219 	&	 440 	&	 273 	&	 82 	&	 104 	&	 -   	&	 -   	&	 -   	\\
			\multicolumn{1}{l}{	\footnotesize	 Agosto 	}&	 Retalhuleu 	&	 38 	&	 12 	&	 13 	&	 27 	&	 70 	&	 31 	&	 13 	&	 22 	&	 -   	&	 -   	&	 -   	\\
			\rowcolor{color1!5!white}\multicolumn{1}{l}{	\footnotesize	 Agosto 	}&	 Sacatepéquez 	&	 60 	&	 11 	&	 26 	&	 42 	&	 104 	&	 77 	&	 12 	&	 30 	&	 -   	&	 -   	&	 -   	\\
			\multicolumn{1}{l}{	\footnotesize	 Agosto 	}&	 San Marcos 	&	 294 	&	 119 	&	 71 	&	 147 	&	 282 	&	 225 	&	 94 	&	 143 	&	 -   	&	 -   	&	 -   	\\
			\rowcolor{color1!5!white}\multicolumn{1}{l}{	\footnotesize	 Agosto 	}&	 Santa Rosa 	&	 85 	&	 21 	&	 15 	&	 57 	&	 137 	&	 104 	&	 20 	&	 37 	&	 -   	&	 -   	&	 -   	\\
			\multicolumn{1}{l}{	\footnotesize	 Agosto 	}&	 Sololá 	&	 134 	&	 59 	&	 33 	&	 62 	&	 94 	&	 109 	&	 56 	&	 68 	&	 -   	&	 -   	&	 -   	\\
			\rowcolor{color1!5!white}\multicolumn{1}{l}{	\footnotesize	 Agosto 	}&	 Suchitepéquez 	&	 116 	&	 43 	&	 38 	&	 57 	&	 142 	&	 116 	&	 46 	&	 41 	&	 -   	&	 -   	&	 -   	\\
			\multicolumn{1}{l}{	\footnotesize	 Agosto 	}&	 Totonicapán 	&	 82 	&	 15 	&	 42 	&	 63 	&	 127 	&	 88 	&	 14 	&	 69 	&	 -   	&	 -   	&	 -   	\\
			\rowcolor{color1!5!white}\multicolumn{1}{l}{	\footnotesize	 Agosto 	}&	 Zacapa 	&	 29 	&	 17 	&	 17 	&	 6 	&	 23 	&	 20 	&	 18 	&	 9 	&	 -   	&	 -   	&	 -   	\\
			\multicolumn{1}{l}{	\footnotesize	 Julio 	}&	 Alta Verapaz 	&	 681 	&	 309 	&	 351 	&	 605 	&	 776 	&	 674 	&	 282 	&	 414 	&	 -   	&	 -   	&	 -   	\\
			\rowcolor{color1!5!white}\multicolumn{1}{l}{	\footnotesize	 Julio 	}&	 Baja Verapaz 	&	 44 	&	 14 	&	 16 	&	 30 	&	 85 	&	 57 	&	 11 	&	 33 	&	 -   	&	 -   	&	 -   	\\
			\multicolumn{1}{l}{	\footnotesize	 Julio 	}&	 Chimaltenango 	&	 217 	&	 100 	&	 31 	&	 72 	&	 148 	&	 201 	&	 98 	&	 128 	&	 -   	&	 -   	&	 -   	\\
			\rowcolor{color1!5!white}\multicolumn{1}{l}{	\footnotesize	 Julio 	}&	 Chiquimula 	&	 114 	&	 10 	&	 34 	&	 109 	&	 239 	&	 84 	&	 9 	&	 21 	&	 -   	&	 -   	&	 -   	\\
			\multicolumn{1}{l}{	\footnotesize	 Julio 	}&	 Petén 	&	 89 	&	 12 	&	 44 	&	 44 	&	 91 	&	 80 	&	 11 	&	 56 	&	 -   	&	 -   	&	 -   	\\
			\rowcolor{color1!5!white}\multicolumn{1}{l}{	\footnotesize	 Julio 	}&	 El Progreso 	&	 10 	&	 4 	&	 2 	&	 5 	&	 35 	&	 11 	&	 3 	&	 2 	&	 -   	&	 -   	&	 -   	\\
			\multicolumn{1}{l}{	\footnotesize	 Julio 	}&	 Escuintla 	&	 135 	&	 42 	&	 12 	&	 41 	&	 104 	&	 130 	&	 44 	&	 37 	&	 -   	&	 -   	&	 -   	\\
			\rowcolor{color1!5!white}\multicolumn{1}{l}{	\footnotesize	 Julio 	}&	 Guatemala 	&	 507 	&	 352 	&	 81 	&	 276 	&	 592 	&	 529 	&	 456 	&	 202 	&	 -   	&	 -   	&	 -   	\\
			\multicolumn{1}{l}{	\footnotesize	 Julio 	}&	 Huehuetenango 	&	 606 	&	 303 	&	 148 	&	 275 	&	 543 	&	 547 	&	 269 	&	 178 	&	 -   	&	 -   	&	 -   	\\
			\rowcolor{color1!5!white}\multicolumn{1}{l}{	\footnotesize	 Julio 	}&	 Izabal 	&	 80 	&	 18 	&	 13 	&	 29 	&	 43 	&	 41 	&	 13 	&	 15 	&	 -   	&	 -   	&	 -   	\\
			\multicolumn{1}{l}{	\footnotesize	 Julio 	}&	 Jalapa 	&	 215 	&	 113 	&	 28 	&	 146 	&	 351 	&	 168 	&	 97 	&	 37 	&	 -   	&	 -   	&	 -   	\\
			\rowcolor{color1!5!white}\multicolumn{1}{l}{	\footnotesize	 Julio 	}&	 Jutiapa 	&	 25 	&	 10 	&	 2 	&	 16 	&	 63 	&	 25 	&	 8 	&	 20 	&	 -   	&	 -   	&	 -   	\\
			\multicolumn{1}{l}{	\footnotesize	 Julio 	}&	 Quetzaltenango 	&	 267 	&	 81 	&	 66 	&	 206 	&	 346 	&	 290 	&	 76 	&	 83 	&	 -   	&	 -   	&	 -   	\\
			\rowcolor{color1!5!white}\multicolumn{1}{l}{	\footnotesize	 Julio 	}&	 Quiché 	&	 321 	&	 96 	&	 97 	&	 238 	&	 517 	&	 288 	&	 80 	&	 202 	&	 -   	&	 -   	&	 -   	\\
			\multicolumn{1}{l}{	\footnotesize	 Julio 	}&	 Retalhuleu 	&	 43 	&	 20 	&	 10 	&	 35 	&	 93 	&	 44 	&	 17 	&	 34 	&	 -   	&	 -   	&	 -   	\\
			\rowcolor{color1!5!white}\multicolumn{1}{l}{	\footnotesize	 Julio 	}&	 Sacatepéquez 	&	 71 	&	 12 	&	 16 	&	 53 	&	 118 	&	 87 	&	 17 	&	 28 	&	 -   	&	 -   	&	 -   	\\
			\multicolumn{1}{l}{	\footnotesize	 Julio 	}&	 San marcos 	&	 335 	&	 83 	&	 139 	&	 177 	&	 357 	&	 257 	&	 76 	&	 192 	&	 -   	&	 -   	&	 -   	\\
			\rowcolor{color1!5!white}\multicolumn{1}{l}{	\footnotesize	 Julio 	}&	 Santa rosa 	&	 57 	&	 12 	&	 11 	&	 29 	&	 88 	&	 61 	&	 11 	&	 22 	&	 -   	&	 -   	&	 -   	\\
			\multicolumn{1}{l}{	\footnotesize	 Julio 	}&	 Sololá 	&	 184 	&	 74 	&	 33 	&	 42 	&	 102 	&	 172 	&	 76 	&	 47 	&	 -   	&	 -   	&	 -   	\\
			\rowcolor{color1!5!white}\multicolumn{1}{l}{	\footnotesize	 Julio 	}&	 Suchitepéquez 	&	 113 	&	 86 	&	 29 	&	 137 	&	 299 	&	 113 	&	 84 	&	 46 	&	 -   	&	 -   	&	 -   	\\
			\multicolumn{1}{l}{	\footnotesize	 Julio 	}&	 Totonicapán 	&	 106 	&	 23 	&	 30 	&	 72 	&	 122 	&	 109 	&	 23 	&	 35 	&	 -   	&	 -   	&	 -   	\\
			\rowcolor{color1!5!white}\multicolumn{1}{l}{	\footnotesize	 Julio 	}&	 Zacapa 	&	 27 	&	 10 	&	 11 	&	 11 	&	 27 	&	 15 	&	 11 	&	 6 	&	 -   	&	 -   	&	 -   	\\
			\multicolumn{1}{l}{	\footnotesize	 Junio 	}&	 Alta Verapaz 	&	 540 	&	 250 	&	 176 	&	 396 	&	 589 	&	 425 	&	 162 	&	 361 	&	 1 	&	 -   	&	 -   	\\
			\rowcolor{color1!5!white}\multicolumn{1}{l}{	\footnotesize	 Junio 	}&	 Baja Verapaz 	&	 70 	&	 37 	&	 20 	&	 47 	&	 94 	&	 28 	&	 27 	&	 9 	&	 -   	&	 -   	&	 -   	\\
			\multicolumn{1}{l}{	\footnotesize	 Junio 	}&	 Chimaltenango 	&	 186 	&	 93 	&	 22 	&	 78 	&	 173 	&	 153 	&	 74 	&	 155 	&	 -   	&	 -   	&	 -   	\\
			\rowcolor{color1!5!white}\multicolumn{1}{l}{	\footnotesize	 Junio 	}&	 Chiquimula 	&	 154 	&	 24 	&	 67 	&	 225 	&	 322 	&	 173 	&	 22 	&	 26 	&	 -   	&	 -   	&	 -   	\\
			\multicolumn{1}{l}{	\footnotesize	 Junio 	}&	 Petén 	&	 122 	&	 19 	&	 39 	&	 98 	&	 176 	&	 131 	&	 14 	&	 52 	&	 -   	&	 -   	&	 -   	\\
			\rowcolor{color1!5!white}\multicolumn{1}{l}{	\footnotesize	 Junio 	}&	 El Progreso 	&	 22 	&	 6 	&	 13 	&	 11 	&	 37 	&	 26 	&	 4 	&	 10 	&	 -   	&	 -   	&	 -   	\\
			\multicolumn{1}{l}{	\footnotesize	 Junio 	}&	 Escuintla 	&	 245 	&	 89 	&	 36 	&	 84 	&	 186 	&	 254 	&	 82 	&	 67 	&	 -   	&	 -   	&	 -   	\\
			\rowcolor{color1!5!white}\multicolumn{1}{l}{	\footnotesize	 Junio 	}&	 Guatemala 	&	 910 	&	 447 	&	 164 	&	 349 	&	 763 	&	 831 	&	 510 	&	 379 	&	 -   	&	 -   	&	 -   	\\
			\multicolumn{1}{l}{	\footnotesize	 Junio 	}&	 Huehuetenango 	&	 733 	&	 391 	&	 148 	&	 209 	&	 430 	&	 665 	&	 367 	&	 162 	&	 -   	&	 -   	&	 -   	\\
			\rowcolor{color1!5!white}\multicolumn{1}{l}{	\footnotesize	 Junio 	}&	 Izabal 	&	 121 	&	 35 	&	 22 	&	 65 	&	 124 	&	 85 	&	 27 	&	 32 	&	 -   	&	 -   	&	 -   	\\
			\multicolumn{1}{l}{	\footnotesize	 Junio 	}&	 Jalapa 	&	 34 	&	 3 	&	 14 	&	 29 	&	 108 	&	 29 	&	 22 	&	 84 	&	 -   	&	 -   	&	 -   	\\
			\rowcolor{color1!5!white}\multicolumn{1}{l}{	\footnotesize	 Junio 	}&	 Jutiapa 	&	 38 	&	 16 	&	 12 	&	 25 	&	 43 	&	 27 	&	 4 	&	 7 	&	 -   	&	 -   	&	 -   	\\
			\multicolumn{1}{l}{	\footnotesize	 Junio 	}&	 Quetzaltenango 	&	 421 	&	 107 	&	 92 	&	 219 	&	 422 	&	 392 	&	 86 	&	 101 	&	 -   	&	 -   	&	 -   	\\
			\rowcolor{color1!5!white}\multicolumn{1}{l}{	\footnotesize	 Junio 	}&	 Quiché 	&	 537 	&	 187 	&	 126 	&	 315 	&	 574 	&	 416 	&	 138 	&	 285 	&	 -   	&	 -   	&	 -   	\\
			\multicolumn{1}{l}{	\footnotesize	 Junio 	}&	 Retalhuleu 	&	 53 	&	 19 	&	 20 	&	 33 	&	 65 	&	 49 	&	 15 	&	 36 	&	 -   	&	 -   	&	 -   	\\
			\rowcolor{color1!5!white}\multicolumn{1}{l}{	\footnotesize	 Junio 	}&	 Sacatepéquez 	&	 127 	&	 22 	&	 30 	&	 88 	&	 134 	&	 106 	&	 25 	&	 29 	&	 -   	&	 -   	&	 -   	\\
			\multicolumn{1}{l}{	\footnotesize	 Junio 	}&	 San Marcos 	&	 373 	&	 159 	&	 163 	&	 238 	&	 356 	&	 336 	&	 148 	&	 320 	&	 -   	&	 -   	&	 -   	\\
			\rowcolor{color1!5!white}\multicolumn{1}{l}{	\footnotesize	 Junio 	}&	 Santa Rosa 	&	 103 	&	 31 	&	 25 	&	 55 	&	 144 	&	 89 	&	 20 	&	 32 	&	 -   	&	 -   	&	 -   	\\
			\multicolumn{1}{l}{	\footnotesize	 Junio 	}&	 Sololá 	&	 215 	&	 65 	&	 31 	&	 68 	&	 124 	&	 215 	&	 64 	&	 35 	&	 -   	&	 -   	&	 -   	\\
			\rowcolor{color1!5!white}\multicolumn{1}{l}{	\footnotesize	 Junio 	}&	 Suchitepéquez 	&	 102 	&	 56 	&	 41 	&	 81 	&	 197 	&	 91 	&	 46 	&	 34 	&	 -   	&	 -   	&	 -   	\\
			\multicolumn{1}{l}{	\footnotesize	 Junio 	}&	 Totonicapán 	&	 80 	&	 21 	&	 35 	&	 75 	&	 144 	&	 76 	&	 20 	&	 46 	&	 -   	&	 -   	&	 -   	\\
			\rowcolor{color1!5!white}\multicolumn{1}{l}{	\footnotesize	 Junio 	}&	 Zacapa 	&	 39 	&	 20 	&	 12 	&	 17 	&	 41 	&	 22 	&	 12 	&	 5 	&	 -   	&	 -   	&	 -   	\\
			\multicolumn{1}{l}{	\footnotesize	 Septiembre 	}&	 Alta Verapaz 	&	 216 	&	 123 	&	 63 	&	 239 	&	 433 	&	 203 	&	 97 	&	 124 	&	 -   	&	 -   	&	 -   	\\
			\rowcolor{color1!5!white}\multicolumn{1}{l}{	\footnotesize	 Septiembre 	}&	 Baja Verapaz 	&	 19 	&	 9 	&	 1 	&	 6 	&	 15 	&	 18 	&	 10 	&	 9 	&	 -   	&	 -   	&	 -   	\\
			\multicolumn{1}{l}{	\footnotesize	 Septiembre 	}&	 Chimaltenango 	&	 76 	&	 20 	&	 14 	&	 32 	&	 107 	&	 57 	&	 21 	&	 31 	&	 -   	&	 -   	&	 -   	\\
			\rowcolor{color1!5!white}\multicolumn{1}{l}{	\footnotesize	 Septiembre 	}&	 Chiquimula 	&	 92 	&	 18 	&	 20 	&	 95 	&	 190 	&	 103 	&	 21 	&	 30 	&	 -   	&	 -   	&	 -   	\\
			\multicolumn{1}{l}{	\footnotesize	 Septiembre 	}&	 Petén 	&	 70 	&	 4 	&	 31 	&	 64 	&	 135 	&	 67 	&	 6 	&	 38 	&	 -   	&	 -   	&	 -   	\\
			\rowcolor{color1!5!white}\multicolumn{1}{l}{	\footnotesize	 Septiembre 	}&	 El Progreso 	&	 15 	&	 6 	&	 4 	&	 15 	&	 26 	&	 11 	&	 2 	&	 9 	&	 -   	&	 -   	&	 -   	\\
			\multicolumn{1}{l}{	\footnotesize	 Septiembre 	}&	 Escuintla 	&	 116 	&	 41 	&	 26 	&	 46 	&	 145 	&	 121 	&	 39 	&	 37 	&	 -   	&	 -   	&	 -   	\\
			\rowcolor{color1!5!white}\multicolumn{1}{l}{	\footnotesize	 Septiembre 	}&	 Guatemala 	&	 409 	&	 291 	&	 69 	&	 147 	&	 388 	&	 336 	&	 270 	&	 192 	&	 -   	&	 -   	&	 -   	\\
			\multicolumn{1}{l}{	\footnotesize	 Septiembre 	}&	 Huehuetenango 	&	 363 	&	 167 	&	 94 	&	 177 	&	 341 	&	 340 	&	 159 	&	 163 	&	 -   	&	 -   	&	 -   	\\
			\rowcolor{color1!5!white}\multicolumn{1}{l}{	\footnotesize	 Septiembre 	}&	 Izabal 	&	 109 	&	 29 	&	 51 	&	 70 	&	 148 	&	 88 	&	 21 	&	 26 	&	 -   	&	 -   	&	 -   	\\
			\multicolumn{1}{l}{	\footnotesize	 Septiembre 	}&	 Jalapa 	&	 39 	&	 16 	&	 1 	&	 7 	&	 8 	&	 8 	&	 3 	&	 22 	&	 -   	&	 -   	&	 -   	\\
			\rowcolor{color1!5!white}\multicolumn{1}{l}{	\footnotesize	 Septiembre 	}&	 Jutiapa 	&	 31 	&	 7 	&	 5 	&	 22 	&	 64 	&	 28 	&	 10 	&	 6 	&	 -   	&	 -   	&	 -   	\\
			\multicolumn{1}{l}{	\footnotesize	 Septiembre 	}&	 Quetzaltenango 	&	 179 	&	 50 	&	 39 	&	 120 	&	 222 	&	 180 	&	 49 	&	 64 	&	 -   	&	 -   	&	 -   	\\
			\rowcolor{color1!5!white}\multicolumn{1}{l}{	\footnotesize	 Septiembre 	}&	 Quiché 	&	 195 	&	 55 	&	 73 	&	 154 	&	 358 	&	 179 	&	 53 	&	 100 	&	 -   	&	 -   	&	 -   	\\
			\multicolumn{1}{l}{	\footnotesize	 Septiembre 	}&	 Retalhuleu 	&	 25 	&	 11 	&	 17 	&	 16 	&	 45 	&	 20 	&	 11 	&	 27 	&	 -   	&	 -   	&	 -   	\\
			\rowcolor{color1!5!white}\multicolumn{1}{l}{	\footnotesize	 Septiembre 	}&	 Sacatepéquez 	&	 92 	&	 22 	&	 16 	&	 58 	&	 92 	&	 96 	&	 27 	&	 21 	&	 -   	&	 -   	&	 -   	\\
			\multicolumn{1}{l}{	\footnotesize	 Septiembre 	}&	 San Marcos 	&	 147 	&	 61 	&	 75 	&	 136 	&	 242 	&	 144 	&	 51 	&	 150 	&	 -   	&	 -   	&	 -   	\\
			\rowcolor{color1!5!white}\multicolumn{1}{l}{	\footnotesize	 Septiembre 	}&	 Santa Rosa 	&	 54 	&	 17 	&	 5 	&	 27 	&	 91 	&	 57 	&	 17 	&	 22 	&	 -   	&	 -   	&	 -   	\\
			\multicolumn{1}{l}{	\footnotesize	 Septiembre 	}&	 Sololá 	&	 119 	&	 76 	&	 24 	&	 50 	&	 84 	&	 105 	&	 73 	&	 45 	&	 -   	&	 -   	&	 -   	\\
			\rowcolor{color1!5!white}\multicolumn{1}{l}{	\footnotesize	 Septiembre 	}&	 Suchitepéquez 	&	 89 	&	 14 	&	 17 	&	 77 	&	 174 	&	 85 	&	 16 	&	 42 	&	 -   	&	 -   	&	 -   	\\
			\multicolumn{1}{l}{	\footnotesize	 Septiembre 	}&	 Totonicapán 	&	 82 	&	 13 	&	 33 	&	 77 	&	 101 	&	 83 	&	 13 	&	 50 	&	 -   	&	 -   	&	 -   	\\
			\rowcolor{color1!5!white}\multicolumn{1}{l}{	\footnotesize	 Septiembre 	}&	 Zacapa 	&	 18 	&	 26 	&	 8 	&	 7 	&	 17 	&	 16 	&	 24 	&	 6 	&	 -   	&	 -   	&	 -   	\\
			\multicolumn{1}{l}{	\footnotesize	 Octubre 	}&	 Alta Verapaz 	&	 478 	&	 168 	&	 143 	&	 510 	&	 881 	&	 497 	&	 132 	&	 222 	&	 -   	&	 -   	&	 -   	\\
			\rowcolor{color1!5!white}\multicolumn{1}{l}{	\footnotesize	 Octubre 	}&	 Baja Verapaz 	&	 7 	&	 10 	&	 1 	&	 7 	&	 16 	&	 8 	&	 11 	&	 1 	&	 -   	&	 -   	&	 -   	\\
			\multicolumn{1}{l}{	\footnotesize	 Octubre 	}&	 Chimaltenango 	&	 148 	&	 38 	&	 25 	&	 100 	&	 197 	&	 142 	&	 38 	&	 53 	&	 -   	&	 -   	&	 -   	\\
			\rowcolor{color1!5!white}\multicolumn{1}{l}{	\footnotesize	 Octubre 	}&	 Chiquimula 	&	 98 	&	 13 	&	 31 	&	 91 	&	 219 	&	 94 	&	 13 	&	 57 	&	 -   	&	 -   	&	 -   	\\
			\multicolumn{1}{l}{	\footnotesize	 Octubre 	}&	 Petén 	&	 85 	&	 12 	&	 32 	&	 47 	&	 120 	&	 93 	&	 15 	&	 45 	&	 -   	&	 -   	&	 -   	\\
			\rowcolor{color1!5!white}\multicolumn{1}{l}{	\footnotesize	 Octubre 	}&	 El Progreso 	&	 12 	&	 4 	&	 6 	&	 12 	&	 22 	&	 13 	&	 4 	&	 1 	&	 -   	&	 -   	&	 -   	\\
			\multicolumn{1}{l}{	\footnotesize	 Octubre 	}&	 Escuintla 	&	 171 	&	 48 	&	 28 	&	 78 	&	 265 	&	 185 	&	 50 	&	 64 	&	 -   	&	 -   	&	 -   	\\
			\rowcolor{color1!5!white}\multicolumn{1}{l}{	\footnotesize	 Octubre 	}&	 Guatemala 	&	 370 	&	 271 	&	 87 	&	 184 	&	 445 	&	 352 	&	 285 	&	 162 	&	 -   	&	 -   	&	 -   	\\
			\multicolumn{1}{l}{	\footnotesize	 Octubre 	}&	 Huehuetenango 	&	 471 	&	 227 	&	 126 	&	 310 	&	 584 	&	 435 	&	 226 	&	 206 	&	 -   	&	 -   	&	 -   	\\
			\rowcolor{color1!5!white}\multicolumn{1}{l}{	\footnotesize	 Octubre 	}&	 Izabal 	&	 102 	&	 34 	&	 51 	&	 91 	&	 142 	&	 102 	&	 25 	&	 32 	&	 -   	&	 -   	&	 -   	\\
			\multicolumn{1}{l}{	\footnotesize	 Octubre 	}&	 Jalapa 	&	 123 	&	 44 	&	 44 	&	 156 	&	 416 	&	 142 	&	 54 	&	 15 	&	 -   	&	 -   	&	 -   	\\
			\rowcolor{color1!5!white}\multicolumn{1}{l}{	\footnotesize	 Octubre 	}&	 Jutiapa 	&	 58 	&	 15 	&	 21 	&	 55 	&	 130 	&	 50 	&	 9 	&	 24 	&	 -   	&	 -   	&	 -   	\\
			\multicolumn{1}{l}{	\footnotesize	 Octubre 	}&	 Quetzaltenango 	&	 197 	&	 56 	&	 72 	&	 107 	&	 273 	&	 231 	&	 62 	&	 59 	&	 -   	&	 -   	&	 -   	\\
			\rowcolor{color1!5!white}\multicolumn{1}{l}{	\footnotesize	 Octubre 	}&	 Quiché 	&	 215 	&	 71 	&	 61 	&	 157 	&	 370 	&	 244 	&	 83 	&	 114 	&	 -   	&	 -   	&	 -   	\\
			\multicolumn{1}{l}{	\footnotesize	 Octubre 	}&	 Retalhuleu 	&	 63 	&	 19 	&	 14 	&	 21 	&	 72 	&	 61 	&	 19 	&	 37 	&	 -   	&	 -   	&	 -   	\\
			\rowcolor{color1!5!white}\multicolumn{1}{l}{	\footnotesize	 Octubre 	}&	 Sacatepéquez 	&	 102 	&	 45 	&	 24 	&	 46 	&	 107 	&	 113 	&	 48 	&	 36 	&	 -   	&	 -   	&	 -   	\\
			\multicolumn{1}{l}{	\footnotesize	 Octubre 	}&	 San Marcos 	&	 118 	&	 47 	&	 50 	&	 98 	&	 200 	&	 73 	&	 26 	&	 138 	&	 -   	&	 -   	&	 -   	\\
			\rowcolor{color1!5!white}\multicolumn{1}{l}{	\footnotesize	 Octubre 	}&	 Santa Rosa 	&	 94 	&	 26 	&	 16 	&	 46 	&	 133 	&	 120 	&	 30 	&	 43 	&	 -   	&	 -   	&	 -   	\\
			\multicolumn{1}{l}{	\footnotesize	 Octubre 	}&	 Sololá 	&	 139 	&	 45 	&	 18 	&	 43 	&	 84 	&	 138 	&	 40 	&	 37 	&	 -   	&	 -   	&	 -   	\\
			\rowcolor{color1!5!white}\multicolumn{1}{l}{	\footnotesize	 Octubre 	}&	 Suchitepéquez 	&	 129 	&	 43 	&	 27 	&	 63 	&	 148 	&	 131 	&	 42 	&	 43 	&	 -   	&	 -   	&	 -   	\\
			\multicolumn{1}{l}{	\footnotesize	 Octubre 	}&	 Totonicapán 	&	 118 	&	 29 	&	 44 	&	 106 	&	 172 	&	 112 	&	 26 	&	 64 	&	 -   	&	 -   	&	 -   	\\
			\rowcolor{color1!5!white}\multicolumn{1}{l}{	\footnotesize	 Octubre 	}&	 Zacapa 	&	 15 	&	 19 	&	 4 	&	 7 	&	 18 	&	 15 	&	 17 	&	 5 	&	 -   	&	 -   	&	 -   	\\
			\multicolumn{1}{l}{	\footnotesize	 Noviembre 	}&	 Alta Verapaz 	&	 247 	&	 99 	&	 87 	&	 199 	&	 422 	&	 257 	&	 129 	&	 93 	&	 -   	&	 -   	&	 -   	\\
			\rowcolor{color1!5!white}\multicolumn{1}{l}{	\footnotesize	 Noviembre 	}&	 Baja Verapaz 	&	 48 	&	 10 	&	 3 	&	 23 	&	 50 	&	 44 	&	 11 	&	 7 	&	 -   	&	 -   	&	 -   	\\
			\multicolumn{1}{l}{	\footnotesize	 Noviembre 	}&	 Chimaltenango 	&	 100 	&	 26 	&	 20 	&	 60 	&	 150 	&	 106 	&	 23 	&	 40 	&	 -   	&	 -   	&	 -   	\\
			\rowcolor{color1!5!white}\multicolumn{1}{l}{	\footnotesize	 Noviembre 	}&	 Chiquimula 	&	 81 	&	 7 	&	 12 	&	 33 	&	 111 	&	 81 	&	 7 	&	 33 	&	 -   	&	 -   	&	 -   	\\
			\multicolumn{1}{l}{	\footnotesize	 Noviembre 	}&	 Petén 	&	 69 	&	 12 	&	 14 	&	 30 	&	 96 	&	 71 	&	 10 	&	 38 	&	 -   	&	 -   	&	 -   	\\
			\rowcolor{color1!5!white}\multicolumn{1}{l}{	\footnotesize	 Noviembre 	}&	 El Progreso 	&	 11 	&	 3 	&	 3 	&	 6 	&	 16 	&	 12 	&	 2 	&	 5 	&	 -   	&	 -   	&	 -   	\\
			\multicolumn{1}{l}{	\footnotesize	 Noviembre 	}&	 Escuintla 	&	 264 	&	 87 	&	 46 	&	 106 	&	 281 	&	 274 	&	 87 	&	 81 	&	 -   	&	 -   	&	 -   	\\
			\rowcolor{color1!5!white}\multicolumn{1}{l}{	\footnotesize	 Noviembre 	}&	 Guatemala 	&	 509 	&	 334 	&	 128 	&	 223 	&	 481 	&	 524 	&	 331 	&	 252 	&	 -   	&	 -   	&	 -   	\\
			\multicolumn{1}{l}{	\footnotesize	 Noviembre 	}&	 Huehuetenango 	&	 272 	&	 140 	&	 96 	&	 175 	&	 263 	&	 265 	&	 128 	&	 147 	&	 -   	&	 -   	&	 -   	\\
			\rowcolor{color1!5!white}\multicolumn{1}{l}{	\footnotesize	 Noviembre 	}&	 Izabal 	&	 93 	&	 2 	&	 13 	&	 16 	&	 42 	&	 80 	&	 2 	&	 12 	&	 -   	&	 -   	&	 -   	\\
			\multicolumn{1}{l}{	\footnotesize	 Noviembre 	}&	 Jalapa 	&	 106 	&	 40 	&	 12 	&	 58 	&	 184 	&	 116 	&	 42 	&	 14 	&	 -   	&	 -   	&	 -   	\\
			\rowcolor{color1!5!white}\multicolumn{1}{l}{	\footnotesize	 Noviembre 	}&	 Jutiapa 	&	 88 	&	 39 	&	 21 	&	 93 	&	 202 	&	 101 	&	 29 	&	 64 	&	 -   	&	 -   	&	 -   	\\
			\multicolumn{1}{l}{	\footnotesize	 Noviembre 	}&	 Quetzaltenango 	&	 172 	&	 53 	&	 59 	&	 91 	&	 212 	&	 192 	&	 55 	&	 61 	&	 -   	&	 -   	&	 -   	\\
			\rowcolor{color1!5!white}\multicolumn{1}{l}{	\footnotesize	 Noviembre 	}&	 Quiché 	&	 138 	&	 46 	&	 43 	&	 93 	&	 183 	&	 152 	&	 39 	&	 69 	&	 -   	&	 -   	&	 -   	\\
			\multicolumn{1}{l}{	\footnotesize	 Noviembre 	}&	 Retalhuleu 	&	 43 	&	 14 	&	 20 	&	 22 	&	 61 	&	 39 	&	 18 	&	 12 	&	 -   	&	 -   	&	 -   	\\
			\rowcolor{color1!5!white}\multicolumn{1}{l}{	\footnotesize	 Noviembre 	}&	 Sacatepéquez 	&	 78 	&	 22 	&	 16 	&	 42 	&	 71 	&	 91 	&	 19 	&	 43 	&	 -   	&	 -   	&	 -   	\\
			\multicolumn{1}{l}{	\footnotesize	 Noviembre 	}&	 San Marcos 	&	 79 	&	 41 	&	 31 	&	 32 	&	 60 	&	 42 	&	 23 	&	 63 	&	 -   	&	 -   	&	 -   	\\
			\rowcolor{color1!5!white}\multicolumn{1}{l}{	\footnotesize	 Noviembre 	}&	 Santa Rosa 	&	 56 	&	 15 	&	 14 	&	 32 	&	 94 	&	 55 	&	 16 	&	 31 	&	 -   	&	 -   	&	 -   	\\
			\multicolumn{1}{l}{	\footnotesize	 Noviembre 	}&	 Sololá 	&	 124 	&	 46 	&	 22 	&	 43 	&	 125 	&	 122 	&	 40 	&	 46 	&	 -   	&	 -   	&	 -   	\\
			\rowcolor{color1!5!white}\multicolumn{1}{l}{	\footnotesize	 Noviembre 	}&	 Suchitepéquez 	&	 118 	&	 52 	&	 24 	&	 65 	&	 131 	&	 119 	&	 55 	&	 57 	&	 -   	&	 -   	&	 -   	\\
			\multicolumn{1}{l}{	\footnotesize	 Noviembre 	}&	 Totonicapán 	&	 116 	&	 35 	&	 40 	&	 92 	&	 159 	&	 118 	&	 35 	&	 76 	&	 -   	&	 -   	&	 -   	\\
			\rowcolor{color1!5!white}\multicolumn{1}{l}{	\footnotesize	 Noviembre 	}&	 Zacapa 	&	 22 	&	 14 	&	 2 	&	 3 	&	 14 	&	 20 	&	 14 	&	 4 	&	 -   	&	 -   	&	 -   	\\
			[-0.28cm]
		\end{longtable}\addtocounter{Cuadro}{1}
	\end{center}




%%%%%%%%%%%%%%%%%%


\hoja{
	{\Bold\Large 5.3 Acceso a servicios básicos}\\
	
	\normalsize
	{\Bold\color{color1!80!black}{Cuadro \theCuadro $\,-$    Hogares conectados a red de distribución de agua y drenajes; }}\\
	{\Bold\color{color1!80!black}{según características varias. }}\\
	{\Bold\color{color1!80!black}{República de Guatemala, año 2014.}}\\
	{\color{color1!80!black}{(Porcentajes)}}\\
	\begin{center}\fontsize{4mm}{1.6em}\selectfont \setlength{\arrayrulewidth}{0.7pt}
		\begin{tabular}{x{5cm}x{4cm}x{4cm}}
			%				\multicolumn{3}{l}{\Bold\color{color1!80!black}{Cuadro \theCuadro $\,-$    Ajonjolí (Sesamum indicum), por área cosechada, producción y rendimiento}}\\
			%				\multicolumn{3}{l}{\Bold\color{color1!80!black}{según año agrícola. República de Guatemala, años varios.}}\\
			%		\multicolumn{4}{l}{(Población de 15 o más años de edad)}
			%[0.4cm]
			\hline &&\\[-0.36cm]  
			\multicolumn{1}{x{2.7cm}}{\textbf{Característica}} &	\multicolumn{1}{c}{\Bold{Agua}}&\multicolumn{1}{c}{\Bold{Drenajes}}\\[0.05cm]\cline{2-3}
			\rowcolor{color1!40!white} \multicolumn{1}{l}{\Bold{	 Total}}&\textbf{78.1 }	 &\textbf{45.2}\\ 
			\rowcolor{color1!20!white} \multicolumn{1}{l}{\Bold{	 Área}}& 		 & 		 \\ 
			\multicolumn{1}{l}{	 Urbana}&89.8&73.4\\ 
			\rowcolor{color1!5!white}\multicolumn{1}{l}{	 Rural}&64.2&11.6\\ 
			\rowcolor{color1!20!white} \multicolumn{1}{l}{\Bold{	 Pobreza}}& 		 & 		 \\ 
			\multicolumn{1}{l}{	 Pobreza extrema}&60.8&12.1\\ 
			\rowcolor{color1!5!white}\multicolumn{1}{l}{	 Pobreza no extrema}&72.6&31.4\\ 
			\multicolumn{1}{l}{	 No pobreza}&87.0&64.3\\ 
			\rowcolor{color1!20!white} \multicolumn{1}{l}{\Bold{	 Departamento}}& 		 & 		 \\ 
			\multicolumn{1}{l}{	 Guatemala}&91.3&78.0\\ 
			\rowcolor{color1!5!white}\multicolumn{1}{l}{	 El Progreso}&82.3&39.1\\ 
			\multicolumn{1}{l}{	 Sacatepéquez}&91.9&81.9\\ 
			\rowcolor{color1!5!white}\multicolumn{1}{l}{	 Chimaltenango}&78.2&52.8\\ 
			\multicolumn{1}{l}{	 Escuintla}&62.6&51.0\\ 
			\rowcolor{color1!5!white}\multicolumn{1}{l}{	 Santa Rosa}&64.2&32.1\\ 
			\multicolumn{1}{l}{	 Sololá}&94.9&21.8\\ 
			\rowcolor{color1!5!white}\multicolumn{1}{l}{	 Totonicapán}&84.7&24.9\\ 
			\multicolumn{1}{l}{	 Quetzaltenango}&81.2&56.1\\ 
			\rowcolor{color1!5!white}\multicolumn{1}{l}{	 Suchitepéquez}&70.2&50.4\\ 
			\multicolumn{1}{l}{	 Retalhuleu}&58.2&26.6\\ 
			\rowcolor{color1!5!white}\multicolumn{1}{l}{	 San Marcos}&76.1&29.1\\ 
			\multicolumn{1}{l}{	 Huehuetenango}&77.2&33.6\\ 
			\rowcolor{color1!5!white}\multicolumn{1}{l}{	 Quiché}&83.3&32.7\\ 
			\multicolumn{1}{l}{	 Baja Verapaz}&82.4&30.7\\ 
			\rowcolor{color1!5!white}\multicolumn{1}{l}{	 Alta Verapaz}&45.6&12.5\\ 
			\multicolumn{1}{l}{	 Petén}&66.9&7.4\\ 
			\rowcolor{color1!5!white}\multicolumn{1}{l}{	 Izabal}&72.2&27.6\\ 
			\multicolumn{1}{l}{	 Zacapa}&87.0&46.3\\ 
			\rowcolor{color1!5!white}\multicolumn{1}{l}{	 Chiquimula}&71.4&32.6\\ 
			\multicolumn{1}{l}{	 Jalapa}&78.6&32.2\\ 
			\rowcolor{color1!5!white}\multicolumn{1}{l}{	 Jutiapa}&79.1&32.8\\ 
			[0.05cm]
			\hline
			&&\\[-0.36cm]
			\multicolumn{3}{l}{\footnotesize Fuente:  Encuesta Nacional de Condiciones de Vida (Encovi), 2014.}\\
		\end{tabular}\addtocounter{Cuadro}{1}
	\end{center}}



%%%%%%%%%%%%%%%%%%


\hoja{
	
	{\Bold\Large 5.4	Educación sanitaria}\\[-.5cm]
	
	No hay información disponible.\\
	
	{\Bold\Large 5.5	Vivienda}
	
	\normalsize
	{\Bold\color{color1!80!black}{Cuadro \theCuadro $\,-$  Hogares por tipo de material predominante en las paredes, techo y piso de la vivienda del hogar;según características varias.}}\\
	{\Bold\color{color1!80!black}{República de Guatemala, año 2014.}}\\
	{\color{color1!80!black}{(Porcentajes)}}\\[-.5cm]
	\begin{center}\fontsize{3.5mm}{1.5em}\selectfont \setlength{\arrayrulewidth}{0.7pt}
		\begin{tabular}{x{3cm}rrrrr}
			\hline &&&&&\\[-0.46cm]  
			\multicolumn{1}{x{2.7cm}}{\textbf{Característica}} &	\multicolumn{1}{c}{\Bold{Casa Formal}}&\multicolumn{1}{c}{\Bold{Viviendas con}}&\multicolumn{1}{c}{\Bold{Viviendas con}}&\multicolumn{2}{c}{\Bold{Material del piso}}\\[0.05cm]\cline{5-6}
			\multicolumn{1}{x{2.7cm}}{ } &	\multicolumn{1}{c}{ }&\multicolumn{1}{c}{\Bold{pared de block}}&\multicolumn{1}{c}{\Bold{techo de lámina}}&\multicolumn{1}{c}{\Bold{Torta de cemento}}&\multicolumn{1}{c}{\Bold{Tierra}}\\[0.05cm]	\rowcolor{color1!40!white} \multicolumn{1}{l}{\Bold{	Total	 }}& 	 88.6 	 & 	 58.2 	 & 	 72.3 	 & 	 39.7 	 & 	 27.5 	 \\ 
			\rowcolor{color1!20!white} \multicolumn{1}{l}{\Bold{	 Área 	 }}& 		 & 		 & 		 & 		 & 		 \\ 
			\multicolumn{1}{l}{	 Urbana 	 }& 	 88.7 	 & 	 72.3 	 & 	 63.4 	 & 	 40.0 	 & 	 14.0 	 \\ 
			\rowcolor{color1!5!white}\multicolumn{1}{l}{	 Rural 	 }& 	 88.5 	 & 	 41.4 	 & 	 82.8 	 & 	 39.5 	 & 	 43.6 	 \\ 
			\rowcolor{color1!20!white} \multicolumn{1}{l}{\Bold{	 Situación de pobreza 	 }}& 		 & 		 & 		 & 		 & 		 \\ 
			\multicolumn{1}{l}{	 Pobreza extrema 	 }& 	 81.1 	 & 	 22.0 	 & 	 84.3 	 & 	 25.3 	 & 	 67.0 	 \\ 
			\rowcolor{color1!5!white}\multicolumn{1}{l}{	 Pobreza no extrema 	 }& 	 87.9 	 & 	 48.5 	 & 	 82.3 	 & 	 46.3 	 & 	 35.0 	 \\ 
			\multicolumn{1}{l}{	 No pobreza 	 }& 	 91.4 	 & 	 75.7 	 & 	 62.2 	 & 	 40.2 	 & 	 10.4 	 \\ 
			\rowcolor{color1!20!white} \multicolumn{1}{l}{\Bold{	 Departamento 	 }}& 		 & 		 & 		 & 		 & 		 \\ 
			\multicolumn{1}{l}{	 Guatemala 	 }& 	 81.3 	 & 	 75.7 	 & 	 57.1 	 & 	 36.5 	 & 	 10.0 	 \\ 
			\rowcolor{color1!5!white}\multicolumn{1}{l}{	 El Progreso 	 }& 	 95.5 	 & 	 72.2 	 & 	 78.2 	 & 	 48.4 	 & 	 15.5 	 \\ 
			\multicolumn{1}{l}{	 Sacatepéquez 	 }& 	 91.8 	 & 	 77.9 	 & 	 74.8 	 & 	 48.8 	 & 	 17.9 	 \\ 
			\rowcolor{color1!5!white}\multicolumn{1}{l}{	 Chimaltenango 	 }& 	 94.4 	 & 	 75.9 	 & 	 81.1 	 & 	 49.4 	 & 	 24.3 	 \\ 
			\multicolumn{1}{l}{	 Escuintla 	 }& 	 86.0 	 & 	 78.2 	 & 	 90.0 	 & 	 64.8 	 & 	 15.1 	 \\ 
			\rowcolor{color1!5!white}\multicolumn{1}{l}{	 Santa Rosa 	 }& 	 82.0 	 & 	 61.6 	 & 	 87.3 	 & 	 41.7 	 & 	 30.9 	 \\ 
			\multicolumn{1}{l}{	 Sololá 	 }& 	 97.9 	 & 	 60.2 	 & 	 78.1 	 & 	 52.9 	 & 	 20.7 	 \\ 
			\rowcolor{color1!5!white}\multicolumn{1}{l}{	 Totonicapán 	 }& 	 99.6 	 & 	 39.7 	 & 	 57.6 	 & 	 29.8 	 & 	 44.6 	 \\ 
			\multicolumn{1}{l}{	 Quetzaltenango 	 }& 	 96.1 	 & 	 76.6 	 & 	 58.3 	 & 	 42.0 	 & 	 12.0 	 \\ 
			\rowcolor{color1!5!white}\multicolumn{1}{l}{	 Suchitepéquez 	 }& 	 76.0 	 & 	 61.1 	 & 	 88.6 	 & 	 51.8 	 & 	 28.4 	 \\ 
			\multicolumn{1}{l}{	 Retalhuleu 	 }& 	 78.9 	 & 	 68.3 	 & 	 87.6 	 & 	 44.3 	 & 	 33.9 	 \\ 
			\rowcolor{color1!5!white}\multicolumn{1}{l}{	 San Marcos 	 }& 	 94.8 	 & 	 53.2 	 & 	 77.4 	 & 	 48.9 	 & 	 29.6 	 \\ 
			\multicolumn{1}{l}{	 Huehuetenango 	 }& 	 95.7 	 & 	 37.1 	 & 	 70.2 	 & 	 31.6 	 & 	 40.7 	 \\ 
			\rowcolor{color1!5!white}\multicolumn{1}{l}{	 Quiché 	 }& 	 96.1 	 & 	 31.7 	 & 	 54.7 	 & 	 26.2 	 & 	 44.4 	 \\ 
			\multicolumn{1}{l}{	 Baja Verapaz 	 }& 	 93.4 	 & 	 34.6 	 & 	 70.7 	 & 	 46.9 	 & 	 39.8 	 \\ 
			\rowcolor{color1!5!white}\multicolumn{1}{l}{	 Alta Verapaz 	 }& 	 81.6 	 & 	 35.4 	 & 	 92.3 	 & 	 29.1 	 & 	 62.4 	 \\ 
			\multicolumn{1}{l}{	 Petén 	 }& 	 88.9 	 & 	 44.6 	 & 	 87.5 	 & 	 40.4 	 & 	 33.8 	 \\ 
			\rowcolor{color1!5!white}\multicolumn{1}{l}{	 Izabal 	 }& 	 84.1 	 & 	 56.2 	 & 	 84.5 	 & 	 41.0 	 & 	 31.3 	 \\ 
			\multicolumn{1}{l}{	 Zacapa 	 }& 	 93.0 	 & 	 54.0 	 & 	 86.2 	 & 	 64.0 	 & 	 16.0 	 \\ 
			\rowcolor{color1!5!white}\multicolumn{1}{l}{	 Chiquimula 	 }& 	 85.5 	 & 	 36.8 	 & 	 76.3 	 & 	 25.9 	 & 	 42.7 	 \\ 
			\multicolumn{1}{l}{	 Jalapa 	 }& 	 96.5 	 & 	 26.2 	 & 	 75.7 	 & 	 18.9 	 & 	 51.8 	 \\ 
			\rowcolor{color1!5!white}\multicolumn{1}{l}{	 Jutiapa 	 }& 	 95.6 	 & 	 47.2 	 & 	 78.3 	 & 	 28.3 	 & 	 30.7 	 \\ 
			[0.05cm]
			\hline
			&&&&&\\[-0.36cm]
			\multicolumn{6}{l}{\footnotesize Fuente:  Encuesta Nacional de Condiciones de Vida (Encovi), 2014.}\\
		\end{tabular}\addtocounter{Cuadro}{1}
	\end{center}}
	

		\INEchaptercarta{Situación y atención a la desnutrición/malnutrición}{}

\addtocounter{Cuadro}{1}
\hoja{{\Bold\color{color1!80!black}{\Large SITUACIÓN NUTRICIONAL DE LA MADRE}}
	
	\normalsize
	{\Bold\color{color1!80!black}{Cuadro \theCuadro $\,-$  Mujeres en edad fértil y menores de 5 años con anemia; según características varias.}}\\
%	{\Bold\color{color1!80!black}{según características varias. }}\\
	{\Bold\color{color1!80!black}{República de Guatemala, año 2008/2009.}}\\
	{\color{color1!80!black}{(Porcentajes)}}\\
	\begin{center}\fontsize{3mm}{1.4em}\selectfont \setlength{\arrayrulewidth}{0.7pt}
		\begin{tabular}{x{5cm}ccc}
			\hline &&&\\[-0.36cm]  
			\multicolumn{1}{x{2.7cm}}{\textbf{Característica}} &	\multicolumn{2}{c}{\Bold{Mujeres en edad fértil}}&\multicolumn{1}{c}{\multirow{2}{*}[1mm]{\Bold{Menores de 5 años}}}\\[0.05cm]\cline{2-3}
			\multicolumn{1}{x{2.7cm}}{ } &	\multicolumn{1}{c}{\textbf{ No embarazadas}}&\multicolumn{1}{c}{\Bold{Embarazadas}}& \multicolumn{1}{c}{ }\\[0.05cm]
\rowcolor{color1!40!white} \multicolumn{1}{l}{\Bold{	Total República	 }}& 	 21.4 	 & 	 29.1 	 & 	 47.7 	 \\ 
\rowcolor{color1!20!white} \multicolumn{1}{l}{\Bold{	Área geográfica	 }}& 		 & 		 & 		 \\ 
\multicolumn{1}{l}{	Urbana	 }& 	 19.1 	 & 	 27.5 	 & 	 46.2 	 \\ 
\rowcolor{color1!5!white}\multicolumn{1}{l}{	Rural	 }& 	 23.1 	 & 	 30.0 	 & 	 48.6 	 \\ 
\rowcolor{color1!20!white} \multicolumn{1}{l}{\Bold{	Departamento	 }}& 		 & 		 & 		 \\ 
\multicolumn{1}{l}{	Guatemala	 }& 	 16.6 	 & 	 30.1 	 & 	 40.7 	 \\ 
\rowcolor{color1!5!white}\multicolumn{1}{l}{	El Progreso	 }& 	 20.8 	 & 	 19.4 	 & 	 37.8 	 \\ 
\multicolumn{1}{l}{	Sacatepéquez	 }& 	 19.9 	 & 	 24.9 	 & 	 54.2 	 \\ 
\rowcolor{color1!5!white}\multicolumn{1}{l}{	Chimaltenango	 }& 	 20.5 	 & 	 24.7 	 & 	 53.5 	 \\ 
\multicolumn{1}{l}{	Escuintla	 }& 	 22.1 	 & 	 27.4 	 & 	 50.2 	 \\ 
\rowcolor{color1!5!white}\multicolumn{1}{l}{	Santa Rosa	 }& 	 12.9 	 & 	 22.3 	 & 	 51.4 	 \\ 
\multicolumn{1}{l}{	Sololá	 }& 	 22.6 	 & 	 19.8 	 & 	 56.1 	 \\ 
\rowcolor{color1!5!white}\multicolumn{1}{l}{	Totonicapán	 }& 	 32.3 	 & 	 36.3 	 & 	 62.2 	 \\ 
\multicolumn{1}{l}{	Quetzaltenango	 }& 	 20.6 	 & 	 35.3 	 & 	 40.2 	 \\ 
\rowcolor{color1!5!white}\multicolumn{1}{l}{	Suchitepéquez	 }& 	 21.1 	 & 	 31.5 	 & 	 37.7 	 \\ 
\multicolumn{1}{l}{	Retalhuleu	 }& 	 23.7 	 & 	 44.8 	 & 	 45.3 	 \\ 
\rowcolor{color1!5!white}\multicolumn{1}{l}{	San Marcos	 }& 	 29.0 	 & 	 34.0 	 & 	 52.6 	 \\ 
\multicolumn{1}{l}{	Huehuetenango	 }& 	 21.1 	 & 	 17.9 	 & 	 47.7 	 \\ 
\rowcolor{color1!5!white}\multicolumn{1}{l}{	Quiché	 }& 	 24.8 	 & 	 30.3 	 & 	 47.4 	 \\ 
\multicolumn{1}{l}{	Baja Verapaz	 }& 	 19.3 	 & 	 31.9 	 & 	 49.8 	 \\ 
\rowcolor{color1!5!white}\multicolumn{1}{l}{	Alta Verapaz	 }& 	 22.2 	 & 	 33.3 	 & 	 46.1 	 \\ 
\multicolumn{1}{l}{	Petén	 }& 	 21.3 	 & 	 34.1 	 & 	 48.5 	 \\ 
\rowcolor{color1!5!white}\multicolumn{1}{l}{	Izabal	 }& 	 35.3 	 & 	 36.9 	 & 	 53.0 	 \\ 
\multicolumn{1}{l}{	Zacapa	 }& 	 20.8 	 & 	 19.4 	 & 	 53.7 	 \\ 
\rowcolor{color1!5!white}\multicolumn{1}{l}{	Chiquimula	 }& 	 27.3 	 & 	 41.3 	 & 	 55.5 	 \\ 
\multicolumn{1}{l}{	Jalapa	 }& 	 15.0 	 & 	 7.3 	 & 	 43.9 	 \\ 
\rowcolor{color1!5!white}\multicolumn{1}{l}{	Jutiapa	 }& 	 13.3 	 & 	 21.3 	 & 	 50.3 	 \\ 
\rowcolor{color1!20!white} \multicolumn{1}{l}{\Bold{	Categoría étnica	 }}& 		 & 		 & 		 \\ 
\multicolumn{1}{l}{	Indígena	 }& 	 24.9 	 & 	 32.2 	 & 	 49.5 	 \\ 
\rowcolor{color1!5!white}\multicolumn{1}{l}{	Ladino	 }& 	 19.0 	 & 	 26.6 	 & 	 46.3 	 \\ 
\rowcolor{color1!20!white} \multicolumn{1}{l}{\Bold{	Nivel de educación	 }}& 		 & 		 & 		 \\ 
\multicolumn{1}{l}{	Sin educación	 }& 	 27.8 	 & 	 33.0 	 & 	 48.3 	 \\ 
\rowcolor{color1!5!white}\multicolumn{1}{l}{	Primaria	 }& 	 20.8 	&	 28.8 	&	 49.2 	 \\ 
\multicolumn{1}{l}{	Secundaria	 }& 	 16.3 	&	 26.2 	&	 44.0 	 \\ 
\rowcolor{color1!5!white}\multicolumn{1}{l}{	Superior	 }& 	 15.6 	&	 21.4 	&	 36.0 	 \\ 
[0.05cm]
			\hline
			&&&\\[-0.36cm]
			\multicolumn{4}{l}{\footnotesize Fuente:  Encuesta Nacional de Condiciones de Vida (Encovi), 2014.}\\
		\end{tabular}\addtocounter{Cuadro}{1}
 	\end{center}}




%%%%5555555555555555%%%%%%%%%%%%%5


%%%%%%%%%%%%%%%%%%


\hoja{
	\normalsize
	{\Bold\color{color1!80!black}{Cuadro \theCuadro $\,-$  Distribución de las mujeres no embarazadas, por valor del Índice de Masa Corporal; }}\\
	{\Bold\color{color1!80!black}{según características varias. }}\\
	{\Bold\color{color1!80!black}{República de Guatemala, año 2008/2009.}}\\
	{\color{color1!80!black}{(Porcentajes)}}\\[-1cm]
	\begin{center}\fontsize{3.5mm}{1.4em}\selectfont \setlength{\arrayrulewidth}{0.7pt}
		\begin{tabular}{x{5cm}ccccc}
			\hline &&&&\\[-0.5cm]  
			\multicolumn{1}{x{2.7cm}}{\raisebox{-.5cm}{\textbf{Característica}}} &	\multicolumn{4}{c}{\raisebox{-.15cm}{\Bold{Índice de masa corporal (IMC)}}}\\[0.05cm]\cline{2-5}
			\multicolumn{1}{x{2.7cm}}{ } &	\multicolumn{1}{x{2.7cm}}{\textbf{Bajo}}&\multicolumn{1}{x{2.7cm}}{\Bold{Normal}}& \multicolumn{1}{x{2.7cm}}{\textbf{Sobre Peso}} &\multicolumn{1}{c}{\textbf{Obesidad}}\\[0.05cm]
			\rowcolor{color1!40!white} \multicolumn{1}{l}{\Bold{	Total	 }}& 	1.6	 & 	47.9	 & 	35.1	 & 	15.4	 \\ 
			\rowcolor{color1!20!white} \multicolumn{1}{l}{\Bold{	Área geográfica	 }}& 		 & 		 & 		 & 		 \\ 
			\multicolumn{1}{l}{	Urbana	 }& 	 1.8 	 & 	 40.5 	 & 	 37.5 	 & 	 20.3 	 \\ 
			\rowcolor{color1!5!white}\multicolumn{1}{l}{	Rural	 }& 	 1.5 	 & 	 53.0 	 & 	 33.4 	 & 	 12.1 	 \\ 
			\rowcolor{color1!20!white} \multicolumn{1}{l}{\Bold{	Departamentos	 }}& 		 & 		 & 		 & 		 \\ 
			\multicolumn{1}{l}{	Guatemala	 }& 	 2.5 	 & 	 38.7 	 & 	 37.3 	 & 	 21.5 	 \\ 
			\rowcolor{color1!5!white}\multicolumn{1}{l}{	El Progreso	 }& 	 1.2 	 & 	 48.7 	 & 	 31.9 	 & 	 18.3 	 \\ 
			\multicolumn{1}{l}{	Sacatepéquez	 }& 	 -   	 & 	 39.2 	 & 	 38.9 	 & 	 21.9 	 \\ 
			\rowcolor{color1!5!white}\multicolumn{1}{l}{	Chimaltenango	 }& 	 1.0 	 & 	 41.1 	 & 	 43.7 	 & 	 14.2 	 \\ 
			\multicolumn{1}{l}{	Escuintla	 }& 	 4.5 	 & 	 38.6 	 & 	 32.4 	 & 	 24.5 	 \\ 
			\rowcolor{color1!5!white}\multicolumn{1}{l}{	Santa Rosa	 }& 	 2.0 	 & 	 44.7 	 & 	 33.2 	 & 	 20.1 	 \\ 
			\multicolumn{1}{l}{	Sololá	 }& 	 1.6 	 & 	 53.3 	 & 	 32.5 	 & 	 12.6 	 \\ 
			\rowcolor{color1!5!white}\multicolumn{1}{l}{	Totonicapán	 }& 	 0.6 	 & 	 51.8 	 & 	 36.5 	 & 	 11.1 	 \\ 
			\multicolumn{1}{l}{	Quetzaltenango	 }& 	 0.8 	 & 	 45.3 	 & 	 39.6 	 & 	 14.3 	 \\ 
			\rowcolor{color1!5!white}\multicolumn{1}{l}{	Suchitepéquez	 }& 	 2.0 	 & 	 39.8 	 & 	 39.0 	 & 	 19.2 	 \\ 
			\multicolumn{1}{l}{	Retalhuleu	 }& 	 1.5 	 & 	 42.5 	 & 	 35.2 	 & 	 20.8 	 \\ 
			\rowcolor{color1!5!white}\multicolumn{1}{l}{	San Marcos	 }& 	 0.8 	 & 	 53.9 	 & 	 34.7 	 & 	 10.7 	 \\ 
			\multicolumn{1}{l}{	Huehuetenango	 }& 	 0.7 	 & 	 62.2 	 & 	 31.3 	 & 	 5.8 	 \\ 
			\rowcolor{color1!5!white}\multicolumn{1}{l}{	Quiché	 }& 	 1.0 	 & 	 54.3 	 & 	 36.7 	 & 	 8.0 	 \\ 
			\multicolumn{1}{l}{	Baja Verapaz	 }& 	 1.6 	 & 	 60.4 	 & 	 28.8 	 & 	 9.1 	 \\ 
			\rowcolor{color1!5!white}\multicolumn{1}{l}{	Alta Verapaz	 }& 	 0.4 	 & 	 51.6 	 & 	 36.3 	 & 	 11.7 	 \\ 
			\multicolumn{1}{l}{	Petén	 }& 	 1.6 	 & 	 43.4 	 & 	 30.6 	 & 	 24.4 	 \\ 
			\rowcolor{color1!5!white}\multicolumn{1}{l}{	Izabal	 }& 	 1.3 	 & 	 42.5 	 & 	 39.0 	 & 	 17.2 	 \\ 
			\multicolumn{1}{l}{	Zacapa	 }& 	 2.0 	 & 	 49.5 	 & 	 31.1 	 & 	 17.4 	 \\ 
			\rowcolor{color1!5!white}\multicolumn{1}{l}{	Chiquimula	 }& 	 0.7 	 & 	 61.7 	 & 	 27.3 	 & 	 10.4 	 \\ 
			\multicolumn{1}{l}{	Jalapa	 }& 	 2.0 	 & 	 56.5 	 & 	 28.8 	 & 	 12.7 	 \\ 
			\rowcolor{color1!5!white}\multicolumn{1}{l}{	Jutiapa	 }& 	 2.4 	 & 	 50.3 	 & 	 32.1 	 & 	 15.2 	 \\ 
			\rowcolor{color1!20!white} \multicolumn{1}{l}{\Bold{	Categoría étnica	 }}& 		 & 		 & 		 & 		 \\ 
			\multicolumn{1}{l}{	Indígena	 }& 	 0.7 	 & 	 52.5 	 & 	 35.3 	 & 	 11.5 	 \\ 
			\rowcolor{color1!5!white}\multicolumn{1}{l}{	Ladino	 }& 	 2.2 	 & 	 44.6 	 & 	 34.9 	 & 	 18.3 	 \\ 
			\rowcolor{color1!20!white} \multicolumn{1}{l}{\Bold{	Nivel de educación	 }}& 		 & 		 & 		 & 		 \\ 
			\multicolumn{1}{l}{	Sin educación	 }& 	 0.8 	 & 	 53.5 	 & 	 33.8 	 & 	 12.0 	 \\ 
			\rowcolor{color1!5!white}\multicolumn{1}{l}{	Primaria	 }& 	 1.6 	 & 	 47.2 	 & 	 35.2 	 & 	 16.0 	 \\ 
			\multicolumn{1}{l}{	Secundaria	 }& 	 2.5 	 & 	 44.3 	 & 	 35.6 	 & 	 17.6 	 \\ 
			\rowcolor{color1!5!white}\multicolumn{1}{l}{	Superior	 }& 	 1.5 	 & 	 38.7 	 & 	 39.8 	 & 	 19.9 	 \\ 
			[0.05cm]
			\hline
			&&&&\\[-0.36cm]
			\multicolumn{5}{l}{\footnotesize Fuente:  INE. Encuesta Nacional de Salud Materno Infantil (Ensmi) 2008/2009.}\\[-.09cm]
			\multicolumn{5}{l}{\parbox{13cm}{\footnotesize \textbf{Nota:} Peso bajo es el menor a 18.4 onzas; peso normal es entre 18.5 y 24.9; el sobre peso al nacer es el mayor a 25.0 y menor 29.9 onzas; obesidad es mayor a 30.0 onzas.}}
		\end{tabular}\addtocounter{Cuadro}{1}
	\end{center}}
	


%%%%%%%%%%%%%%%%%%%



%%%%%%%%%%%%%%%%%%


\hoja{
	\normalsize
	{\Bold\color{color1!80!black}{Cuadro \theCuadro $\,-$ Indicadores de talla de las madres de niños y niñas menores de 5 años; según características varias. }}\\
	%	{\Bold\color{color1!80!black}{según características varias. }}\\
	{\Bold\color{color1!80!black}{República de Guatemala, año 2008/2009.}}\\
	{\color{color1!80!black}{(Centímetros)}}\\%[-1cm]
	\begin{center}\fontsize{3.5mm}{1.4em}\selectfont \setlength{\arrayrulewidth}{0.7pt}
		\begin{tabular}{x{4cm}cc}
			\hline &&\\[-0.5cm]  
			\multicolumn{1}{x{2.7cm}}{\textbf{Característica}} &	\multicolumn{1}{x{3cm}}{\textbf{Talla promedio}}&\multicolumn{1}{x{4cm}}{\Bold{Mide menos de 145cms}}\\[0.05cm]
			\multicolumn{1}{x{2.7cm}}{} &	\multicolumn{1}{x{2.7cm}}{(en centímetros)}&\multicolumn{1}{x{3cm}}{(en porcentaje)}\\[0.05cm]
			\rowcolor{color1!40!white} \multicolumn{1}{l}{\Bold{	Total 	 }}& 	 148.0 	 & 	 31.2 	 \\ 
			\rowcolor{color1!20!white} \multicolumn{1}{l}{\Bold{	Área geográfica	 }}& 		 & 		 \\ 
			\multicolumn{1}{l}{	     Urbana	 }& 	 149.4 	 & 	 25.0 	 \\ 
			\rowcolor{color1!5!white}\multicolumn{1}{l}{	     Rural	 }& 	 147.1 	 & 	 35.4 	 \\ 
			\rowcolor{color1!20!white} \multicolumn{1}{l}{\Bold{	Departamentos	 }}& 		 & 		 \\ 
			\multicolumn{1}{l}{	     Guatemala	 }& 	 149.9 	 & 	 20.5 	 \\ 
			\rowcolor{color1!5!white}\multicolumn{1}{l}{	     El Progreso	 }& 	 151.8 	 & 	 9.9 	 \\ 
			\multicolumn{1}{l}{	     Sacatepéquez	 }& 	 147.5 	 & 	 32.8 	 \\ 
			\rowcolor{color1!5!white}\multicolumn{1}{l}{	     Chimaltenango	 }& 	 146.9 	 & 	 42.1 	 \\ 
			\multicolumn{1}{l}{	     Escuintla	 }& 	 150.6 	 & 	 13.0 	 \\ 
			\rowcolor{color1!5!white}\multicolumn{1}{l}{	     Santa Rosa	 }& 	 150.8 	 & 	 14.7 	 \\ 
			\multicolumn{1}{l}{	     Sololá	 }& 	 144.4 	 & 	 55.7 	 \\ 
			\rowcolor{color1!5!white}\multicolumn{1}{l}{	     Totonicapán	 }& 	 145.3 	 & 	 50.4 	 \\ 
			\multicolumn{1}{l}{	     Quetzaltenango	 }& 	 147.7 	 & 	 34.0 	 \\ 
			\rowcolor{color1!5!white}\multicolumn{1}{l}{	     Suchitepéquez	 }& 	 148.2 	 & 	 31.4 	 \\ 
			\multicolumn{1}{l}{	     Retalhuleu	 }& 	 150.1 	 & 	 18.2 	 \\ 
			\rowcolor{color1!5!white}\multicolumn{1}{l}{	     San Marcos	 }& 	 146.9 	 & 	 33.5 	 \\ 
			\multicolumn{1}{l}{	     Huehuetenango	 }& 	 145.2 	 & 	 47.3 	 \\ 
			\rowcolor{color1!5!white}\multicolumn{1}{l}{	     Quiché	 }& 	 144.5 	 & 	 53.9 	 \\ 
			\multicolumn{1}{l}{	     Baja Verapaz	 }& 	 148.0 	 & 	 32.9 	 \\ 
			\rowcolor{color1!5!white}\multicolumn{1}{l}{	     Alta Verapaz	 }& 	 146.4 	 & 	 40.8 	 \\ 
			\multicolumn{1}{l}{	     Petén	 }& 	 149.0 	 & 	 22.4 	 \\ 
			\rowcolor{color1!5!white}\multicolumn{1}{l}{	     Izabal	 }& 	 149.8 	 & 	 21.1 	 \\ 
			\multicolumn{1}{l}{	     Zacapa	 }& 	 150.5 	 & 	 18.3 	 \\ 
			\rowcolor{color1!5!white}\multicolumn{1}{l}{	     Chiquimula	 }& 	 147.5 	 & 	 34.4 	 \\ 
			\multicolumn{1}{l}{	     Jalapa	 }& 	 149.2 	 & 	 20.8 	 \\ 
			\rowcolor{color1!5!white}\multicolumn{1}{l}{	     Jutiapa	 }& 	 151.4 	 & 	 17.3 	 \\ 
			\rowcolor{color1!20!white} \multicolumn{1}{l}{\Bold{	Categoría étnica	 }}& 		 & 		 \\ 
			\multicolumn{1}{l}{	     Indígena	 }& 	 145.3 	 & 	 48.3 	 \\ 
			\rowcolor{color1!5!white}\multicolumn{1}{l}{	     Ladino	 }& 	 150.0 	 & 	 19.0 	 \\ 
			\rowcolor{color1!20!white} \multicolumn{1}{l}{\Bold{	Nivel de educación	 }}& 		 & 		 \\ 
			\multicolumn{1}{l}{	     Sin educación	 }& 	 145.3 	 & 	 48.8 	 \\ 
			\rowcolor{color1!5!white}\multicolumn{1}{l}{	     Primaria	 }& 	 147.8 	 & 	 30.6 	 \\ 
			\multicolumn{1}{l}{	     Secundaria o más	 }& 	 151.6 	 & 	 13.8 	 \\[0.05cm]
			\hline
			&&\\[-0.36cm]
			\multicolumn{3}{l}{\footnotesize Fuente:  INE. Encuesta Nacional de Salud Materno Infantil (Ensmi) 2008/2009.}\\[-.09cm]
			%			\multicolumn{3}{l}{\parbox{13cm}{\footnotesize \textbf{Nota:} Peso bajo es el menor a 18.4 onzas; peso normal es entre 18.5 y 24.9; el sobre peso al nacer es el mayor a 25.0 y menor 29.9 onzas; obesidad es mayor a 30.0 onzas.}}
		\end{tabular}\addtocounter{Cuadro}{1}
	\end{center}}
	


%%%%%%%%%%%%%%%%%%%%%%%%%



%%%%%%%%%%%%%%%%%%


\hoja{
	\normalsize
	{\Bold\color{color1!80!black}{Cuadro \theCuadro $\,-$Menores de 5 años por tipo de desnutrición; según área, región, categoría étnica y }}\\
	{\Bold\color{color1!80!black}{nivel de educación de la madre.}}\\
	{\Bold\color{color1!80!black}{República de Guatemala, año 2008/2009.}}\\
	{\color{color1!80!black}{(Porcentaje)}}\\[-0.3cm]
	\begin{center}\fontsize{3.0mm}{1.3em}\selectfont \setlength{\arrayrulewidth}{0.7pt}
		\begin{tabular}{x{2.0cm}cccccc}
			\hline &&&&&&\\[-0.5cm]  
			\multicolumn{1}{x{2.0cm}}{\multirow{3}{*}[1mm]{\textbf{Característica}}} &	\multicolumn{2}{x{3.2cm}}{\textbf{Desnutrición crónica}}&\multicolumn{2}{x{3.2cm}}{\Bold{Desnutrición aguda}}&\multicolumn{2}{x{3.2cm}}{\Bold{Desnutrición global}}\\[-0.05cm]
			\multicolumn{1}{x{2.0cm}}{ } &	\multicolumn{2}{x{3.2cm}}{\textbf{(Talla para la edad)}}&\multicolumn{2}{x{3.2cm}}{\Bold{(Peso para la talla)}}&\multicolumn{2}{x{3.2cm}}{\Bold{(Peso para la edad)}}\\[0.05cm]\cline{2-7}
			\multicolumn{1}{x{2.0cm}}{ } &	\multicolumn{1}{x{1.5cm}}{\textbf{Severa}}&\multicolumn{1}{x{1.5cm}}{\Bold{Total}}&	\multicolumn{1}{x{1.5cm}}{\textbf{Severa}}&\multicolumn{1}{x{1.5cm}}{\Bold{Total}}&	\multicolumn{1}{x{1.5cm}}{\textbf{Severa}}&\multicolumn{1}{x{1.5cm}}{\Bold{Total}}\\[0.05cm]
			\rowcolor{color1!40!white} \multicolumn{1}{l}{\Bold{	Total	 }}& 	21.2	 & 	49.8	 & 	0.5	 & 	1.4	 & 	2.1	 & 	13.1	 \\ 
			\rowcolor{color1!20!white} \multicolumn{1}{l}{\Bold{	Área geográfica	 }}& 		 & 		 & 		 & 		 & 		 & 		 \\ 
			\multicolumn{1}{l}{	Urbana	 }& 	11.6	 & 	34.3	 & 	0.4	 & 	1	 & 	1.1	 & 	8.2	 \\ 
			\rowcolor{color1!5!white}\multicolumn{1}{l}{	Rural	 }& 	26.7	 & 	58.6	 & 	0.6	 & 	1.6	 & 	2.6	 & 	15.9	 \\ 
			\rowcolor{color1!20!white} \multicolumn{1}{l}{\Bold{	Departamento	 }}& 		 & 		 & 		 & 		 & 		 & 		 \\ 
			\multicolumn{1}{l}{	Guatemala	 }& 	7.9	 & 	26.3	 & 	0.4	 & 	1.4	 & 	1.2	 & 	7.3	 \\ 
			\rowcolor{color1!5!white}\multicolumn{1}{l}{	El Progreso	 }& 	9.2	 & 	25.3	 & 	1	 & 	1.7	 & 	1.3	 & 	8	 \\ 
			\multicolumn{1}{l}{	Sacatepéquez	 }& 	17.7	 & 	51.4	 & 	1.4	 & 	1.6	 & 	0.8	 & 	8.5	 \\ 
			\rowcolor{color1!5!white}\multicolumn{1}{l}{	Chimaltenango	 }& 	23.8	 & 	61.2	 & 	0.8	 & 	1.2	 & 	3.3	 & 	14.5	 \\ 
			\multicolumn{1}{l}{	Escuintla	 }& 	10.3	 & 	32.4	 & 	0.2	 & 	0.8	 & 	0.5	 & 	10.2	 \\ 
			\rowcolor{color1!5!white}\multicolumn{1}{l}{	Santa Rosa	 }& 	10.1	 & 	28.9	 & 	0.8	 & 	1.7	 & 	0.6	 & 	7.7	 \\ 
			\multicolumn{1}{l}{	Sololá	 }& 	36.4	 & 	72.3	 & 	1	 & 	1	 & 	2.6	 & 	17.3	 \\ 
			\rowcolor{color1!5!white}\multicolumn{1}{l}{	Totonicapán	 }& 	42.8	 & 	82.2	 & 	0.5	 & 	0.9	 & 	3.1	 & 	24.5	 \\ 
			\multicolumn{1}{l}{	Quetzaltenango	 }& 	13.1	 & 	43.1	 & 	0.6	 & 	1.5	 & 	1	 & 	10	 \\ 
			\rowcolor{color1!5!white}\multicolumn{1}{l}{	Suchitepéquez	 }& 	13.3	 & 	43.5	 & 	1	 & 	2.3	 & 	1.7	 & 	12.5	 \\ 
			\multicolumn{1}{l}{	Retalhuleu	 }& 	10.5	 & 	34.6	 & 	0	 & 	2.3	 & 	2.5	 & 	11.5	 \\ 
			\rowcolor{color1!5!white}\multicolumn{1}{l}{	San Marcos	 }& 	21.5	 & 	53.5	 & 	1.8	 & 	2.9	 & 	1	 & 	14.4	 \\ 
			\multicolumn{1}{l}{	Huehuetenango	 }& 	36.5	 & 	69.5	 & 	0.1	 & 	1	 & 	4.2	 & 	20.8	 \\ 
			\rowcolor{color1!5!white}\multicolumn{1}{l}{	Quiché	 }& 	39.4	 & 	72.2	 & 	0.6	 & 	1	 & 	4.5	 & 	21.5	 \\ 
			\multicolumn{1}{l}{	Baja Verapaz	 }& 	29	 & 	59.4	 & 	0	 & 	1.6	 & 	3	 & 	14.9	 \\ 
			\rowcolor{color1!5!white}\multicolumn{1}{l}{	Alta Verapaz	 }& 	24.6	 & 	59.4	 & 	0.1	 & 	1.1	 & 	1.3	 & 	9.3	 \\ 
			\multicolumn{1}{l}{	Petén	 }& 	13.2	 & 	41.9	 & 	0.4	 & 	1	 & 	0.6	 & 	9	 \\ 
			\rowcolor{color1!5!white}\multicolumn{1}{l}{	Izabal	 }& 	12.3	 & 	40.4	 & 	0.4	 & 	2.8	 & 	3.5	 & 	13.2	 \\ 
			\multicolumn{1}{l}{	Zacapa	 }& 	21.3	 & 	45.9	 & 	0	 & 	0.4	 & 	3.7	 & 	16	 \\ 
			\rowcolor{color1!5!white}\multicolumn{1}{l}{	Chiquimula	 }& 	29.1	 & 	61.8	 & 	0.4	 & 	1.2	 & 	3.7	 & 	16.9	 \\ 
			\multicolumn{1}{l}{	Jalapa	 }& 	22.9	 & 	49.3	 & 	0	 & 	0.2	 & 	1.4	 & 	11.6	 \\ 
			\rowcolor{color1!5!white}\multicolumn{1}{l}{	Jutiapa	 }& 	14	 & 	36.8	 & 	0.5	 & 	1.8	 & 	0.8	 & 	10.5	 \\ 
			\rowcolor{color1!20!white} \multicolumn{1}{l}{\Bold{	Categoría étnica	 }}& 		 & 		 & 		 & 		 & 		 & 		 \\ 
			\multicolumn{1}{l}{	Indígena	 }& 	31.3	 & 	65.9	 & 	0.5	 & 	1.3	 & 	3	 & 	16.8	 \\ 
			\rowcolor{color1!5!white}\multicolumn{1}{l}{	Ladino	 }& 	12.7	 & 	36.2	 & 	0.5	 & 	1.5	 & 	1.3	 & 	10.1	 \\ 
			\rowcolor{color1!20!white} \multicolumn{1}{l}{\Bold{	Nivel de educación	 }}& 		 & 		 & 		 & 		 & 		 & 		 \\ 
			\multicolumn{1}{l}{	Sin educación	 }& 	35.9	 & 	69.3	 & 	0.8	 & 	1.6	 & 	3.5	 & 	19.9	 \\ 
			\rowcolor{color1!5!white}\multicolumn{1}{l}{	Primaria	 }& 	19.1	 & 	50.3	 & 	0.5	 & 	1.4	 & 	1.8	 & 	12.6	 \\ 
			\multicolumn{1}{l}{	Secundaria	 }& 	5.7	 & 	21.2	 & 	0.2	 & 	1.1	 & 	0.8	 & 	5.1	 \\ 
			\rowcolor{color1!5!white}\multicolumn{1}{l}{	Superior	 }& 	3.7	 & 	14.1	 & 	0.6	 & 	0.6	 & 	0.5	 & 	2.1	 \\[0.05cm]
			\hline
			&&&&&&\\[-0.36cm]
			\multicolumn{7}{l}{\footnotesize Fuente:  INE. Encuesta Nacional de Salud Materno Infantil (Ensmi) 2008/2009.}\\[.1cm]
			\multicolumn{7}{l}{\parbox{13cm}{\footnotesize \textbf{Notas:} Se denomina desnutrición severa cuando los niños están 3 desviaciones estándar o más por debajo de la media, de acuerdo a la tabla de medidas de la OMS. }}\\[.3cm]
			\multicolumn{7}{l}{\parbox{13cm}{\footnotesize Se denomina desnutrición total cuando los niños están dos desviaciones estándar o más por debajo de la media. Incluye a los niños que están 3 desviaciones estándar o más por debajo de la media.}}\\
		\end{tabular}\addtocounter{Cuadro}{1}
	\end{center}}
	


%%%%%%%%%%%%%%%%%%


\hoja{
	\normalsize
	{\Bold\color{color1!80!black}{Cuadro \theCuadro $\,-$Niños y niñas de 6 a 59 meses con anemia; según características varias. }}\\
	%	{\Bold\color{color1!80!black}{nivel de educación de la madre.}}\\
	{\Bold\color{color1!80!black}{República de Guatemala, año 2008/2009.}}\\
	{\color{color1!80!black}{(Porcentaje)}}\\[-0.3cm]
	\begin{center}\fontsize{3.0mm}{1.3em}\selectfont \setlength{\arrayrulewidth}{0.7pt}
		\begin{tabular}{x{4.0cm}c}
			\hline &\\[-0.5cm]  
			\multicolumn{1}{x{2.0cm}}{\textbf{Característica}} &	\multicolumn{1}{x{5cm}}{\textbf{Niños menores de 5 años}}\\[-0.05cm]\hline
			%			\multicolumn{1}{x{2.0cm}}{ } &	\multicolumn{2}{x{3.2cm}}{\textbf{(Talla para la edad)}}&\multicolumn{2}{x{3.2cm}}{\Bold{(Peso para la talla)}}&\multicolumn{2}{x{3.2cm}}{\Bold{(Peso para la edad)}}\\[0.05cm]\cline{2-7}
			%			\multicolumn{1}{x{2.0cm}}{ } &	\multicolumn{1}{x{1.5cm}}{\textbf{Severa}}&\multicolumn{1}{x{1.5cm}}{\Bold{Total}}&	\multicolumn{1}{x{1.5cm}}{\textbf{Severa}}&\multicolumn{1}{x{1.5cm}}{\Bold{Total}}&	\multicolumn{1}{x{1.5cm}}{\textbf{Severa}}&\multicolumn{1}{x{1.5cm}}{\Bold{Total}}\\[0.05cm]
			\rowcolor{color1!40!white} \multicolumn{1}{l}{\Bold{	Total República	 }}& 	 47.7 	 \\ 
			\rowcolor{color1!20!white} \multicolumn{1}{l}{\Bold{	Área geográfica	 }}& 		 \\ 
			\multicolumn{1}{l}{	Urbana	 }& 	 46.2 	 \\ 
			\rowcolor{color1!5!white}\multicolumn{1}{l}{	Rural	 }& 	 48.6 	 \\ 
			\rowcolor{color1!20!white} \multicolumn{1}{l}{\Bold{	Departamento	 }}& 		 \\ 
			\multicolumn{1}{l}{	Guatemala	 }& 	 40.7 	 \\ 
			\rowcolor{color1!5!white}\multicolumn{1}{l}{	El Progreso	 }& 	 37.8 	 \\ 
			\multicolumn{1}{l}{	Sacatepéquez	 }& 	 54.2 	 \\ 
			\rowcolor{color1!5!white}\multicolumn{1}{l}{	Chimaltenango	 }& 	 53.5 	 \\ 
			\multicolumn{1}{l}{	Escuintla	 }& 	 50.2 	 \\ 
			\rowcolor{color1!5!white}\multicolumn{1}{l}{	Santa Rosa	 }& 	 51.4 	 \\ 
			\multicolumn{1}{l}{	Sololá	 }& 	 56.1 	 \\ 
			\rowcolor{color1!5!white}\multicolumn{1}{l}{	Totonicapán	 }& 	 62.2 	 \\ 
			\multicolumn{1}{l}{	Quetzaltenango	 }& 	 40.2 	 \\ 
			\rowcolor{color1!5!white}\multicolumn{1}{l}{	Suchitepéquez	 }& 	 37.7 	 \\ 
			\multicolumn{1}{l}{	Retalhuleu	 }& 	 45.3 	 \\ 
			\rowcolor{color1!5!white}\multicolumn{1}{l}{	San Marcos	 }& 	 52.6 	 \\ 
			\multicolumn{1}{l}{	Huehuetenango	 }& 	 47.7 	 \\ 
			\rowcolor{color1!5!white}\multicolumn{1}{l}{	Quiché	 }& 	 47.4 	 \\ 
			\multicolumn{1}{l}{	Baja Verapaz	 }& 	 49.8 	 \\ 
			\rowcolor{color1!5!white}\multicolumn{1}{l}{	Alta Verapaz	 }& 	 46.1 	 \\ 
			\multicolumn{1}{l}{	Petén	 }& 	 48.5 	 \\ 
			\rowcolor{color1!5!white}\multicolumn{1}{l}{	Izabal	 }& 	 53.0 	 \\ 
			\multicolumn{1}{l}{	Zacapa	 }& 	 53.7 	 \\ 
			\rowcolor{color1!5!white}\multicolumn{1}{l}{	Chiquimula	 }& 	 55.5 	 \\ 
			\multicolumn{1}{l}{	Jalapa	 }& 	 43.9 	 \\ 
			\rowcolor{color1!5!white}\multicolumn{1}{l}{	Jutiapa	 }& 	 50.3 	 \\ 
			\rowcolor{color1!20!white} \multicolumn{1}{l}{\Bold{	Categoría étnica	 }}& 		 \\ 
			\multicolumn{1}{l}{	Indígena	 }& 	 49.5 	 \\ 
			\rowcolor{color1!5!white}\multicolumn{1}{l}{	Ladino	 }& 	 46.3 	 \\ 
			\rowcolor{color1!20!white} \multicolumn{1}{l}{\Bold{	Nivel de educación	 }}& 		 \\ 
			\multicolumn{1}{l}{	Sin educación	 }& 	 48.3 	 \\ 
			\rowcolor{color1!5!white}\multicolumn{1}{l}{	Primaria	 }& 	 49.2 	 \\ 
			\multicolumn{1}{l}{	Secundaria	 }& 	 44.0 	 \\ 
			\rowcolor{color1!5!white}\multicolumn{1}{l}{	Superior	 }& 	 36.0 	 \\ 
			[0.05cm]
			\hline
			&\\[-0.36cm]\end{tabular}\addtocounter{Cuadro}{1}
	\end{center}
	{\footnotesize Fuente:  INE. Encuesta Nacional de Salud Materno Infantil (Ensmi) 2008/2009.}\\[.1cm]
	{\parbox{13cm}{\footnotesize \textbf{Notas:} Se denomina desnutrición severa cuando los niños están 3 desviaciones estándar o más por debajo de la media, de acuerdo a la tabla de medidas de la OMS. }}\\[.3cm]
	{\parbox{13cm}{\footnotesize Se denomina desnutrición total cuando los niños están dos desviaciones estándar o más por debajo de la media. Incluye a los niños que están 3 desviaciones estándar o más por debajo de la media.}}\\
}

%%%%%%%%%%%%%%%%%%


\hoja{
	\normalsize
	{\Bold\color{color1!80!black}{Cuadro \theCuadro $\,-$ Distribución de los niños recién nacidos; según peso reportado por la madre. }}\\
	%	{\Bold\color{color1!80!black}{nivel de educación de la madre.}}\\
	{\Bold\color{color1!80!black}{República de Guatemala, año 2008/2009.}}\\
	{\color{color1!80!black}{(Porcentaje)}}\\[-0.3cm]
	\begin{center}\fontsize{3.0mm}{1.3em}\selectfont \setlength{\arrayrulewidth}{0.7pt}
		\begin{tabular}{x{4.0cm}ccc}
			\hline &\\[-0.5cm]  
			\multicolumn{1}{x{2.0cm}}{\textbf{Característica}} &	\multicolumn{1}{x{2.5cm}}{\textbf{Menos de 2.5 kilos}}&\multicolumn{1}{x{2cm}}{\textbf{2.5 kilos o más}}&\multicolumn{1}{x{2cm}}{\textbf{No sabe}}\\[0.05cm]\hline
			%			\multicolumn{1}{x{2.0cm}}{ } &	\multicolumn{2}{x{3.2cm}}{\textbf{(Talla para la edad)}}&\multicolumn{2}{x{3.2cm}}{\Bold{(Peso para la talla)}}&\multicolumn{2}{x{3.2cm}}{\Bold{(Peso para la edad)}}\\[0.05cm]\cline{2-7}
			%			\multicolumn{1}{x{2.0cm}}{ } &	\multicolumn{1}{x{1.5cm}}{\textbf{Severa}}&\multicolumn{1}{x{1.5cm}}{\Bold{Total}}&	\multicolumn{1}{x{1.5cm}}{\textbf{Severa}}&\multicolumn{1}{x{1.5cm}}{\Bold{Total}}&	\multicolumn{1}{x{1.5cm}}{\textbf{Severa}}&\multicolumn{1}{x{1.5cm}}{\Bold{Total}}\\[0.05cm]
			\rowcolor{color1!40!white} \multicolumn{1}{l}{\Bold{	Total	 }}& 	11.4	 & 	88.1	 & 	0.5	 \\ 
			\rowcolor{color1!20!white} \multicolumn{1}{l}{\Bold{	Área geográfica	 }}& 		 & 		 & 		 \\ 
			\multicolumn{1}{l}{	Urbana	 }& 	 12.6 	 & 	 86.9 	 & 	 0.5 	 \\ 
			\rowcolor{color1!5!white}\multicolumn{1}{l}{	Rural	 }& 	 10.6 	 & 	 88.9 	 & 	 0.5 	 \\ 
			\rowcolor{color1!20!white} \multicolumn{1}{l}{\Bold{	Departamentos	 }}& 		 & 		 & 		 \\ 
			\multicolumn{1}{l}{	Guatemala	 }& 	 13.7 	 & 	 85.7 	 & 	 0.6 	 \\ 
			\rowcolor{color1!5!white}\multicolumn{1}{l}{	El Progreso	 }& 	 13.7 	 & 	 85.1 	 & 	 1.2 	 \\ 
			\multicolumn{1}{l}{	Sacatepéquez	 }& 	 14.6 	 & 	 85.4 	 & 	 -   	 \\ 
			\rowcolor{color1!5!white}\multicolumn{1}{l}{	Chimaltenango	 }& 	 9.1 	 & 	 90.9 	 & 	 -   	 \\ 
			\multicolumn{1}{l}{	Escuintla	 }& 	 10.3 	 & 	 89.2 	 & 	 0.6 	 \\ 
			\rowcolor{color1!5!white}\multicolumn{1}{l}{	Santa Rosa	 }& 	 6.8 	 & 	 92.4 	 & 	 0.8 	 \\ 
			\multicolumn{1}{l}{	Sololá	 }& 	 10.8 	 & 	 88.4 	 & 	 0.8 	 \\ 
			\rowcolor{color1!5!white}\multicolumn{1}{l}{	Totonicapán	 }& 	 11.7 	 & 	 88.2 	 & 	 0.2 	 \\ 
			\multicolumn{1}{l}{	Quetzaltenango	 }& 	 14.7 	 & 	 85.3 	 & 	 -   	 \\ 
			\rowcolor{color1!5!white}\multicolumn{1}{l}{	Suchitepéquez	 }& 	 5.6 	 & 	 94.3 	 & 	 0.2 	 \\ 
			\multicolumn{1}{l}{	Retalhuleu	 }& 	 6.7 	 & 	 93.3 	 & 	 -   	 \\ 
			\rowcolor{color1!5!white}\multicolumn{1}{l}{	San Marcos	 }& 	 9.4 	 & 	 90.0 	 & 	 0.6 	 \\ 
			\multicolumn{1}{l}{	Huehuetenango	 }& 	 10.1 	 & 	 89.1 	 & 	 0.8 	 \\ 
			\rowcolor{color1!5!white}\multicolumn{1}{l}{	Quiché	 }& 	 11.8 	 & 	 86.7 	 & 	 1.5 	 \\ 
			\multicolumn{1}{l}{	Baja Verapaz	 }& 	 18.7 	 & 	 80.7 	 & 	 0.6 	 \\ 
			\rowcolor{color1!5!white}\multicolumn{1}{l}{	Alta Verapaz	 }& 	 12.9 	 & 	 86.9 	 & 	 0.1 	 \\ 
			\multicolumn{1}{l}{	Petén	 }& 	 9.7 	 & 	 89.6 	 & 	 0.8 	 \\ 
			\rowcolor{color1!5!white}\multicolumn{1}{l}{	Izabal	 }& 	 7.5 	 & 	 92.5 	 & 	 -   	 \\ 
			\multicolumn{1}{l}{	Zacapa	 }& 	 11.0 	 & 	 89.0 	 & 	 -   	 \\ 
			\rowcolor{color1!5!white}\multicolumn{1}{l}{	Chiquimula	 }& 	 15.0 	 & 	 84.7 	 & 	 0.3 	 \\ 
			\multicolumn{1}{l}{	Jalapa	 }& 	 14.6 	 & 	 85.4 	 & 	 -   	 \\ 
			\rowcolor{color1!5!white}\multicolumn{1}{l}{	Jutiapa	 }& 	 9.6 	 & 	 90.4 	 & 	 -   	 \\ 
			\rowcolor{color1!20!white} \multicolumn{1}{l}{\Bold{	Categoría étnica	 }}& 		 & 		 & 		 \\ 
			\multicolumn{1}{l}{	Indígena	 }& 	 11.7 	 & 	 87.7 	 & 	 0.6 	 \\ 
			\rowcolor{color1!5!white}\multicolumn{1}{l}{	Ladino	 }& 	 11.2 	 & 	 88.4 	 & 	 0.4 	 \\ 
			\rowcolor{color1!20!white} \multicolumn{1}{l}{\Bold{	Nivel de educación	 }}& 		 & 		 & 		 \\ 
			\multicolumn{1}{l}{	Sin educación	 }& 	 12.1 	 & 	 87.1 	 & 	 0.9 	 \\ 
			\rowcolor{color1!5!white}\multicolumn{1}{l}{	Primaria	 }& 	 10.7 	 & 	 88.9 	 & 	 0.4 	 \\ 
			\multicolumn{1}{l}{	Secundaria	 }& 	 12.6 	 & 	 87.0 	 & 	 0.4 	 \\ 
			\rowcolor{color1!5!white}\multicolumn{1}{l}{	Superior	 }& 	 10.2 	 & 	 89.8 	 & 		 \\ 
			[0.05cm]
			\hline
			&&&\\[-0.36cm]\end{tabular}\addtocounter{Cuadro}{1}
	\end{center}
	{\footnotesize Fuente:  INE. Encuesta Nacional de Salud Materno Infantil (Ensmi) 2008/2009.}\\[.1cm]
	%	{\parbox{13cm}{\footnotesize \textbf{Notas:} Se denomina desnutrición severa cuando los niños están 3 desviaciones estándar o más por debajo de la media, de acuerdo a la tabla de medidas de la OMS. }}\\[.3cm]
	%	{\parbox{13cm}{\footnotesize Se denomina desnutrición total cuando los niños están dos desviaciones estándar o más por debajo de la media. Incluye a los niños que están 3 desviaciones estándar o más por debajo de la media.}}\\
}


  %%%Revisar la nota en la tabla de bajo peso y talla 
		

\INEchaptercarta{Inversión pública en SAN}{}

\addtocounter{Cuadro}{1}
%%%%%%%%%%%%%%%%%%


\hoja{
	{\Bold\Large 7.1	Integración del presupuesto en seguridad alimentaria y nutricional del Plan del Pacto Hambre Cero a nivel nacional. }\\
	\normalsize
	{\Bold\color{color1!80!black}\parbox{15cm}{Cuadro \theCuadro $\,-$ Presupuesto en Seguridad Alimentaria y Nutricional (SAN), del Plan del Pacto Hambre Cero (PH0), por criterios de seguimiento y porcentaje de ejecución; según institución.}}\\
	{\Bold\color{color1!80!black}{República de Guatemala, año 2012.}}\\
	{\color{color1!80!black}{(Quetzales y porcentaje)}}\\[-0.3cm]
	\begin{center}\fontsize{3.0mm}{1.3em}\selectfont \setlength{\arrayrulewidth}{0.7pt}
		\begin{tabular}{x{4.0cm}rrrc}
			\hline &&&&\\[-0.4cm]  
			\multicolumn{1}{x{2.0cm}}{\raisebox{-.3cm}{\textbf{Institución}}} &	\multicolumn{3}{x{6.3cm}}{\textbf{Criterios de seguimiento (en quetzales)}}&\multicolumn{1}{x{2cm}}{\multirow{2}{*}[-.5mm]{\textbf{Ejecución}}}\\[0.05cm]\cline{2-4}
			\multicolumn{1}{l}{ }&\multicolumn{4}{l}{ }\\[-.35cm]
			\multicolumn{1}{x{2.0cm}}{ } &	\multicolumn{1}{x{2cm}}{\textbf{Asignado}}&\multicolumn{1}{x{2cm}}{\textbf{Vigente}}&\multicolumn{1}{x{2cm}}{\textbf{Ejecutado}}&\multicolumn{1}{x{2cm}}{\textbf{(Porcentaje)}}\\[0.05cm]\hline
			\rowcolor{color1!40!white} \multicolumn{1}{l}{\Bold{	Total presupuesto	 }}& 	4,733,725,081	&	5,601,097,205	&	4,995,879,003	&	89.2	\\
			\rowcolor{color1!20!white} \multicolumn{1}{l}{\Bold{	Ministerios	 }}& 	4,008,928,485	&	5,028,707,801	&	4,526,928,708	&	90.0	\\
			\multicolumn{1}{l}{	Ministerio de Educación Pública -Mineduc-	 }& 	638,253,918	&	560,143,362	&	558,267,850	&	99.7	\\
			\rowcolor{color1!5!white}\multicolumn{1}{l}{	Ministerio de Salud Pública y Asistencia Social -Mspas-	 }& 	590,752,798	&	669,883,381	&	552,014,678	&	82.4	\\
			\multicolumn{1}{l}{	Ministerio de Economía -Mineco- 	 }& 	56,469,411	&	58,456,590	&	54,607,485	&	93.4	\\
			\rowcolor{color1!5!white}\multicolumn{1}{l}{	Ministerio de Trabajo -Mintrab-	 }& 	10,344,362	&	3,181,159	&	2,898,509	&	91.1	\\
			\multicolumn{1}{l}{	Ministerio de Agricultura, Ganadería y Alimentación -MAGA-	 }& 	1,180,732,672	&	696,660,339	&	647,975,496	&	93.0	\\
			\rowcolor{color1!5!white}\multicolumn{1}{l}{	Ministerio de Comunicaciones, Infraestructura y Vivienda  -Micivi-	 }& 	1,445,807,023	&	1,896,224,766	&	1,671,325,521	&	88.1	\\
			\multicolumn{1}{l}{	Ministerio de Ambiente y Recursos Naturales  -MARN-	 }& 	86,568,301	&	51,592,577	&	48,916,964	&	94.8	\\
			\rowcolor{color1!5!white}\multicolumn{1}{l}{	Ministerio de Desarrollo Social -Mides-	 }& 	0	&	1,092,565,628	&	990,922,206	&	90.7	\\
			\rowcolor{color1!20!white} \multicolumn{1}{l}{\Bold{	Secretarías	 }}& 	449,454,765	&	299,213,569	&	288,137,447	&	96.3	\\
			\multicolumn{1}{l}{	Secretaría de Coordinación Ejecutiva de la Presidencia -SCEP-	 }& 	45,954,770	&	44,450,835	&	42,821,670	&	96.3	\\
			\rowcolor{color1!5!white}\multicolumn{1}{l}{	Secretaría de Bienestar Social -SBS-	 }& 	204,945,078	&	114,626,067	&	107,713,856	&	94.0	\\
			\multicolumn{1}{l}{	Secretaría de Obras Sociales de la Esposa del Presidente -Sosep-	 }& 	155,892,084	&	102,139,270	&	100,071,030	&	98.0	\\
			\rowcolor{color1!5!white}\multicolumn{1}{l}{	Secretaría de Seguridad Aliementaria y Nutricional -Sesan-	 }& 	42,662,833	&	37,997,397	&	37,530,890	&	98.8	\\
			\rowcolor{color1!20!white} \multicolumn{1}{l}{\Bold{	Descentralizadas	 }}& 	275,341,831	&	273,175,835	&	180,812,849	&	66.2	\\
			\multicolumn{1}{l}{	Coordinadora Nacional para la Reducción de Desastres -Conred-	 }& 	8,466,582	&	6,300,586	&	6,162,996	&	97.8	\\
			\rowcolor{color1!5!white}\multicolumn{1}{l}{	Instituto de Ciencia y Tecnología Agrícola -ICTA-	 }& 	42,000,000	&	42,000,000	&	28,170,923	&	67.1	\\
			\multicolumn{1}{l}{	Comisión Nacional de Alfabetización -Conalfa-	 }& 	204,875,249	&	204,875,249	&	137,433,878	&	67.1	\\
			\rowcolor{color1!5!white}\multicolumn{1}{l}{	Instituto Nacional de Comercialización Agrícola -Indeca-	 }& 	20,000,000	&	20,000,000	&	9,045,052	&	45.2	\\[0.05cm]
			\hline
			&&&\\[-0.36cm]\end{tabular}\addtocounter{Cuadro}{1}
	\end{center}
	{\footnotesize Fuente:  Elaborado por Sesan con datos de Sicoin-Minfin, 2016.}\\[.1cm]
	%	{\parbox{13cm}{\footnotesize \textbf{Notas:} Se denomina desnutrición severa cuando los niños están 3 desviaciones estándar o más por debajo de la media, de acuerdo a la tabla de medidas de la OMS. }}\\[.3cm]
	%	{\parbox{13cm}{\footnotesize Se denomina desnutrición total cuando los niños están dos desviaciones estándar o más por debajo de la media. Incluye a los niños que están 3 desviaciones estándar o más por debajo de la media.}}\\
}



%%%%%%%%%%%%%%%%%%%%%%

%%%%%%%%%%%%%%%%%%


\hoja{
	\normalsize
	{\Bold\color{color1!80!black}{Cuadro \theCuadro $\,-$Presupuesto en Seguridad Alimentaria y Nutricional (SAN), del Pacto Hambre Cero (PH0);}}\\
	{\Bold\color{color1!80!black}{ por criterios de seguimiento y porcentaje de ejecución, según institución}}\\
	{\Bold\color{color1!80!black}{República de Guatemala, año 2013.}}\\
	{\color{color1!80!black}{(Quetzales y porcentaje)}}\\[-0.3cm]
	\begin{center}\fontsize{3.0mm}{1.3em}\selectfont \setlength{\arrayrulewidth}{0.7pt}
		\begin{tabular}{x{5.5cm}rrc}
			\hline &&&\\[-0.4cm]  
			\multicolumn{1}{x{5.5cm}}{\raisebox{-.3cm}{\textbf{Instituciones}}} &	\multicolumn{2}{x{4.6cm}}{\textbf{Criterios de seguimiento (en quetzales)}}&\multicolumn{1}{x{2cm}}{\multirow{2}{*}[-.5mm]{\textbf{Ejecución}}}\\[0.05cm]\cline{2-3}
			\multicolumn{1}{l}{ }&\multicolumn{3}{l}{ }\\[-.35cm]
			\multicolumn{1}{c}{ }&\multicolumn{1}{x{2cm}}{\textbf{Vigente}}&\multicolumn{1}{x{2cm}}{\textbf{Ejecutado}}&\multicolumn{1}{x{2cm}}{\textbf{(Porcentaje)}}\\[0.05cm]\hline
			\rowcolor{color1!40!white} \multicolumn{1}{l}{\Bold{	Total presupuesto	 }}& 	6,099,516,355	&	4,575,679,524	&	75.0	\\
			\rowcolor{color1!20!white} \multicolumn{1}{l}{\Bold{	Ministerios	 }}& 	5,444,656,944	&	4,300,095,736	&	79.0	\\
			\multicolumn{1}{l}{	Ministerio de Educación Pública -Mineduc-	 }& 	737,101,217	&	584,016,414	&	79.2	\\
			\rowcolor{color1!5!white}\multicolumn{1}{l}{	Ministerio de Salud Pública y Asistencia Social -Mspas-	 }& 	948,552,777	&	875,150,770	&	92.3	\\
			\multicolumn{1}{l}{	Ministerio de Economía -Mineco-	 }& 	9,420,886	&	9,420,885	&	100.0	\\
			\rowcolor{color1!5!white}\multicolumn{1}{l}{	Ministerio de Trabajo -Mintrab- 	 }& 	24,770,815	&	24,117,799	&	97.4	\\
			\multicolumn{1}{l}{	Ministerio de Agricultura Ganadería y Alimentación -MAGA-	 }& 	1,054,903,855	&	632,593,189	&	60.0	\\
			\rowcolor{color1!5!white}\multicolumn{1}{l}{	Ministerio de Comunicaciones, Infraestructura y Vivienda -Micivi-	 }& 	1,611,284,421	&	1,423,438,995	&	88.3	\\
			\multicolumn{1}{l}{	Ministerio de Ambiente y Recursos Naturales -MARN-	 }& 	1,304,562	&	914,439	&	70.1	\\
			\rowcolor{color1!5!white}\multicolumn{1}{l}{	Ministerio de Desarrollo Social -Midex-	 }& 	1,057,318,411	&	750,443,245	&	71.0	\\
			\rowcolor{color1!20!white} \multicolumn{1}{l}{\Bold{	Secretarías	 }}& 	130,448,713	&	104,688,447	&	80.3	\\
			\multicolumn{1}{l}{	Secretaría de Coordinación Ejecutiva de la Presidencia -SCEP-	 }& 	473,205	&	395,608	&	83.6	\\
			\rowcolor{color1!5!white}\multicolumn{1}{l}{	Secretaría de Bienestar Social -SBS-	 }& 	2,003,362	&	1,965,564	&	98.1	\\
			\multicolumn{1}{l}{	Secretaría de Obras Sociales de la Esposa del Presidente -Sosep-	 }& 	34,548,326	&	34,548,325	&	100.0	\\
			\rowcolor{color1!5!white}\multicolumn{1}{l}{	Secretaria Seprem	 }& 	823,184	&	822,361	&	99.9	\\
			\multicolumn{1}{l}{	Sesan	 }& 	92,600,636	&	66,956,590	&	72.3	\\
			\rowcolor{color1!20!white} \multicolumn{1}{l}{\Bold{	Descentralizadas	 }}& 	524,410,699	&	170,895,340	&	32.6	\\
			\multicolumn{1}{l}{	ICTA	 }& 	45,182,076	&	34,292,886	&	75.9	\\
			\rowcolor{color1!5!white}\multicolumn{1}{l}{	Infom	 }& 	399,883,688	&	70,170,245	&	17.5	\\
			\multicolumn{1}{l}{	Conalfa 	 }& 	60,344,935	&	52,208,071	&	86.5	\\
			\rowcolor{color1!5!white}\multicolumn{1}{l}{	Indeca	 }& 	19,000,000	&	14,224,138	&	74.9	\\
			[0.05cm]
			\hline
			&&&\\[-0.36cm]\end{tabular}\addtocounter{Cuadro}{1}
	\end{center}
	{\footnotesize Fuente:  Elaborado por Sesan con datos de Sicoin-Minfin, 2016.}\\[.1cm]
	%	{\parbox{13cm}{\footnotesize \textbf{Notas:} Se denomina desnutrición severa cuando los niños están 3 desviaciones estándar o más por debajo de la media, de acuerdo a la tabla de medidas de la OMS. }}\\[.3cm]
	%	{\parbox{13cm}{\footnotesize Se denomina desnutrición total cuando los niños están dos desviaciones estándar o más por debajo de la media. Incluye a los niños que están 3 desviaciones estándar o más por debajo de la media.}}\\
}






%%%%%%%%%%%%%%%%%%


\hoja{
	\normalsize
	{\Bold\color{color1!80!black}{Cuadro \theCuadro $\,-$ Presupuesto en Seguridad Alimentaria y Nutricional, del Plan del Pacto Hambre Cero (PH0), }}\\
	{\Bold\color{color1!80!black}{por criterios de seguimiento y porcentaje de ejecución; según institución.}}\\
	{\Bold\color{color1!80!black}{República de Guatemala, año 2014.}}\\
	{\color{color1!80!black}{(Quetzales y porcentaje)}}\\[-0.3cm]
	\begin{center}\fontsize{3.0mm}{1.3em}\selectfont \setlength{\arrayrulewidth}{0.7pt}
		\begin{tabular}{x{4.0cm}rrrc}
			\hline &&&&\\[-0.4cm]  
			\multicolumn{1}{x{2.0cm}}{\raisebox{-.3cm}{\textbf{Institución}}} &	\multicolumn{3}{x{6.3cm}}{\textbf{Criterios de seguimiento (en quetzales)}}&\multicolumn{1}{x{2cm}}{\multirow{2}{*}[-.5mm]{\textbf{Ejecución}}}\\[0.05cm]\cline{2-4}
			\multicolumn{1}{l}{ }&\multicolumn{4}{l}{ }\\[-.35cm]
			\multicolumn{1}{x{2.0cm}}{ } &	\multicolumn{1}{x{2cm}}{\textbf{Asignado}}&\multicolumn{1}{x{2cm}}{\textbf{Vigente}}&\multicolumn{1}{x{2cm}}{\textbf{Ejecutado}}&\multicolumn{1}{x{2cm}}{\textbf{(Porcentaje)}}\\[0.05cm]\hline
			\rowcolor{color1!40!white} \multicolumn{1}{l}{\Bold{	Total presupuesto	 }}& 	5,271,613,438	&	6,587,699,083	&	5,615,711,668	&	85.2	\\
			\rowcolor{color1!20!white} \multicolumn{1}{l}{\Bold{	Ministerios 	 }}& 	4,507,844,851	&	5,778,816,812	&	5,175,638,909	&	89.6	\\
			\multicolumn{1}{l}{	Mineduc	 }& 	699,552,593	&	700,740,593	&	618,271,098	&	88.2	\\
			\rowcolor{color1!5!white}\multicolumn{1}{l}{	Mspas	 }& 	663,857,018	&	990,638,317	&	841,129,249	&	84.9	\\
			\multicolumn{1}{l}{	MAGA	 }& 	955,572,082	&	877,715,651	&	790,560,249	&	90.1	\\
			\rowcolor{color1!5!white}\multicolumn{1}{l}{	Micivi	 }& 	1,265,437,403	&	2,133,468,319	&	1,874,645,098	&	87.9	\\
			\multicolumn{1}{l}{	MARN	 }& 	1,304,562	&	1,350,231	&	525,833	&	38.9	\\
			\rowcolor{color1!5!white}\multicolumn{1}{l}{	Mides	 }& 	922,121,193	&	1,074,903,701	&	1,050,507,382	&	97.7	\\
			\rowcolor{color1!20!white} \multicolumn{1}{l}{\Bold{	Secretarías	 }}& 	93,163,866	&	120,720,925	&	100,761,731	&	83.5	\\
			\multicolumn{1}{l}{	SCEP	 }& 	544,200	&	544,200	&	510,836	&	93.9	\\
			\rowcolor{color1!5!white}\multicolumn{1}{l}{	Sosep	 }& 	43,937,153	&	52,030,887	&	44,785,106	&	86.1	\\
			\multicolumn{1}{l}{	Sesan	 }& 	45,977,554	&	65,440,879	&	53,033,906	&	81.0	\\
			\rowcolor{color1!5!white}\multicolumn{1}{l}{	SBS	 }& 	2,704,959	&	2,704,959	&	2,431,882	&	89.9	\\
			\rowcolor{color1!20!white} \multicolumn{1}{l}{\Bold{	Descentralizadas	 }}& 	670,604,721	&	688,161,346	&	339,311,027	&	49.3	\\
			\multicolumn{1}{l}{	ICTA	 }& 	38,000,000	&	40,430,500	&	35,150,813	&	86.9	\\
			\rowcolor{color1!5!white}\multicolumn{1}{l}{	Infom	 }& 	362,182,524	&	405,108,963	&	107,945,244	&	26.6	\\
			\multicolumn{1}{l}{	Conalfa	 }& 	194,688,339	&	145,382,916	&	107,601,793	&	74.0	\\
			\rowcolor{color1!5!white}\multicolumn{1}{l}{	Indeca	 }& 	12,000,000	&	22,300,000	&	14,489,251	&	65.0	\\
			\multicolumn{1}{l}{	Fontierras	 }& 	63,733,858	&	74,938,967	&	74,123,927	&	98.9	\\[0.05cm]
			\hline
			&&&&\\[-0.36cm]\end{tabular}\addtocounter{Cuadro}{1}
	\end{center}
	{\footnotesize Fuente:  Elaborado por Sesan con datos de Sicoin-Minfin, 2016.}\\[.1cm]
	%	{\parbox{13cm}{\footnotesize \textbf{Notas:} Se denomina desnutrición severa cuando los niños están 3 desviaciones estándar o más por debajo de la media, de acuerdo a la tabla de medidas de la OMS. }}\\[.3cm]
	%	{\parbox{13cm}{\footnotesize Se denomina desnutrición total cuando los niños están dos desviaciones estándar o más por debajo de la media. Incluye a los niños que están 3 desviaciones estándar o más por debajo de la media.}}\\
}


%%%%%%%%%%%%%%%%%%%4



\hoja{
	\normalsize
	{\Bold\color{color1!80!black}{Cuadro \theCuadro $\,-$ Presupuesto en Seguridad Alimentaria y Nutricional (SAN), del Pacto Hambre Cero (PH0);}}\\
	{\Bold\color{color1!80!black}{ por criterios de seguimiento y porcentaje de ejecución, según institución. }}\\
	{\Bold\color{color1!80!black}{República de Guatemala, año 2015.}}\\
	{\color{color1!80!black}{(Quetzales y porcentaje)}}\\[-0.3cm]
	\begin{center}\fontsize{3.0mm}{1.3em}\selectfont \setlength{\arrayrulewidth}{0.7pt}
		\begin{tabular}{x{4.0cm}rrrc}
			\hline &&&&\\[-0.4cm]  
			\multicolumn{1}{x{2.0cm}}{\raisebox{-.3cm}{\textbf{Institución}}} &	\multicolumn{3}{x{6.3cm}}{\textbf{Criterios de seguimiento (en quetzales)}}&\multicolumn{1}{x{2cm}}{\multirow{2}{*}[-.5mm]{\textbf{Ejecución}}}\\[0.05cm]\cline{2-4}
			\multicolumn{1}{l}{ }&\multicolumn{4}{l}{ }\\[-.35cm]
			\multicolumn{1}{x{2.0cm}}{ } &	\multicolumn{1}{x{2cm}}{\textbf{Asignado}}&\multicolumn{1}{x{2cm}}{\textbf{Vigente}}&\multicolumn{1}{x{2cm}}{\textbf{Ejecutado}}&\multicolumn{1}{x{2cm}}{\textbf{(Porcentaje)}}\\[0.05cm]\hline
			\rowcolor{color1!40!white} \multicolumn{1}{l}{\Bold{	Total presupuesto	 }}& 	5,433,883,259	&	5,342,538,764	&	3,560,292,421	&	66.6	\\
			\rowcolor{color1!20!white} \multicolumn{1}{l}{\Bold{	Ministerios	 }}& 	4,833,238,682	&	4,687,601,799	&	3,235,010,865	&	69.0	\\
			\multicolumn{1}{l}{	Mineduc	 }& 	733,498,088	&	573,507,563	&	573,099,895	&	99.9	\\
			\rowcolor{color1!5!white}\multicolumn{1}{l}{	Mspas	 }& 	1,225,659,293	&	1,590,091,837	&	1,251,483,963	&	78.7	\\
			\multicolumn{1}{l}{	MAGA	 }& 	735,423,108	&	738,297,821	&	368,724,638	&	49.9	\\
			\rowcolor{color1!5!white}\multicolumn{1}{l}{	Micivi	 }& 	1,315,185,738	&	1,211,547,974	&	613,552,025	&	50.6	\\
			\multicolumn{1}{l}{	MARN	 }& 	6,064,342	&	6,264,563	&	5,616,550	&	89.7	\\
			\rowcolor{color1!5!white}\multicolumn{1}{l}{	Mides	 }& 	817,408,113	&	567,892,041	&	422,533,794	&	74.4	\\
			\rowcolor{color1!20!white} \multicolumn{1}{l}{\Bold{	Secretarías	 }}& 	 90,240,558 	&	 113,289,469 	&	 103,343,636 	&	91.2	\\
			\multicolumn{1}{l}{	SCEP	 }& 	3,068,045	&	2,863,616	&	2,540,717	&	88.7	\\
			\rowcolor{color1!5!white}\multicolumn{1}{l}{	Sosep	 }& 	0	&	54,448,531	&	47,897,317	&	88.0	\\
			\multicolumn{1}{l}{	Sesan	 }& 	85,191,335	&	54,052,253	&	50,980,532	&	94.3	\\
			\rowcolor{color1!5!white}\multicolumn{1}{l}{	SBS	 }& 	1,981,178	&	1,925,069	&	1,925,069	&	100.0	\\
			\rowcolor{color1!20!white} \multicolumn{1}{l}{\Bold{	Descentralizadas	 }}& 	 510,404,019 	&	 541,647,496 	&	 221,937,920 	&	41.0	\\
			\multicolumn{1}{l}{	ICTA	 }& 	37,500,000	&	37,500,000	&	32,331,296	&	86.2	\\
			\rowcolor{color1!5!white}\multicolumn{1}{l}{	Infom	 }& 	208,288,358	&	249,401,014	&	36,765,567	&	14.7	\\
			\multicolumn{1}{l}{	Conalfa	 }& 	174,973,497	&	162,404,318	&	125,585,659	&	77.3	\\
			\rowcolor{color1!5!white}\multicolumn{1}{l}{	Indeca	 }& 	17,000,000	&	19,700,000	&	11,148,774	&	56.6	\\
			\multicolumn{1}{l}{	Fontierras	 }& 	72,642,164	&	72,642,164	&	16,106,625	&	22.2	\\
			[0.05cm]
			\hline
			&&&&\\[-0.36cm]\end{tabular}\addtocounter{Cuadro}{1}
	\end{center}
	{\footnotesize Fuente:  Elaborado por Sesan con datos de Sicoin-Minfin, 2016.}\\[.1cm]
	%	{\parbox{13cm}{\footnotesize \textbf{Notas:} Se denomina desnutrición severa cuando los niños están 3 desviaciones estándar o más por debajo de la media, de acuerdo a la tabla de medidas de la OMS. }}\\[.3cm]
	%	{\parbox{13cm}{\footnotesize Se denomina desnutrición total cuando los niños están dos desviaciones estándar o más por debajo de la media. Incluye a los niños que están 3 desviaciones estándar o más por debajo de la media.}}\\
}





%%%%%%%%%%%%%%%%%%% 5

\newpage
{\Bold\Large 7.2	Seguimiento mensual del gasto del Plan del Pacto Hambre Cero. Años 2013 a 2015}
$\ $\\[1cm]
\fontsize{4mm}{1.9em}\selectfont \setlength{\arrayrulewidth}{01pt}
$\ $\\[-1.8cm]
%	{\Bold\color{color1!80!black}{Cuadro \theCuadro $\,-$  Mujeres embarazadas al momento de la encuesta, que recibieron atención pre natal, por establecimiento o lugar a donde asistieron; según características varias. }}\\
%	{\Bold\color{color1!80!black}{República de Guatemala, año 2008/2009. }}\\
%	\normalsize (Porcentajes)\\[0.4cm]
\begin{center}\fontsize{4mm}{1.8em}
	\selectfont \setlength{\arrayrulewidth}{1pt}
	$\ $\\[-3.5cm]
	$\!$\begin{longtable}{x{4.0cm}rrrc}
		\begin{tabular}{l}\fontsize{3mm}{0.6em}
			\selectfont \setlength{\arrayrulewidth}{1pt}
			$\ $\\[-1.0cm]
			\multicolumn{1}{p{14cm}}{\Bold\color{color1!80!black}{Cuadro \theCuadro $\,-$   Presupuesto en Seguridad Alimentaria y Nutricional (SAN), del Plan del Pacto Hambre Cero (PH0), por criterios de seguimiento asignado, vigente y ejecutado y porcentaje de ejecución; según actividad presupuestaria.  }}\\[-0.09cm]
			\multicolumn{1}{p{14cm}}{\Bold\color{color1!80!black}{República de Guatemala, año 2012. }}\\[-0.1cm]
			\multicolumn{1}{p{14cm}}{\normalsize (Porcentajes)}\\[0.4cm]
		\end{tabular}\\
		%			\fontsize{4mm}{1.8em}
		%			\selectfont \setlength{\arrayrulewidth}{1pt}	
		\hline &&&&\\[-0.4cm]  
		\multicolumn{1}{x{2.0cm}}{\raisebox{-.3cm}{\textbf{Actividad presupuestaria}}} &	\multicolumn{3}{x{6.3cm}}{\textbf{Criterios de seguimiento (en quetzales)}}&\multicolumn{1}{x{2cm}}{\multirow{2}{*}[-.5mm]{\textbf{Ejecución}}}\\[0.05cm]\cline{2-4}
		\multicolumn{1}{l}{ }&\multicolumn{4}{l}{ }\\[-.35cm]
		\multicolumn{1}{x{2.0cm}}{ } &	\multicolumn{1}{x{2cm}}{\textbf{Asignado}}&\multicolumn{1}{x{2cm}}{\textbf{Vigente}}&\multicolumn{1}{x{2cm}}{\textbf{Ejecutado}}&\multicolumn{1}{x{2cm}}{\textbf{(Porcentaje)}}\\[0.05cm]\hline\\[-0.1cm]
		%			&&&&&&&&&&&&&&&& \\[-0.1cm]
		\multicolumn{1}{l}{$\ $} &  \multicolumn{4}{c}{$\ $} \\[-0.48cm]				     
		\hline\endhead
		\hline \multicolumn{5}{r}{\textit{Continúa en la siguiente página}} \\[2cm]
		\endfoot
		%			&&&&\tiny&&&&\tiny&&&&\tiny&&&& \\[-0.3cm]
		\multicolumn{5}{l}{\footnotesize Fuente: Elaborado por Sesan con datos de Sicoin-Minfin, 2016.}\\[-0.1cm]
		%			\multicolumn{5}{l}{\footnotesize * Menos de 25 casos. }\\[-0.1cm]
		\endlastfoot
		\rowcolor{color1!20!white} \multicolumn{1}{p{5.5cm}}{\Bold{	Total presupuesto PPH0	}}&	4,733,725,081	&	5,601,097,205	&	4,995,879,003	&	89.2	\\
		\multicolumn{1}{p{5.5cm}}{	Provisión de alimentación escolar	}&	638,253,918	&	560,143,362	&	558,267,850	&	99.7	\\
		\rowcolor{color1!5!white}\multicolumn{1}{p{5.5cm}}{	Inmunizaciones 	}&	277,051,555	&	294,015,937	&	236,181,729	&	80.3	\\
		\multicolumn{1}{p{5.5cm}}{	Prevención y control de la desnutrición 	}&	45,192,430	&	105,903,807	&	97,396,345	&	92.0	\\
		\rowcolor{color1!5!white}\multicolumn{1}{p{5.5cm}}{	Prevención y promoción de la salud reproductiva 	}&	141,319,770	&	159,183,027	&	112,598,441	&	70.7	\\
		\multicolumn{1}{p{5.5cm}}{	Fortalecimiento a medicamentos (ventana de los 100 días)	}&	78,000,000	&	82,642,733	&	79,234,957	&	95.9	\\
		\rowcolor{color1!5!white}\multicolumn{1}{p{5.5cm}}{	Vigilancia del sistema de salud 	}&	46,822,839	&	24,779,646	&	23,246,682	&	93.8	\\
		\multicolumn{1}{p{5.5cm}}{	Administración de la salud ambiental 	}&	2,366,204	&	3,358,231	&	3,356,524	&	99.9	\\
		\rowcolor{color1!5!white}\multicolumn{1}{p{5.5cm}}{	Desarrollo de la micro, pequeña y mediana empresa	}&	38,135,934	&	41,728,283	&	37,973,903	&	91.0	\\
		\multicolumn{1}{p{5.5cm}}{	Servicios de protección al consumidor	}&	18,333,477	&	16,728,307	&	16,633,582	&	99.4	\\
		\rowcolor{color1!5!white}\multicolumn{1}{p{5.5cm}}{	Regulaciones de asuntos laborales y de empleo 	}&	10,344,362	&	3,181,159	&	2,898,509	&	91.1	\\
		\multicolumn{1}{p{5.5cm}}{	Seguridad alimentaria nutricional	}&	155,818,837	&	58,181,006	&	54,201,097	&	93.2	\\
		\rowcolor{color1!5!white}\multicolumn{1}{p{5.5cm}}{	Disponibilidad de alimentos	}&	91,170,462	&	38,938,703	&	34,972,121	&	89.8	\\
		\multicolumn{1}{p{5.5cm}}{	Desarrollo económico rural agropecuario	}&	904,271,872	&	578,255,366	&	538,698,022	&	93.2	\\
		\rowcolor{color1!5!white}\multicolumn{1}{p{5.5cm}}{	Servicios de coordinación regional y extensión rural	}&	29,471,501	&	21,285,264	&	20,104,255	&	94.5	\\
		\multicolumn{1}{p{5.5cm}}{	Dirección y coordinación desarrollo de la infraestructura vial	}&	67,829,781	&	89,748,090	&	82,554,562	&	92.0	\\
		\rowcolor{color1!5!white}\multicolumn{1}{p{5.5cm}}{	Construcción, ampliación, rehabilitación y pavimentación de carreteras primarias, puentes y distribuidores de tránsito	}&	1,377,977,242	&	1,806,476,676	&	1,588,770,958	&	87.9	\\
		\multicolumn{1}{p{5.5cm}}{	Sistema integrado de gestión ambiental nacional	}&	15,284,512	&	24,472,330	&	22,225,484	&	90.8	\\
		\rowcolor{color1!5!white}\multicolumn{1}{p{5.5cm}}{	Gestión socio-ambiental desconcentrada	}&	26,054,147	&	21,588,533	&	21,288,913	&	98.6	\\
		\multicolumn{1}{p{5.5cm}}{	Conservación y protección de los recursos naturales	}&	32,194,642	&	4,697,990	&	4,695,813	&	100.0	\\
		\rowcolor{color1!5!white}\multicolumn{1}{p{5.5cm}}{	Sustentabilidad ambiental	}&	105,000	&	105,000	&	102,097	&	97.2	\\
		\multicolumn{1}{p{5.5cm}}{	Adaptación y mitigación al cambio climático	}&	12,930,000	&	728,724	&	604,658	&	83.0	\\
		\rowcolor{color1!5!white}\multicolumn{1}{p{5.5cm}}{	Actividades centrales de desarrollo social (Mides)	}&	0	&	878,049,679	&	809,147,283	&	92.2	\\
		\multicolumn{1}{p{5.5cm}}{	Protección al adulto mayor	}&	0	&	25,936,736	&	25,258,180	&	97.4	\\
		\rowcolor{color1!5!white}\multicolumn{1}{p{5.5cm}}{	Hambre cero (Mides)	}&	0	&	133,566,365	&	123,696,743	&	92.6	\\
		\multicolumn{1}{p{5.5cm}}{	Familias seguras	}&	0	&	24,870,407	&	19,321,687	&	77.7	\\
		\rowcolor{color1!5!white}\multicolumn{1}{p{5.5cm}}{	Empleabilidad de jóvenes 	}&	0	&	22,498,890	&	7,711,993	&	34.3	\\
		\multicolumn{1}{p{5.5cm}}{	Productividad rural	}&	0	&	5,699,508	&	5,343,419	&	93.8	\\
		\rowcolor{color1!5!white}\multicolumn{1}{p{5.5cm}}{	Rectoria y coordinación social	}&	0	&	1,944,044	&	442,901	&	22.8	\\
		\multicolumn{1}{p{5.5cm}}{	Coordinación de politicas y proyectos de desarrollo	}&	45,954,770	&	44,450,835	&	42,821,670	&	96.3	\\
		\rowcolor{color1!5!white}\multicolumn{1}{p{5.5cm}}{	Actividadaes de bienestar social 	}&	16,241,606	&	15,418,284	&	15,037,478	&	97.5	\\
		\multicolumn{1}{p{5.5cm}}{	Fortalecimiento y apoyo familiar y comunitario 	}&	98,628,712	&	48,674,625	&	47,360,182	&	97.3	\\
		\rowcolor{color1!5!white}\multicolumn{1}{p{5.5cm}}{	Proteccion, abrigo y rehabilitación familiar	}&	90,074,760	&	50,533,158	&	45,316,195	&	89.7	\\
		\multicolumn{1}{p{5.5cm}}{	Obras sociales (Sosep)	}&	155,892,084	&	102,139,270	&	100,071,030	&	98.0	\\
		\rowcolor{color1!5!white}\multicolumn{1}{p{5.5cm}}{	Coordinación de seguridad alimentaria y nutricional	}&	42,662,833	&	37,997,397	&	37,530,890	&	98.8	\\
		\multicolumn{1}{p{5.5cm}}{	Actividades centrales de gestión de riesgo (Conred)	}&	4,092,582	&	6,184,586	&	6,085,051	&	98.4	\\
		\rowcolor{color1!5!white}\multicolumn{1}{p{5.5cm}}{	Gestión de riesgo 	}&	1,944,000	&	0	&	0	&	0.0	\\
		\multicolumn{1}{p{5.5cm}}{	Emergencia alimentaria en Jalapa	}&	2,430,000	&	116,000	&	77,945	&	67.2	\\
		\rowcolor{color1!5!white}\multicolumn{1}{p{5.5cm}}{	Ciencia, tecnología e innovación agrícola	}&	42,000,000	&	42,000,000	&	28,170,923	&	67.1	\\
		\multicolumn{1}{p{5.5cm}}{	Alfabetización	}&	204,875,249	&	204,875,249	&	137,433,878	&	67.1	\\
		\rowcolor{color1!5!white}\multicolumn{1}{p{5.5cm}}{	Actividades centrales de asistencia alimentaria  (Indeca)	}&	4,302,355	&	4,395,530	&	3,049,136	&	69.4	\\
		\multicolumn{1}{p{5.5cm}}{	Apoyo a la asistencia alimentaria	}&	13,535,328	&	9,442,153	&	5,892,500	&	62.4	\\
		\rowcolor{color1!5!white}\multicolumn{1}{p{5.5cm}}{	Beneficiado de granos básicos	}&	2,162,317	&	6,162,317	&	103,416	&	1.7	\\
		\hline
		&&&&\\[-0.28cm]
	\end{longtable}\addtocounter{Cuadro}{1}
\end{center}




%%%%%%%%%%%%%%%%%%%6

%%%%%%%%%%%%%%%%%%


\hoja{
	\normalsize
	{\Bold\color{color1!80!black}{Cuadro \theCuadro $\,-$Ejecución del Plan del Pacto Hambre Cero (PPH0);según componente y porcentaje de ejecución.}}\\
	{\Bold\color{color1!80!black}{República de Guatemala, año 2013.}}\\
	{\color{color1!80!black}{(Quetzales y porcentaje)}}\\[-0.3cm]
	\begin{center}\fontsize{3.0mm}{1.3em}\selectfont \setlength{\arrayrulewidth}{0.7pt}
		\begin{tabular}{x{5.5cm}rrc}
			\hline &&&\\[-0.4cm]  
			\multicolumn{1}{x{5.5cm}}{\raisebox{-.3cm}{\textbf{Componentes}}} &	\multicolumn{2}{x{4.6cm}}{\textbf{Criterios de seguimiento (en quetzales)}}&\multicolumn{1}{x{2cm}}{\multirow{2}{*}[-.5mm]{\textbf{Ejecución}}}\\[0.05cm]\cline{2-3}
			\multicolumn{1}{l}{ }&\multicolumn{3}{l}{ }\\[-.35cm]
			\multicolumn{1}{c}{ }&\multicolumn{1}{x{2cm}}{\textbf{Vigente}}&\multicolumn{1}{x{2cm}}{\textbf{Ejecutado}}&\multicolumn{1}{x{2cm}}{\textbf{(Porcentaje)}}\\[0.05cm]\hline
			\rowcolor{color1!40!white} \multicolumn{1}{l}{\Bold{	Total presupuesto	}}&	6,099,516,355	&	4,544,224,464	&	74.5	\\
			\multicolumn{1}{l}{	Provisión de servicios básicos de salud y nutrición	}&	947,390,200	&	874,218,707	&	92.3	\\
			\rowcolor{color1!5!white}\multicolumn{1}{l}{	Promoción de lactancia materna y alimentación complementaria	}&	33,783,811	&	33,783,811	&	100.0	\\
			\multicolumn{1}{l}{	Alimentos fortificados	}&	71,166,002	&	69,866,549	&	98.2	\\
			\rowcolor{color1!5!white}\multicolumn{1}{l}{	Atención a población vulnerable a la inseguridad alimentaria	}&	690,718,853	&	503,874,164	&	72.9	\\
			\multicolumn{1}{l}{	Mejoramiento de los ingresos y la economía familia	}&	2,872,537,433	&	2,142,793,380	&	74.6	\\
			\rowcolor{color1!5!white}\multicolumn{1}{l}{	Agua y saneamiento	}&	404,554,198	&	74,182,320	&	18.3	\\
			\multicolumn{1}{l}{	Gobernanza local	}&	34,886,190	&	36,719,499	&	105.3	\\
			\rowcolor{color1!5!white}\multicolumn{1}{l}{	Escuelas saludables	}&	771,997,976	&	603,452,041	&	78.2	\\
			\multicolumn{1}{l}{	Hogar saludable	}&	153,125,923	&	153,125,923	&	100.0	\\
			\rowcolor{color1!5!white}\multicolumn{1}{l}{	Alfabetización	}&	60,344,935	&	52,208,071	&	86.5	\\
			\multicolumn{1}{l}{	Ejes transversales 	}&	59,010,835	&	31,455,060	&	53.3	\\
			[0.05cm]
			\hline
			&&&\\[-0.36cm]\end{tabular}\addtocounter{Cuadro}{1}
	\end{center}
	{\footnotesize Fuente:  Elaborado por Sesan con datos de Sicoin-Minfin, 2016.}\\[.1cm]
	%	{\parbox{13cm}{\footnotesize \textbf{Notas:} Se denomina desnutrición severa cuando los niños están 3 desviaciones estándar o más por debajo de la media, de acuerdo a la tabla de medidas de la OMS. }}\\[.3cm]
	%	{\parbox{13cm}{\footnotesize Se denomina desnutrición total cuando los niños están dos desviaciones estándar o más por debajo de la media. Incluye a los niños que están 3 desviaciones estándar o más por debajo de la media.}}\\
}

%%%%%%%%%%%%%%%%%%%%%%7



\hoja{
	\normalsize
	{\Bold\color{color1!80!black}{Cuadro \theCuadro $\,-$ Ejecución del Plan del Pacto Hambre Cero (PPH0); por criterios de seguimiento y porcentaje de ejecución, según componentes y eje transversal. }}\\
	{\Bold\color{color1!80!black}{República de Guatemala, año 2014.}}\\
	{\color{color1!80!black}{(Quetzales y porcentaje)}}\\[-0.3cm]
	\begin{center}\fontsize{3.0mm}{1.3em}\selectfont \setlength{\arrayrulewidth}{0.7pt}
		\begin{tabular}{x{4.0cm}rrrc}
			\hline &&&&\\[-0.4cm]  
			\multicolumn{1}{x{2.0cm}}{\raisebox{-.3cm}{\textbf{Componente, eje transversal y grupo institucional }}} &	\multicolumn{3}{x{6.3cm}}{\textbf{Criterios de seguimiento (en quetzales)}}&\multicolumn{1}{x{2cm}}{\multirow{2}{*}[-.5mm]{\textbf{Ejecución}}}\\[0.05cm]\cline{2-4}
			\multicolumn{1}{l}{ }&\multicolumn{4}{l}{ }\\[-.35cm]
			\multicolumn{1}{x{2.0cm}}{ } &	\multicolumn{1}{x{2cm}}{\textbf{Asignado}}&\multicolumn{1}{x{2cm}}{\textbf{Vigente}}&\multicolumn{1}{x{2cm}}{\textbf{Ejecutado}}&\multicolumn{1}{x{2cm}}{\textbf{(Porcentaje)}}\\[0.05cm]\hline
			\rowcolor{color1!40!white} \multicolumn{1}{l}{\Bold{	Total presupuesto	}}&	5,249,682,137	&	6,587,699,083	&	5,615,711,668	&	85.2	\\
			\rowcolor{color1!20!white} \multicolumn{1}{l}{\Bold{	Componentes directos	}}&	1,259,691,823	&	1,728,988,117	&	1,545,343,020	&	89.4	\\
			\multicolumn{1}{l}{	Provisión de servicios básicos de salud y nutrición	}&	594,114,453	&	869,376,567	&	736,564,012	&	84.7	\\
			\rowcolor{color1!5!white}\multicolumn{1}{l}{	Promoción de lactancia materna y alimentación complementaria	}&	114,604,369	&	111,669,386	&	89,710,589	&	80.3	\\
			\multicolumn{1}{l}{	Alimentos fortificados	}&	0	&	60,202,921	&	58,046,796	&	96.4	\\
			\rowcolor{color1!5!white}\multicolumn{1}{l}{	Atención a población vulnerable a la inseguridad alimentaria	}&	550,973,001	&	687,739,243	&	661,021,623	&	96.1	\\
			\rowcolor{color1!20!white} \multicolumn{1}{l}{\Bold{	Componente de viabilidad	}}&	3,944,012,760	&	4,793,270,087	&	4,017,334,741	&	83.8	\\
			\multicolumn{1}{l}{	Mejoramiento de los ingresos y la economía familiar	}&	2,650,710,234	&	3,323,679,636	&	3,008,484,214	&	90.5	\\
			\rowcolor{color1!5!white}\multicolumn{1}{l}{	Agua y saneamiento	}&	365,267,394	&	410,584,483	&	112,495,918	&	27.4	\\
			\multicolumn{1}{l}{	Gobernanza local	}&	544,200	&	544,200	&	510,836	&	93.9	\\
			\rowcolor{color1!5!white}\multicolumn{1}{l}{	Escuelas saludables	}&	732,802,593	&	734,161,512	&	629,167,958	&	85.7	\\
			\multicolumn{1}{l}{	Hogar saludable	}&	0	&	178,917,340	&	159,074,023	&	88.9	\\
			\rowcolor{color1!5!white}\multicolumn{1}{l}{	Alfabetización	}&	194,688,339	&	145,382,916	&	107,601,793	&	74.0	\\
			\rowcolor{color1!20!white} \multicolumn{1}{l}{\Bold{	Eje transversal	}}&	45977554	&	65440879	&	53,033,906	&	81.0	\\
			\multicolumn{1}{l}{	Coordinación interinstitucional	}&	12,398,669	&	25,359,495	&	19,708,563	&	77.7	\\
			\rowcolor{color1!5!white}\multicolumn{1}{l}{	Comunicación para la seguridad alimentaria y nutricional	}&	2,165,900	&	3,399,000	&	1,960,205	&	57.7	\\
			\multicolumn{1}{l}{	Participación comunitaria	}&	0	&	31,262	&	25,737	&	82.3	\\
			\rowcolor{color1!5!white}\multicolumn{1}{l}{	Sistema de monitoreo y evaluación	}&	31,412,985	&	36,651,122	&	31,339,402	&	85.5	\\
			[0.05cm]
			\hline
			&&&&\\[-0.36cm]\end{tabular}\addtocounter{Cuadro}{1}
	\end{center}
	{\footnotesize Fuente:  Elaborado por Sesan con datos de Sicoin-Minfin, 2016.}\\[.1cm]}


%%%%%%%%%%%%%%%%%%%%7.1




\begin{landscape}%\fontsize{4mm}{1.9em}\selectfont \setlength{\arrayrulewidth}{01pt}
	%	$\ $\\[-1.8cm]
	%	{\Bold\color{color1!80!black}{Cuadro \theCuadro $\,-$  Mujeres embarazadas al momento de la encuesta, que recibieron atención pre natal, por establecimiento o lugar a donde asistieron; según características varias. }}\\
	%	{\Bold\color{color1!80!black}{República de Guatemala, año 2008/2009. }}\\
	%	\normalsize (Porcentajes)\\[0.4cm]
	
	{\Bold\color{color1!80!black}{Cuadro \theCuadro $\,-$ Ejecución del Plan del Pacto Hambre Cero (PPH0); por mes, según componentes y eje transversal. }}\\
	{\Bold\color{color1!80!black}{República de Guatemala, año 2013.}}\\
	{\color{color1!80!black}{(Quetzales y porcentaje)}}\\[-0.3cm]
	%			\begin{center}\fontsize{3.0mm}{1.3em}\selectfont \setlength{\arrayrulewidth}{0.7pt}
	\begin{center}\fontsize{3mm}{1.5em}
		\selectfont \setlength{\arrayrulewidth}{1pt}
		%		$\ $\\[-2.0cm]
		$\!$\begin{longtable}{x{2.0cm}rrrrrrrrrrrr}
			&&&&\\[-2.5cm] 
			\hline &&&&\\[-0.45cm]  
			\multicolumn{1}{x{3.5cm}}{{\textbf{Componente, eje transversal}}} &	\multicolumn{12}{c}{\textbf{Mes}}\\[0.05cm]\cline{2-13}
			\multicolumn{1}{c}{\textbf{ y grupo institucional }}&\multicolumn{1}{l}{\textbf{Enero}}&\multicolumn{1}{l}{\textbf{Febrero}}&\multicolumn{1}{l}{\textbf{Marzo}}&\multicolumn{1}{l}{\textbf{Abril}}&\multicolumn{1}{l}{\textbf{Mayo}}&\multicolumn{1}{l}{\textbf{Junio}}&\multicolumn{1}{l}{\textbf{Julio}}&\multicolumn{1}{l}{\textbf{Agosto}}&\multicolumn{1}{l}{\textbf{Septiembre}}&\multicolumn{1}{l}{\textbf{Octubre}}&\multicolumn{1}{l}{\textbf{Noviembre}}&\multicolumn{1}{l}{\textbf{Diciembre}}\\%[-.01cm]
			\hline\endhead
			\hline \multicolumn{5}{r}{\textit{Continúa en la siguiente página}} \\
			\endfoot
				\hline \multicolumn{5}{l}{\textbf{Fuente}: Elaborado por Sesan con datos de Sicoin-Minfin, 2016.}\endlastfoot
			\rowcolor{color1!40!white} \multicolumn{1}{p{3.5cm}}{\Bold{	Total presupuesto	}}&	54,616,315	&	130,112,066	&	430,724,684	&	325,802,737	&	671,686,015	&	654,173,041	&	448,632,476	&	1,020,932,077	&	180,229,138	&	339,831,765	&	398,130,050	&	960,841,304	\\
			\rowcolor{color1!20!white} \multicolumn{1}{p{3.5cm}}{\Bold{	Componentes directos	}}&	11,683,854	&	75,988,473	&	132,594,381	&	88,894,411	&	83,287,142	&	219,571,238	&	111,590,673	&	225,220,545	&	105,394,399	&	86,636,049	&	173,916,657	&	230,565,198	\\
			\multicolumn{1}{p{3.5cm}}{	Provisión de servicios básicos de salud y nutrición	}&	3,440,258	&	41,909,945	&	55,047,845	&	36,451,003	&	34,808,538	&	162,997,252	&	51,403,705	&	107,006,824	&	55,532,142	&	40,259,307	&	86,394,594	&	61,312,599	\\
			\rowcolor{color1!5!white}\multicolumn{1}{p{3.5cm}}{	Promoción de lactancia materna y alimentación complementaria	}&	1,239,084	&	291,631	&	9,858,805	&	7,812,985	&	6,427,600	&	9,674,710	&	10,655,321	&	11,178,842	&	8,884,598	&	4,261,486	&	9,529,662	&	9,895,867	\\
			\multicolumn{1}{p{3.5cm}}{	Alimentos fortificados	}&	0	&	0	&	10,000	&	0	&	10,065,682	&	1,704,410	&	11,539,375	&	9,137,907	&	23,017	&	12,631,817	&	855,278	&	12,079,310	\\
			\rowcolor{color1!5!white}\multicolumn{1}{p{3.5cm}}{	Atención a población vulnerable a la inseguridad alimentaria	}&	7,004,512	&	33,786,898	&	67,677,732	&	44,630,423	&	31,985,321	&	45,194,866	&	37,992,273	&	97,896,972	&	40,954,642	&	29,483,440	&	77,137,123	&	147,277,422	\\
			\rowcolor{color1!20!white} \multicolumn{1}{p{3.5cm}}{\Bold{	Componente de viabilidad	}}&	39,725,031	&	51,050,677	&	294,691,653	&	233,515,519	&	584,904,826	&	430,646,690	&	329,301,433	&	788,450,750	&	71,210,934	&	249,705,000	&	220,674,722	&	723,457,506	\\
			\multicolumn{1}{p{3.5cm}}{	Mejoramiento de los ingresos y la economía familiar	}&	4,146,574	&	27,525,533	&	83,185,166	&	70,625,999	&	485,392,233	&	276,613,158	&	282,765,407	&	635,189,237	&	50,488,643	&	230,519,191	&	179,361,674	&	682,671,398	\\
			\rowcolor{color1!5!white}\multicolumn{1}{p{3.5cm}}{	Agua y saneamiento	}&	1,081,808	&	1,411,793	&	5,542,666	&	7,915,141	&	11,851,358	&	7,205,700	&	11,916,707	&	17,584,470	&	9,493,964	&	8,464,478	&	7,742,226	&	22,285,605	\\
			\multicolumn{1}{p{3.5cm}}{	Gobernanza local	}&	650	&	0	&	70,295	&	56,570	&	0	&	32,668	&	38,110	&	111,779	&	18,024	&	22,525	&	68,007	&	92,208	\\
			\rowcolor{color1!5!white}\multicolumn{1}{p{3.5cm}}{	Escuelas saludables	}&	0	&	15,400,665	&	162,830,031	&	125,498,367	&	47,861,592	&	109,059,820	&	16,266,118	&	117,813,212	&	2,127,154	&	5,215,473	&	23,430,344	&	3,665,182	\\
			\multicolumn{1}{p{3.5cm}}{	Hogar saludable	}&	34,496,000	&	0	&	34,281,096	&	23,136,500	&	27,566,175	&	25,345,100	&	4,983,252	&	5,236,000	&	5,914,500	&	0	&		&	-1,884,600	\\
			\rowcolor{color1!5!white}\multicolumn{1}{p{3.5cm}}{	Alfabetización	}&	0	&	6,712,685	&	8,782,399	&	6,282,942	&	12,233,468	&	12,390,243	&	13,331,839	&	12,516,052	&	3,168,650	&	5,483,332	&	10,072,470	&	16,627,712	\\
			\rowcolor{color1!20!white} \multicolumn{1}{p{3.5cm}}{\Bold{	Eje transversal	}}&	3,207,430	&	3,072,916	&	3,438,650	&	3,392,807	&	3,494,047	&	3,955,113	&	7,740,369	&	7,260,781	&	3,623,805	&	3,490,716	&	3,538,671	&	6,818,600	\\
			\multicolumn{1}{p{3.5cm}}{	Coordinación interinstitucional	}&	903,773	&	888,860	&	1,179,139	&	1,081,071	&	1,203,211	&	1,412,805	&	2,894,328	&	2,808,843	&	1,345,139	&	1,138,756	&	1,145,240	&	3,707,397	\\
			\rowcolor{color1!5!white}\multicolumn{1}{p{3.5cm}}{	Comunicación para la seguridad alimentaria y nutricional	}&	147,543	&	132,877	&	113,173	&	107,796	&	118,931	&	117,641	&	195,551	&	154,062	&	195,999	&	209,199	&	152,763	&	314,671	\\
			\multicolumn{1}{p{3.5cm}}{	Participación comunitaria	}&	0	&	0	&	0	&	0	&	0	&	0	&	0	&	1,737	&	0	&	3,250	&	0	&	20,750	\\
			\rowcolor{color1!5!white}\multicolumn{1}{p{3.5cm}}{	Sistema de monitoreo y evaluación	}&	2,156,113	&	2,051,178	&	2,146,338	&	2,203,939	&	2,171,905	&	2,424,666	&	4,650,491	&	4,296,140	&	2,082,668	&	2,139,512	&	2,240,668	&	2,775,783	\\[-0.1cm]
		\end{longtable}\addtocounter{Cuadro}{1}
	\end{center}
\end{landscape}






%%%%%%%%%%%%%%%%%%%8

%
%
%\hoja{
%	\normalsize
%	{\Bold\color{color1!80!black}{Cuadro \theCuadro $\,-$ Ejecución del plan del pacto hambre cero (PPH0);  por criterios de seguimiento, mes y porcentaje de ejecución, según institución (ministerios, secretarías, descentralizadas) y grupo institucional. }}\\
%	{\Bold\color{color1!80!black}{República de Guatemala, año 2014.}}\\
%	{\color{color1!80!black}{(Quetzales y porcentaje)}}\\[-0.3cm]
%	\begin{center}\fontsize{3.0mm}{1.3em}\selectfont \setlength{\arrayrulewidth}{0.7pt}
%		\begin{tabular}{x{4.0cm}rrrc}
%			\hline &&&&\\[-0.4cm]  
%			\multicolumn{1}{x{2.0cm}}{\raisebox{-.3cm}{\textbf{Componente, eje transversal y grupo institucional }}} &	\multicolumn{3}{x{6.3cm}}{\textbf{Criterios de seguimiento (en quetzales)}}&\multicolumn{1}{x{2cm}}{\multirow{2}{*}[-.5mm]{\textbf{Ejecución}}}\\[0.05cm]\cline{2-4}
%			\multicolumn{1}{l}{ }&\multicolumn{4}{l}{ }\\[-.35cm]
%			\multicolumn{1}{x{2.0cm}}{ } &	\multicolumn{1}{x{2cm}}{\textbf{Asignado}}&\multicolumn{1}{x{2cm}}{\textbf{Vigente}}&\multicolumn{1}{x{2cm}}{\textbf{Ejecutado}}&\multicolumn{1}{x{2cm}}{\textbf{(Porcentaje)}}\\[0.05cm]\hline
%			\rowcolor{color1!40!white} \multicolumn{1}{l}{\Bold{	Total presupuesto	}}&	5,249,682,137	&	6,587,699,083	&	5,615,711,668	&	85.2	\\
%			\rowcolor{color1!20!white} \multicolumn{1}{l}{\Bold{	Ministerios	}}&	4,507,844,851	&	5,778,816,812	&	5,175,638,909	&	89.6	\\
%			\multicolumn{1}{l}{	Ministerio de Educación	}&	699,552,593	&	700,740,593	&	618,271,098	&	88.2	\\
%			\rowcolor{color1!5!white}\multicolumn{1}{l}{	Ministerio de Salud Pública y Asistencia Social	}&	663,857,018	&	990,638,317	&	841,129,249	&	84.9	\\
%			\multicolumn{1}{l}{	Ministerio de Agricultura, Ganadería y Alimentación	}&	955,572,082	&	877,715,651	&	790,560,249	&	90.1	\\
%			\rowcolor{color1!5!white}\multicolumn{1}{l}{	Ministerio de Comunicaciones, Infraestructura y Vivienda	}&	1,265,437,403	&	2,133,468,319	&	1,874,645,098	&	87.9	\\
%			\multicolumn{1}{l}{	Ministerio de Ambiente y Recursos Naturales	}&	1,304,562	&	1,350,231	&	525,833	&	38.9	\\
%			\rowcolor{color1!5!white}\multicolumn{1}{l}{	Ministerio de Desarrollo Social	}&	922,121,193	&	1,074,903,701	&	1,050,507,382	&	97.7	\\
%			\rowcolor{color1!20!white} \multicolumn{1}{l}{\Bold{	Secretarías	}}&	93,163,866	&	120,720,925	&	100,761,731	&	83.5	\\
%			\multicolumn{1}{l}{	Secretaría de Coordinación Ejecutiva de la Presidencia	}&	544,200	&	544,200	&	510,836	&	93.9	\\
%			\rowcolor{color1!5!white}\multicolumn{1}{l}{	Secretaría de Obras Sociales de la Esposa del Presidente	}&	43,937,153	&	52,030,887	&	44,785,106	&	86.1	\\
%			\multicolumn{1}{l}{	Secretaría de Seguridad Alimentaria y Nutricional	}&	45,977,554	&	65,440,879	&	53,033,906	&	81.0	\\
%			\rowcolor{color1!5!white}\multicolumn{1}{l}{	Secretaría de Bienestar Social de la Presidencia	}&	2,704,959	&	2,704,959	&	2,431,882	&	89.9	\\
%			\rowcolor{color1!20!white} \multicolumn{1}{l}{\Bold{	Descentralizadas	}}&	670,604,721	&	688,161,346	&	339,311,027	&	49.3	\\
%			\multicolumn{1}{l}{	Instituto de Ciencia y Tecnología Agrícola -ICTA-	}&	38,000,000	&	40,430,500	&	35,150,813	&	86.9	\\
%			\rowcolor{color1!5!white}\multicolumn{1}{l}{	Instituto de Fomento Municipal -Infom-	}&	362,182,524	&	405,108,963	&	107,945,244	&	26.6	\\
%			\multicolumn{1}{l}{	Consejo Nacional de Alfabetización -Conalfa-	}&	194,688,339	&	145,382,916	&	107,601,793	&	74.0	\\
%			\rowcolor{color1!5!white}\multicolumn{1}{l}{	Instituto Nacional de Comercialización Agrícola -Indeca-	}&	12,000,000	&	22,300,000	&	14,489,251	&	65.0	\\
%			\multicolumn{1}{l}{	Fondo de Tierras 	}&	63,733,858	&	74,938,967	&	74,123,927	&	98.9	\\
%			\rowcolor{color1!20!white} \multicolumn{1}{l}{\Bold{	Grupo institucional (total)	}}&	5,249,682,137	&	6,587,699,083	&	5,615,711,668	&	85.2	\\
%			\multicolumn{1}{l}{	Gobierno central	}&	4,579,077,416	&	5,899,537,737	&	5,276,400,640	&	89.4	\\
%			\rowcolor{color1!5!white}\multicolumn{1}{l}{	Descentralizadas	}&	670,604,721	&	688,161,346	&	339,311,027	&	49.5	\\
%			[0.05cm]
%			\hline
%			&&&&\\[-0.36cm]\end{tabular}\addtocounter{Cuadro}{1}
%	\end{center}
%	{\footnotesize Fuente:  Elaborado por Sesan con datos de Sicoin-Minfin, 2016.}\\[.1cm]
%	%	{\parbox{13cm}{\footnotesize \textbf{Notas:} Se denomina desnutrición severa cuando los niños están 3 desviaciones estándar o más por debajo de la media, de acuerdo a la tabla de medidas de la OMS. }}\\[.3cm]
%	%	{\parbox{13cm}{\footnotesize Se denomina desnutrición total cuando los niños están dos desviaciones estándar o más por debajo de la media. Incluye a los niños que están 3 desviaciones estándar o más por debajo de la media.}}\\
%}




%%%%%%%%%%%%%%%%%%%%%8.1




%
%\begin{landscape}%\fontsize{4mm}{1.9em}\selectfont \setlength{\arrayrulewidth}{01pt}
%	%	$\ $\\[-1.8cm]
%	%	{\Bold\color{color1!80!black}{Cuadro \theCuadro $\,-$  Mujeres embarazadas al momento de la encuesta, que recibieron atención pre natal, por establecimiento o lugar a donde asistieron; según características varias. }}\\
%	%	{\Bold\color{color1!80!black}{República de Guatemala, año 2008/2009. }}\\
%	%	\normalsize (Porcentajes)\\[0.4cm]
%	
%	{\Bold\color{color1!80!black}{Cuadro \theCuadro $\,-$ Ejecución del plan del pacto hambre cero (PPH0);  por mes, según institución (ministerios, secretarías, descentralizadas) y grupo institucional. }}\\
%	{\Bold\color{color1!80!black}{República de Guatemala, año 2014.}}\\
%	{\color{color1!80!black}{(Quetzales y porcentaje)}}\\[-0.3cm]
%	%			\begin{center}\fontsize{3.0mm}{1.3em}\selectfont \setlength{\arrayrulewidth}{0.7pt}
%	\begin{center}\fontsize{2.5mm}{1.1em}
%		\selectfont \setlength{\arrayrulewidth}{1pt}
%		%		$\ $\\[-2.0cm]
%		$\!$\begin{longtable}{x{4.5cm}rrrrrrrrrrrr}
%			&&&&\\[-3cm] 
%			\hline &&&&\\[-0.45cm]  
%			\multicolumn{1}{x{4.5cm}}{{\textbf{Componente, eje transversal}}} &	\multicolumn{12}{c}{\textbf{Mes}}\\[0.05cm]\cline{2-13}
%			\multicolumn{1}{c}{\textbf{ y grupo institucional }}&\multicolumn{1}{l}{\textbf{Enero}}&\multicolumn{1}{l}{\textbf{Febrero}}&\multicolumn{1}{l}{\textbf{Marzo}}&\multicolumn{1}{l}{\textbf{Abril}}&\multicolumn{1}{l}{\textbf{Mayo}}&\multicolumn{1}{l}{\textbf{Junio}}&\multicolumn{1}{l}{\textbf{Julio}}&\multicolumn{1}{l}{\textbf{Agosto}}&\multicolumn{1}{l}{\textbf{Septiembre}}&\multicolumn{1}{l}{\textbf{Octubre}}&\multicolumn{1}{l}{\textbf{Noviembre}}&\multicolumn{1}{l}{\textbf{Diciembre}}\\%[-.01cm]
%			\hline\endhead
%			\hline \multicolumn{5}{r}{\textit{Continúa en la siguiente página}} \\
%			\endfoot
%			\rowcolor{color1!40!white} \multicolumn{1}{l}{\Bold{	Total presupuesto	}}&	54,616,315	&	130,112,066	&	430,724,684	&	325,802,737	&	671,686,015	&	654,173,041	&	448,632,476	&	1,020,932,077	&	180,229,138	&	339,831,765	&	398,130,050	&	960,841,304	\\
%			\rowcolor{color1!20!white} \multicolumn{1}{l}{\Bold{	Ministerios	}}&	49,915,812	&	114,745,209	&	399,966,258	&	299,846,069	&	623,249,065	&	623,207,493	&	404,593,566	&	973,170,083	&	154,315,521	&	317,928,950	&	364,997,344	&	849,703,538	\\
%			\multicolumn{1}{p{4.5cm}}{	Ministerio de Educación	}&	0	&	15,400,665	&	162,830,031	&	122,612,017	&	47,861,592	&	108,299,182	&	13,104,148	&	116,621,413	&	2,127,154	&	2,495,519	&	23,254,195	&	3,665,182	\\
%			\rowcolor{color1!5!white}\multicolumn{1}{p{4.5cm}}{	Ministerio de Salud Pública y Asistencia Social	}&	4,694,250	&	42,131,646	&	60,142,163	&	39,362,235	&	46,878,576	&	169,961,384	&	67,104,701	&	121,118,339	&	57,919,578	&	57,286,683	&	92,434,766	&	82,094,929	\\
%			\multicolumn{1}{p{4.5cm}}{	Ministerio de Agricultura, Ganadería y Alimentación	}&	4,153,691	&	7,877,764	&	36,047,278	&	32,708,713	&	62,582,924	&	141,113,324	&	41,268,171	&	167,159,108	&	19,573,670	&	48,408,589	&	30,935,362	&	198,731,656	\\
%			\rowcolor{color1!5!white}\multicolumn{1}{p{4.5cm}}{	Ministerio de Comunicaciones, Infraestructura y Vivienda	}&	34,496,000	&	0	&	34,281,096	&	40,562,850	&	459,942,118	&	158,189,566	&	231,596,109	&	417,990,005	&	36,029,745	&	185,679,440	&	96,139,495	&	179,738,675	\\
%			\multicolumn{1}{p{4.5cm}}{	Ministerio de Ambiente y Recursos Naturales	}&	52,762	&	38,956	&	38,956	&	38,956	&	38,956	&	38,956	&	67,683	&	38,956	&	38,581	&	38,956	&	38,956	&	55,159	\\
%			\rowcolor{color1!5!white}\multicolumn{1}{p{4.5cm}}{	Ministerio de Desarrollo Social	}&	6,519,110	&	49,296,178	&	106,626,734	&	64,561,298	&	5,944,899	&	45,605,081	&	51,452,754	&	150,242,262	&	38,626,793	&	24,019,764	&	122,194,570	&	385,417,939	\\
%			\rowcolor{color1!20!white} \multicolumn{1}{l}{\Bold{	Secretarías	}}&	3,214,562	&	3,162,583	&	9,250,723	&	8,629,457	&	8,152,496	&	8,666,529	&	14,327,745	&	13,921,074	&	10,354,092	&	3,729,548	&	8,366,781	&	8,986,141	\\
%			\multicolumn{1}{p{4.5cm}}{	Secretaría de Coordinación Ejecutiva de la Presidencia	}&	650	&	0	&	70,295	&	56,570	&	0	&	32,668	&	38,110	&	111,779	&	18,024	&	22,525	&	68,007	&	92,208	\\
%			\rowcolor{color1!5!white}\multicolumn{1}{p{4.5cm}}{	Secretaría de Obras Sociales de la Esposa del Presidente	}&	0	&	0	&	5,559,749	&	5,010,036	&	4,373,970	&	4,239,443	&	6,336,026	&	6,296,133	&	6,530,592	&	0	&	4,469,392	&	1,969,765	\\
%			\multicolumn{1}{p{4.5cm}}{	Secretaría de Seguridad Alimentaria y Nutricional	}&	3,207,430	&	3,072,916	&	3,438,650	&	3,392,807	&	3,494,047	&	3,955,113	&	7,740,369	&	7,260,781	&	3,623,805	&	3,490,716	&	3,538,671	&	6,818,600	\\
%			\rowcolor{color1!5!white}\multicolumn{1}{p{4.5cm}}{	Secretaría de Bienestar Social de la Presidencia	}&	6,481	&	89,667	&	182,029	&	170,044	&	284,479	&	439,305	&	213,240	&	252,381	&	181,671	&	216,307	&	290,711	&	105,566	\\
%			\rowcolor{color1!20!white} \multicolumn{1}{l}{\Bold{	Descentralizadas	}}&	1,485,942	&	12,204,273	&	21,507,703	&	17,327,211	&	40,284,454	&	22,299,019	&	29,711,164	&	33,840,919	&	15,559,525	&	18,173,266	&	24,765,925	&	102,151,625	\\
%			\multicolumn{1}{p{4.5cm}}{	Instituto de Ciencia y Tecnología Agrícola -ICTA-	}&		&	3,467,714	&	2,001,345	&	2,534,196	&	3,408,171	&	2,116,657	&	3,312,351	&	2,594,584	&	2,244,874	&	2,953,323	&	2,884,161	&	7,633,436	\\
%			\rowcolor{color1!5!white}\multicolumn{1}{p{4.5cm}}{	Instituto de Fomento Municipal -Infom-	}&	1,007,656	&	1,353,099	&	4,536,419	&	7,597,857	&	11,577,199	&	6,902,985	&	11,793,458	&	17,202,234	&	9,263,298	&	8,075,141	&	7,287,936	&	21,347,962	\\
%			\multicolumn{1}{p{4.5cm}}{	Consejo Nacional de Alfabetización -Conalfa-	}&	0	&	6,712,685	&	8,782,399	&	6,282,942	&	12,233,468	&	12,390,243	&	13,331,839	&	12,516,052	&	3,168,650	&	5,483,332	&	10,072,470	&	16,627,712	\\
%			\rowcolor{color1!5!white}\multicolumn{1}{p{4.5cm}}{	Instituto Nacional de Comercialización Agrícola -Indeca-	}&	478,285	&	542,452	&	640,798	&	785,982	&	830,547	&	760,291	&	1,079,616	&	1,393,829	&	759,220	&	1,535,874	&	866,091	&	4,816,266	\\
%			\multicolumn{1}{p{4.5cm}}{	Fondo de Tierras 	}&	0	&	128,322	&	5,546,741	&	126,234	&	12,235,070	&	128,843	&	193,900	&	134,220	&	123,483	&	125,596	&	3,655,268	&	51,726,251	\\
%			\rowcolor{color1!20!white} \multicolumn{1}{l}{\Bold{	Grupo institucional (total)	}}&	54,616,315	&	130,112,066	&	430,724,684	&	325,802,737	&	671,686,015	&	654,173,041	&	448,632,476	&	1,020,932,077	&	180,229,138	&	339,831,765	&	398,130,050	&	960,841,304	\\
%			\multicolumn{1}{p{4.5cm}}{	Gobierno central	}&	53,130,374	&	117,907,792	&	409,216,982	&	308,475,526	&	631,401,560	&	631,874,021	&	418,921,311	&	987,091,158	&	164,669,613	&	321,658,498	&	373,364,125	&	858,689,679	\\
%			\rowcolor{color1!5!white}\multicolumn{1}{p{4.5cm}}{	Descentralizadas	}&	1,485,942	&	12,204,273	&	21,507,703	&	17,327,211	&	40,284,454	&	22,299,019	&	29,711,164	&	33,840,919	&	15,559,525	&	18,173,266	&	24,765,925	&	102,151,625	\\
%			[-0.28cm]
%		\end{longtable}\addtocounter{Cuadro}{1}
%	\end{center}
%\end{landscape}
%
%
%


%%%%%%%%%%%%%%%%%%%09



\hoja{
	\normalsize
	{\Bold\color{color1!80!black}{Cuadro \theCuadro $\,-$Ejecución del plan del pacto hambre cero (PPH0); por criterios de seguimiento y porcentaje de ejecución, según componentes y eje transversal. }}\\
	%	{\Bold\color{color1!80!black}{ por criterios de seguimiento, mes y porcentaje de ejecución, según institución (ministerios, secretarías, descentralizadas) y grupo institucional. }}\\
	{\Bold\color{color1!80!black}{República de Guatemala, año 2015.}}\\
	{\color{color1!80!black}{(Quetzales y porcentaje)}}\\[-0.3cm]
	\begin{center}\fontsize{3.0mm}{1.3em}\selectfont \setlength{\arrayrulewidth}{0.7pt}
		\begin{tabular}{x{4.0cm}rrrc}
			\hline &&&&\\[-0.4cm]  
			\multicolumn{1}{x{2.0cm}}{\raisebox{-.3cm}{\textbf{Componente, eje transversal y grupo institucional }}} &	\multicolumn{3}{x{6.3cm}}{\textbf{Criterios de seguimiento (en quetzales)}}&\multicolumn{1}{x{2cm}}{\multirow{2}{*}[-.5mm]{\textbf{Ejecución}}}\\[0.05cm]\cline{2-4}
			\multicolumn{1}{l}{ }&\multicolumn{4}{l}{ }\\[-.35cm]
			\multicolumn{1}{x{2.0cm}}{ } &	\multicolumn{1}{x{2cm}}{\textbf{Asignado}}&\multicolumn{1}{x{2cm}}{\textbf{Vigente}}&\multicolumn{1}{x{2cm}}{\textbf{Ejecutado}}&\multicolumn{1}{x{2cm}}{\textbf{(Porcentaje)}}\\[0.05cm]\hline
			\rowcolor{color1!40!white} \multicolumn{1}{l}{\Bold{	Total presupuesto	}}&	5,433,883,259	&	5,342,538,764	&	3,560,292,421	&	66.6	\\
			\rowcolor{color1!20!white} \multicolumn{1}{l}{\Bold{	Componentes directos	}}&	2,107,359,915	&	2,134,720,536	&	1,677,489,889	&	78.6	\\
			\multicolumn{1}{l}{	Provisión de servicios básicos de salud y nutrición	}&	1,097,107,441	&	1,404,609,220	&	1,133,929,226	&	80.7	\\
			\rowcolor{color1!5!white}\multicolumn{1}{l}{	Promoción de lactancia materna y alimentación complementaria	}&	43,944,951	&	109,715,710	&	92,065,076	&	83.9	\\
			\multicolumn{1}{l}{	Alimentos fortificados	}&	71,420,195	&	103,562,337	&	55,948,828	&	54.0	\\
			\rowcolor{color1!5!white}\multicolumn{1}{l}{	Atención a población vulnerable a la inseguridad alimentaria	}&	894,887,328	&	516,833,270	&	395,546,758	&	76.5	\\
			\rowcolor{color1!20!white} \multicolumn{1}{l}{\Bold{	Componente de viabilidad	}}&	3,241,332,009	&	3,153,765,975	&	1,831,822,000	&	58.1	\\
			\multicolumn{1}{l}{	Mejoramiento de los ingresos y la economía familiar	}&	1,689,819,145	&	1,742,949,859	&	863,638,954	&	49.6	\\
			\rowcolor{color1!5!white}\multicolumn{1}{l}{	Agua y saneamiento	}&	229,520,584	&	284,243,748	&	61,745,335	&	21.7	\\
			\multicolumn{1}{l}{	Gobernanza local	}&	3,068,045	&	2,863,616	&	2,540,717	&	88.7	\\
			\rowcolor{color1!5!white}\multicolumn{1}{l}{	Escuelas saludables	}&	752,174,615	&	578,239,444	&	574,875,121	&	99.4	\\
			\multicolumn{1}{l}{	Hogar saludable	}&	391,776,123	&	383,064,990	&	203,436,214	&	53.1	\\
			\rowcolor{color1!5!white}\multicolumn{1}{l}{	Alfabetización	}&	174,973,497	&	162,404,318	&	125,585,659	&	77.3	\\
			\rowcolor{color1!20!white} \multicolumn{1}{l}{\Bold{	Eje transversal	}}&	 85,191,335 	 & 	 54,052,253 	 & 	 50,980,532 	 & 	94.3	 \\ 
			\multicolumn{1}{l}{	Coordinación interinstitucional	}&	48,607,780	&	24,237,708	&	23,613,395	&	97.4	\\
			\rowcolor{color1!5!white}\multicolumn{1}{l}{	Comunicación para la seguridad alimentaria y nutricional	}&	2,992,056	&	1,286,695	&	1,221,818	&	95.0	\\
			\multicolumn{1}{l}{	Participación comunitaria	}&	3,972,884	&	14,070,218	&	13,043,810	&	92.7	\\
			\rowcolor{color1!5!white}\multicolumn{1}{l}{	Sistema de información de seguridad alimentaria y nutricional	}&	2,054,179	&	2,160,938	&	924,523	&	42.8	\\
			\multicolumn{1}{l}{	Sistema de monitoreo y evaluación	}&	27,564,436	&	12,296,694	&	12,176,988	&	99.0	\\
			[0.05cm]
			\hline
			&&&&\\[-0.36cm]\end{tabular}\addtocounter{Cuadro}{1}
	\end{center}
	{\footnotesize Fuente:  Elaborado por Sesan con datos de Sicoin-Minfin, 2016.}\\[.1cm]
	%	{\parbox{13cm}{\footnotesize \textbf{Notas:} Se denomina desnutrición severa cuando los niños están 3 desviaciones estándar o más por debajo de la media, de acuerdo a la tabla de medidas de la OMS. }}\\[.3cm]
	%	{\parbox{13cm}{\footnotesize Se denomina desnutrición total cuando los niños están dos desviaciones estándar o más por debajo de la media. Incluye a los niños que están 3 desviaciones estándar o más por debajo de la media.}}\\
}




%%%%%%%%%%%%%%%%%%%10


%
%\hoja{
%	\normalsize
%	{\Bold\color{color1!80!black}{Cuadro \theCuadro $\,-$ Ejecución del plan del pacto hambre cero (PPH0); por criterios de seguimiento, mes y porcentaje de ejecución, }}\\
%	{\Bold\color{color1!80!black}{según institución (ministerios, secretarías, descentralizadas) y grupo institucional. }}\\
%	{\Bold\color{color1!80!black}{República de Guatemala, año 2015.}}\\
%	{\color{color1!80!black}{(Quetzales y porcentaje)}}\\[-0.3cm]
%	\begin{center}\fontsize{3.0mm}{1.3em}\selectfont \setlength{\arrayrulewidth}{0.7pt}
%		\begin{tabular}{x{4.0cm}rrrc}
%			\hline &&&&\\[-0.4cm]  
%			\multicolumn{1}{x{2.0cm}}{\raisebox{-.3cm}{\textbf{Componente, eje transversal y grupo institucional }}} &	\multicolumn{3}{x{6.3cm}}{\textbf{Criterios de seguimiento (en quetzales)}}&\multicolumn{1}{x{2cm}}{\multirow{2}{*}[-.5mm]{\textbf{Ejecución}}}\\[0.05cm]\cline{2-4}
%			\multicolumn{1}{l}{ }&\multicolumn{4}{l}{ }\\[-.35cm]
%			\multicolumn{1}{x{2.0cm}}{ } &	\multicolumn{1}{x{2cm}}{\textbf{Asignado}}&\multicolumn{1}{x{2cm}}{\textbf{Vigente}}&\multicolumn{1}{x{2cm}}{\textbf{Ejecutado}}&\multicolumn{1}{x{2cm}}{\textbf{(Porcentaje)}}\\[0.05cm]\hline
%			\rowcolor{color1!40!white} \multicolumn{1}{l}{\Bold{	Total presupuesto	}}&	5,433,883,259	&	5,342,538,764	&	3,560,292,421	&	66.6	\\
%			\rowcolor{color1!20!white} \multicolumn{1}{l}{\Bold{	Ministerios	}}&	4,833,238,682	&	4,687,601,799	&	3,235,010,865	&	69.0	\\
%			\multicolumn{1}{l}{	Ministerio de Educación	}&	733,498,088	&	573,507,563	&	573,099,895	&	99.9	\\
%			\rowcolor{color1!5!white}\multicolumn{1}{l}{	Ministerio de Salud Pública y Asistencia Social	}&	1,225,659,293	&	1,590,091,837	&	1,251,483,963	&	78.7	\\
%			\multicolumn{1}{l}{	Ministerio de Agricultura, Ganadería y Alimentación	}&	735,423,108	&	738,297,821	&	368,724,638	&	49.9	\\
%			\rowcolor{color1!5!white}\multicolumn{1}{l}{	Ministerio de Comunicaciones, Infraestructura y Vivienda	}&	1,315,185,738	&	1,211,547,974	&	613,552,025	&	50.6	\\
%			\multicolumn{1}{l}{	Ministerio de Ambiente y Recursos Naturales	}&	6,064,342	&	6,264,563	&	5,616,550	&	89.7	\\
%			\rowcolor{color1!5!white}\multicolumn{1}{l}{	Ministerio de Desarrollo Social	}&	817,408,113	&	567,892,041	&	422,533,794	&	74.4	\\
%			\rowcolor{color1!20!white} \multicolumn{1}{l}{\Bold{	Secretarías	}}&	90,240,558	&	113,289,469	&	103,343,636	&	91.2	\\
%			\multicolumn{1}{l}{	Secretaría de Coordinación Ejecutiva de la Presidencia	}&	3,068,045	&	2,863,616	&	2,540,717	&	88.7	\\
%			\rowcolor{color1!5!white}\multicolumn{1}{l}{	Secretaría de Obras Sociales de la Esposa del Presidente	}&	0	&	54,448,531	&	47,897,317	&	88.0	\\
%			\multicolumn{1}{l}{	Secretaría de Seguridad Alimentaria y Nutricional	}&	85,191,335	&	54,052,253	&	50,980,532	&	94.3	\\
%			\rowcolor{color1!5!white}\multicolumn{1}{l}{	Secretaría de Bienestar Social de la Presidencia	}&	1,981,178	&	1,925,069	&	1,925,069	&	100.0	\\
%			\rowcolor{color1!20!white} \multicolumn{1}{l}{\Bold{	Descentralizadas	}}&	510,404,019	&	541,647,496	&	221,937,920	&	41.0	\\
%			\multicolumn{1}{l}{	Instituto de Ciencia y Tecnología Agrícola -ICTA-	}&	37,500,000	&	37,500,000	&	32,331,296	&	86.2	\\
%			\rowcolor{color1!5!white}\multicolumn{1}{l}{	Instituto de Fomento Municipal -Infom-	}&	208,288,358	&	249,401,014	&	36,765,567	&	14.7	\\
%			\multicolumn{1}{l}{	Consejo Nacional de Alfabetización -Conalfa-	}&	174,973,497	&	162,404,318	&	125,585,659	&	77.3	\\
%			\rowcolor{color1!5!white}\multicolumn{1}{l}{	Instituto Nacional de Comercialización Agrícola -Indeca-	}&	17,000,000	&	19,700,000	&	11,148,774	&	56.6	\\
%			\multicolumn{1}{l}{	Fondo de Tierras 	}&	72,642,164	&	72,642,164	&	16,106,625	&	22.2	\\
%			\rowcolor{color1!20!white} \multicolumn{1}{l}{\Bold{	Grupo institucional (total)	}}&	5,433,883,259	&	5,342,538,764	&	3,560,292,421	&	66.6	\\
%			\multicolumn{1}{l}{	Gobierno central	}&	4,923,479,240	&	4,800,891,268	&	3,338,354,501	&	69.5	\\
%			\rowcolor{color1!5!white}\multicolumn{1}{l}{	Descentralizadas	}&	510,404,019	&	541,647,496	&	221,937,920	&	41.0	\\
%			[0.05cm]
%			\hline
%			&&&&\\[-0.36cm]\end{tabular}\addtocounter{Cuadro}{1}
%	\end{center}
%	{\footnotesize Fuente:  Elaborado por Sesan con datos de Sicoin-Minfin, 2016.}\\[.1cm]
%	%	{\parbox{13cm}{\footnotesize \textbf{Notas:} Se denomina desnutrición severa cuando los niños están 3 desviaciones estándar o más por debajo de la media, de acuerdo a la tabla de medidas de la OMS. }}\\[.3cm]
%	%	{\parbox{13cm}{\footnotesize Se denomina desnutrición total cuando los niños están dos desviaciones estándar o más por debajo de la media. Incluye a los niños que están 3 desviaciones estándar o más por debajo de la media.}}\\
%}


%%%%%%%%%%%%%11



\hoja{
	{\Bold\Large 7.3	Seguimiento Especial del Gasto de la Ventana de los Mil Días. Años 2013 a 2015}\\
	\normalsize
	{\Bold\color{color1!80!black}{Cuadro \theCuadro $\,-$ Presupuesto de la ventana de los mil días, por criterios de seguimiento y porcentaje de ejecución;  según departamento. }}\\
	{\Bold\color{color1!80!black}{República de Guatemala, año 2013.}}\\
	{\color{color1!80!black}{(Quetzales y porcentaje)}}\\[-0.3cm]
	\begin{center}\fontsize{3.0mm}{1.3em}\selectfont \setlength{\arrayrulewidth}{0.7pt}
		\begin{tabular}{x{4.0cm}rrrc}
			\hline &&&&\\[-0.4cm]  
			\multicolumn{1}{x{2.0cm}}{\raisebox{-.3cm}{\textbf{Departamento}}} &	\multicolumn{3}{x{6.3cm}}{\textbf{Criterios de seguimiento (en quetzales)}}&\multicolumn{1}{x{2cm}}{\multirow{2}{*}[-.5mm]{\textbf{Ejecución}}}\\[0.05cm]\cline{2-4}
			\multicolumn{1}{l}{ }&\multicolumn{4}{l}{ }\\[-.35cm]
			\multicolumn{1}{x{2.0cm}}{ } &	\multicolumn{1}{x{2cm}}{\textbf{Asignado}}&\multicolumn{1}{x{2cm}}{\textbf{Vigente}}&\multicolumn{1}{x{2cm}}{\textbf{Ejecutado}}&\multicolumn{1}{x{2cm}}{\textbf{(Porcentaje)}}\\[0.05cm]\hline
			\rowcolor{color1!20!white} \multicolumn{1}{l}{\Bold{	Total presupuesto	 }}& 	130,246,108	&	213,921,494	&	188,260,411	&	88.0	\\
			\multicolumn{1}{l}{	Guatemala	 }& 	73,631,550	&	66,841,657	&	43,962,366	&	65.8	\\
			\rowcolor{color1!5!white}\multicolumn{1}{l}{	El Progreso 	 }& 	856,331	&	2,457,716	&	2,248,041	&	91.5	\\
			\multicolumn{1}{l}{	Sacatepéquez	}&	1,475,900	&	2,084,354	&	2,214,189	&	106.2	\\
			\rowcolor{color1!5!white}\multicolumn{1}{l}{	Chimaltenango 	}&	0	&	4,766,529	&	4,764,157	&	100.0	\\
			\multicolumn{1}{l}{	Escuintla 	}&	94,152	&	3,391,274	&	2,895,890	&	85.4	\\
			\rowcolor{color1!5!white}\multicolumn{1}{l}{	Santa Rosa 	}&	1,029,512	&	3,603,877	&	3,452,175	&	95.8	\\
			\multicolumn{1}{l}{	Sololá	}&	2,982,273	&	4,650,488	&	4,484,018	&	96.4	\\
			\rowcolor{color1!5!white}\multicolumn{1}{l}{	Totonicapán	}&	3,107,727	&	6,233,668	&	6,187,812	&	99.3	\\
			\multicolumn{1}{l}{	Quetzaltenango 	}&	1,742,000	&	9,543,728	&	9,542,685	&	100.0	\\
			\rowcolor{color1!5!white}\multicolumn{1}{l}{	Suchitepéquez	}&	412,491	&	7,450,304	&	7,345,584	&	98.6	\\
			\multicolumn{1}{l}{	Retalhuleu	}&	387,987	&	3,303,147	&	2,920,526	&	88.4	\\
			\rowcolor{color1!5!white}\multicolumn{1}{l}{	San Marcos	}&	10,368,812	&	9,178,458	&	9,093,055	&	99.1	\\
			\multicolumn{1}{l}{	Huehuetenango	}&	3,405,856	&	25,812,645	&	25,496,204	&	98.8	\\
			\rowcolor{color1!5!white}\multicolumn{1}{l}{	Quiché	}&	10,644,589	&	30,445,304	&	29,887,076	&	98.2	\\
			\multicolumn{1}{l}{	Baja Verapaz 	}&	1,886,944	&	3,040,773	&	3,141,200	&	103.3	\\
			\rowcolor{color1!5!white}\multicolumn{1}{l}{	Alta Verapaz	}&	128,647	&	4,004,900	&	3,630,618	&	90.7	\\
			\multicolumn{1}{l}{	Petén	}&	1,420,128	&	7,651,000	&	7,553,364	&	98.7	\\
			\rowcolor{color1!5!white}\multicolumn{1}{l}{	Izabal 	}&	2,778,596	&	2,367,325	&	2,345,316	&	99.1	\\
			\multicolumn{1}{l}{	Zacapa 	}&	6,461,513	&	492,027	&	386,740	&	78.6	\\
			\rowcolor{color1!5!white}\multicolumn{1}{l}{	Chiquimula 	}&	4,938,411	&	14,137,688	&	14,296,696	&	101.1	\\
			\multicolumn{1}{l}{	Jalapa 	}&	1,297,783	&	1,837,448	&	1,829,803	&	99.6	\\
			\rowcolor{color1!5!white}\multicolumn{1}{l}{	Jutiapa 	}&	1,194,906	&	627,184	&	582,896	&	92.9	\\
			[0.05cm]
			\hline
			&&&&\\[-0.36cm]\end{tabular}\addtocounter{Cuadro}{1}
	\end{center}
	{\footnotesize Fuente:  Elaborado por Sesan con datos de Sicoin-Minfin, 2016.}\\[.1cm]
	%	{\parbox{13cm}{\footnotesize \textbf{Notas:} Se denomina desnutrición severa cuando los niños están 3 desviaciones estándar o más por debajo de la media, de acuerdo a la tabla de medidas de la OMS. }}\\[.3cm]
	%	{\parbox{13cm}{\footnotesize Se denomina desnutrición total cuando los niños están dos desviaciones estándar o más por debajo de la media. Incluye a los niños que están 3 desviaciones estándar o más por debajo de la media.}}\\
}




%%%%%%%%%%%%%%%%%%12



\hoja{
	\normalsize
	{\Bold\color{color1!80!black}{Cuadro \theCuadro $\,-$ Presupuesto de la ventana de los mil días, por criterios de seguimiento y porcentaje de ejecución; según tipo de acción.}}\\
	%	{\Bold\color{color1!80!black}{ según departamento. }}\\
	{\Bold\color{color1!80!black}{República de Guatemala, año 2013.}}\\
	{\color{color1!80!black}{(Quetzales y porcentaje)}}\\[-0.3cm]
	\begin{center}\fontsize{3.0mm}{1.3em}\selectfont \setlength{\arrayrulewidth}{0.7pt}
		\begin{tabular}{x{4.0cm}rrrc}
			\hline &&&&\\[-0.4cm]  
			\multicolumn{1}{x{6.0cm}}{\raisebox{-.3cm}{\textbf{ Acción de la ventana de los mil días }}} &	\multicolumn{3}{x{6.3cm}}{\textbf{Criterios de seguimiento (en quetzales)}}&\multicolumn{1}{x{2cm}}{\multirow{2}{*}[-.5mm]{\textbf{Ejecución}}}\\[0.05cm]\cline{2-4}
			\multicolumn{1}{l}{ }&\multicolumn{4}{l}{ }\\[-.35cm]
			\multicolumn{1}{x{6.0cm}}{ } &	\multicolumn{1}{x{2cm}}{\textbf{Asignado}}&\multicolumn{1}{x{2cm}}{\textbf{Vigente}}&\multicolumn{1}{x{2cm}}{\textbf{Ejecutado}}&\multicolumn{1}{x{2cm}}{\textbf{(Porcentaje)}}\\[0.05cm]\hline
			\rowcolor{color1!40!white} \multicolumn{1}{l}{\Bold{	Total presupuesto	}}&	130,246,108	&	213,921,494	&	188,260,411	&	88.0	\\
			\multicolumn{1}{l}{	Promoción y apoyo de la lactancia materna 	}&	70,667,216	&	79,405,083	&	78,010,257	&	98.2	\\
			\rowcolor{color1!5!white}\multicolumn{1}{l}{	Suplementación de zinc terapéutico en el manejo de la diarrea 	}&	12,227,798	&	14,288,689	&	12,221,569	&	85.5	\\
			\multicolumn{1}{l}{	Desparasitación y vacunación de niños y niñas	}&	947,306	&	15,907,914	&	15,466,429	&	97.2	\\
			\rowcolor{color1!5!white}\multicolumn{1}{l}{	Suplementación de vitamina "A" y micronutrientes a niños y niñas	}&	16,268,643	&	72,906,156	&	53,401,734	&	73.2	\\
			\multicolumn{1}{p{6cm}}{	Suplementación con micronutrientes, hierro y ácido fólico a mujer en edad fértil 	}&	30,135,145	&	31,413,652	&	29,160,423	&	92.8	\\
			[0.05cm]
			\hline
			&&&&\\[-0.36cm]\end{tabular}\addtocounter{Cuadro}{1}
	\end{center}
	{\footnotesize Fuente:  Elaborado por Sesan con datos de Sicoin-Minfin, 2016.}\\[.1cm]}











%%%%%%%%%%%%%%%%%%13



%%%%%%%%%%%%%%%%%%


\hoja{
	\normalsize
	{\Bold\color{color1!80!black}{Cuadro \theCuadro $\,-$ Presupuesto de la ventana de los mil días, por criterios de seguimiento y porcentaje de ejecución; según tipo de acción.}}\\
	%	{\Bold\color{color1!80!black}{ según departamento. }}\\
	{\Bold\color{color1!80!black}{República de Guatemala, año 2014.}}\\
	{\color{color1!80!black}{(Quetzales y porcentaje)}}\\[-0.3cm]
	\begin{center}\fontsize{3.0mm}{1.3em}\selectfont \setlength{\arrayrulewidth}{0.7pt}
		\begin{tabular}{x{4.0cm}rrrc}
			\hline &&&&\\[-0.4cm]  
			\multicolumn{1}{x{6.0cm}}{\raisebox{-.3cm}{\textbf{ Acción de la ventana de los mil días }}} &	\multicolumn{3}{x{6.3cm}}{\textbf{Criterios de seguimiento (en quetzales)}}&\multicolumn{1}{x{2cm}}{\multirow{2}{*}[-.5mm]{\textbf{Ejecución}}}\\[0.05cm]\cline{2-4}
			\multicolumn{1}{l}{ }&\multicolumn{4}{l}{ }\\[-.35cm]
			\multicolumn{1}{x{6.0cm}}{ } &	\multicolumn{1}{x{2cm}}{\textbf{Asignado}}&\multicolumn{1}{x{2cm}}{\textbf{Vigente}}&\multicolumn{1}{x{2cm}}{\textbf{Ejecutado}}&\multicolumn{1}{x{2cm}}{\textbf{(Porcentaje)}}\\[0.05cm]\hline
			\rowcolor{color1!20!white} \multicolumn{1}{m{6.0cm}}{\Bold{	Total presupuesto	}}&	380,172,611	&	619,385,547	&	528,769,202	&	85.4	\\
			\multicolumn{1}{l}{	Promoción y Apoyo de la Lactancia Materna 	}&	72,447,524	&	58,967,472	&	44,759,502	&	75.9	\\
			\rowcolor{color1!5!white}\multicolumn{1}{m{6.0cm}}{	Suplementación de Zinc Terapéutico en el Manejo de la Diarrea 	}&	12,227,798	&	11,729,436	&	10,919,687	&	93.1	\\
			\multicolumn{1}{l}{	Desparasitación y Vacunación de Niños y Niñas	}&	250,917,305	&	413,246,999	&	347,907,510	&	84.2	\\
			\rowcolor{color1!5!white}\multicolumn{1}{m{6.0cm}}{	Suplementación de Vitamina "A " y Micronutrientes a Niños y Niñas	}&	14,444,839	&	50,575,279	&	44,675,947	&	88.3	\\
			\multicolumn{1}{m{6.0cm}}{	Suplementación con Micronutrientes, Hierro y Ácido Fólico a Mujer en Edad Fértil 	}&	30,135,145	&	19,343,869	&	17,617,592	&	91.1	\\
			\rowcolor{color1!5!white}\multicolumn{1}{m{6.0cm}}{	Mejoramiento de la Alimentación Complementaría a partir de los seis meses	}&	0	&	61,265,841	&	59,109,715	&	96.5	\\
			\multicolumn{1}{m{6.0cm}}{	Tratamiento. Oportuno de Desnutrición Aguda Moderada y Severa Utilizando ATLC en Comunidad con Orientación  y Seguimiento  del Personal de Salud	}&	0	&	4,256,651	&	3,779,250	&	88.8	\\
			[0.05cm]
			\hline
			&&&&\\[-0.36cm]\end{tabular}\addtocounter{Cuadro}{1}
	\end{center}
	{\footnotesize Fuente:  Elaborado por Sesan con datos de Sicoin-Minfin, 2016.}\\[.1cm]





%%%%%%%%%%%%%%%%%%14



	\normalsize
	{\Bold\color{color1!80!black}{Cuadro \theCuadro $\,-$Presupuesto de la ventana de los mil días, por criterios de seguimiento y porcentaje de ejecución; según departamento. }}\\
	{\Bold\color{color1!80!black}{República de Guatemala, año 2015.}}\\
	{\color{color1!80!black}{(Quetzales y porcentaje)}}\\[-0.3cm]
	\begin{center}\fontsize{3.0mm}{1.3em}\selectfont \setlength{\arrayrulewidth}{0.7pt}
		\begin{tabular}{x{4.0cm}rrrc}
			\hline &&&&\\[-0.4cm]  
			\multicolumn{1}{x{4.0cm}}{\raisebox{-.3cm}{\textbf{ Departamento}}} &	\multicolumn{3}{x{6.3cm}}{\textbf{Criterios de seguimiento (en quetzales)}}&\multicolumn{1}{x{2cm}}{\multirow{2}{*}[-.5mm]{\textbf{Ejecución}}}\\[0.05cm]\cline{2-4}
			\multicolumn{1}{l}{ }&\multicolumn{4}{l}{ }\\[-.35cm]
			\multicolumn{1}{c}{ } &	\multicolumn{1}{x{2cm}}{\textbf{Asignado}}&\multicolumn{1}{x{2cm}}{\textbf{Vigente}}&\multicolumn{1}{x{2cm}}{\textbf{Ejecutado}}&\multicolumn{1}{x{2cm}}{\textbf{(Porcentaje)}}\\[0.05cm]\hline
			\rowcolor{color1!20!white} \multicolumn{1}{l}{\Bold{	Total presupuesto	}}&	705,840,558	&	1,008,060,707	&	786,358,785	&	78.0	\\
			\multicolumn{1}{l}{	Guatemala	}&	408,306,813	&	537,848,177	&	411,191,010	&	76.5	\\
			\rowcolor{color1!5!white}\multicolumn{1}{l}{	El Progreso 	}&	1,204,080	&	5,621,153	&	3,336,247	&	59.4	\\
			\multicolumn{1}{l}{	Sacatepéquez	}&	5,845,429	&	7,542,807	&	6,566,065	&	87.1	\\
			\rowcolor{color1!5!white}\multicolumn{1}{l}{	Chimaltenango 	}&	13,722,485	&	16,447,663	&	13,390,492	&	81.4	\\
			\multicolumn{1}{l}{	Escuintla 	}&	10,385,375	&	11,548,853	&	9,906,879	&	85.8	\\
			\rowcolor{color1!5!white}\multicolumn{1}{l}{	Santa Rosa 	}&	5,939,642	&	18,638,051	&	12,168,429	&	65.3	\\
			\multicolumn{1}{l}{	Sololá	}&	17,311,633	&	26,007,672	&	25,349,570	&	97.5	\\
			\rowcolor{color1!5!white}\multicolumn{1}{l}{	Totonicapán	}&	12,382,402	&	7,669,164	&	5,631,303	&	73.4	\\
			\multicolumn{1}{l}{	Quetzaltenango 	}&	13,819,115	&	25,290,106	&	21,919,804	&	86.7	\\
			\rowcolor{color1!5!white}\multicolumn{1}{l}{	Suchitepéquez	}&	17,305,413	&	20,870,716	&	14,276,700	&	68.4	\\
			\multicolumn{1}{l}{	Retalhuleu	}&	4,456,939	&	10,180,064	&	6,692,518	&	65.7	\\
			\rowcolor{color1!5!white}\multicolumn{1}{l}{	San Marcos	}&	25,675,011	&	30,959,245	&	26,263,304	&	84.8	\\
			\multicolumn{1}{l}{	Huehuetenango	}&	41,779,295	&	72,814,289	&	59,903,539	&	82.3	\\
			\rowcolor{color1!5!white}\multicolumn{1}{l}{	Quiché	}&	42,194,200	&	62,881,390	&	54,537,007	&	86.7	\\
			\multicolumn{1}{l}{	Baja Verapaz 	}&	8,808,891	&	19,214,922	&	16,288,905	&	84.8	\\
			\rowcolor{color1!5!white}\multicolumn{1}{l}{	Alta Verapaz	}&	33,737,037	&	44,675,523	&	32,516,167	&	72.8	\\
			\multicolumn{1}{l}{	Petén	}&	8,300,739	&	16,855,128	&	13,852,908	&	82.2	\\
			\rowcolor{color1!5!white}\multicolumn{1}{l}{	Izabal 	}&	3,875,689	&	13,092,068	&	6,616,147	&	50.5	\\
			\multicolumn{1}{l}{	Zacapa 	}&	3,175,446	&	7,909,827	&	4,432,697	&	56.0	\\
			\rowcolor{color1!5!white}\multicolumn{1}{l}{	Chiquimula 	}&	20,055,885	&	24,222,130	&	23,224,325	&	95.9	\\
			\multicolumn{1}{l}{	Jalapa 	}&	1,248,744	&	6,947,052	&	3,251,617	&	46.8	\\
			\rowcolor{color1!5!white}\multicolumn{1}{l}{	Jutiapa 	}&	6,310,295	&	20,824,707	&	15,043,153	&	72.2	\\
			[0.05cm]
			\hline
			&&&&\\[-0.36cm]\end{tabular}\addtocounter{Cuadro}{1}
	\end{center}
	{\footnotesize Fuente:  Elaborado por Sesan con datos de Sicoin-Minfin, 2016.}\\[.1cm]}


%%%%%%%%%%%%%%%%%%15


\hoja{
	\normalsize
	{\Bold\color{color1!80!black}{Cuadro \theCuadro $\,-$Presupuesto de la ventana de los mil días, por criterios de seguimiento y porcentaje de ejecución; según departamento. }}\\
	{\Bold\color{color1!80!black}{República de Guatemala, año 2015.}}\\
	{\color{color1!80!black}{(Quetzales y porcentaje)}}\\[-0.3cm]
	\begin{center}\fontsize{3.0mm}{1.3em}\selectfont \setlength{\arrayrulewidth}{0.7pt}
		\begin{tabular}{x{5.5cm}rrrc}
			\hline &&&&\\[-0.4cm]  
			\multicolumn{1}{x{5.5cm}}{\raisebox{-.3cm}{\textbf{Acción de la ventana de los mil días }}} &	\multicolumn{3}{x{6.6cm}}{\textbf{Criterios de seguimiento (en quetzales)}}&\multicolumn{1}{x{2cm}}{\multirow{2}{*}[-.5mm]{\textbf{Ejecución}}}\\[0.05cm]\cline{2-4}
			\multicolumn{1}{l}{ }&\multicolumn{4}{l}{ }\\[-.35cm]
			\multicolumn{1}{c}{ } &	\multicolumn{1}{x{2cm}}{\textbf{Asignado}}&\multicolumn{1}{x{2cm}}{\textbf{Vigente}}&\multicolumn{1}{x{2cm}}{\textbf{Ejecutado}}&\multicolumn{1}{x{2cm}}{\textbf{(Porcentaje)}}\\[0.05cm]\hline
			\rowcolor{color1!20!white} \multicolumn{1}{p{5.5cm}}{\Bold{	Total presupuesto	}}&	705,840,558	&	1,008,060,707	&	786,358,785	&	78.0	\\
			\multicolumn{1}{p{5.5cm}}{	Promoción y apoyo de la lactancia materna 	}&	43,944,951	&	55,267,179	&	44,701,659	&	80.9	\\
			\rowcolor{color1!5!white}\multicolumn{1}{p{5.5cm}}{	Suplementación de zinc terapéutico en el manejo de la diarrea 	}&	13,937,987	&	12,429,349	&	9,003,089	&	72.4	\\
			\multicolumn{1}{p{5.5cm}}{	Suplementación de vitamina "A" y micronutrientes a niños y niñas	}&	250,608,099	&	205,354,606	&	99,898,656	&	48.6	\\
			\rowcolor{color1!5!white}\multicolumn{1}{p{5.5cm}}{	Desparasitación y vacunación de niños y niñas	}&	299,792,527	&	611,228,619	&	561,531,908	&	91.9	\\
			\multicolumn{1}{p{5.5cm}}{	Suplementación con micronutrientes, hierro y ácido fólico a mujer en edad fértil 	}&	26,136,799	&	20,218,617	&	15,274,645	&	75.5	\\
			\rowcolor{color1!5!white}\multicolumn{1}{p{5.5cm}}{	Mejoramiento de la alimentación complementaría a partir de los seis meses	}&	69,719,062	&	102,921,967	&	55,819,161	&	54.2	\\
			\multicolumn{1}{p{5.5cm}}{	Fortificción de los alimentos básicos con micronutrientes	}&	1,701,133	&	640,370	&	129,667	&	20.2	\\
			[0.05cm]
			\hline
			&&&&\\[-0.36cm]\end{tabular}\addtocounter{Cuadro}{1}
	\end{center}
	{\footnotesize Fuente:  Elaborado por Sesan con datos de Sicoin-Minfin, 2016.}\\[.1cm]}



\newpage
	$\ $\\
 \textcolor{color2}{\Huge Literatura citada}\\
	\normalsize	$\ $\\
	
%
\begin{enumerate}\setlength{\textwidth}{25mm}
	\item {Instituto Nacional de Estadística (INE) y Banco Mundial. 2013. \textit{Mapas de Pobreza Rural en Guatemala 2011.}  Guatemala. En: {http://www.ine.gob.gt/sistema/uploads/2014/01/10/ifRRpEnf0cjUfRZGhyXD7RQjf7EQH2Er.pdf} (Consultado: Febrero, 2016).}
%	
	\item {CONEVAL (Consejo Nacional de Evaluación de la Política de Desarrollo Social). (2010). Dimensiones de la seguridad alimentaria: \textit{Dimensiones de la seguridad alimentaria: Evaluación estratégica de nutrición y abasto} (Vol. 2).}
%	
	\item {DevTech Systems, I. \& I. (2014). \textit{Manual de estadísticas e indicadores de seguridad alimentaria y nutricional en Programa de Monitoreo y Evaluación Manual de estadísticas e indicadores de seguridad alimentaria y nutricional en Guatemala.} Guatemala.}
%
	\item {FAO (Food and Agriculture Organization). (2015). \textit{Seguridad alimentaria y nutrición en el marco post-2015.} Roma. En: http://www.
			fao.org/fsnforum/post2015/sites/post2015/files/resources/Post2015\_SANversionFINAL.pdf}
%
	\item {$\ $---------------. (2015a). \textit{Estadísticas sobre seguridad alimentaria.} En: {http://www.fao.org/economic/ess/ess-fs/es/} (Consultado: 26.oct.2015).}

	\item { $\ $---------------. (2014). \textit{Food security indicators.} Roma. En: http://www.fao.org/economic/ess/ess-fs/indicadores-de-la-segu ridad-alimentaria/es/\#.VfMDkdJ\_Okq}
%
	\item {$\ $---------------. (s.f.).\textit{Cumbre Mundial sobre la Alimentación.} En: {http://www.fao.org/wfs/index\_es.htm} (Consultado: 26.oct.2015).}
%
	\item {FAO, FIDA, \& PMA (Organización de las Naciones Unidas para la Alimentación y la Agricultura; Fondo Internacional de Desarrollo Agrícola; Programa Mundial de Alimentos). (2015). \textit{El estado de la Inseguridad Alimentaria en el Mundo 2014. Fortalecimiento de un entorno favorable para la seguridad alimentaria y la nutrición.} Roma. doi:9789251073179}
%
	\item {Gobierno de Guatemala. (2012). \textit{El plan del pacto hambre cero. Guatemala.} Guatemala. En: http://www.sesan.gob.gt/index.php/ descargas/17-plan-del-pacto-hambre-cero/file (Consultado: 26.oct.2015). }
%
	\item {Global Strategy. (2015). \textit{Resultados, pilares y productos.} En: {http://www.gsars.org/es/about/} (Consultado: 26.oct.2015).}
%
	\item{IFPRI (International Food Policy Research Institute) (2015). \textit{The Global Hunger Index Interactive Map.} En: {http://ghi.ifpri.org/} (Consultado: 26.oct.2015).}
	
	\item {Ministerio de Agricultura, Ganadería y Alimentación (MAGA). 2013. \textit{El agro en cifras 2015.} En: {http://web.maga.gob.gt/download /1agro-cifras2014.pdf}  (Consultado: febrero, 2016).}
	
	\item {INE (Instituto de Estadística Nacional). (2013). \textit{Resumen ejecutivo del compendio estadístico sobre la situación de niñas adolescentes.} Guatemala. En: {http://osarguatemala.org/userfiles/Resumen ejecutivo nin\_as y adolescentes.pdf}	}
	
	\item {URL, Iarna, IICA, \& M. U. (2015). \textit{Modelo nacional de la desnutrición crónica infantil.} Guatemala.}
\end{enumerate} 
$\ $\\

	
\newpage
	$\ $\\[-1cm]
 \textcolor{color2}{\Huge Acrónimos y abreviaturas}\\
	\normalsize	$\ $\\
\begin{center}\fontsize{3.8mm}{1.4em}\selectfont \setlength{\arrayrulewidth}{0.7pt}
	$\ $\\[-1.5cm]
	$\!$\begin{tabular}{p{4.3cm}l}	\hline
	ANAM	&	Asociación Nacional de Municipalidades	\\
\rowcolor{color1!5!white}	Asies	&	Asociación de Investigación y Estudios Sociales	\\
	Banguat	&	Banco de Guatemala	\\
\rowcolor{color1!5!white}	Cacif	&	Comité Coordinador de Asociaciones Agrícolas, Comerciales, Industriales y Financieras	\\
	CMA	&	Cumbre Mundial sobre la Alimentación	\\
\rowcolor{color1!5!white}	Diplan	&	Dirección de Planeamiento	\\
	Encovi	&	Encuesta Nacional de Condiciones de Vida	\\
\rowcolor{color1!5!white}	ENEI	&	Encuesta Nacional de Empleo e Ingresos	\\
	Ensmi	&	Encuesta Nacional de Salud Materno Infantil	\\
\rowcolor{color1!5!white}	FAO	&	Organización de las Naciones Unidas para la Alimentación y la Agricultura, por sus siglas en inglés	\\
	FIDA	&	Fondo Internacional de Desarrollo Agrícola	\\
\rowcolor{color1!5!white}	GHI	&	Índice del Hambre Global, siglas en inglés	\\
	GSARS	&	Estrategia Global para el Mejoramiento de las Estadísticas	\\
	\rowcolor{color1!5!white}Iarna	&	Instituto de Agricultura, Recursos Naturales y Ambiente	\\
		IDH	&	Índice de Desarrollo Humano\\
\rowcolor{color1!5!white}	Ifpri	&	Instituto de Investigación de Política Alimentaria	\\
	IGSS	&	Instituto Guatemalteco de Seguridad Social	\\
		\rowcolor{color1!5!white} INE	&	Instituto Nacional de Estadística\\
	MAGA	&	Ministerio de Agricultura, Ganadería y Alimentación	\\
	\rowcolor{color1!5!white}MARN	&	Ministerio de Ambiente y Recursos Naturales	\\
	MEM	&	Ministerio de Energía y Minas	\\
	\rowcolor{color1!5!white}Mineco	&	Ministerio de Economía de Guatemala	\\
	Mineduc	&	Ministerio de Educación	\\
	\rowcolor{color1!5!white}Minfin	&	Ministerio de Finanzas Públicas	\\
	Mspas	&	Ministerio de Salud Pública y Asistencia Social	\\
	\rowcolor{color1!5!white}Mintrab	&	Ministerio de Trabajo y Previsión Social	\\
	Ocsesan	&	Oficina Coordinadora Sectorial de Estadísticas de Seguridad Alimentaria y Nutricional	\\
	\rowcolor{color1!5!white} ODM	&	Objetivos del Desarrollo del Milenio	\\
	PDH	&	Procuraduría de los Derechos Humanos de Guatemala	\\
	\rowcolor{color1!5!white}PMA	&	Programa Mundial de Alimentos	\\
	PPH0	&	Plan del Pacto Hambre Cero	\\
	\rowcolor{color1!5!white}SAN	&	Seguridad Alimentaria y Nutricional	\\
	Segeplan	&	Secretaría General de Planificación y Programación de la Presidencia	\\
	\rowcolor{color1!5!white}Sesan	&	Secretaría de Seguridad Alimentaria y Nutricional	\\
	Sicoin	&	Sistema de Contabilidad Integrada (Sicoin)	\\
	\rowcolor{color1!5!white}Sigsa	&	Sistema de Información Gerencial de Salud	\\
	URL	&	Universidad Rafael Landívar	\\
	\rowcolor{color1!5!white}Usaid	&	Agencia de los Estados Unidos para el Desarrollo Internacional, siglas en inglés	\\
	Visan	&	Viceministerio de Seguridad Alimentaria y Nutricional	\\
		\hline
		&\\[-0.28cm]
		%			\multicolumn{9}{l}{\footnotesize Fuente: Informe Nacional de Desarrollo Humano (PNUD), con base en las Encuestas Nacionales de Condiciones de Vida (Encovi).}
	\end{tabular}
\end{center}

\end{document}
