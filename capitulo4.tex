%#########################1########################

 \cajita{%
Lactancia en niños menores de 5 años }%
{%
De acuerdo a los datos de la Encuesta Nacional de Salud Materno-Infantil, en el período de análisis 2008/2009, mas del 90\% de los niños menores de 5 años habían recibido lactancia materna alguna vez.

Sin embargo, en el área urbana el porcentaje fue del 94.4\%, 2.7 puntos porcentuales menor a lo encontrado en niños del área rural.

La región con el mayor porcentaje de niños que no recibió lactancia materna fue el Suroriente, donde fue del 6.3\%.}%
{%
 Proporción de niños menores de cinco años que recibieron lactancia alguna vez, según varias características} %
{%
 República de Guatemala, 2008/2009, en porcentaje} %
{%
\small
%\ra{1.2}$\ $\\
\begin{tabular}{p{4cm}c}\hline
	\rowcolor{color2!0!white}&\\[-3mm]
	{\Bold   Característica}  & \textbf{Alguna vez recibió lactancia}\\
	\hline
	\rowcolor{color1!30!white} &\\[-4mm]
	\rowcolor{color1!30!white}\textbf{Área geográfica}	&		\\
	\rowcolor{color1!0!white}Urbana	&	94.4	\\
	Rural	&	97.1	\\
	\rowcolor{color1!30!white}\textbf{Región}	&		\\
	\rowcolor{color1!0!white}Metropolitana	&	94.2	\\
	Norte	&	95.9	\\
	Nororiente	&	95.4	\\
	Suroriente	&	93.7	\\
	Central	&	96.5	\\
	Suroccidente	&	97.1	\\
	Noroccidente	&	97.3	\\
	Petén	&	97.6	\\
	\rowcolor{color1!30!white}\textbf{Categoría étnica de la madre}	&		\\
	\rowcolor{color1!0!white}Indígena	&	97.1	\\
	Ladino	&	95.2	\\
	\hline
	
	\rowcolor{color1!0!white}	&\\[0.05cm]
\end{tabular}}%
{%
 ENSMI 2008/2009} %


%#########################2########################

\cajita{%
	Duración de la lactancia según área de residencia}%
{%
En cuanto a la duración de la lactancia materna en cualquiera de sus tipos\llamada, según el área geográfica de residencia, esta fue de 19.6 meses para los niños menores de 2 años en el área urbana y de 21.8 meses para el área rural.

\textollamada{Lactancia exclusiva, donde solo reciben pecho. Lactancia completa, pecho y agua.}}%
{%
	Duración de cualquier tipo de lactancia en niños menores de 2 años según área de residencia de la madre} %
{%
	República de Guatemala, 2008/2009, en meses} %
{%
	\begin{tikzpicture}[x=1pt,y=1pt]  \input{graficas/4_02.tex}  \end{tikzpicture}
}%
{%
	ENSMI 2008/2009} %


%#########################3########################

\cajita{%
	Duración de la lactancia según etnia}%
{%
En cuanto a la duración de la lactancia materna en cualquiera de sus tipos\llamada en niños menores de 2 años, según la etnicidad de la madre, esta fue de 22.6 meses para los niños con madres que se autoidentifican como indígenas y de 17.9 para las que se autoidentifican como ladinas.

\textollamada{Lactancia exclusiva, donde solo reciben pecho. Lactancia completa, pecho y agua.}}%
{%
	Duración de cualquier tipo de lactancia en niños menores de 2 años según etnia de la madre} %
{%
	República de Guatemala, 2008/2009, en meses} %
{%
	\begin{tikzpicture}[x=1pt,y=1pt]  \input{graficas/4_03.tex}  \end{tikzpicture}
}%
{%
	ENSMI 2008/2009} %

%#########################4########################

\cajita{%
	Duración de la lactancia por educación}%
{%
De acuerdo a la duración de la lactancia de cualquier tipo en niños menores de 2 años, esta presenta diferencias según el nivel educativo de la madre.

Para los niños con madres sin educación, la duración promedio de lactancia fue de 21.9 meses y para los que tenían madre con educación superior esta fue de 8.1 meses.}%
{%
	Duración de cualquier tipo de lactancia en niños menores de 2 años según el nivel de educación de la madre} %
{%
	República de Guatemala, 2008/2009, en meses} %
{%
	\begin{tikzpicture}[x=1pt,y=1pt]  \input{graficas/4_04.tex}  \end{tikzpicture}
}%
{%
	ENSMI 2008/2009} %


%#########################5########################

\cajita{%
Niños de 0 a 3 meses según lactancia}%
{%
La proporción de niños de 0 a 3 meses del área urbana, según el tipo de lactancia recibida, muestra que el 5.5\% no estaba lactando, el 37.9\% recibía lactancia exclusiva, en el 24.4\% el estado era predominantemente lactancia. El resto de niños del área urbana recibía otros tipos de lactancia\llamada. \textollamada{Se incluyen los niños que lactan y consumen líquidos no lácteos, lacta y consume otra leche y lacta y consume alimentos complementarios.}

Por otro lado, en el área urbana, recibían lactancia exclusiva el 67.4\% de los niños y lactancia predominante el 18.6\%. El 3.3\% no lactaba, 2.2 puntos porcentuales por debajo de los niños en el área urbana.}%
{%
	Proporción de niños de 0 a 3 meses por área de residencia, según principales situaciones de la lactancia
	} %
{%
	República de Guatemala, 2008/2009, en porcentaje} %
{%
	\begin{tikzpicture}[x=1pt,y=1pt]  \input{graficas/4_05.tex}  \end{tikzpicture}
}%
{%
	ENSMI 2008/2009} %


%#########################6########################

\cajita{%
	Niños de 0 a 3 meses según lactancia por etnia}%
{%
En cuanto a los niños de 0 a 3 meses según la etnicidad de la madre, 7 de cada 10 los niños con madre indígena recibían lactancia materna exclusiva y el 14.4\% recibían lactancia predominante. 

%La proporción de niños de 0 a 3 meses del área urbana, según el tipo de lactancia recibida, muestra que el 5.5\% no estaba lactando, el 37.9\% recibía lactancia exclusiva, en el 24.4\% el estado era predominantemente lactancia. El resto de niños del área urbana recibía otros tipos de lactancia\llamada. \textollamada{Acá se incluyen los niños que lactan y consumen líquidos no lácteos, lacta y consume otra leche y lacta y consume alimentos complementarios.}

Por otro lado, los niños con madre con etnicidad no indígena, el 4.5\% no estaba lactando, el 41.3\%  recibía lactancia exclusiva.}%
{%
 Proporción de niños de 0 a 3 meses según etnicidad de la madre, por tipo de lactancia recibido
} %
{%
	República de Guatemala, 2008/2009, en porcentaje} %
{%
	\begin{tikzpicture}[x=1pt,y=1pt]  \input{graficas/4_06.tex}  \end{tikzpicture}
}%
{%
	ENSMI 2008/2009} %


%#########################7########################

\cajita{%
	Niños de 0 a 3 meses según lactancia y educación de la madre}%
{%
De los niños de 0 a 3 meses cuyas madres no tenían educación formal, el 67.3\% recibió lactancia exclusiva, el 22.5\% lactancia predominante y el 3.6\% no lactaba; el resto recibía otros tipos de lactancia.

De las madres con educación superior, el 12.\% daba lactancia de forma predominante, el 5.7\% lactancia exclusiva y el 3.9\% no lactaba. El resto de porcentaje se distribuyen entre los tipos de lactancia que incluyen fórmula, que consumen líquidos no lácteos y otros.}%
{%
	Proporción de niños de 0 a 3 meses por educación de la madre, según situación de lactancia
} %
{%
	República de Guatemala, 2008/2009, en porcentaje} %
{%
	\begin{tikzpicture}[x=1pt,y=1pt]  \input{graficas/4_07.tex}  \end{tikzpicture}
}%
{%
	ENSMI 2008/2009} %

%#########################8########################

\cajita{%
	Niños de 0 a 5 meses según lactancia y área}%
{%
De acuerdo al área geográfica de residencia, el 32.5\% de los niños de 0 a 5 meses cuya madre vivía en área urbana recibía lactancia exclusiva, el 22.6\% lactancia predominante y el 9.9\% no estaba lactando.

Por otro lado, de los niños de 0 a 5 cuya madre residía en el área rural, el 60.4\% recibía lactancia exclusiva.}%
{%
	Proporción de niños de 0 a 5 meses según área geográfica de residencia por situación de lactancia
} %
{%
	República de Guatemala, 2008/2009, en porcentaje} %
{%
	\begin{tikzpicture}[x=1pt,y=1pt]  % Created by tikzDevice version 0.9 on 2016-03-03 04:58:32
% !TEX encoding = UTF-8 Unicode
\definecolor{fillColor}{RGB}{255,255,255}
\path[use as bounding box,fill=fillColor,fill opacity=0.00] (0,0) rectangle (289.08,198.74);
\begin{scope}
\path[clip] (  0.00,  0.00) rectangle (289.08,198.74);

\path[] (  0.00,  0.00) rectangle (289.08,198.74);
\end{scope}
\begin{scope}
\path[clip] (  0.00,  0.00) rectangle (289.08,198.74);

\path[] (  0.00, 18.46) rectangle (289.08,166.57);

\path[] ( 54.20, 18.46) --
	( 54.20,166.57);

\path[] (144.54, 18.46) --
	(144.54,166.57);

\path[] (234.88, 18.46) --
	(234.88,166.57);
\definecolor{drawColor}{RGB}{0,0,255}
\definecolor{fillColor}{RGB}{0,0,255}

\path[draw=drawColor,line width= 0.6pt,line join=round,fill=fillColor] ( 15.81, 18.46) rectangle ( 51.94, 42.73);
\definecolor{drawColor}{RGB}{157,187,255}
\definecolor{fillColor}{RGB}{157,187,255}

\path[draw=drawColor,line width= 0.6pt,line join=round,fill=fillColor] ( 56.46, 18.46) rectangle ( 92.60, 26.80);
\definecolor{drawColor}{RGB}{0,0,255}
\definecolor{fillColor}{RGB}{0,0,255}

\path[draw=drawColor,line width= 0.6pt,line join=round,fill=fillColor] (106.15, 18.46) rectangle (142.28, 98.16);
\definecolor{drawColor}{RGB}{157,187,255}
\definecolor{fillColor}{RGB}{157,187,255}

\path[draw=drawColor,line width= 0.6pt,line join=round,fill=fillColor] (146.80, 18.46) rectangle (182.93,166.57);
\definecolor{drawColor}{RGB}{0,0,255}
\definecolor{fillColor}{RGB}{0,0,255}

\path[draw=drawColor,line width= 0.6pt,line join=round,fill=fillColor] (196.48, 18.46) rectangle (232.62, 73.88);
\definecolor{drawColor}{RGB}{157,187,255}
\definecolor{fillColor}{RGB}{157,187,255}

\path[draw=drawColor,line width= 0.6pt,line join=round,fill=fillColor] (237.14, 18.46) rectangle (273.27, 61.13);
\definecolor{drawColor}{RGB}{0,0,0}

\path[draw=drawColor,line width= 0.6pt,line join=round] (  0.00, 18.46) -- (289.08, 18.46);

\node[text=drawColor,rotate= 90.00,anchor=base west,inner sep=0pt, outer sep=0pt, scale=  0.83] at ( 37.11, 44.56) {9.9};

\node[text=drawColor,rotate= 90.00,anchor=base west,inner sep=0pt, outer sep=0pt, scale=  0.83] at ( 77.76, 28.62) {3.4};

\node[text=drawColor,rotate= 90.00,anchor=base west,inner sep=0pt, outer sep=0pt, scale=  0.83] at (127.45,100.70) {32.5};

\node[text=drawColor,rotate= 90.00,anchor=base west,inner sep=0pt, outer sep=0pt, scale=  0.83] at (168.10,169.12) {60.4};

\node[text=drawColor,rotate= 90.00,anchor=base west,inner sep=0pt, outer sep=0pt, scale=  0.83] at (217.79, 76.43) {22.6};

\node[text=drawColor,rotate= 90.00,anchor=base west,inner sep=0pt, outer sep=0pt, scale=  0.83] at (258.44, 63.68) {17.4};

\path[] (  0.00, 18.46) rectangle (289.08,166.57);
\end{scope}
\begin{scope}
\path[clip] (  0.00,  0.00) rectangle (289.08,198.74);

\path[] (  0.00, 18.46) --
	(289.08, 18.46);
\end{scope}
\begin{scope}
\path[clip] (  0.00,  0.00) rectangle (289.08,198.74);

\path[] ( 54.20, 15.71) --
	( 54.20, 18.46);

\path[] (144.54, 15.71) --
	(144.54, 18.46);

\path[] (234.88, 15.71) --
	(234.88, 18.46);
\end{scope}
\begin{scope}
\path[clip] (  0.00,  0.00) rectangle (289.08,198.74);
\definecolor{drawColor}{RGB}{0,0,0}

\node[text=drawColor,anchor=base,inner sep=0pt, outer sep=0pt, scale=  1.00] at ( 54.20,  5.69) {No lactando};

\node[text=drawColor,anchor=base,inner sep=0pt, outer sep=0pt, scale=  1.00] at (144.54,  5.69) {Lactancia Exclusiva};

\node[text=drawColor,anchor=base,inner sep=0pt, outer sep=0pt, scale=  1.00] at (234.88,  5.69) {Lactancia Predominante};
\end{scope}
\begin{scope}
\path[clip] (  0.00,  0.00) rectangle (289.08,198.74);
\coordinate (apoyo) at (57.27,191.13);
\coordinate (longitudFicticia) at (7.11,7.61);
\coordinate (longitud) at (7.11,7.11);
\coordinate (desX) at (142.24,0);
\coordinate (desY) at (0,0.25);
\definecolor[named]{ct1}{HTML}{
0000FF
}
\definecolor[named]{ct2}{HTML}{
9DBBFF
}
\definecolor[named]{ctb1}{HTML}{
0000FF
}
\definecolor[named]{ctb2}{HTML}{
9DBBFF
}
\path [fill=none] (apoyo) rectangle ($(apoyo)+(longitudFicticia)$)
node [xshift=0.3cm,inner sep=0pt, outer sep=0pt,midway,right,scale = 0.9]{Urbana};
\draw [color = ctb1,fill=ct1] ( $(apoyo)  + (desY) $) rectangle ($(apoyo)+ (desY) +(longitud)$);
\path [fill=none] ($(apoyo)+(desX)$) rectangle ($(apoyo)+(desX)+(longitudFicticia)$)
node [xshift=0.3cm,inner sep=0pt, outer sep=0pt,midway,right,scale = 0.9]{Rural};
\draw [color = ctb2 ,fill=ct2] ( $(apoyo)  + (desY) + (desX) $) rectangle ($(apoyo)+ (desY)+ (desX) +(longitud)$);
\end{scope}
  \end{tikzpicture}
}%
{%
	ENSMI 2008/2009} %


%#########################9########################

\cajita{%
	Niños de 0 a 5 meses según lactancia por etnia}%
{%
De acuerdo a la etnia de la madre, el 66.4\% de los niños de 0 a 5 meses cuya madre se autoidenficiaba como indígena recibía lactancia exclusiva, el 15.5\% lactancia predominante y el 3.2\% no estaba lactando.

Por otro lado, de los niños de 0 a 5 de madre no indígena, el 34.4\% recibía lactancia exclusiva.}%
{%
	Proporción de niños de 0 a 5 meses según etnicidad de la madre, por situación de lactancia
} %
{%
	República de Guatemala, 2008/2009, en porcentaje} %
{%
	\begin{tikzpicture}[x=1pt,y=1pt]  % Created by tikzDevice version 0.9 on 2016-03-03 04:58:38
% !TEX encoding = UTF-8 Unicode
\definecolor{fillColor}{RGB}{255,255,255}
\path[use as bounding box,fill=fillColor,fill opacity=0.00] (0,0) rectangle (289.08,198.74);
\begin{scope}
\path[clip] (  0.00,  0.00) rectangle (289.08,198.74);

\path[] (  0.00,  0.00) rectangle (289.08,198.74);
\end{scope}
\begin{scope}
\path[clip] (  0.00,  0.00) rectangle (289.08,198.74);

\path[] (  0.00, 18.46) rectangle (289.08,164.23);

\path[] ( 54.20, 18.46) --
	( 54.20,164.23);

\path[] (144.54, 18.46) --
	(144.54,164.23);

\path[] (234.88, 18.46) --
	(234.88,164.23);
\definecolor{drawColor}{RGB}{0,0,255}
\definecolor{fillColor}{RGB}{0,0,255}

\path[draw=drawColor,line width= 0.6pt,line join=round,fill=fillColor] ( 15.81, 18.46) rectangle ( 51.94, 25.48);
\definecolor{drawColor}{RGB}{157,187,255}
\definecolor{fillColor}{RGB}{157,187,255}

\path[draw=drawColor,line width= 0.6pt,line join=round,fill=fillColor] ( 56.46, 18.46) rectangle ( 92.60, 36.90);
\definecolor{drawColor}{RGB}{0,0,255}
\definecolor{fillColor}{RGB}{0,0,255}

\path[draw=drawColor,line width= 0.6pt,line join=round,fill=fillColor] (106.15, 18.46) rectangle (142.28,164.23);
\definecolor{drawColor}{RGB}{157,187,255}
\definecolor{fillColor}{RGB}{157,187,255}

\path[draw=drawColor,line width= 0.6pt,line join=round,fill=fillColor] (146.80, 18.46) rectangle (182.93, 93.98);
\definecolor{drawColor}{RGB}{0,0,255}
\definecolor{fillColor}{RGB}{0,0,255}

\path[draw=drawColor,line width= 0.6pt,line join=round,fill=fillColor] (196.48, 18.46) rectangle (232.62, 52.49);
\definecolor{drawColor}{RGB}{157,187,255}
\definecolor{fillColor}{RGB}{157,187,255}

\path[draw=drawColor,line width= 0.6pt,line join=round,fill=fillColor] (237.14, 18.46) rectangle (273.27, 68.95);
\definecolor{drawColor}{RGB}{0,0,0}

\path[draw=drawColor,line width= 0.6pt,line join=round] (  0.00, 18.46) -- (289.08, 18.46);

\node[text=drawColor,rotate= 90.00,anchor=base west,inner sep=0pt, outer sep=0pt, scale=  0.83] at ( 37.11, 27.31) {3.2};

\node[text=drawColor,rotate= 90.00,anchor=base west,inner sep=0pt, outer sep=0pt, scale=  0.83] at ( 77.76, 38.72) {8.4};

\node[text=drawColor,rotate= 90.00,anchor=base west,inner sep=0pt, outer sep=0pt, scale=  0.83] at (127.45,166.78) {66.4};

\node[text=drawColor,rotate= 90.00,anchor=base west,inner sep=0pt, outer sep=0pt, scale=  0.83] at (168.10, 96.53) {34.4};

\node[text=drawColor,rotate= 90.00,anchor=base west,inner sep=0pt, outer sep=0pt, scale=  0.83] at (217.79, 55.03) {15.5};

\node[text=drawColor,rotate= 90.00,anchor=base west,inner sep=0pt, outer sep=0pt, scale=  0.83] at (258.44, 71.50) {23.0};

\path[] (  0.00, 18.46) rectangle (289.08,164.23);
\end{scope}
\begin{scope}
\path[clip] (  0.00,  0.00) rectangle (289.08,198.74);

\path[] (  0.00, 18.46) --
	(289.08, 18.46);
\end{scope}
\begin{scope}
\path[clip] (  0.00,  0.00) rectangle (289.08,198.74);

\path[] ( 54.20, 15.71) --
	( 54.20, 18.46);

\path[] (144.54, 15.71) --
	(144.54, 18.46);

\path[] (234.88, 15.71) --
	(234.88, 18.46);
\end{scope}
\begin{scope}
\path[clip] (  0.00,  0.00) rectangle (289.08,198.74);
\definecolor{drawColor}{RGB}{0,0,0}

\node[text=drawColor,anchor=base,inner sep=0pt, outer sep=0pt, scale=  1.00] at ( 54.20,  5.69) {No lactando};

\node[text=drawColor,anchor=base,inner sep=0pt, outer sep=0pt, scale=  1.00] at (144.54,  5.69) {Lactancia Exclusiva};

\node[text=drawColor,anchor=base,inner sep=0pt, outer sep=0pt, scale=  1.00] at (234.88,  5.69) {Lactancia Predominante};
\end{scope}
\begin{scope}
\path[clip] (  0.00,  0.00) rectangle (289.08,198.74);
\coordinate (apoyo) at (55.19,188.79);
\coordinate (longitudFicticia) at (7.11,9.95);
\coordinate (longitud) at (7.11,7.11);
\coordinate (desX) at (133.24,0);
\coordinate (desY) at (0,1.42);
\definecolor[named]{ct1}{HTML}{
0000FF
}
\definecolor[named]{ct2}{HTML}{
9DBBFF
}
\definecolor[named]{ctb1}{HTML}{
0000FF
}
\definecolor[named]{ctb2}{HTML}{
9DBBFF
}
\path [fill=none] (apoyo) rectangle ($(apoyo)+(longitudFicticia)$)
node [xshift=0.3cm,inner sep=0pt, outer sep=0pt,midway,right,scale = 0.9]{Indígena};
\draw [color = ctb1,fill=ct1] ( $(apoyo)  + (desY) $) rectangle ($(apoyo)+ (desY) +(longitud)$);
\path [fill=none] ($(apoyo)+(desX)$) rectangle ($(apoyo)+(desX)+(longitudFicticia)$)
node [xshift=0.3cm,inner sep=0pt, outer sep=0pt,midway,right,scale = 0.9]{No indígena};
\draw [color = ctb2 ,fill=ct2] ( $(apoyo)  + (desY) + (desX) $) rectangle ($(apoyo)+ (desY)+ (desX) +(longitud)$);
\end{scope}
  \end{tikzpicture}
}%
{%
	ENSMI 2008/2009} %


%#########################10########################

\cajita{%
	Niños de 0 a 5 meses según lactancia y educación de la madre}%
{%
De acuerdo a la situación de la lactancia en niños de 0 a 5 meses, ésta presenta diferencias según el nivel educativo de la madre.

Para los niños con madres sin educación, el 64.0\% recibía lactancia exclusiva, el 19.9\% lactancia predominante y el 3.0\% no estaba lactando. 

Por otro lado, los niños de 0 a 5 meses de madre con educación superior, el 9.9\% recibía lactancia predominante, el 8.9\% no lactaba y el 4.7\% lactancia exclusiva. El resto de niños con madre con educación superior se distribuía entre los estados de: recibir fórmula, combinación con líquidos no lácteos y otros.}%
{%
	Proporción de niños de 0 a 5 meses según educación de la madre, por situación lactancia
} %
{%
	República de Guatemala, 2008/2009, en porcentaje} %
{%
	\begin{tikzpicture}[x=1pt,y=1pt]  % Created by tikzDevice version 0.9 on 2016-03-03 04:58:43
% !TEX encoding = UTF-8 Unicode
\definecolor{fillColor}{RGB}{255,255,255}
\path[use as bounding box,fill=fillColor,fill opacity=0.00] (0,0) rectangle (289.08,198.74);
\begin{scope}
\path[clip] (  0.00,  0.00) rectangle (289.08,198.74);

\path[] (  0.00,  0.00) rectangle (289.08,198.74);
\end{scope}
\begin{scope}
\path[clip] (  0.00,  0.00) rectangle (289.08,198.74);

\path[] (  0.00, 18.46) rectangle (289.08,164.65);

\path[] ( 54.20, 18.46) --
	( 54.20,164.65);

\path[] (144.54, 18.46) --
	(144.54,164.65);

\path[] (234.88, 18.46) --
	(234.88,164.65);
\definecolor{drawColor}{RGB}{0,0,255}
\definecolor{fillColor}{RGB}{0,0,255}

\path[draw=drawColor,line width= 0.6pt,line join=round,fill=fillColor] ( 15.81, 18.46) rectangle ( 51.94, 25.31);
\definecolor{drawColor}{RGB}{157,187,255}
\definecolor{fillColor}{RGB}{157,187,255}

\path[draw=drawColor,line width= 0.6pt,line join=round,fill=fillColor] ( 56.46, 18.46) rectangle ( 92.60, 38.79);
\definecolor{drawColor}{RGB}{0,0,255}
\definecolor{fillColor}{RGB}{0,0,255}

\path[draw=drawColor,line width= 0.6pt,line join=round,fill=fillColor] (106.15, 18.46) rectangle (142.28,164.65);
\definecolor{drawColor}{RGB}{157,187,255}
\definecolor{fillColor}{RGB}{157,187,255}

\path[draw=drawColor,line width= 0.6pt,line join=round,fill=fillColor] (146.80, 18.46) rectangle (182.93, 29.19);
\definecolor{drawColor}{RGB}{0,0,255}
\definecolor{fillColor}{RGB}{0,0,255}

\path[draw=drawColor,line width= 0.6pt,line join=round,fill=fillColor] (196.48, 18.46) rectangle (232.62, 63.91);
\definecolor{drawColor}{RGB}{157,187,255}
\definecolor{fillColor}{RGB}{157,187,255}

\path[draw=drawColor,line width= 0.6pt,line join=round,fill=fillColor] (237.14, 18.46) rectangle (273.27, 41.07);
\definecolor{drawColor}{RGB}{0,0,0}

\path[draw=drawColor,line width= 0.6pt,line join=round] (  0.00, 18.46) -- (289.08, 18.46);

\node[text=drawColor,rotate= 90.00,anchor=base west,inner sep=0pt, outer sep=0pt, scale=  0.83] at ( 37.11, 27.13) {3.0};

\node[text=drawColor,rotate= 90.00,anchor=base west,inner sep=0pt, outer sep=0pt, scale=  0.83] at ( 77.76, 40.61) {8.9};

\node[text=drawColor,rotate= 90.00,anchor=base west,inner sep=0pt, outer sep=0pt, scale=  0.83] at (127.45,167.20) {64.0};

\node[text=drawColor,rotate= 90.00,anchor=base west,inner sep=0pt, outer sep=0pt, scale=  0.83] at (168.10, 31.02) {4.7};

\node[text=drawColor,rotate= 90.00,anchor=base west,inner sep=0pt, outer sep=0pt, scale=  0.83] at (217.79, 66.46) {19.9};

\node[text=drawColor,rotate= 90.00,anchor=base west,inner sep=0pt, outer sep=0pt, scale=  0.83] at (258.44, 42.89) {9.9};

\path[] (  0.00, 18.46) rectangle (289.08,164.65);
\end{scope}
\begin{scope}
\path[clip] (  0.00,  0.00) rectangle (289.08,198.74);

\path[] (  0.00, 18.46) --
	(289.08, 18.46);
\end{scope}
\begin{scope}
\path[clip] (  0.00,  0.00) rectangle (289.08,198.74);

\path[] ( 54.20, 15.71) --
	( 54.20, 18.46);

\path[] (144.54, 15.71) --
	(144.54, 18.46);

\path[] (234.88, 15.71) --
	(234.88, 18.46);
\end{scope}
\begin{scope}
\path[clip] (  0.00,  0.00) rectangle (289.08,198.74);
\definecolor{drawColor}{RGB}{0,0,0}

\node[text=drawColor,anchor=base,inner sep=0pt, outer sep=0pt, scale=  1.00] at ( 54.20,  5.69) {No lactando};

\node[text=drawColor,anchor=base,inner sep=0pt, outer sep=0pt, scale=  1.00] at (144.54,  5.69) {Lactancia Exclusiva};

\node[text=drawColor,anchor=base,inner sep=0pt, outer sep=0pt, scale=  1.00] at (234.88,  5.69) {Lactancia Predominante};
\end{scope}
\begin{scope}
\path[clip] (  0.00,  0.00) rectangle (289.08,198.74);
\coordinate (apoyo) at (46.95,189.2);
\coordinate (longitudFicticia) at (7.11,9.54);
\coordinate (longitud) at (7.11,7.11);
\coordinate (desX) at (147.1,0);
\coordinate (desY) at (0,1.21);
\definecolor[named]{ct1}{HTML}{
0000FF
}
\definecolor[named]{ct2}{HTML}{
9DBBFF
}
\definecolor[named]{ctb1}{HTML}{
0000FF
}
\definecolor[named]{ctb2}{HTML}{
9DBBFF
}
\path [fill=none] (apoyo) rectangle ($(apoyo)+(longitudFicticia)$)
node [xshift=0.3cm,inner sep=0pt, outer sep=0pt,midway,right,scale = 0.9]{Sin educación};
\draw [color = ctb1,fill=ct1] ( $(apoyo)  + (desY) $) rectangle ($(apoyo)+ (desY) +(longitud)$);
\path [fill=none] ($(apoyo)+(desX)$) rectangle ($(apoyo)+(desX)+(longitudFicticia)$)
node [xshift=0.3cm,inner sep=0pt, outer sep=0pt,midway,right,scale = 0.9]{Superior};
\draw [color = ctb2 ,fill=ct2] ( $(apoyo)  + (desY) + (desX) $) rectangle ($(apoyo)+ (desY)+ (desX) +(longitud)$);
\end{scope}
  \end{tikzpicture}
}%
{%
	ENSMI 2008/2009} %