$\ $

\cleardoublepage



\indent\titulo{Introducción}
\\


El Compendio Estadístico de Seguridad Alimentaria y Nutricional (SAN) 2015 se integra bajo el enfoque de dimensiones/pilares de la seguridad alimentaria y nutricional, con la finalidad de brindar información de utilidad para evaluar el estado de SAN para Guatemala, y apoyar de esta forma las políticas públicas sobre el tema, así como fortalecer la toma de decisiones informada y la participación ciudadana. 

Es por ello que el Compendio se ha integrado en las dimensiones/pilares de población, disponibilidad de alimentos, acceso a los alimentos, consumo de los alimentos, utilización biológica de los alimentos,  situación y atención a la desnutrición/malnutrición e inversión pública en SAN.

Para la recopilación de la información se llevó a cabo revisión bibliográfica, consultas con expertos, y socialización de la información con los representantes de la Oficina Coordinadora Sectorial de Estadísticas de Seguridad Alimentaria y Nutricional (Ocsesan).

Como resultado, se presenta un Compendio Estadístico de SAN 2015 que reúne información sobre los indicadores y sub-indicadores identificados. No obstante, la información presentada en este Compendio dependió de la disponibilidad y calidad de los datos, por lo que no todos los indicadores se encuentran representados en este informe. El detalle de la información faltante se da al inicio de cada dimensión, y se espera que la misma pueda integrarse a futuras versiones del compendio.\\[16mm]
