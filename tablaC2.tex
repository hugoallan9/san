


\begin{center}
	\begin{tabular}{lS[table-format=8]S[table-format=8]S[table-format=8]}
		\multicolumn{4}{l}{\Bold\color{color1!80!black}{Cuadro \theCuadro $\,-$   Maíz (Zea mays) por área cosechada, producción y rendimiento}}\\
		\multicolumn{4}{l}{\Bold\color{color1!80!black}{según año agrícola. República de Guatemala, años varios.}}\\
		%		\multicolumn{4}{l}{(Población de 15 o más años de edad)}
		\\[0.4cm]
		\hline &&&\\[-0.36cm]  
		\multicolumn{1}{x{2.7cm}}{ } &	\multicolumn{3}{c}{\Bold{Maíz en grano}}\\[0.05cm]\cline{2-4}
		\multicolumn{1}{x{2.7cm}}{\Bold{Año agrícola 1/}} &	\multicolumn{1}{x{2.7cm}}{\Bold{Área cosechada}} & \multicolumn{1}{x{2.7cm}}{\Bold{Producción }} & \multicolumn{1}{x{2.4cm}}{\Bold{Rendimiento}}\\[0.05cm]
		\multicolumn{1}{x{2.7cm}}{} &	\multicolumn{1}{x{2.7cm}}{\Bold{(manzanas)}} & \multicolumn{1}{x{2.7cm}}{\Bold{(quintales)}} & \multicolumn{1}{x{2.4cm}}{\Bold{(qq/mz)}}\\[0.05cm]
		\hline
		\rowcolor{color1!10!white}	&&&\\[-0.35cm]
		\rowcolor{color1!10!white}2009/2010	&	1,174,955	&	35,842,974	&	30.5	\\[0.05cm]
		2010/2011	&	1,175,255	&	36,117,212	&	30.7	\\[0.05cm]
		\rowcolor{color1!10!white}2011/2012	&	1,199,900	&	36,932,600	&	30.8	\\[0.05cm]
		2012/2013	&	1,211,900	&	37,995,900	&	31.4	\\[0.05cm]
		\rowcolor{color1!10!white}2013/2014 p/	&	1,233,300	&	39,576,500	&	32.1	\\[0.05cm]
		2014/2015 e/	&	1,247,100	&	40,724,100	&	32.7	\\
		
		\hline
		&&&\\[-0.36cm]
		\multicolumn{4}{l}{\footnotesize Fuente: Diplan-MAGA con datos de Banguat (MAGA, 2013).}\\
		\multicolumn{4}{l}{\footnotesize 1/ De mayo de un año a abril del siguiente.}\\
		\multicolumn{4}{l}{\footnotesize p/ Cifras preliminares.  e/ Cifras estimadas.}\\	
		%		\multicolumn{4}{m{15cm}}{\footnotesize	Ministerio de Agricultura, Ganadería y Alimentación (MAGA). 2013. El agro en cifras 2015. En: \url{http://web.maga.gob.gt/download/1agro-cifras2014.pdf}  (Consultado: febrero 2016).}
	\end{tabular}\addtocounter{Cuadro}{1}
\end{center}
{\footnotesize	Ministerio de Agricultura, Ganadería y Alimentación (MAGA). 2013. El agro en cifras 2015. En: \url{http://web.maga.gob.gt/download/1agro-cifras2014.pdf}  (Consultado: febrero 2016).}




%%% 2-2



\begin{center}
	\begin{tabular}{lS[table-format=8]S[table-format=8]S[table-format=3]}
		\multicolumn{4}{l}{\Bold\color{color1!80!black}{Cuadro \theCuadro $\,-$   Frijol (Phaseolus vulgaris) por área cosechada, producción y rendimiento}}\\
		\multicolumn{4}{l}{\Bold\color{color1!80!black}{según año agrícola. República de Guatemala, años varios.}}\\
		%		\multicolumn{4}{l}{(Población de 15 o más años de edad)}
		\\[0.4cm]
		\hline &&&\\[-0.36cm]  
		\multicolumn{1}{x{2.7cm}}{ } &	\multicolumn{3}{c}{\Bold{Frijol}}\\[0.05cm]\cline{2-4}
		\multicolumn{1}{x{2.7cm}}{\Bold{Año agrícola 1/}} &	\multicolumn{1}{x{2.7cm}}{\Bold{Área cosechada}} & \multicolumn{1}{x{2.7cm}}{\Bold{Producción }} & \multicolumn{1}{x{2.4cm}}{\Bold{Rendimiento}}\\[0.05cm]
		\multicolumn{1}{x{2.7cm}}{} &	\multicolumn{1}{x{2.7cm}}{\Bold{(manzanas)}} & \multicolumn{1}{x{2.7cm}}{\Bold{(quintales)}} & \multicolumn{1}{x{2.4cm}}{\Bold{(qq/mz)}}\\[0.05cm]
		\hline
		\rowcolor{color1!10!white}	&&&\\[-0.35cm]
		\rowcolor{color1!10!white}	2009/2010	&	336,500	&	4,367,660	&	13.0	\\[0.05cm]
		2010/2011	&	336,756	&	4,610,828	&	13.7	\\[0.05cm]
		\rowcolor{color1!10!white}	2011/2012	&	339,200	&	4,704,200	&	13.9	\\[0.05cm]
		2012/2013	&	345,400	&	4,845,500	&	14.0	\\[0.05cm]
		\rowcolor{color1!10!white}	2013/2014 p/	&	352,500	&	5,026,200	&	14.3	\\[0.05cm]
		2014/2015 e/	&	358,300	&	5,181,500	&	14.5	\\[0.05cm]
		\hline
		&&&\\[-0.36cm]
		\multicolumn{4}{l}{\footnotesize Fuente: Diplan-MAGA con datos de Banguat (MAGA, 2013).}\\
		\multicolumn{4}{l}{\footnotesize 1/ De mayo de un año a abril del siguiente.}\\
		\multicolumn{4}{l}{\footnotesize p/ Cifras preliminares.  e/ Cifras estimadas.}\\	
		%					\multicolumn{4}{m{15cm}}{\footnotesize	Ministerio de Agricultura, Ganadería y Alimentación (MAGA). 2013. El agro en cifras 2015. En: \url{http://web.maga.gob.gt/download/1agro-cifras2014.pdf}  (Consultado: febrero 2016).}
	\end{tabular}\addtocounter{Cuadro}{1}
\end{center}
{\footnotesize	Ministerio de Agricultura, Ganadería y Alimentación (MAGA). 2013. El agro en cifras 2015. En: \url{http://web.maga.gob.gt/download/1agro-cifras2014.pdf}  (Consultado: febrero 2016).}




%%%%%  2-3

\hoja{
	{\Bold\color{color1!80!black}{Cuadro \theCuadro $\,-$    Arroz (Oryza sativa), por área cosechada, producción y rendimiento}}\\
	{\Bold\color{color1!80!black}{según año agrícola. República de Guatemala, años varios.}}\\
	\begin{center}
		\begin{tabular}{lccc}
			%		\multicolumn{4}{l}{(Población de 15 o más años de edad)}
			\hline &&&\\[-0.36cm]  
			\multicolumn{1}{r}{ } &	\multicolumn{3}{c}{\Bold{Arroz}}\\%[0.05cm]
			&&&\\[-0.3cm]\cline{2-4}
			&&&\\[-0.15cm]
			\multicolumn{1}{c}{\Bold{Año agrícola 1/}} &	\multicolumn{1}{c}{\Bold{Área cosechada}} & \multicolumn{1}{c}{\Bold{Producción \2}} & \multicolumn{1}{c}{\Bold{Rendimiento}}\\[0.05cm]
			\multicolumn{1}{c}{} &	\multicolumn{1}{c}{\Bold{(manzanas)}} & \multicolumn{1}{c}{\Bold{(quintales)}} & \multicolumn{1}{c}{\Bold{(qq/mz)}}\\[0.05cm]
			\hline
			\rowcolor{color1!10!white}	&&&\\[-0.35cm]
			\rowcolor{color1!10!white}	2009/2010	&	14,200	&	632,765	&	44.6	\\[0.05cm]
			2010/2011	&	15,012	&	653,140	&	43.5	\\[0.05cm]
			\rowcolor{color1!10!white}	2011/2012	&	15,200	&	670,300	&	44.1	\\[0.05cm]
			2012/2013	&	15,400	&	686,400	&	44.6	\\[0.05cm]
			\rowcolor{color1!10!white}	2013/2014 p/	&	15,700	&	710,900	&	45.3	\\[0.05cm]
			2014/2015 e/	&	16,000	&	732,900	&	45.8	\\[0.05cm]
			\hline
			&&&\\[-0.36cm]
			\multicolumn{4}{l}{\footnotesize Fuente: Diplan-MAGA con datos de Banguat (MAGA, 2013).}\\
			\multicolumn{4}{l}{\footnotesize 1/ De mayo de un año a abril del siguiente.}\\
			\multicolumn{4}{l}{\footnotesize 2/ Se refiere al grano en granza.}\\
			\multicolumn{4}{l}{\footnotesize p/ Cifras preliminares.  e/ Cifras estimadas.}\\	
			%					\multicolumn{4}{m{15cm}}{\footnotesize	Ministerio de Agricultura, Ganadería y Alimentación (MAGA). 2013. El agro en cifras 2015. En: \url{http://web.maga.gob.gt/download/1agro-cifras2014.pdf}  (Consultado: febrero 2016).}
		\end{tabular}\addtocounter{Cuadro}{1}
	\end{center}
	{\footnotesize	Ministerio de Agricultura, Ganadería y Alimentación (MAGA). 2013. El agro en cifras 2015. En: \url{http://web.maga.gob.gt/download/1agro-cifras2014.pdf}  (Consultado: febrero 2016).}
	
	%%%%%%%%%%%
	$\ $\\[1cm]
	{\Bold\color{color1!80!black}{Cuadro \theCuadro $\,-$    Trigo (Triticum spp.), por área cosechada, producción y rendimiento}}\\
	{\Bold\color{color1!80!black}{según año agrícola. República de Guatemala, años varios.}}\\
	$\ $\\[-1cm]
	\begin{center}
		\begin{tabular}{lS[table-format=8]S[table-format=8]S[table-format=3]}
			
			%		\multicolumn{4}{l}{(Población de 15 o más años de edad)}
			&&&\\[0.6cm]
			\hline &&&\\[-0.36cm]  
			\multicolumn{1}{x{2.7cm}}{ } &	\multicolumn{3}{c}{\Bold{Trigo}}\\[0.01cm]\cline{2-4}
			\multicolumn{1}{x{2.7cm}}{\Bold{Año agrícola 1/}} &	\multicolumn{1}{x{2.7cm}}{\Bold{Área cosechada}} & \multicolumn{1}{x{2.7cm}}{\Bold{Producción \2}} & \multicolumn{1}{x{2.4cm}}{\Bold{Rendimiento}}\\[0.05cm]
			\multicolumn{1}{x{2.7cm}}{} &	\multicolumn{1}{x{2.7cm}}{\Bold{(manzanas)}} & \multicolumn{1}{x{2.7cm}}{\Bold{(quintales)}} & \multicolumn{1}{x{2.4cm}}{\Bold{(qq/mz)}}\\[0.05cm]
			\hline
			\rowcolor{color1!10!white}	&&&\\[-0.35cm]
			\rowcolor{color1!10!white}	2008/2009	&	1,005	&	35,633	&	35.5	\\[0.05cm]
			2009/2010	&	975	&	34,125	&	35.0	\\[0.05cm]
			\rowcolor{color1!10!white}	2010/2011	&	968	&	31,681	&	32.7	\\[0.05cm]
			2011/2012	&	1,000	&	31,600	&	31.6	\\[0.05cm]
			\rowcolor{color1!10!white}	2012/2013	&	1,000	&	33,400	&	33.4	\\[0.05cm]
			2013/2014 p/	&	1,100	&	34,300	&	31.2	\\[0.05cm]
			\rowcolor{color1!10!white}	2014/2015 e/	&	1,100	&	35,800	&	32.5	\\[0.05cm]
			\hline
			&&&\\[-0.36cm]
			\multicolumn{4}{l}{\footnotesize Fuente: Diplan-MAGA con datos de Banguat (MAGA, 2013).}\\
			\multicolumn{4}{l}{\footnotesize 1/ De mayo de un año a abril del siguiente.}\\
			%			\multicolumn{4}{l}{\footnotesize 2/ Se refiere al grano en granza.}\\
			\multicolumn{4}{l}{\footnotesize p/ Cifras preliminares.  e/ Cifras estimadas.}\\	
			%					\multicolumn{4}{m{15cm}}{\footnotesize	Ministerio de Agricultura, Ganadería y Alimentación (MAGA). 2013. El agro en cifras 2015. En: \url{http://web.maga.gob.gt/download/1agro-cifras2014.pdf}  (Consultado: febrero 2016).}
		\end{tabular}\addtocounter{Cuadro}{1}
	\end{center}
	{\footnotesize	Ministerio de Agricultura, Ganadería y Alimentación (MAGA). 2013. El agro en cifras 2015. En: \url{http://web.maga.gob.gt/download/1agro-cifras2014.pdf}  (Consultado: febrero 2016).}
	
}

\hoja{
	\begin{center}
		\begin{tabular}{lS[table-format=8]S[table-format=8]S[table-format=8]}
			\multicolumn{4}{l}{\Bold\color{color1!80!black}{Cuadro \theCuadro $\,-$    Ajonjolí (Sesamum indicum), por área cosechada, producción y rendimiento}}\\
			\multicolumn{4}{l}{\Bold\color{color1!80!black}{según año agrícola. República de Guatemala, años varios.}}\\
			%		\multicolumn{4}{l}{(Población de 15 o más años de edad)}
			%[0.4cm]
			\hline &&&\\[-0.36cm]  
			\multicolumn{1}{x{2.7cm}}{ } &	\multicolumn{3}{c}{\Bold{Ajonjolí}}\\[0.05cm]\cline{2-4}
			\multicolumn{1}{x{2.7cm}}{\Bold{Año agrícola 1/}} &	\multicolumn{1}{x{2.7cm}}{\Bold{Área cosechada}} & \multicolumn{1}{x{2.7cm}}{\Bold{Producción 2/}} & \multicolumn{1}{x{2.4cm}}{\Bold{Rendimiento}}\\[0.05cm]
			\multicolumn{1}{x{2.7cm}}{} &	\multicolumn{1}{x{2.7cm}}{\Bold{(manzanas)}} & \multicolumn{1}{x{2.7cm}}{\Bold{(quintales)}} & \multicolumn{1}{x{2.4cm}}{\Bold{(qq/mz)}}\\[0.05cm]
			\hline
			\rowcolor{color1!10!white}	&&&\\[-0.35cm]
			\rowcolor{color1!10!white}	2008/2009	&	48,900	&	838,000	&	17.1	\\[0.05cm]
			2009/2010	&	50,378	&	1,099,422	&	21.8	\\[0.05cm]
			\rowcolor{color1!10!white}	2010/2011	&	43,200	&	870,100	&	20.1	\\[0.05cm]
			2011/2012	&	54,000	&	1,077,600	&	20.0	\\[0.05cm]
			\rowcolor{color1!10!white}	2012/2013 p/	&	56,200	&	1,304,500	&	23.2	\\[0.05cm]
			2013/2014 e/	&	54,900	&	1,119,800	&	20.4	\\[0.05cm]
			\hline
			&&&\\[-0.36cm]
			\multicolumn{4}{l}{\footnotesize Fuente: Diplan-MAGA con datos de Banguat (MAGA, 2013).}\\
			\multicolumn{4}{l}{\footnotesize 1/ De octubre de un año a septiembre del siguiente.}\\
			%			\multicolumn{4}{l}{\footnotesize 2/ Se refiere al grano en granza.}\\
			\multicolumn{4}{l}{\footnotesize p/ Cifras preliminares.  e/ Cifras estimadas.}\\	
			%					\multicolumn{4}{m{15cm}}{\footnotesize	Ministerio de Agricultura, Ganadería y Alimentación (MAGA). 2013. El agro en cifras 2015. En: \url{http://web.maga.gob.gt/download/1agro-cifras2014.pdf}  (Consultado: febrero 2016).}
		\end{tabular}\addtocounter{Cuadro}{1}
	\end{center}
	{\footnotesize	Ministerio de Agricultura, Ganadería y Alimentación (MAGA). 2013. El agro en cifras 2015. En: \url{http://web.maga.gob.gt/download/1agro-cifras2014.pdf}  (Consultado: febrero 2016).}
	
	
	%%%%%%%%%%%%%%%%%%
	$\ $\\[1cm]
	
	{\Bold\color{color1!80!black}{Cuadro \theCuadro $\,-$    Comercio exterior de maíz blanco}}\\
	{\Bold\color{color1!80!black}{según año. República de Guatemala,  2005 - 2014.}}\\
	{(En toneladas métricas.)}\\[-3cm]
	
	\begin{center}
		\begin{tabular}{cS[table-format=8]S[table-format=8]}
			
			%[0.4cm]
			\hline &&\\[-0.36cm]  
			\multicolumn{1}{x{2.7cm}}{ } &	\multicolumn{2}{c}{\Bold{Maíz blanco}}\\[0.05cm]\cline{2-3}
			\multicolumn{1}{x{2.7cm}}{\Bold{\raisebox{.4cm}{Año}}}&\multicolumn{1}{x{2.7cm}}{\Bold{Exportaciones}} &	\multicolumn{1}{x{2.7cm}}{\Bold{Importaciones}} \\[0.05cm]
			\hline
			\rowcolor{color1!10!white}	&&\\[-0.35cm]
			\rowcolor{color1!10!white}	2005	&	457.09	&	78,206.93	\\[0.05cm]
			2006	&	8.78	&	80,426.15	\\[0.05cm]
			\rowcolor{color1!10!white}	2007	&	4,094.29	&	58,143.62	\\[0.05cm]
			2008	&	11,977.73	&	19,558.90	\\[0.05cm]
			\rowcolor{color1!10!white}	2009	&	2,153.46	&	39,092.91	\\[0.05cm]
			2010	&	2,127.54	&	24,745.31	\\[0.05cm]
			\rowcolor{color1!10!white}	2011	&	14,164.00	&	41,547.83	\\[0.05cm]
			2012	&	2,568.63	&	36,393.62	\\[0.05cm]
			\rowcolor{color1!10!white}	2013	&	8,214.93	&	18,422.06	\\[0.05cm]
			2014*	&	1,994.95	&	23,081.27	\\[0.05cm]
			\hline
			&&\\[-0.36cm]
			\multicolumn{3}{l}{\footnotesize Fuente: Diplan-MAGA con datos de Banguat (MAGA, 2013).}\\
			\multicolumn{3}{l}{\footnotesize */ Cifras al mes de agosto.}\\
		\end{tabular}\addtocounter{Cuadro}{1}
	\end{center}
	{\footnotesize	Ministerio de Agricultura, Ganadería y Alimentación (MAGA). 2013. El agro en cifras 2015. En: \url{http://web.maga.gob.gt/download/1agro-cifras2014.pdf}  (Consultado: febrero 2016).}
	
}





\begin{center}
	\begin{tabular}{cS[table-format=8]S[table-format=8]S[table-format=8]}
		\multicolumn{4}{l}{\Bold\color{color1!80!black}{Cuadro \theCuadro $\,-$    Comercio exterior de maíz amarillo}}\\
		\multicolumn{4}{l}{\Bold\color{color1!80!black}{según año. República de Guatemala,  2005 - 2014.}}\\
		\multicolumn{4}{l}{(En toneladas métricas.)}\\
		%[0.4cm]
		\hline &&&\\[-0.36cm]  
		\multicolumn{1}{x{2.7cm}}{ } &	\multicolumn{3}{c}{\Bold{Maíz amarillo}}\\[0.05cm]\cline{2-4}
		\multicolumn{1}{x{2.7cm}}{\Bold{\raisebox{.4cm}{Año}}}&\multicolumn{1}{x{2.7cm}}{\Bold{Exportaciones}} &	\multicolumn{1}{x{2.7cm}}{\Bold{Importaciones}} &	\multicolumn{1}{x{2.7cm}}{\Bold{Balanza}} \\[0.05cm]
		\hline
		\rowcolor{color1!10!white}&	&&\\[-0.35cm]
		\rowcolor{color1!10!white}	2005	&	12.9	&	585,177.2	&	-585,164.4	\\[0.05cm]
		2006	&	0.0	&	686,018.6	&	-686,018.5	\\[0.05cm]
		\rowcolor{color1!10!white}	2007	&	62.1	&	641,780.6	&	-641,718.5	\\[0.05cm]
		2008	&	0.8	&	574,103.6	&	-574,102.8	\\[0.05cm]
		\rowcolor{color1!10!white}	2009	&	2,437.2	&	618,840.8	&	-616,403.6	\\[0.05cm]
		2010	&	782.3	&	602,003.1	&	-601,220.8	\\[0.05cm]
		\rowcolor{color1!10!white}	2011	&	0.0	&	666,495.0	&	-666,495.0	\\[0.05cm]
		2012	&	2.5	&	652,455.8	&	-652,453.3	\\[0.05cm]
		\rowcolor{color1!10!white}	2013	&	4.1	&	667,311.8	&	-667,307.8	\\[0.05cm]
		2014*	&	25.2	&	519,406.2	&	-519,381.0	\\[0.05cm]
		
		\hline
		&&&\\[-0.36cm]
		\multicolumn{4}{l}{\footnotesize Fuente: Diplan-MAGA con datos de Banguat (MAGA, 2013).}\\
		\multicolumn{4}{l}{\footnotesize */ Cifras al mes de agosto.}\\
	\end{tabular}\addtocounter{Cuadro}{1}
\end{center}
{\footnotesize	Ministerio de Agricultura, Ganadería y Alimentación (MAGA). 2013. El agro en cifras 2015. En: \url{http://web.maga.gob.gt/download/1agro-cifras2014.pdf}  (Consultado: febrero 2016).}







\begin{center}
	\begin{tabular}{cS[table-format=8]S[table-format=8]S[table-format=8]}
		\multicolumn{4}{l}{\Bold\color{color1!80!black}{Cuadro \theCuadro $\,-$    Comercio exterior de frijol (Phaseolus vulgaris)}}\\
		\multicolumn{4}{l}{\Bold\color{color1!80!black}{según año. República de Guatemala,  2005 - 2014.}}\\
		\multicolumn{4}{l}{(En toneladas métricas.)}\\
		%[0.4cm]
		\hline &&&\\[-0.36cm]  
		\multicolumn{1}{x{2.7cm}}{ } &	\multicolumn{3}{c}{\Bold{Frijol}}\\[0.05cm]\cline{2-4}
		\multicolumn{1}{x{2.7cm}}{\Bold{\raisebox{.4cm}{Año}}}&\multicolumn{1}{x{2.7cm}}{\Bold{Exportaciones}} &	\multicolumn{1}{x{2.7cm}}{\Bold{Importaciones}} &	\multicolumn{1}{x{2.7cm}}{\Bold{Balanza}} \\[0.05cm]
		\hline
		\rowcolor{color1!10!white}&	&&\\[-0.35cm]
		\rowcolor{color1!10!white}	2005	&	1,093.8	&	4,778.9	&	-3,685.1	\\[0.05cm]
		2006	&	55.8	&	11,547.6	&	-11,491.8	\\[0.05cm]
		\rowcolor{color1!10!white}	2007	&	2,306.1	&	7,975.8	&	-5,669.7	\\[0.05cm]
		2008	&	2,187.1	&	5,148.7	&	-2,961.6	\\[0.05cm]
		\rowcolor{color1!10!white}	2009	&	478.1	&	8,134.5	&	-7,656.4	\\[0.05cm]
		2010	&	1,246.3	&	11,913.9	&	-10,667.6	\\[0.05cm]
		\rowcolor{color1!10!white}	2011	&	1,568.0	&	20,531.0	&	-18,963.0	\\[0.05cm]
		2012	&	216.9	&	10,980.1	&	-10,763.2	\\[0.05cm]
		\rowcolor{color1!10!white}	2013	&	1,508.1	&	6,414.1	&	-4,906.1	\\[0.05cm]
		2014*	&	2,222.7	&	3,839.2	&	-1,616.5	\\[0.05cm]
		\hline
		&&&\\[-0.36cm]
		\multicolumn{4}{l}{\footnotesize Fuente: Diplan-MAGA con datos de Banguat (MAGA, 2013).}\\
		\multicolumn{4}{l}{\footnotesize */ Cifras al mes de agosto.}\\
	\end{tabular}\addtocounter{Cuadro}{1}
\end{center}
{\footnotesize	Ministerio de Agricultura, Ganadería y Alimentación (MAGA). 2013. El agro en cifras 2015. En: \url{http://web.maga.gob.gt/download/1agro-cifras2014.pdf}  (Consultado: febrero 2016).}







\begin{center}
	\begin{tabular}{cS[table-format=8]S[table-format=8]S[table-format=8]}
		\multicolumn{4}{l}{\Bold\color{color1!80!black}{Cuadro \theCuadro $\,-$    Comercio exterior de arroz (Oriza sativa)}}\\
		\multicolumn{4}{l}{\Bold\color{color1!80!black}{según año. República de Guatemala,  2005 - 2014.}}\\
		\multicolumn{4}{l}{(En toneladas métricas.)}\\
		%[0.4cm]
		\hline &&&\\[-0.36cm]  
		\multicolumn{1}{x{2.7cm}}{ } &	\multicolumn{3}{c}{\Bold{Arroz}}\\[0.05cm]\cline{2-4}
		\multicolumn{1}{x{2.7cm}}{\Bold{\raisebox{.4cm}{Año}}}&\multicolumn{1}{x{2.7cm}}{\Bold{Exportaciones}} &	\multicolumn{1}{x{2.7cm}}{\Bold{Importaciones}} &	\multicolumn{1}{x{2.7cm}}{\Bold{Balanza}} \\[0.05cm]
		\hline
		\rowcolor{color1!10!white}&	&&\\[-0.35cm]
		\rowcolor{color1!10!white}	2005	&	3,204.51	&	91,283.85	&	-88,079.34	\\[0.05cm]
		2006	&	4,720.84	&	108,190.15	&	-103,469.31	\\[0.05cm]
		\rowcolor{color1!10!white}	2007	&	6,351.59	&	96,218.06	&	-89,866.47	\\[0.05cm]
		2008	&	5,338.60	&	88,951.71	&	-83,613.11	\\[0.05cm]
		\rowcolor{color1!10!white}	2009	&	4,112.57	&	83,050.01	&	-78,937.44	\\[0.05cm]
		2010	&	2,445.67	&	71,041.94	&	-68,596.27	\\[0.05cm]
		\rowcolor{color1!10!white}	2011	&	1,472.00	&	77,464.00	&	-75,992.00	\\[0.05cm]
		2012	&	2,261.72	&	101,424.27	&	-99,162.55	\\[0.05cm]
		\rowcolor{color1!10!white}	2013	&	490.57	&	97,845.52	&	-97,354.95	\\[0.05cm]
		2014*	&	55.76	&	64,289.78	&	-64,234.02	\\[0.05cm]
		\hline
		&&&\\[-0.36cm]
		\multicolumn{4}{l}{\footnotesize Fuente: Diplan-MAGA con datos de Banguat (MAGA, 2013).}\\
		\multicolumn{4}{l}{\footnotesize */ Cifras al mes de agosto.}\\
	\end{tabular}\addtocounter{Cuadro}{1}
\end{center}
{\footnotesize	Ministerio de Agricultura, Ganadería y Alimentación (MAGA). 2013. El agro en cifras 2015. En: \url{http://web.maga.gob.gt/download/1agro-cifras2014.pdf}  (Consultado: febrero 2016).}






\begin{center}
	\begin{tabular}{cS[table-format=8]S[table-format=8]S[table-format=8]}
		\multicolumn{4}{l}{\Bold\color{color1!80!black}{Cuadro \theCuadro $\,-$    Comercio exterior de trigo (Triticum spp.)}}\\
		\multicolumn{4}{l}{\Bold\color{color1!80!black}{según año. República de Guatemala,  2005 - 2014.}}\\
		\multicolumn{4}{l}{(En toneladas métricas.)}\\
		%[0.4cm]
		\hline &&&\\[-0.36cm]  
		\multicolumn{1}{x{2.7cm}}{ } &	\multicolumn{3}{c}{\Bold{Trigo}}\\[0.05cm]\cline{2-4}
		\multicolumn{1}{x{2.7cm}}{\Bold{\raisebox{.4cm}{Año}}}&\multicolumn{1}{x{2.7cm}}{\Bold{Exportaciones}} &	\multicolumn{1}{x{2.7cm}}{\Bold{Importaciones}} &	\multicolumn{1}{x{2.7cm}}{\Bold{Balanza}} \\[0.05cm]
		\hline
		\rowcolor{color1!10!white}&	&&\\[-0.35cm]
		\rowcolor{color1!10!white}	2005	&	2,041.0	&	487,422.7	&	-485,381.7	\\[0.05cm]
		2006	&	262.3	&	450,196.7	&	-449,934.4	\\[0.05cm]
		\rowcolor{color1!10!white}	2007	&	1,774.0	&	493,626.5	&	-491,852.5	\\[0.05cm]
		2008	&	1,709.8	&	473,767.1	&	-472,057.3	\\[0.05cm]
		\rowcolor{color1!10!white}	2009	&	215.7	&	444,051.5	&	-443,835.8	\\[0.05cm]
		2010	&	3.0	&	492,354.0	&	-492,351.0	\\[0.05cm]
		\rowcolor{color1!10!white}	2011	&	201.0	&	516,907.0	&	-516,706.0	\\[0.05cm]
		2012	&	252.9	&	514,445.6	&	-514,192.7	\\[0.05cm]
		\rowcolor{color1!10!white}	2013	&	1,092.6	&	462,758.5	&	-461,665.9	\\[0.05cm]
		2014*	&	706.0	&	364,685.9	&	-363,980.0	\\[0.05cm]
		\hline
		&&&\\[-0.36cm]
		\multicolumn{4}{l}{\footnotesize Fuente: Diplan-MAGA con datos de Banguat (MAGA, 2013).}\\
		\multicolumn{4}{l}{\footnotesize */ Cifras al mes de agosto.}\\
	\end{tabular}\addtocounter{Cuadro}{1}
\end{center}
{\footnotesize	Ministerio de Agricultura, Ganadería y Alimentación (MAGA). 2013. El agro en cifras 2015. En: \url{http://web.maga.gob.gt/download/1agro-cifras2014.pdf}  (Consultado: febrero 2016).}






\begin{center}
	\begin{tabular}{cS[table-format=8]S[table-format=8]S[table-format=8]}
		\multicolumn{4}{l}{\Bold\color{color1!80!black}{Cuadro \theCuadro $\,-$    Comercio exterior de ajonjolí (Sesamum indicum)}}\\
		\multicolumn{4}{l}{\Bold\color{color1!80!black}{según año. República de Guatemala,  2005 - 2014.}}\\
		\multicolumn{4}{l}{(En toneladas métricas.)}\\
		%[0.4cm]
		\hline &&&\\[-0.36cm]  
		\multicolumn{1}{x{2.7cm}}{ } &	\multicolumn{3}{c}{\Bold{Ajonjolí}}\\[0.05cm]\cline{2-4}
		\multicolumn{1}{x{2.7cm}}{\Bold{\raisebox{.4cm}{Año}}}&\multicolumn{1}{x{2.7cm}}{\Bold{Exportaciones}} &	\multicolumn{1}{x{2.7cm}}{\Bold{Importaciones}} &	\multicolumn{1}{x{2.7cm}}{\Bold{Balanza}} \\[0.05cm]
		\hline
		\rowcolor{color1!10!white}&	&&\\[-0.35cm]
		\rowcolor{color1!10!white}	2005	&	28,295.3	&	14,523.6	&	13,771.7	\\[0.05cm]
		2006	&	22,306.7	&	11,313.0	&	10,993.7	\\[0.05cm]
		\rowcolor{color1!10!white}	2007	&	26,652.3	&	6,642.2	&	20,010.1	\\[0.05cm]
		2008	&	13,344.6	&	9,754.8	&	3,589.8	\\[0.05cm]
		\rowcolor{color1!10!white}	2009	&	19,261.9	&	9,591.2	&	9,670.6	\\[0.05cm]
		2010	&	23,143.9	&	8,764.1	&	14,379.8	\\[0.05cm]
		\rowcolor{color1!10!white}	2011	&	17,977.0	&	18,812.0	&	-835.0	\\[0.05cm]
		2012	&	24,812.0	&	9,282.0	&	15,530.1	\\[0.05cm]
		\rowcolor{color1!10!white}	2013	&	34,078.0	&	11,061.2	&	23,016.9	\\[0.05cm]
		2014*	&	17,799.0	&	14,645.6	&	3,153.4	\\[0.05cm]
		\hline
		&&&\\[-0.36cm]
		\multicolumn{4}{l}{\footnotesize Fuente: Diplan-MAGA con datos de Banguat (MAGA, 2013).}\\
		\multicolumn{4}{l}{\footnotesize */ Cifras al mes de agosto.}\\
	\end{tabular}\addtocounter{Cuadro}{1}
\end{center}
{\footnotesize	Ministerio de Agricultura, Ganadería y Alimentación (MAGA). 2013. El agro en cifras 2015. En: \url{http://web.maga.gob.gt/download/1agro-cifras2014.pdf}  (Consultado: febrero 2016).}








{\Bold\color{color1!80!black}{\normalsize Cuadro \theCuadro $\,-$  Disponibilidad per cápita, estimación de pérdidas de los productos de la hoja de balance de alimentos; según producto.}}\\
{\Bold\color{color1!80!black}{\normalsize República de Guatemala, año 2013.}}\\
%(Porcentaje de personas)\\
\begin{center}\fontsize{3.8mm}{1.6em}\selectfont \setlength{\arrayrulewidth}{0.7pt}
	$\ $\\[-2.5cm]
	$\!$\begin{longtable}{x{0.65cm}m{4.3cm}x{2cm}x{3cm}x{2.2cm}x{2cm}}
		\multicolumn{6}{l}{$\ $}\\[-.2cm]
		%			\multicolumn{9}{l}{\Bold\color{color1!80!black}{\normalsize Cuadro \theCuadro $\,-$ Número de habitantes total y por sexo; según grupos quinquenales de edad.}}\\
		%			\multicolumn{9}{l}{\normalsize (Personas)}
		%			\\[-0.1cm]
		\multicolumn{1}{l}{\small \Bold{No.}} & \multicolumn{1}{c}{\small \Bold{{Productos}}}& \multicolumn{1}{c}{\small \Bold{{Precio de}}} &\multicolumn{1}{c}{\small \Bold{Unidad de }}&\multicolumn{1}{c}{\small \Bold{Disponibilidad}}&\multicolumn{1}{c}{\small \Bold{Pérdidas}}\\
		\multicolumn{1}{l}{ } & \multicolumn{1}{c}{ }& \multicolumn{1}{c}{\small \Bold{productos a/}} &\multicolumn{1}{c}{\small \Bold{medida}}&\multicolumn{1}{c}{\small \Bold{per cápita (kg)}}&\multicolumn{1}{c}{\small \Bold{mermas (Tm)}}\\
		&&&&& \\[-0.6cm]
		\multicolumn{1}{l}{$\ $} &  \multicolumn{5}{c}{$\ $} \\[-0.48cm]
		%			\multicolumn{1}{c}{$\ $} & \multicolumn{1}{c}{\Bold{IDH}} & \multicolumn{2}{c}{\Bold{IDH Salud}} & \multicolumn{2}{c}{\Bold{IDH Educación}} & \multicolumn{2}{c}{\Bold{IDH Ingresos}} \\
		%		\multicolumn{1}{c}{} &  \multicolumn{1}{c}{\Bold{Incidencia}} & \multicolumn{1}{c}{\Bold{Error estándar}} &  \multicolumn{1}{c}{ } &\multicolumn{1}{c}{\Bold{Incidencia}} & \multicolumn{1}{c}{\Bold{Error estándar}} &  \multicolumn{1}{c}{ }\\						     
		\hline\endhead
		\hline \multicolumn{6}{r}{\textit{Continúa en la siguiente página}} \\
		\endfoot
		&&&&& \\[-0.9cm]
		\multicolumn{6}{l}{\footnotesize Fuente: INE, 2014 y MAGA 2015.}\\
		\multicolumn{6}{l}{\parbox{15cm}{\footnotesize \textbf{Notas:} - Solo áreas rurales. La tasa de pobreza total se calculó usando la línea oficial de Q.8,283.a/ los precios son promedios anuales. En la medida de lo posible se ha utilizado datos del IPC, donde no datos de precios a mayoristas.}}\\[0.1cm]
		\multicolumn{6}{l}{\parbox{15cm}{\footnotesize n.d: dato no disponible.}}\\[0.1cm]
		\multicolumn{6}{l}{\parbox{15cm}{\footnotesize n.a: no aplica}}\\[0.1cm]
		\multicolumn{6}{l}{\parbox{15cm}{\footnotesize Instituto Nacional de Estadística (INE). 2014. Hoja de Balance de Alimentos 2013. Guatemala. autor. }}\\[0.1cm]
		\multicolumn{6}{l}{\parbox{15cm}{\footnotesize  Ministerio de Agricultura, Ganadería y Alimentación (MAGA). 2015. El Agro en Cifras 2014. Índice de precios al consumidor, diciembre 2013.}}
		\endlastfoot
		\rowcolor{color1!40!white}				     &&&&& \\[-0.5cm]
		%						\multicolumn{1}{l}{\multirow{3}[0]{*}{\Bold{\raisebox{-0.6cm}{ }}}} & \multicolumn{7}{c}{\Bold{Año}} \\\cline{2-8}
		%						\multicolumn{1}{l}{$\ $} &  \multicolumn{7}{c}{$\ $} \\[-0.28cm]
		%						\rowcolor{color1!0!white} &&&&&& \\[-0.28cm]     
		%						\rowcolor{color1!0!white} { } & \multicolumn{1}{c}{2008} & \multicolumn{1}{c}{2009} & \multicolumn{1}{c}{2010} & \multicolumn{1}{c}{2011} & \multicolumn{1}{c}{2012} & \multicolumn{1}{c}{2013} & \multicolumn{1}{c}{2014}  \\ \hline
		%						\rowcolor{color1!40!white} &&&&&&& \\[-0.28cm]
		%							&&&&&& \\[-0.58cm]
				\rowcolor{color1!40!white} \multicolumn{6}{l}{\Bold{	1. Cereales	}}		\\
		\multicolumn{1}{l}{	1	}&		Maíz blanco	&	131.71	&	Quintal	&		&	58066	\\
		\rowcolor{color1!5!white}\multicolumn{1}{l}{	2	}&		Maíz amarillo	&	144.70	&	Quintal	&		&	9453	\\
		\multicolumn{1}{l}{	3	}&		Harina de maíz	&	n.d	&	n.d	&	9.9	&	20863	\\
		\rowcolor{color1!5!white}\multicolumn{1}{l}{	4	}&		Tortilla	&	6.27	&	460 gramos	&	173.1	&		\\
		\multicolumn{1}{l}{	5	}&		Trigo	&	n.d	&	n.d	&		&	47	\\
		\rowcolor{color1!5!white}\multicolumn{1}{l}{	6	}&		Harina de trigo	&	n.d	&	n.d	&	10.4	&	1673	\\
		\multicolumn{1}{l}{	7	}&		Pan y galleta	&	n.d	&	n.d	&	11.8	&		\\
		\rowcolor{color1!5!white}\multicolumn{1}{l}{	8	}&		Pastas alimenticias	&	n.d	&	n.d	&	1.3	&		\\
		\multicolumn{1}{l}{	9	}&		Avena 	&	n.d	&	n.d	&	0	&		\\
		\rowcolor{color1!5!white}\multicolumn{1}{l}{	10	}&		Arroz en granza	&	n.d	&	n.d	&		&	1090	\\
		\multicolumn{1}{l}{	11	}&		Arroz oro	&	322.96	&	Quintal	&	4.8	&	10032	\\
		\rowcolor{color1!5!white}\multicolumn{1}{l}{	12	}&		Maicillo (sorgo)	&	n.d	&	n.d	&		&	1752	\\
		\multicolumn{1}{l}{	13	}&		Tortilla (maicillo)	&	n.d	&	n.d	&	1.3	&		\\
		\rowcolor{color1!40!white} \multicolumn{6}{l}{\Bold{	2.  Leguminosas	}}		\\
		\rowcolor{color1!5!white} \multicolumn{1}{l}{	1	}&		Frijoles	&	5.74	&	460 gramos	&	11.8	&	34688	\\
		\rowcolor{color1!40!white} \multicolumn{6}{l}{\Bold{	3.  Azúcares	}}	\\
		\multicolumn{1}{l}{	1	}&		Caña de azúcar	&	n.d	&	n.d	&		&	269136	\\
		\multicolumn{1}{l}{	2	}&		Azúcar cruda	&	n.d	&	n.d	&		&	633	\\
		\rowcolor{color1!5!white}\multicolumn{1}{l}{	3	}&		Azúcar blanca y refinada	&	3.66	&	460 gramos	&	35.7	&	153	\\
		\multicolumn{1}{l}{	4	}&		Materiales azucarados	&	n.d	&	n.d	&	2	&	7	\\
		\rowcolor{color1!5!white}\multicolumn{1}{l}{	5	}&		Melazas  	&	n.d	&	n.d	&		&	35111	\\
		\rowcolor{color1!40!white} \multicolumn{6}{l}{\Bold{	4.  Tubérculos y raíces	}}	\\
		\multicolumn{1}{l}{	1	}&		Papa	&	6.72	&	460 gramos	&	24.6	&	57914	\\
		\rowcolor{color1!5!white} \multicolumn{1}{l}{	2	}&		Yuca	&	n.d	&	n.d	&	0.3	&	440	\\
		\rowcolor{color1!40!white} \multicolumn{6}{l}{\Bold{	5.  Hortalizas	}}	\\
		\multicolumn{1}{l}{	1	}&		Cebolla	&	5.70	&	460 gramos	&	7.8	&	12169	\\
		\rowcolor{color1!5!white}\multicolumn{1}{l}{	2	}&		Tomate	&	6.49	&	460 gramos	&	13.7	&	37788	\\
		\multicolumn{1}{l}{	3	}&		Zanahoria	&	31.69	&	Red (7 a 8 docenas)	&	2.2	&	8638	\\
		\rowcolor{color1!5!white}\multicolumn{1}{l}{	4	}&		Chile pimiento	&	91.72	&	Caja (90 a 100 unidades)	&	2.6	&	5570	\\
		\multicolumn{1}{l}{	5	}&		Güicoy	&	n.d	&	n.d	&	3.5	&	6502	\\
		\rowcolor{color1!5!white}\multicolumn{1}{l}{	6	}&		Otras hortalizas (lechuga, repollo y coliflor)	&	n.a	&	n.a	&	2.6	&	15786	\\
		\rowcolor{color1!40!white} \multicolumn{6}{l}{\Bold{	6.  Frutas	}}		\\
		\multicolumn{1}{l}{	1	}&		Plátano	&	4.56	&	460 gramos	&	1.9	&	17934	\\
		\rowcolor{color1!5!white} \multicolumn{1}{l}{	2	}&		Banano	&	4.53	&	460 gramos	&	51.2	&	160054	\\
		\multicolumn{1}{l}{	3	}&		Cítricos (limón, naranja)	&	n.d	&	n.d	&	18.1	&	42141	\\
		\rowcolor{color1!5!white} \multicolumn{1}{l}{	4	}&		Aguacate	&	209.74	&	Red (90 a 100 unidades)	&	4.9	&	19192	\\
		\multicolumn{1}{l}{	5	}&		Melón	&	501.11	&	Ciento	&	5.6	&	79652	\\
		\rowcolor{color1!5!white} \multicolumn{1}{l}{	6	}&		Piña	&	328.70	&	Ciento	&	11.2	&	48705	\\
		\multicolumn{1}{l}{	7	}&		Otras frutas (sandía, papaya, manzana y mango)	&	n.a	&	n.a	&	6.2	&	34072	\\
		\rowcolor{color1!40!white} \multicolumn{6}{l}{\Bold{	7.  Carnes	}}		\\
		\multicolumn{1}{l}{	1	}&		Carne con y sin hueso (vacunos)	&	14.18	&	Libra	&	6.7	&	9274	\\
		\rowcolor{color1!5!white}\multicolumn{1}{l}{	2	}&		Vísceras y menudos (vacunos)	&	n.d	&	n.d	&	0.9	&	1179	\\
		\multicolumn{1}{l}{	3	}&		Carne con y sin hueso (cerdos)	&	12.62	&	Libra	&	2.1	&	2722	\\
		\rowcolor{color1!5!white}\multicolumn{1}{l}{	4	}&		Vísceras y menudos (cerdos) 	&	n.d	&	n.d	&	0.2	&	203	\\
		\multicolumn{1}{l}{	5	}&		carne de ave	&	n.a	&	n.a	&	5.4	&	7725	\\
		\rowcolor{color1!5!white}\multicolumn{1}{l}{	6	}&		Embutidos de toda clase	&	n.d	&	n.d	&	1.8	&	2113	\\
		\rowcolor{color1!40!white} \multicolumn{6}{l}{\Bold{	8.  Huevos	}}	\\
		\rowcolor{color1!5!white} \multicolumn{1}{l}{	1	}&		Gallinas en postura (cabezas) / Huevos	&	17,33	&	648 gramos	&	7.6	&	11867	\\
		\rowcolor{color1!40!white} \multicolumn{6}{l}{\Bold{	9.  Pescado y mariscos	}}	\\
		\multicolumn{1}{l}{	1	}&		Pescado	&	n.d	&	n.d	&	2.5	&	7881	\\
		\rowcolor{color1!5!white}\multicolumn{1}{l}{	2	}&		Camarón	&	n.d	&	n.d	&	0.00	&	30	\\
		\rowcolor{color1!40!white} \multicolumn{6}{l}{\Bold{	10.  Productos lácteos	}}	\\
		\multicolumn{1}{l}{	1	}&		Leche fluida cruda de vaca	&	n.d	&	n.d	&	2.2	&	57815	\\
		\rowcolor{color1!5!white} \multicolumn{1}{l}{	2	}&		Leche fluida entera pasteurizada	&	11,84	&	Litro	&	17.5	&		\\
		\multicolumn{1}{l}{	3	}&		Leche fluida semidescremada	&	n.d	&	n.d	&	1.6	&		\\
		\rowcolor{color1!5!white} \multicolumn{1}{l}{	4	}&		Leche en polvo semidescremada	&	n.d	&	n.d	&	0	&		\\
		\multicolumn{1}{l}{	5	}&		Leche pasteurizada/ Leche fluida descremada	&	n.d	&	n.d	&	0.1	&		\\
		\rowcolor{color1!5!white} \multicolumn{1}{l}{	6	}&		Lecha en polvo descremada	&	28,11	&	Bolsa (360 gramos)	&	0.1	&		\\
		\multicolumn{1}{l}{	7	}&		Leche en polvo entera 	&	26,92	&	Bolsa (360 gramos)	&	0.6	&		\\
		\rowcolor{color1!5!white} \multicolumn{1}{l}{	8	}&		Quesos	&	33.98	&	250 ml	&	1.6	&		\\
		\multicolumn{1}{l}{	9	}&		Leche pasteurizada/crema de leche	&	8,76	&	250 ml	&	0.5	&		\\
		\rowcolor{color1!5!white} \multicolumn{1}{l}{	10	}&		Leche pasteurizada/yogur	&	n.d	&	n.d	&	0.3	&		\\
		\rowcolor{color1!40!white} \multicolumn{6}{l}{\Bold{	11.  Aceites y grasas	}}	\\
		\rowcolor{color1!15!white} \multicolumn{6}{l}{\Bold{	Aceites vegetales:	}}		\\
		\multicolumn{1}{l}{	1	}&		Aceite refinado/palma africana	&	n.d	&	n.d	&	1.4	&	2838	\\
		\rowcolor{color1!5!white}\multicolumn{1}{l}{	2	}&		Aceite refinado/soya	&	n.d	&	n.d	&	4	&	7991	\\
		\multicolumn{1}{l}{	3	}&		Aceite refinado/semilla de algodón	&	n.d	&	n.d	&	0.00	&	2	\\
		\rowcolor{color1!5!white}\multicolumn{1}{l}{	4	}&		Aceite refinado/semilla de girasol	&	n.d	&	n.d	&	0.8	&	1690	\\
		\multicolumn{1}{l}{	5	}&		Aceite refinado/aceituna (oliva)	&	n.d	&	n.d	&	0.00	&	100	\\
		\rowcolor{color1!15!white} \multicolumn{6}{l}{\Bold{	Grasas animales:	}}	\\
		\multicolumn{1}{l}{	1	}&		Leche pasteurizada/mantequilla	&	n.d	&	n.d	&	0.1	&	155	\\
		\rowcolor{color1!5!white} \multicolumn{1}{l}{	2	}&		Vacunos faenados/grasa de res	&	n.d	&	n.d	&	0.1	&	5069	\\
		\multicolumn{1}{l}{	3	}&		Cerdos faenados/manteca	&	n.d	&	n.d	&	0.1	&		\\
		\rowcolor{color1!40!white} \multicolumn{6}{l}{\Bold{	12. Alimentos gratificantes	}}	\\
		\multicolumn{1}{l}{	1	}&		Cerveza	&	n.d	&	n.d	&	11.9	&		\\
		\rowcolor{color1!5!white}\multicolumn{1}{l}{	2	}&		melazas/licores	&	n.d	&	n.d	&	3.2	&	805	\\
		\multicolumn{1}{l}{	3	}&		azúcar/bebidas gaseosas	&	10,39	&	1000 ml	&	68.2	&		\\
		\hline
		&&&&&\\[-0.28cm]
		%			\multicolumn{9}{l}{\footnotesize Fuente: Informe Nacional de Desarrollo Humano (PNUD), con base en las Encuestas Nacionales de Condiciones de Vida (Encovi).}
	\end{longtable}
\end{center}



